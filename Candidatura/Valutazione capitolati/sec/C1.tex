\subsection{Capitolato C1 - Knowledge management AI}


\subsubsection{Descrizione}
\begin{itemize}
    \item \textbf{Proponente}: \href{https://www.azzurrodigitale.com/}{AzzurroDigitale};
    \item \textbf{Obiettivo}: realizzazione di una piattaforma \textit{web} per la gestione dei documenti e per l'interazione con il motore di intelligenza artificiale tramite \textit{chat}.
\end{itemize}
L'obiettivo è di facilitare l'accesso alle informazioni, alle regolamentazioni e alle direttive aziendali per i dipendenti. 
Si prevedono nuovi approcci nella formazione e nell'assistenza sul posto di lavoro, mirando a superare la rigidità e la gerarchia tipiche di un tradizionale archivio documentale, concentrandosi sulla fruibilità delle informazioni piuttosto che sull'ordine strutturale. 
Si intende anche favorire una comunicazione più naturale con le macchine e i processi, in modo che sia accessibile anche a persone con un livello di istruzione inferiore. 
Infine, si punta a ridurre o addirittura eliminare gli ostacoli all'ingresso, agevolando l'adempimento di compiti e il rispetto delle regole aziendali. \\
\\
\noindent
Tra i requisiti opzionali emersi durante il colloquio con l'azienda vi sono:
\begin{itemize}
    \item esplorazione degli aspetti legati alla privacy del cliente, proteggendo il contenuto dei documenti inseriti nell'applicativo;
    \item possibilità di interazione vocale con il sistema in situazioni dove non è possibile adoperare una tastiera.
\end{itemize}

\subsubsection{Tecnologie}
\begin{itemize}
    \item \textbf{Node.js}: \textit{open-source, cross-platform JavaScript runtime environment};
    \item \textbf{OpenAI API}: API per l'accesso ai nuovi modelli di intelligenza artificiale sviluppati da OpenAI;
    \item \textbf{Angular}: \textit{framework} per lo sviluppo di applicazioni \textit{web};
    \item \textbf{LangChain}: \textit{framework} che semplifica la creazione di applicazioni utilizzando modelli linguistici di grandi dimensioni.
\end{itemize}


\subsubsection{Considerazioni}
\begin{minipage}[t]{0.45\linewidth}
    \vspace{0pt}
    {\renewcommand{\arraystretch}{1.5}
    \begin{tabular}{p{1\linewidth}}
        \multicolumn{1}{c}{\textbf{Pro}}                        \\
        \midrule
        Ambiti di applicazione molto interessanti per il gruppo \\
        Tecnologie consigliate molto diffuse ed interessanti    \\
        Attenzione su aspetti opzionali quali il rispetto della privacy    \\
        Può migliorare l'esperienza lavorativa                  \\
        \hline
    \end{tabular}
    }
\end{minipage}
\hspace{0.05\linewidth}
\begin{minipage}[t]{0.45\linewidth}
    \vspace{0pt}
    {\renewcommand{\arraystretch}{1.5}
    \begin{tabular}{p{1\linewidth}}
        \multicolumn{1}{c}{\textbf{Contro}} \\
        \midrule
        Il gruppo non ha riscontrato particolari aspetti negativi \\
        \hline
    \end{tabular}
    }
\end{minipage}
\vspace{1em}

Attualmente il gruppo non ha esperienza con alcune delle tecnologie consigliate ma, pur non prevedendo attività di formazione, l'azienda ha messo a disposizione i suoi tecnici per rispondere ad eventuali domande e supportare lo sviluppo.
L'azienda si è mostrata disponibile ad effettuare incontri a scadenza regolare ed è disponibile a fornire strumenti utili alla gestione dell'organizzazione del progetto. 
Tra le successive implementazioni opzionali proposte vi è una particolare attenzione alla privacy dei clienti che dovranno utilizzare l'applicativo, il gruppo ritiene che l'esplorazione di tali tematiche sarà utile a livello formativo.

\subsubsection{Valutazione finale}
Il gruppo ha trovato il capitolato molto interessante e stimolante per le sue applicazioni nel mondo reale. 
Le tecnologie consigliate sono molto diffuse e il gruppo è curioso di approfondirle. 
Il progetto è stato quindi scelto come prima opzione.