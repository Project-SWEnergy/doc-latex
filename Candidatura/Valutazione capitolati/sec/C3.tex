\subsection{Capitolato C3 - Easy meal}


\subsubsection{Descrizione}
\begin{itemize}
    \item \textbf{Proponente}: \href{https://imolainformatica.it/}{Imola Informatica}.
    \item \textbf{Obiettivo}: sviluppo di una applicazione \textit{web responsive} per la gestione di prenotazioni e ordinazioni nei ristoranti.
\end{itemize}
Lo scopo del progetto è quello di migliorare l'esperienza nei ristoranti sia dei clienti che dei ristoratori, operando sulle criticità legate alle operazioni di prenotazione ed ordinazione e semplificando questi processi attraverso un'applicazione \textit{web responsive}.\\ 
Tale applicazione deve quindi permettere agli utenti di prenotare tavoli in modo intuitivo e di personalizzare gli ordini in base alle proprie esigenze e preferenze alimentari. 
Inoltre, dovrà supportare l'interazione collaborativa tra utenti ed il personale del ristorante, semplificare la divisione del conto e incoraggiare le recensioni.\\ 
L'applicativo sviluppato richiede tra i requisiti obbligatori di ottenere una copertura dei test di almeno l'$80\%$. 
Sarà richiesta anche una documentazione relativa alle scelte di progettazione e implementazione, nonché un'analisi del carico massimo supportato e del servizio \textit{cloud} più adatto.

\subsubsection{Tecnologie}
Non sono state fornite indicazioni in merito a tecnologie consigliate, lasciando libertà di scelta al gruppo. 
In particolare una delle richieste riguarda una valutazione approfondita dei diversi servizi di \textit{cloud} disponibili e la conseguente scelta di quello che si adatterebbe meglio al progetto.


\subsubsection{Considerazioni}
\begin{minipage}[t]{0.45\linewidth}
    \vspace{0pt}
    {\renewcommand{\arraystretch}{1.5}
    \begin{tabular}{p{1\linewidth}}
        \multicolumn{1}{c}{\textbf{Pro}} \\
        \midrule
        Lo scopo del progetto suscita l'interesse del gruppo \\
        Gli obiettivi sono chiari ed in gerarchia \\
        L'applicativo è realizzabile sfruttando tecnologie già note al gruppo \\
        \hline
    \end{tabular}
    }
\end{minipage}
\hspace{0.05\linewidth}
\begin{minipage}[t]{0.45\linewidth}
    \vspace{0pt}
    {\renewcommand{\arraystretch}{1.5}
    \begin{tabular}{p{1\linewidth}}
        \multicolumn{1}{c}{\textbf{Contro}} \\
        \midrule
        Scarsa flessibilità rispetto agli obiettivi \\
        La mancata presentazione di tecnologie consigliate potrebbe nascondere criticità per lo sviluppo\\
        \hline
    \end{tabular}
    }
\end{minipage}
\vspace{1em}

\noindent
Il progetto offre l'opportunità di acquisire esperienza nella progettazione e nello sviluppo di un'applicazione complessa, che coinvolge una vasta gamma di tecnologie e competenze, permettendo di affinare le proprie conoscenze \textit{full-stack}, inclusi sviluppo \textit{frontend}, \textit{backend} e \textit{database}.
Inoltre, lo sviluppo richiede una copertura di test elevata, il che consente ai membri del gruppo di acquisire esperienza nella scrittura di test e nel controllo della qualità del \textit{software}.
La gestione dei dati sensibili richiede un'attenzione rigorosa alla sicurezza dei dati, il che potrebbe comportare sfide aggiuntive per il gruppo.


\subsubsection{Valutazione finale}
Il gruppo ha trovato il progetto molto interessante sia perchè presenta una diretta applicazione nel mondo reale, sia perchè l'esperienza maturata durante il suo sviluppo risulterebbe facilmente spendibile in futuro.\\
La possibilità di organizzare incontri con l'azienda a cadenza settimanale consentirebbe al gruppo una buona flessibilità nell'organizzazione del lavoro e nella distribuzione degli impegni durante uno \textit{sprint}, inoltre rende più semplice la comunicazione in caso si abbia necessità di supporto durante lo sviluppo. \\ 
Infine, alcuni membri del gruppo potranno condividere la propria esperienza nell'uso di tecnologie utilizzabili per la realizzazione del progetto, diminuendo il tempo complessivo necessario per la formazione individuale. \\
Il gruppo ha quindi scelto di candidarsi per il capitolato C3.