\subsection{Capitolato C5 - Warehouse Management 3D}


\subsubsection{Descrizione}
\begin{itemize}
    \item \textbf{Proponente}: San Marco Informatica;
    \item \textbf{Obiettivo}: 
	    Creare un'applicazione per visualizzare e simulare gli spazi fisici di
		un magazzino, in modo da monitorare le performance, migliorare lo
		sfruttamento degli spazi e ottimizzare i processi di logistica.

\end{itemize}


\subsubsection{Tecnologie}
\begin{itemize}
    \item Three.js: libreria per la creazione di grafica 3D, in un
	browser web.
\end{itemize}

Per cui il linguaggio consigliato è JavaScript (oppure typescript poi compilato 
in JavaScript). \\
Alternativamente, sono state proposte le seguenti tecnologie:

\begin{itemize}
	\item Unity: C\#;
	\item Unreal engine: C++;
\end{itemize}


\subsubsection{Considerazioni}
\begin{minipage}[t]{0.45\linewidth}
    \vspace{0pt}
    {\renewcommand{\arraystretch}{1.5}
    \begin{tabular}{p{1\linewidth}}
        \multicolumn{1}{c}{\textbf{Pro}} \\
        \midrule
		Il campo di sviluppo ci incuriosisce \\
		Le tecnologie consigliate suscitano il nostro interesse \\
		Gli obiettivi sono chiari ed in gerarchia \\

        \hline
    \end{tabular}
    }
\end{minipage}
\hspace{0.05\linewidth}
\begin{minipage}[t]{0.45\linewidth}
    \vspace{0pt}
    {\renewcommand{\arraystretch}{1.5}
    \begin{tabular}{p{1\linewidth}}
        \multicolumn{1}{c}{\textbf{Contro}} \\
        \midrule

		Abbiamo dubbi sulle applicazioni pratiche del progetto \\

        \hline
    \end{tabular}
    }
\end{minipage}
\vspace{1em}

Le tecnologie proposte sono interessanti. Siamo curiosi di imparare a sviluppare
un'applicazione che gestisce una grafica 3D. Non solo, il programma è pensato
per essere eseguito sul web: una caratteristica che gli permette di essere
\textit{crossplatform}; e che permette a noi di
mostrare l'applicazione sviluppata molto facilmente ad un pubblico futuro. \\
Tuttavia, abbiamo qualche dubbio sul campo di applicazione del progetto. Abbiamo
l'impressione che esistano già soluzioni adeguate, come SketchUp
Web\footnote{\label{footnote:sketchup}
\href{https://www.sketchup.com/it/products/sketchup-for-web}
{https://www.sketchup.com/it/products/sketchup-for-web.}}.
