\subsection{Capitolato C7 - ChatGPT vs BedRock developer analysis}


\subsubsection{Descrizione}
\begin{itemize}
    \item \textbf{Proponente}: \href{https://www.zero12.it/}{Zero12}.
    \item \textbf{Obiettivo}: creazione di un \textit{middleware} per la produzione di \textit{user stories} associate ai requisiti di \textit{business} tramite ChatGPT e AWS BedRock, creazione di \textit{plugin} per VisualStudio Code ed Apple Xcode, comparazione tra le capacità di ChatGPT e AWS BedRock.
\end{itemize}
Il progetto prevede la possibilità dell'utente di caricare documenti, ad esempio requisiti di \textit{business}, all'interno di una \textit{Web Interface}, per i quali, dopo un processo di normalizzazione, ChatGPT e/o AWS BedRock creearanno le \textit{user epic} e le \textit{user stories}, che verranno memorizzate in un \textit{database} ed infine mostrate all'utente tramite \textit{Web Interface}.\\
Uno dei compiti dell'utente sarà quello di fornire dei \textit{feedback} per permettere a ChatGPT e AWS BedRock di migliorare i loro \textit{output} futuri.


\subsubsection{Tecnologie}
\begin{itemize}
    \item \textbf{Amazon Web Servicies}: servizi di \textit{cloud computing} su piattaforma \textit{on demand}.
    \item \textbf{AWS Fargate}: motore di calcolo \textit{Serverless} per \textit{container} funzionante con Amazon Elastic Container Service ed Amazon Elastic Kubernetes.
    \item \textbf{MongoDB}: DBMS non relazionale orientato ai documenti.
\end{itemize}
La tecnologia raccomandata dall'azienda è Amazon Web Servicies. 
In particolare si richiede di utilizzare servizi come AWS Fargate che permette una gestione a container serverless e MongoDB, un database documentale per la gestione di progetti ad eventi.
I linguaggi di programmazione consigliati sono NodeJS, Python e Typescript.


\subsubsection{Considerazioni}
\begin{minipage}[t]{0.45\linewidth}
    \vspace{0pt}
    {\renewcommand{\arraystretch}{1.5}
    \begin{tabular}{p{1\linewidth}}
        \multicolumn{1}{c}{\textbf{Pro}} \\
        \midrule
        Formazione e disponibilità da parte dell'azienda su tecnologie moderne \\
        L'azienda ha suscitato interesse nel gruppo \\
        Uso di tecnologie nuove come ChatGPT e servizi di AWS \\
        \hline
    \end{tabular}
    }
\end{minipage}
\hspace{0.05\linewidth}
\begin{minipage}[t]{0.45\linewidth}
    \vspace{0pt}
    {\renewcommand{\arraystretch}{1.5}
    \begin{tabular}{p{1\linewidth}}
        \multicolumn{1}{c}{\textbf{Contro}} \\
        \midrule
        Lo sviluppo lato Apple non suscita interesse \\
        MVP di difficile individuazione     \\
        \hline
    \end{tabular}
    }
\end{minipage}
\vspace{1em} 

\noindent
Il progetto si rivela interessante soprattutto grazie alle tecnologie proposte e all'opportunità di applicare le competenze acquisite. 
Inoltre, è stata valutata positivamente la disponibilità dichiarata dell'azienda a offrire supporto e formazione durante lo sviluppo.
Alcune specifiche come la creazione di un \textit{plugin} per Apple Xcode si sono rivelate poco stimolanti. 
In aggiunta il progetto e le sue finalità non sono risultate totalmente chiare, la presenza di capitolati ritenuti più stimolanti ha portato il gruppo a scegliere di non avviare un colloquio con l'azienda proponente per risolvere i dubbi presenti.
