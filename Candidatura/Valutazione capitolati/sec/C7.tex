\subsection{Capitolato C7 - ChatGPT vs BedRock developer analysis}


\subsubsection{Descrizione}
\begin{itemize}
    \item \textbf{Proponente}: Zero12.
    \item \textbf{Committenti}: Prof. Tullio Vardanega e Prof. Riccardo Cardin.
    \item \textbf{Obiettivo}: Creazione di un middleware per la produzione di user stories associate ai requisiti di business tramite ChatGPT e AWS BedRock, creazione di plugin per VisualStudio Code e Apple Xcode, comparazione tra le capacità di ChatGPT e AWS BedRock.
\end{itemize}
Si vuole dare la possibilità all'utente di caricare dei documenti come dei requisiti di business all'interno di una Web Interface. Attraverso poi un processo di normalizzazione di tali dati inseriti,
ChatGPT e/o AWS BedRock creearanno le user epic e le user stories, le quali verranno memorizzate in un database ed infine mostrate all'utente tramite la Web Interface citata prima.
Uno dei compiti dell'utente sarà quello di fornire dei feedback per permettere a ChatGPT e AWS BedRock di migliorare i loro output futuri.


\subsubsection{Tecnologie}
\begin{itemize}
    \item Amazon Web Servicies.
    \item AWS fargate.
    \item MongoDB.
\end{itemize}
La tecnologia raccomandata dall'azienda è Amazon Web Servicies. In particolare si richiede di utilizzare servizi come
AWS Fargate che permette una gestione a container serverless e MongoDB, un database documentale per la gestione di progetti ad eventi. \\
I linguaggi di programmazione consigliati sono: NodeJS, Python e Typescript.


\subsubsection{Considerazioni}
\begin{minipage}[t]{0.45\linewidth}
    \vspace{0pt}
    {\renewcommand{\arraystretch}{1.5}
    \begin{tabular}{p{1\linewidth}}
        \multicolumn{1}{c}{\textbf{Pro}} \\
        \midrule
        Formazione e disponibilità da parte dell'azienda su tecnologie moderne. \\
        L'azienda ha suscitato interesse nel gruppo. \\
        Uso di tecnologie nuove come ChatGPT e servizi di AWS. \\
        \hline
    \end{tabular}
    }
\end{minipage}
\hspace{0.05\linewidth}
\begin{minipage}[t]{0.45\linewidth}
    \vspace{0pt}
    {\renewcommand{\arraystretch}{1.5}
    \begin{tabular}{p{1\linewidth}}
        \multicolumn{1}{c}{\textbf{Contro}} \\
        \midrule
        Lo sviluppo lato Apple non risultava interessante. \\
        MVP di difficile individuazione.\\
        \hline
    \end{tabular}
    }
\end{minipage}
\vspace{1em} \\
Il progetto si rivela interessante soprattutto grazie alle tecnologie proposte e all'opportunità di applicare le competenze acquisite. Inoltre, è stata valutata positivamente la disponibilità dichiarata dell'azienda a offrire supporto e formazione durante l'implementazione del progetto.
Tuttavia alcune specifiche, come la creazione di un plugin per Apple Xcode, si sono rivelate poco stimolanti. In aggiunta il progetto e le sue finalità nel suo insieme non sono risultate totalmente chiare.
