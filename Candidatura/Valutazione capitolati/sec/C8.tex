\subsection{Capitolato C8 - JMAP: il nuovo protocollo per la posta elettronica}


\subsubsection{Descrizione}
\begin{itemize}
    \item \textbf{Proponente}: \href{https://zextras.com/it}{Zexstras}.
    \item \textbf{Obiettivo}: sviluppo di un servizio di \textit{demo} testabile e valutazione delle prestazioni e della completezza del protocollo rispetto a quelli attuali in Carbonio.
\end{itemize}
Si vuole comprendere se l'estensione del protocollo JMAP in Carbonio sia vantaggiosa per l'azienda.
Per raggiungere questo obiettivo si richiede lo sviluppo di un servizio di \textit{demo} che copra i requisiti presentati nel capitolato tra cui invio e ricezione di \textit{mail}, gestione di oggetti e cartelle, implementazione di un servizio sincronizzazione.
L'azienda utilizzerà poi il servizio al fine di valutare le prestazioni, la manutenibilità e la completezza del protocollo JMAP, paragonandolo agli attuali protocolli adottati in Carbonio.

\subsubsection{Tecnologie}
\begin{itemize}
    \item \textbf{Java}: linguaggio di programmazione preferenziale.
    \item \textbf{iNPUTmice/jmap}: libreria per l'implementazione del protocollo JMAP.
    \item \textbf{Docker}: \textit{software} per l'esecuzione di processi in ambienti isolabili, minimali e distribuibili.
\end{itemize}
Si ha libertà nella scelta del linguaggio, tuttavia è consigliato Java in quanto linguaggio principalmente utilizzato in Carbonio.
La libreria \href{https://github.com/iNPUTmice/jmap}{iNPUTmice/jmap} è necessaria per l'implementazione del protocollo JMAP.
L'utilizzo di Docker è necessario al fine di avviare più istanze del servizio, consentendo verifiche più efficienti delle funzionalità e delle prestazioni.

\subsubsection{Considerazioni}
\begin{minipage}[t]{0.45\linewidth}
    \vspace{0pt}
    {\renewcommand{\arraystretch}{1.5}
    \begin{tabular}{p{1\linewidth}}
        \multicolumn{1}{c}{\textbf{Pro}} \\
        \midrule
        Campo di applicazione ulteriormente estendibile \\
        Possibilità di usare un linguaggio a scelta     \\
        Prodotto di utilizzo reale che si basa su uno standard nuovo \\
        \hline
    \end{tabular}
    }
\end{minipage}
\hspace{0.05\linewidth}
\begin{minipage}[t]{0.45\linewidth}
    \vspace{0pt}
    {\renewcommand{\arraystretch}{1.5}
    \begin{tabular}{p{1\linewidth}}
        \multicolumn{1}{c}{\textbf{Contro}} \\
        \midrule
        Rispetto ad altri progetti, il capitolato ha suscitato minore interesse \\
        \hline
    \end{tabular}
    }
\end{minipage}
\vspace{1em}

\noindent
Le competenze acquisibili tramite la partecipazione al progetto sono considerate utili in diversi ambiti, soprattutto considerando la necessità di utilizzare \textit{standard} moderni.
L'assenza di vincoli rigidi relativi al linguaggio di programmazione consente di sfruttare al meglio le competenze attuali del gruppo e di adattarsi alle esigenze specifiche del progetto, migliorando le prestazioni e accelerando lo sviluppo.
Nonostante tutto il progetto non è riuscito a stimolare l'interesse del gruppo, che ha deciso di concentrarsi su altri capitolati.