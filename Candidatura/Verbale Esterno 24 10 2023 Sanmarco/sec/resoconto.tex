\section{Resoconto}
Inizio della call: 15:00. Fine della call: 15:30. \\
La call si è tenuta su Google Meet con il rappresentate di Sanmarco Matteo Bassani.
\subsection{Domande}
Il gruppo aveva precedentemente contattato l'azienda via email per fissare l'incontro, riportando le domande più tecniche le quali hanno
trovato risposta sempre via mail, poi successivamente approfondite durante la chiamata.
Le domande esposte al rappresentante di capitolato sono le seguenti:

\begin{enumerate}
    \item Sono forniti degli svg? Oppure c'è bisogno che vengano costruiti dal gruppo?
    \item Su quali dispositivi viene utilizzata l'applicazione?
    \item Vengono forniti dati di utilizzo di qualche magazzino per effettuare le simulazioni?
    \item Vengono offerti dei corsi di formazione per la tecnologia consigliata?
    \item Qual è il target di utilizzo di tale applicazione?
    \item Qual è l'assistenza che l'azienda ci mette a disposizione?
    \item In che modo è possibile simulare gli avg o mezzi di movimento?
    \item Come consigliate di gestire la documentazione? In quali parti il gruppo dovrebbe concentrarsi di più?
\end{enumerate}

\subsection{Risultati dell'incontro}
Il gruppo è rimasto soddisfatto dalla disponibilità dimostrata sia nel
rispondere alle domande che a fornire eventuale aiuto con tecnologie che non sono mai state adoperate da nessun membro.
\\ Da questo meeting è risultato che il rappresentante di Sanmarco Matteo Bassani e in generale il suo team non ha avuto modo di utilizzare la tecnologia consigliata Three.js, ma nonostante questo 
si sono rivelati più che disposti ad aiutare il gruppo, eventualmente mettendo a disposizione una figura apposita per questo tipo di problematica.
\\ Un altro risvolto positivo emerso è stato che da parte dell'azienda viene fornita disponibilità di incontri ogni una o due settimane, per verificare lo stato di avanzamento. Non solo, ma
è venuto fuori che l'azienda è molto flessibile per quanto riguarda i vincoli relativi al progetto: non interessano cose specifiche, ma anche soluzioni e approcci diversi.
\\ \\ Per quanto riguarda le risposte alle domande, di seguito vengono riportate brevemente:

\begin{enumerate}
    \item Gli svg da utilizzare come base possono essere recuperati online da file di esempio. (via mail il gruppo ha ricevuto in allegato due esempi.)
    \item I dispotivi su cui verrebbe utilizzata l'applicazione sono PC con mouse e tastiera. Risulta interessante una proposta funzionante in un tablet.
    \item Non vengono forniti dati di questo carattere per non complicare la situazione con eccessivi dettagli reali.
    \item Non vengono forniti dei veri e proprio corsi di formazione, piuttosto un aiuto da una figura professionale che sa qualcosa in più dal lato tecnico.
    \item Il target a cui si rivolge l'applicazione è relativo a operai e magazzinieri i quali hanno un minimo di formazione, ma non sono utenti esperti. Ecco perchè viene consigliato lo sviluppo di qualcosa di intuitivo e semplice.
    \item L'assistenza dell'azienda si concretizza in incontri ogni una o due settimane per quanto riguarda la gestione del progetto. Poi viene anche fornito aiuto riguadante il lato tecnico.
    \item Si è scoperto che gli avg fanno parte degli obiettivi opzionali, fungono da esempio solo per far capire quello che c'è internamente e attorno al magazzino.
    \item Al gruppo viene data completa libertà per quanto riguarda la gestione della documentazione. \\
\end{enumerate}

\subsection{Decisioni prese}
Al termine dell'incontro i componenti del gruppo si sono ritrovati per discutere tra loro.\\
Non sono state prese decisioni ufficiali, però siccome l'azienda ha fatto una buona impressione è emerso un eventuale cambio di decisione relativo alla preferenza di che progetto svolgere.