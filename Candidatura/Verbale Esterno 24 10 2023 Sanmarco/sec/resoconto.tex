\section{Resoconto}
Inizio della call: 15:00. Fine della call: 15:30. \\
La call si è tenuta su Google Meet con il rappresentate di Sanmarco, Matteo Bassani.
\subsection{Domande}
Il gruppo aveva precedentemente contattato l'azienda via email per organizzare l'incontro, sottoponendo le domande più tecniche alle quali avevano ricevuto risposta tramite email. Successivamente, durante la chiamata, questi temi sono stati ulteriormente approfonditi.
Le domande esposte al rappresentante sono le seguenti:

\begin{enumerate}
    \item Sono forniti degli svg? Oppure c'è bisogno che vengano costruiti dal gruppo?
    \item Su quali dispositivi viene utilizzata l'applicazione?
    \item Vengono forniti dati di utilizzo di qualche magazzino per effettuare le simulazioni?
    \item Vengono offerti dei corsi di formazione per la tecnologia consigliata?
    \item Qual è il target di utilizzo di tale applicazione?
    \item Qual è l'assistenza che l'azienda ci mette a disposizione?
    \item In che modo è possibile simulare gli avg o mezzi di movimento?
    \item Come consigliate di gestire la documentazione? In quali parti il gruppo dovrebbe concentrarsi di più?
\end{enumerate}

\subsection{Risultati dell'incontro}
Il gruppo è rimasto estremamente soddisfatto della disponibilità dimostrata, sia nel rispondere alle domande, sia nel fornire eventuale assistenza per l'utilizzo di tecnologie mai sperimentate prima da nessun membro. 
\\ Durante questo incontro, è emerso che il rappresentante di San Marco, Matteo Bassani, e il suo team, non hanno esperienza nell'uso della tecnologia Three.js, ma nonostante ciò si sono dimostrati più che disposti ad aiutare il gruppo, addirittura mettendo a disposizione una figura esterna specializzata per affrontare questa sfida.
\\ Un altro aspetto estremamente positivo è emerso durante la discussione, ovvero l'azienda offre la possibilità di incontri regolari, ogni una o due settimane, per verificare lo stato di avanzamento del progetto. Inoltre, è stato evidente che l'azienda è estremamente flessibile riguardo ai requisiti del progetto; non si concentrano esclusivamente su soluzioni specifiche, ma sono aperti ad esplorare diverse proposte e approcci.
\\ \\ Per quanto riguarda le risposte alle domande, di seguito vengono riportate brevemente:

\begin{enumerate}
    \item Gli svg da utilizzare come base possono essere recuperati online da file di esempio. (via mail il gruppo ha ricevuto in allegato due esempi.)
    \item I dispotivi su cui verrebbe utilizzata l'applicazione sono PC con mouse e tastiera. Risulta interessante una proposta funzionante in un tablet.
    \item Non vengono forniti dati di questo carattere per non complicare la situazione con eccessivi dettagli reali.
    \item Non vengono forniti dei veri e proprio corsi di formazione, piuttosto un aiuto da una figura professionale che sa qualcosa in più dal lato tecnico.
    \item Il target a cui si rivolge l'applicazione è relativo a operai e magazzinieri i quali hanno un minimo di formazione, ma non sono utenti esperti. Ecco perchè viene consigliato lo sviluppo di qualcosa di intuitivo e semplice.
    \item L'assistenza dell'azienda si concretizza in incontri ogni una o due settimane per quanto riguarda la gestione del progetto. Poi viene anche fornito aiuto riguadante il lato tecnico.
    \item Si è scoperto che gli avg fanno parte degli obiettivi opzionali, fungono da esempio solo per far capire quello che c'è internamente e attorno al magazzino.
    \item Al gruppo viene data completa libertà per quanto riguarda la gestione della documentazione. \\
\end{enumerate}

\subsection{Decisioni prese}
Al termine dell'incontro, i membri del gruppo si sono riuniti per una discussione informale tra loro. \\
Sebbene non siano state prese decisioni ufficiali, l'azienda ha lasciato un'ottima impressione, aprendo la possibilità di riconsiderare la preferenza riguardo a quale progetto svolgere.
