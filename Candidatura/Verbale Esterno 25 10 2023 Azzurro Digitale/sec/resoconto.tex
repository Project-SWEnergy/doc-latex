\section{Resoconto}
La chiamata è iniziata alle 15:00 e si è conclusa alle 15:30. \\
La \textit{call} si è tenuta su Google Meet con i rappresentati di Azzurro
Digitale: Giuseppe Caliendo e Carlo Davanzo.
\subsection{Domande}
Precedentemente il gruppo ha contattato l'azienda per organizzare l'incontro, 
anticipando le domande più tecniche.
Le domande esposte ai rappresentanti sono le seguenti:

\begin{enumerate}
	\item Qual è l'assistenza che l'azienda ci mette a disposizione? Tempo medio a
  settimana.
	\item Ci sono alcuni parti della progettazione che vorreste che fossero più precise?
	\item I documenti da usare, da cui estrarre le informazioni delle risposte, li
	  fornite voi, oppure consigliate qualche template?
	\item Ci sarà qualche tipo di formazione sulle tecnologie consigliate?
	\item In quale modo sarà gestita la \textit{repository} di github che ci mettete a 
		disposizione? Si tratta solo di una repo?
	\item Avete già pensato a qualche libreria per sviluppare l'interfaccia?
\end{enumerate}

\subsection{Risultati dell'incontro}
Il gruppo è rimasto molto soddisfatto della disponibilità dimostrata nel 
rispondere alle domande. \\ 
L'azienda ha espresso chiaramente gli obiettivi del progetto: Caliendo ha già
sviluppato un prototipo del software; per questo motivo i requisiti minimi
risultano facilmente raggiungibili. Piuttosto, l'azienda richiede al gruppo di
concentrarsi sullo sviluppo di funzionalità aggiuntive, che possano rendere il
software più completo e appetibile. \\
Di seguito sono riassunte le risposte alle domande:

\begin{enumerate}
	\item L'azienda è disponibile a organizzare degli incontri bisettimanali
		della durata di circa un'ora e mezza.

	\item Non ci sono particolari aspetti della progettazione che interessano
		l'azienda, invece richiedono che il codice prodotto sia ben
		commentato.

	\item Sono disponibili a fornire dei documenti di esempio e a consigliare
		qualche fonte dal quale reperire eventuali documenti utili.

	\item L'azienda non offre dei corsi di formazione, tuttavia le persone che
		seguono il gruppo conoscono le tecnologie consigliate e sono disponibili
		a rispondere a eventuali domande in modo tempestivo, preciso ed
		efficace.

	\item L'azienda mette a disposizione una \textit{repository github} interna, 
		perché ha dei server ad essa collegati. In effetti, l'azienda offre la
		possibilità di lavorare direttamente sui loro server attraverso le
		\textit{github action}.

	\item Il gruppo ha totale autonomia nella scelta dell'interfaccia che
		preferisce implementare. Purché rimangano ben distinti due ruoli: 
		un utente che carica i documenti; e un utente che li consulta mediante un
		\textit{chatbot}.
\end{enumerate}