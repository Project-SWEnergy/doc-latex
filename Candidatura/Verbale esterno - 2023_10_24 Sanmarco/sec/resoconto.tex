\section{Resoconto}
La chiamata è iniziata alle 15:00 e si è conclusa alle 15:30. \\
La \textit{call} si è tenuta su Google Meet con il rappresentate di Sanmarco 
Matteo Bassani.
\subsection{Domande}
Precedentemente il gruppo ha contattato l'azienda per organizzare l'incontro, 
anticipando le domande più tecniche.
Le domande esposte al rappresentante sono le seguenti:

\begin{enumerate}
    \item Sono forniti degli svg? Oppure c'è bisogno che vengano costruiti dal 
		gruppo?
    \item Su quali dispositivi viene utilizzata l'applicazione?
    \item Vengono forniti dati di utilizzo di qualche magazzino per effettuare 
		le simulazioni?
    \item Vengono offerti dei corsi di formazione per la tecnologia consigliata?
	\item Qual è il \textit{target} di utilizzo di tale applicazione?
    \item Qual è l'assistenza che l'azienda ci mette a disposizione?
    \item In che modo è possibile simulare gli avg o mezzi di movimento?
    \item Come consigliate di gestire la documentazione? In quali parti il 
		gruppo dovrebbe concentrarsi di più?
\end{enumerate}

\subsection{Risultati dell'incontro}
Il gruppo è rimasto molto soddisfatto della disponibilità dimostrata nel 
rispondere alle domande. \\ 
Durante questo incontro, è emerso che Bassani e il suo team non hanno 
esperienza nell'uso di \textit{Three.js}; comunque il rappresentante si è reso 
disponibile
a contattare qualche suo collega che ha già lavorato con questa tecnologia. \\
Il gruppo ha apprezzato che Bassani si sia offerto a tenere incontri 
regolari fino ad un'ora alla settimana. Inoltre, il \textit{product owner} è
molto flessibile rispetto ai requisiti minimi del progetto: non impone soluzioni
specifiche, ma è aperto a proposte e approcci diversi. \\

Di seguito sono brevemente riportate le risposte alle domande:

\begin{enumerate}
    \item Gli svg da utilizzare come base possono essere recuperati online da 
		file di esempio. (via mail il gruppo ha ricevuto in allegato due 
		esempi.)
    \item I dispotivi su cui verrebbe utilizzata l'applicazione sono PC con 
		mouse e tastiera. 
		Considerando la possibilità di estendere l'applicazione ai tablet.
    \item Non vengono forniti dati di questo carattere per non complicare 
		lo sviluppo con eccessivi dettagli reali.
	\item Non sono forniti dei corsi di formazione, ma è possibile richiedere un
		qualche tipo di colloquio con un tecnico specializzato.
	\item Il \textit{target} a cui si rivolge l'applicazione è formato da operai 
		e magazzinieri, che hanno una qualche forma di formazione, ma non sono 
		utenti esperti. Per questo motivo, viene consigliato lo sviluppo di 
		un'interfaccia utente intuitiva e semplice.
    \item L'assistenza dell'azienda si concretizza in incontri 
		con cadenza settimanale o bisettimanale per discutere inerentemente al
		progresso del progetto.
	\item Gli avg sono un obiettivo opzionale, che si può implementare.
    \item Al gruppo viene data completa libertà per quanto riguarda la gestione 
		della documentazione. \\
\end{enumerate}
