\section{Resoconto}
La chiamata è iniziata alle 15:00 e si è conclusa alle 15:30, si è svolta nella piattaforma Google Meet con il rappresentate di Sanmarco Informatica Matteo Bassani.
\subsection{Domande}
Precedentemente il gruppo ha contattato l'azienda per organizzare l'incontro, anticipando le domande più tecniche.
Le domande esposte al referente sono le seguenti:
\begin{enumerate}
    \item "Sono forniti dei \textit{file} svg? Oppure è necessario che vengano costruiti dal gruppo?";
    \item "Su quali dispositivi verrà utilizzata l'applicazione?";
    \item "Vengono forniti dati reali di utilizzo di qualche magazzino per effettuare le simulazioni?";
    \item "Vengono offerti dei corsi di formazione per la tecnologia consigliata?";
	\item "Qual è il \textit{target} di utilizzo di tale applicazione?"
    \item "Qual è l'assistenza che l'azienda può metterci a disposizione?";
    \item "In che modo è possibile simulare gli avg o mezzi di movimento?";
    \item "Come consigliate di gestire la documentazione? In quali parti il gruppo dovrebbe concentrarsi di più?".
\end{enumerate}

\subsection{Esito dell'incontro}
Il gruppo è rimasto molto soddisfatto della disponibilità dimostrata nel rispondere alle domande. \\ 
Durante l'incontro è emerso che, a differenza di altri \textit{team} all'interno della loro azienda, il nostro riferimento non vanta un'esperienza pratica nell'uso della libreria \textit{Three.js}, il referente ha però confermato la disponibilità dei colleghi nel fornire assistenza durante lo sviluppo. \\
Il gruppo ha apprezzato che l'azienda si sia mostrata disponibile a tenere incontri regolari, fino ad un massimale di un'ora alla settimana. \\
Il \textit{product owner} è risultato essere molto flessibile rispetto ai requisiti minimi del progetto: non impone soluzioni specifiche, ma è aperto a proposte ed approcci diversi. \\

\noindent
Di seguito sono brevemente riportate le risposte alle domande:
\begin{enumerate}
    \item I \textit{file} svg da utilizzare come base possono facilmente essere recuperati \textit{online}, inoltre il referente ha inviato tramite \textit{email} due esempi.
    \item I dispotivi su cui verrebbe utilizzata l'applicazione sono \textit{computer} dotati di \textit{mouse} e tastiera. Viene valutata positivamente la possibilità di estendere l'applicazione a dispositivi quali \textit{tablet}.
    \item Non vengono forniti dati di questa tipologia per non complicare lo sviluppo con eccessivi dettagli reali.
	\item Non sono previsti corsi di formazione, tuttavia è possibile richiedere colloqui con i tecnici presenti in azienda per risolvere le problematiche relative allo sviluppo.
	\item Il \textit{target} a cui si rivolge l'applicazione è formato da magazzinieri ma anche impiegati che, pur non essendo utenti esperti, dispongono di una buona formazione. Per questo motivo viene consigliato lo sviluppo di un'interfaccia utente intuitiva e semplice sia per la navigazione degli spazi che per le attività di pianificazione.
    \item L'assistenza dell'azienda si concretizza in incontri con cadenza settimanale o bisettimanale per discutere il progresso del progetto.
	\item L'integrazione con gli avg sono un obiettivo opzionale, di cui si può valutare l'implementazione.
    \item Al gruppo viene data completa libertà per quanto riguarda la gestione della documentazione. 
\end{enumerate}

\noindent
L'incontro si è concluso con una conversazione informale volta a conoscere meglio il profilo dell'azienda proponente.

\section{Conclusioni}
Le informazioni raccolte durante l'incontro verranno riorganizzate e discusse dai membri del gruppo.
Il confronto avverrà in data 27/10/2023 e le sue conclusioni saranno riportate nel Verbale interno 05.