\section{Ordine del giorno}
\begin{itemize}
    \item Redazione domande per le aziende di interesse;
    \item Redazione domande per il docente;
    \item Condivisione delle conoscenze in merito all'utilizzo di GitHub e LaTeX;
    \item Definizione delle convenzioni relative all'utilizzo di GitHub e LaTeX.
\end{itemize}

\section{Resoconto}
Unendo quanto discusso durante la (INSERIRE LINK VERBALE 02) e nelle conversazioni avvenute nei canali di comunicazione, il gruppo ha concretizzato un elenco di domande da porre alle aziende proponenti i capitolati di interesse.\\
\noindent
Sono stati quindi contattati tramite \textit{email} i referenti delle tre aziende AzzurroDigitale, Imola Informatica e SanMarco Informatica, comunicando loro in anticipo alcune delle domande del gruppo, e chiedendo disponibilità per fissare un colloquio.\\
\noindent
\\
Essendo sorti dubbi in merito alla metodologia di aggiudicazione degli appalti, si è scelto di contattare con le stesse modalità anche il docente Tullio Vardanega.\\
\noindent
\\
Terminata l'analisi dei capitolati si è iniziato a suddividere il lavoro inerente alla stesura dei documenti richiesti per la partecipazione alle gare d'appalto.\\
\noindent
Per l'assegnazione dei compiti si è scelto di utilizzare lo strumento delle \textit{Issues}, disponibile sulla piattaforma GitHub, le specifiche relative al suo utilizzo saranno inserite nel documento \textit{Way of Working} insieme alle convenzioni definite nelle riunioni precedenti.\\
\noindent
Nel documento "Valutazione capitolati" la redazione delle sezioni inerenti alle aziende di interesse con cui si terranno i colloqui è stata preventivamente suddivisa tra i membri del gruppo, per l'effettivo avvio della redazione si attenderà però lo svolgersi di tali incontri.\\
\noindent
\\
Si è deciso infine di sfruttare il tempo a disposizione per avviare un processo formativo in cui sono state condivise conoscenze su argomenti quali l'utilizzo di GitHub e LaTeX, con cui alcuni membri del gruppo non avevano dimestichezza.

\section{Assegnazione degli incarichi}
In tabella vengono riportate le attività assegnate ai membri del gruppo, sono stati inoltre assegnati i ruoli relativi ai documenti da redigere.

\begin{center}
    {
    \renewcommand{\arraystretch}{1.5}
    \begin{tabular}{p{0.55\linewidth}|p{0.30\linewidth}|p{0.05\linewidth}}
        \textbf{Descrizione}    &   \textbf{Assegnatario}   & \textbf{Rif.}     \\
        \hline
        Redazione valutazione capitolato C1 & Matteo Bando              & \#14  \\
        \hline
        Redazione valutazione capitolato C2 & Giacomo Gualato           & \#6   \\
        \hline
        Redazione valutazione capitolato C3 & Alessandro Tigani Sava    & \#7   \\
        \hline
        Redazione valutazione capitolato C4 & Niccolò Carlesso          & \#8   \\
        \hline
        Redazione valutazione capitolato C5 & Carlo Rosso               & \#9   \\
        \hline
        Redazione valutazione capitolato C6 & Carlo Rosso               & \#10  \\
        \hline
        Redazione valutazione capitolato C7 & Davide Maffei             & \#11  \\
        \hline
        Redazione valutazione capitolato C8 & Alessandro Tigani Sava    & \#12  \\
        \hline
        Redazione valutazione capitolato C9 & Niccolò Carlesso          & \#13  \\
        \hline
        Redazione verbale interno 03 del 18/10/2023    & Alessandro Tigani Sava & \#3 \\
        \hline
        Verifica verbale interno 03 del 18/10/2023     & Carlo Rosso    & \#4   \\
        \hline
        Approvazione verbale interno 03 del 18/10/2023 & Davide Maffei  & \#5   \\
        \hline
        \textit{Way of Working}: redazione convenzioni per la scrittura di documenti e glossario & Alessandro Tigani Sava & \#19 \\
        \hline
        \textit{Way of Working}: redazione convenzioni per le pratiche di versionamento ed assegnazione degli incarichi & Matteo Bando & \#18 \\
    \end{tabular}
    }
    \end{center}