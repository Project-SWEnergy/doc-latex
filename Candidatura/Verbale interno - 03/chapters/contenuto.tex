\section{Ordine del giorno}
\begin{itemize}
    \item Redazione domande per le aziende di interesse.
    \item Redazione domande per il docente.
    \item Condivisione delle conoscenze in merito all'utilizzo di GitHub e LaTeX.
    \item Definizione delle convenzioni relative all'utilizzo di GitHub e LaTeX.
\end{itemize}

\section{Resoconto}
Unendo quanto discusso durante la riunione del 18/10/2023 (in riferimento al verbale interno 02, sezione \S3) e nelle conversazioni avvenute nei canali di comunicazione, il gruppo ha concretizzato un elenco di domande da porre alle aziende proponenti i capitolati di interesse.\\
\noindent
Sono stati quindi contattati tramite \textit{email} i referenti delle tre aziende AzzurroDigitale, Imola Informatica e SanMarco Informatica, comunicando loro in anticipo alcune delle domande del gruppo, e chiedendo disponibilità per fissare un colloquio.\\
\noindent
\\
Essendo sorti dubbi in merito alla metodologia di aggiudicazione degli appalti, si è scelto di contattare con le stesse modalità anche il docente Tullio Vardanega.\\
\noindent
\\
Terminata l'analisi dei capitolati si è iniziato a suddividere il lavoro inerente alla stesura dei documenti richiesti per la partecipazione alle gare d'appalto.\\
\noindent
Per l'assegnazione dei compiti si è scelto di utilizzare lo strumento delle \textit{Issues}, disponibile sulla piattaforma GitHub, le specifiche relative al suo utilizzo saranno inserite nel documento \textit{Way of Working} insieme alle convenzioni definite nelle riunioni precedenti.\\
\noindent
Nel documento "Valutazione capitolati" la redazione delle sezioni inerenti alle aziende di interesse con cui si terranno i colloqui è stata preventivamente suddivisa tra i membri del gruppo, per l'effettivo avvio della redazione si attenderà però lo svolgersi di tali incontri.\\
\noindent
\\
Si è deciso infine di sfruttare il tempo a disposizione per avviare un processo formativo in cui sono state condivise conoscenze su argomenti quali l'utilizzo di GitHub e LaTeX, con cui alcuni membri del gruppo non avevano dimestichezza.

\section{Assegnazione degli incarichi}
In tabella vengono riportate le attività assegnate ai membri del gruppo, sono stati inoltre assegnati i ruoli relativi ai documenti da redigere.

\begin{center}
    {
    \renewcommand{\arraystretch}{1.5}
    \begin{tabular}{p{0.30\linewidth}|p{0.55\linewidth}|p{0.05\linewidth}}
        \textbf{Assegnatario}                   &   \textbf{Descrizione}                        & \textbf{Rif.}     \\
        \hline
        \multirow{2}{*}{Alessandro Tigani Sava} 
						& Redazione valutazione capitolati C3, C7 e C8							& \#7, \#11, \#12	\\
        \cline{2-3}
                        & \textit{Way of Working}: redazione convenzioni per la scrittura di documenti e glossario & \#19 \\
        \cline{2-3}
        	           	& Approvazione verbale interno 03										& \#5				\\
        \hline 
        Carlo Rosso     & Approvazione valutazione capitolati									& \#31  			\\
        \hline
		Davide Maffei	& Redazione valutazione capitolati C1 e C5								& \#9, \#14 		\\
        \hline
        \multirow{2}{*}{Giacomo Gualato}        & Redazione valutazione capitolati C2 e C6  	& \#6, \#10 		\\
        \cline{2-3}
                        & Redazione verbale interno 03											& \#3				\\
        \hline
        \multirow{2}{*}{Matteo Bando}           & Verifica valutazione capitolati   			& \#14				\\
        \cline{2-3}
		& \textit{Way of Working}: redazione convenzioni per le pratiche di versionamento ed assegnazione degli incarichi 		
																								& \#18				\\
        \hline
        \multirow{2}{*}{Niccolò Carlesso}       & Redazione valutazione capitolati C4 e C9		& \#8, \#13 		\\
        \cline{2-3}
						& Verifica verbale interno 03											& \#4				\\
    \end{tabular}
    }
    \end{center}
