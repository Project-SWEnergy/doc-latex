\section{Introduzione}

\subsection{Scopo del documento}
Nel presente documento viene presentata una descrizione dettagliata del prodotto, basata sull'analisi dei bisogni dell'utente emersi durante l'esaminazione del capitolato$^G$ e attraverso gli incontri con l'azienda proponente. 
La modellazione viene realizzata tramite UML$^G$, identificando in modo approfondito requisiti e attori presenti nel progetto. 
Questo approccio consente di descrivere in dettaglio le varie componenti del prodotto e di indicare la struttura di ciascuna funzionalità.

\subsection{Glossario}
Al fine di evitare ambiguità linguistiche e garantire un'utilizzazione coerente delle terminologie nei documenti, il gruppo ha redatto un documento interno chiamato "Glossario". Questo
documento definisce in modo chiaro e preciso i termini che potrebbero generare ambiguità
o incomprensione nel testo. I termini presenti nel Glossario sono identificati da una 'G' ad
apice (per esempio parola\g).

\subsection{Riferimenti}
A seguito di ogni risorsa \textit{web} viene riportata la data di ultima verifica. 
\subsubsection{Riferimenti normativi}
\begin{itemize}
    \item \href{https://www.math.unipd.it/~tullio/IS-1/2023/Progetto/C3.pdf}{Capitolato C3 - \textit{Easy Meal}} [19/05/2024].
    \item \href{https://project-swenergy.github.io/}{Norme di progetto v3.0.0} [19/05/2024]. 
    \item \href{https://www.math.unipd.it/~tullio/IS-1/2023/Dispense/PD2.pdf}{Regolamento progetto didattico} [19/05/2024]. 
\end{itemize}

\subsubsection{Riferimenti informativi}
\begin{itemize}
    \item \href{https://project-swenergy.github.io/}{Glossario v2.0.0} [19/05/2024].
    \item \href{https://www.math.unipd.it/~tullio/IS-1/2023/Dispense/T5.pdf}{Lezione T05 - Analisi dei requisiti} [19/05/2024].
    \item \href{https://www.math.unipd.it/~rcardin/swea/2023/Diagrammi%20delle%20Classi.pdf}{Diagrammi delle classi} [19/05/2024].
    \item \href{https://www.math.unipd.it/~rcardin/swea/2022/Diagrammi%20Use%20Case.pdf}{Diagrammi dei casi d'uso} [19/05/2024].
    \item \href{https://bsituos.weebly.com/uploads/2/5/2/5/25253721/applying-uml-and-patterns-3rd.pdf}{Linguaggio UML} [19/05/2024].
\end{itemize}
