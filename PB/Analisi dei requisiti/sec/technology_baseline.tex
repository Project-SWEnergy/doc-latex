\section{\textit{Technology baseline}}
Le tecnologie scelte per l'analisi sono le seguenti:

\begin{itemize}
	\item \textbf{\textit{TypeScript}}: un linguaggio di programmazione che
	      estende \textit{JavaScript} aggiungendo il supporto ai tipi. La scelta
	      è motivata dalla possibilità di sviluppare sia il \textit{frontend}
	      che il \textit{backend} dell'applicazione \textit{web} utilizzando un
	      unico linguaggio; dalla capacità dell'azienda proponente di fornire
	      supporto; e dalla maggiore comprensibilità rispetto a
	      \textit{JavaScript}, oltre ad una più accurata analisi statica del
	      codice.

	\item \textbf{\textit{Angular}}: un \textit{framework} per lo sviluppo di
	      applicazioni \textit{web single page} in \textit{TypeScript}. La
	      scelta si basa sull'enfasi sull'esperienza pregressa di uno dei membri
	      del gruppo e sulla consapevolezza che esistono alternative simili e
	      ugualmente funzionali.

	\item \textbf{\textit{Node.js}}: un \textit{framework} per lo sviluppo di
	      applicazioni \textit{web} in \textit{JavaScript}. Si opta per questo
	      \textit{framework} siccome il proponente lo consiglia esplicitamente per lo
	      sviluppo del \textit{backend} in \textit{TypeScript}.

	\item \textbf{\textit{PostgreSQL}}: un \textit{database} relazionale
	      \textit{open source}. La decisione di utilizzare questo
	      \textit{database} è supportata dalla familiarità dell'intero gruppo
	      con questa tecnologia e dalla valutazione di SWEnergy, secondo cui le
	      tecnologie elencate fino a questo punto comportano un grado di
	      complessità adeguato per il progetto.
\end{itemize}

Fino a questo punto, sono state elencate le tecnologie scelte dal gruppo.
Tuttavia, SWEnergy non si sta impegnando attivamente allo studio delle
tecnologie, in quanto il gruppo ritiene che sia precoce. Piuttosto, può darsi
il caso che qualche membro del gruppo nel tempo libero approfondisca le proprie
conoscenze.
