\subusecasebase{Selezione della modalità di divisione del conto}
\label{usecase:Selezione della modalità di divisione del conto}

\begin{figure}[h]
	\centering
	\includegraphics[width=0.9\textwidth]{./uml/UCB15.png} 
	\caption{Selezione della modalità di divisione del conto}
	\label{fig:UCB14}
  \end{figure}

\begin{itemize}
	\item \textbf{Attore principale:} Utente base.
	
	\item \textbf{Precondizione:}
	\begin{itemize}
		\item L'Utente base si trova nella sezione di creazione di una prenotazione (vedi \autoref{usecase:Prenotazione di un tavolo}).
	\end{itemize}

	\item \textbf{Postcondizione:}
	      L'Utente base ha selezionato la modalità di divisione del conto.
	\item \textbf{Scenario principale:}
	      \begin{enumerate}
		      \item L'Utente base seleziona la modalità di divisione del conto
		            tra quelle disponibili:
					\begin{itemize}
						\item \textbf{Divisione equa:} il conto viene diviso in parti
							  uguali tra tutti gli Utenti base che hanno condiviso la
							  prenotazione. L'Utente base può scegliere di pagare più di
							  una quota.
		  
						\item \textbf{Proporzionale:} l'Utente base paga tutti i piatti che ha ordinato.
					\end{itemize}

		      \item L'Utente base conferma la modalità di divisione del conto;

		      \item Il Sistema memorizza la modalità di divisione del conto.
	      \end{enumerate}

	\item \textbf{Scenario secondario:}
		  \begin{itemize}
			  \item \autoref{usecase:Visualizzazione errore divisione del conto già effettuata} Modalità di divisione del conto già effettuata:
				\begin{enumerate}
					\item L'Utente base seleziona la modalità di divisione del conto
						tra quelle disponibili;
	
					\item L'Utente base conferma la modalità di divisione del conto;
	
					\item Il Sistema mostra un messaggio di errore e spiega che la
						modalità di divisione del conto è già stata scelta.
				\end{enumerate}
		  \end{itemize}

\end{itemize}

