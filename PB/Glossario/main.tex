\documentclass[a4paper, 12pt]{article}

\newcommand{\template}{../../Templates}
\usepackage{tocloft}
\usepackage{\template/package}
\graphicspath{{../../Assets}}

\newcommand{\subsec}[1]{%
    \phantomsection
    \section*{#1}%
    \addcontentsline{toc}{section}{#1}%
}

\newcommand{\Titolo}{Glossario}
\newcommand{\Data}{19/05/2024}
\newcommand{\Versione}{2.0.0}
\newcommand{\Descrizione}{Documento esplicativo del significato di tutti i termini tecnici o poco chiari presenti nella documentazione.}
\newcommand{\Stato}{Approvato}
\newcommand{\Verificatori}{
    Alessandro Tigani Sava \\
    & Giacomo Gualato \\
    & Matteo Bando \\
    & Niccolò Carlesso
}
\newcommand{\Destinatari}{
    Prof. Tullio Vardanega \\ 
    & Prof. Riccardo Cardin
}
\newcommand{\Redattori}{
    Alessandro Tigani Sava \\
    & Matteo Bando\\
    & Niccolò Carlesso
}
\newcommand{\Approvatori}{Davide Maffei}

\newcommand{\Gruppo}{SWEnergy}
\newcommand{\Mail}{\href{mailto:project.swenergy@gmail.com}{project.swenergy@gmail.com}}
\newcommand{\g}{$^G$ }

\renewcommand\familydefault{\sfdefault} % Set default font family to sans-serif
\linespread{1.5}

\hypersetup{
	pdfmenubar=true,            % show Acrobat’s menu?
	pdfstartview={FitH},        % fits the width of the page to the window
	colorlinks=true,            % false: boxed links; true: colored links
	linkcolor=black,            % color of internal links (change box color with linkbordercolor)
	% citecolor=green,          % color of links to bibliography
	% filecolor=magenta,        % color of file links
	urlcolor=[RGB]{156,1,198}   % color of external links
}

% Define the subsubsubsection command
\newcounter{subsubsubsection}[subsubsection]
\renewcommand{\thesubsubsubsection}{\thesubsubsection.\arabic{subsubsubsection}}
\newcommand{\subsubsubsection}[1]{\refstepcounter{subsubsubsection}\subsubsection*{\thesubsubsubsection\quad#1}}

% Define the subsubsubsubsection command
\newcounter{subsubsubsubsection}[subsubsubsection]
\renewcommand{\thesubsubsubsubsection}{\thesubsubsubsection.\arabic{subsubsubsubsection}}
\newcommand{\subsubsubsubsection}[1]{\refstepcounter{subsubsubsubsection}\subsubsection*{\thesubsubsubsubsection\quad#1}}


% Define cref references
\crefname{section}{Sezione \S}{Sezioni}
\crefname{subsection}{Sottosezione \S}{Sottosezioni}
\crefname{subsubsection}{Sottosezione \S}{Sottosezioni}
\crefname{subsubsubsection}{Sottosezione \S}{Sottosezioni}
\crefname{subsubsubsubsection}{Sottosezione \S}{Sottosezioni}

\setcounter{secnumdepth}{5} % Set the section numbering depth

\newcommand{\copertina}{
	\begin{titlepage}
		\vspace*{-3.5cm}
		\makebox[\textwidth]{\includegraphics[width=\paperwidth]{header.png}}
		\begin{center}
			\includegraphics[width=1\textwidth]{logo.png}	\\
			\vspace{1cm}
			\Mail	\\
			\vspace{0.5cm}
			\textbf{\begin{LARGE} \Titolo \end{LARGE}}		\\
			\vspace{1cm}
			\textbf{Descrizione:} \Descrizione{}			\\
			\vspace{1cm}
			\begin{tabular}{ll}
				\textbf{Stato}               & \Stato              \\
				\textbf{Data}                & \Data               \\
				\midrule
				\textbf{Redattori}           & \Redattori          \\
				\textbf{Verificatori}        & \Verificatori       \\

				\ifdefined\Approvatori
				\textbf{Approvatori}         & \Approvatori        \\
				\fi

				\ifdefined\ApprovatoriInterni
				\textbf{Approvatori interni} & \ApprovatoriInterni \\
				\fi

				\ifdefined\ApprovatoriEsterni
				\textbf{Approvatori esterni} & \ApprovatoriEsterni \\
				\fi

				\ifdefined\Destinatari
				\textbf{Destinatari}         & \Destinatari        \\
				\fi

				\midrule

				\ifdefined\Versione
				\textbf{Versione}            & \Versione           \\
				\fi
			\end{tabular}
		\end{center}
		\vspace{4cm}
	\end{titlepage}
	\newpage
}

\fancypagestyle{plain}{
	\fancyhf{}
	\rhead{ \includegraphics[scale=0.05]{horizontal_logo.png}}
	\lhead{\Titolo \ifdefined\Versione \ \Versione \fi}
	%\lfoot{\Titolo}
	\rfoot{\thepage{} di \pageref{LastPage}}
	\renewcommand{\headrulewidth}{0.2pt}
	\renewcommand{\footrulewidth}{0.2pt}
}
\pagestyle{plain}



\begin{document}

\copertina{}
\section*{Registro delle modifiche}
 {
  \renewcommand{\arraystretch}{1.5}
  \scriptsize
  \begin{longtable}{p{0.10\linewidth}p{0.10\linewidth}p{0.15\linewidth}p{0.15\linewidth}p{0.10\linewidth}p{0.24\linewidth}}
	  \textbf{Versione} & \textbf{Data} & \textbf{Redattore} & \textbf{Verificatore} & \textbf{Approvatore} & \textbf{Modifiche}                                 \\
	  \toprule
	  0.2.1             & 2024-05-16    & Carlo Rosso        & /
	  & /                    & Conclusione della descrizione dei pattern usati
	  nel frontend \\
	  \hline
	  0.2.0             & 2024-05-15    & Carlo Rosso        & /
	  & /                    & Ridefinizione della struttura del documento.
	  Descrizione dell'architettura di deployment e dei pattern architetturali.
	  Inizio della descrizione dei pattern usati nel frontend \\
	  \hline
	  0.1.0             & 2024-04-03    & Carlo Rosso        & /                     & /                    & Prima stesura delle sezioni 2 e 3                  \\
	  \hline
	  0.1.0             & 2024-03-30    & Carlo Rosso        & /                     & /                    & definizione della struttura generale del documento \\
	  \bottomrule
  \end{longtable}
 }

\newpage
\tableofcontents

\section{Introduzione}
\subsection{Descrizione del documento}
Il  seguente documento ha lo scopo di indicare il \textit{way of working} del gruppo di lavoro specificando gli strumenti utilizzati e le convenzioni interne adottate.\\
\noindent
Ogni particolare termine utilizzato all'interno dei documenti viene raccolto in di un apposito documento denominato "Glossario", contenente un glossario ed un registro degli acronimi, al fine di eliminare ogni possibile ambiguità.
\section{Glossario}
\subsec{A}

\subsubsection*{API}
Acronimo di \textit{Application Programming Interface}, si riferisce a un 
insieme di definizioni e protocolli che permettono la creazione e l'integrazione di software applicativi.
Le \textit{API} consentono ai componenti del software di comunicare tra loro e con altri componenti esterni, 
facilitando l'interazione tra diversi sistemi e applicazioni.

\subsubsection*{Area personale}
Si riferisce a una sezione dedicata all'utente registrato all'interno di un sito. 
Questa area fornisce all'utente un accesso riservato e personalizzato, consentendogli di visualizzare e gestire i propri dati personali in modo sicuro e privato. 

\subsubsection*{Attore}
Si tratta di un'entità che interagisce con il sistema svolgendo delle azioni.
Può essere una persona o un sistema esterno. Ciascuna entità è caratterizzata
dall'insieme delle azioni che può compiere.

\newpage

\subsection{C}
\subsubsection{Capitolato}
Si tratta di un documento tecnico, solitamente allegato a un contratto di appalto, a cui si fa riferimento per definire le specifiche tecniche delle opere che andranno a eseguirsi per effetto del contratto stesso, di cui in genere è parte integrante. \\
In questo specifico caso fa riferimento agli accordi presi tra due soggetti privati: il gruppo SWEnergy e l'azienda Imola Informatica. \\
Al suo interno si precisano diritti e doveri delle due parti, oltre che alle particolarità relative all'esecuzione dei lavori.


\subsection{D}
\subsubsection{Discord}
Discord è una piattaforma statunitense di VoIP, messaggistica istantanea e distribuzione digitale inizialmente progettata per la comunicazione tra comunità di videogiocatori.\\
Gli utenti comunicano con chiamate vocali, videochiamate, messaggi di testo, media e \textit{file} in chat private o come membri di un \textit{server} Discord. Quest'ultimi sono una raccolta di canali di tipo vocale e/o testuale.\\
Ulteriori informazioni sono disponibili su: \href{https://discord.com/}{discord.com}.
\subsection{F}

\subsubsection{Feedback}
I \textit{feedback} sono le informazioni che un cliente inserisce relativamente
all'esperienza d'uso del prodotto e del servizio.

\newpage

\subsection{G}

\subsubsection{Git}
\label{git}
Git è un \textit{software} per il controllo di versione distribuito utilizzabile
da interfaccia a riga di comando.
Nacque per essere uno strumento volto a facilitare lo sviluppo del
\textit{kernel} Linux ed è diventato uno degli strumenti di controllo versione
più diffusi.\\
Viene distribuito con licenza \texttt{GNU GPL v2} (licenza libera). \\
Ulteriori informazioni sono disponibili su:
\href{https://git-scm.com/}{git-scm.com}.

\subsubsection{GitHub}
\label{github}
GitHub è un servizio di \textit{hosting} per progetti \textit{software}, di
proprietà della società GitHub Inc.
Il nome deriva dal fatto che "GitHub" è una implementazione dello strumento di
controllo versione distribuito Git (vedi \S\ref{git}). \\
Viene utilizzato da sviluppatori che caricano il codice sorgente di programmi e
lo rendono scaricabile e migliorabile da altre persone.
Questi ultimi possono interagire con gli sviluppatori tramite un sistema per
inviare segnalazione di \textit{bug} o funzionalità (\textit{issue tracker}), un sistema
per copiare il software in una versione modificabile (\textit{fork}), un sistema
per proporre modifiche agli sviluppatori originali (\textit{pull request}) e un
sistema di discussione legato al codice del \textit{repository}.
Viene incluso anche un \textit{hosting} per pagine \textit{web} statiche, che
possono essere modificate sempre tramite un \textit{repository} git.\\
In questo specifico ambito viene sfruttata la possibilità di sviluppare software
collaborativamente, utilizzando le funzionalità fornite da
Git (vedi \S\ref{git}). \\
Ulteriori informazioni sono disponibili su:
\href{https://github.com/}{github.com}.

\newpage

\subsec{I}

\subsubsection*{Issue}
Una Issue è uno strumento disponibile in GitHub (vedi \S\ref{github}), viene
utilizzato per tenere in ordine e assegnare le attività da completare per il
raggiungimento di un obiettivo.\\
Viene caratterizzata da un progetto a cui fa riferimento, un insieme di
etichette, le persone a cui è stata assegnata ed una descrizione esplicativa
dell'attività da svolgere.

\newpage

\subsec{L}

\subsubsection*{Login}
Si tratta della procedura con cui un utente viene identificato e entra in un 
sistema informatico o in una applicazione informatica.
Il termine significa letteralmente "entrata nel log", ovvero il registro, di un 
determinato sistema informativo.

\subsubsection*{Language Server Protocol}
Il Language Server Protocol (LSP) è un protocollo di comunicazione standard utilizzato tra editor di testo o IDE e server di linguaggio. 
L'obiettivo principale del LSP è di fornire un modo uniforme per supportare le funzionalità di sviluppo come l'autocompletamento del codice, 
la navigazione delle definizioni, la documentazione in linea e il refactoring, indipendentemente dal linguaggio di programmazione utilizzato. 
Il protocollo è stato sviluppato da Microsoft ed è ora ampiamente adottato, facilitando l'integrazione di nuovi linguaggi e migliorando 
l'interoperabilità tra strumenti di sviluppo.

\newpage

\subsection{O}

\subsubsection{Ordinazione}
Insieme di prodotti che un ristoratore riceve da un utente\g

\subsubsection{Ordine}
Insieme di prodotti che un utente\g ha ordinato 
Instanza specifica in cui un cliente\g seleziona prodotti da acquistare e completa il processo di transazione



\subsec{P}

\subsubsection*{Partecipanti (ad un ordine)}
Insieme di clienti che partecipano o hanno partecipato allo stesso ordine.

\subsubsection*{PoC}
\label{poc}
Una PoC è la realizzazione di una bozza del progetto al fine di dimostrarne la fattibilità.

\subsubsection*{Proof of Concept}
Si rimanda alla voce PoC (vedi \S\ref{poc}).

\subsubsection*{Pattern Architetturali}
Soluzioni riutilizzabili a problemi comuni nell'architettura dei software. Questi pattern forniscono linee guida strutturate per organizzare i componenti di un sistema, 
facilitando la progettazione di architetture solide e scalabili. Esempi di pattern architetturali includono il Model-View-Controller (MVC), 
che separa la logica di presentazione dalla logica di business, e il Microservizi, che suddivide un'applicazione in servizi indipendenti e distribuiti. 
L'uso di pattern architetturali aiuta a standardizzare le pratiche di sviluppo, migliorare la manutenibilità del codice e facilitare l'integrazione tra diversi sistemi.

\newpage

\subsection{R}

\subsubsection{Repository}
Ambiente di un sistema informatico in cui vengono gestiti i metadati attraverso
tabelle relazionali. L'insieme di tabelle, regole e motori di calcolo tramite
cui si gestiscono i metadati prende il nome di metabase.

\newpage

\subsection{S}
\subsubsection{Sistema di autenticazione esterno}
È un meccanismo utilizzato per verificare l'identità di un utente in un sistema informatico o applicazione mediante l'uso di risorse esterne. 
Invece di gestire autonomamente il processo di autenticazione, il sistema si affida a un servizio esterno specializzato per verificare le credenziali dell'utente.

\subsubsection{Stato (di un ordine)}
Condizione corrente in cui si trova un ordine (vedi \S\ref{ordine}) nell'ambito
di un processo di acquisto.

\newpage

\subsection{T}

\subsubsection{Telegram}
Si tratta di un servizio di messaggistica istantanea e \textit{broadcasting}
basato su \textit{cloud} ed erogato senza fini di lucro dalla società Telegram
LLC.

\newpage

\subsection{U}

\subsubsection{UML}
Si tratta dell'acronimo di \textit{Unified Modeling Language}.
\textit{Unified Modeling Language} è un linguaggio di modellazione visuale
standardizzato utilizzato per rappresentare e documentare la progettazione di un
sistema software.
Fornisce un insieme di elementi di notazione grafica che possono essere
utilizzati per creare modelli visivi della struttura, del comportamento e delle
relazioni di un sistema.

\subsubsection{Utente autenticato}
Si tratta di un Utente base o di un Utente ristoratore.

\subsubsection{Utente base}
\label{utenteBase}
Si tratta di un attore autenticato, rappresenta l'insieme delle azioni che
possono essere eseguite da un cliente di un ristorante.

\subsubsection{Utente generico}
Si tratta di un utente non registrato o un utente registrato che non ha ancora
effettuato l’accesso, dunque di un attore di cui il sistema non ha informazioni
o non sa di avere informazioni.

\subsubsection{Utente ristoratore}
Si tratta di un attore autenticato, rappresenta l'amministratore di un
ristorante.

\newpage

\subsec{W}

\subsubsection*{Webapp}
Si un'applicazione \textit{web} caricata da un \textit{web} \textit{server} ed eseguita in un \textit{browser}.

\newpage


\end{document}
