\subsec{G}

\subsubsection*{Git}
\label{git}
Git è un \textit{software} per il controllo di versione distribuito utilizzabile
da interfaccia a riga di comando.
Nacque per essere uno strumento volto a facilitare lo sviluppo del
\textit{kernel} Linux ed è diventato uno degli strumenti di controllo versione
più diffusi.\\
Viene distribuito con licenza \texttt{GNU GPL v2} (licenza libera). \\
Ulteriori informazioni sono disponibili su:
\href{https://git-scm.com/}{git-scm.com} (ultimo accesso 21/11/2023).

\subsubsection*{GitHub}
\label{github}
GitHub è un servizio di \textit{hosting} per progetti \textit{software}, di
proprietà della società GitHub Inc.
Il nome deriva dal fatto che "GitHub" è una implementazione dello strumento di
controllo versione distribuito Git (vedi \S\ref{git}). \\
Viene utilizzato da sviluppatori che caricano il codice sorgente di programmi e
lo rendono scaricabile e migliorabile da altre persone.
Questi ultimi possono interagire con gli sviluppatori tramite un sistema per
inviare segnalazione di \textit{bug} o funzionalità (\textit{issue tracker}), un sistema
per copiare il software in una versione modificabile (\textit{fork}), un sistema
per proporre modifiche agli sviluppatori originali (\textit{pull request}) e un
sistema di discussione legato al codice del \textit{repository}.
Viene incluso anche un \textit{hosting} per pagine \textit{web} statiche, che
possono essere modificate sempre tramite un \textit{repository} git.\\
In questo specifico ambito viene sfruttata la possibilità di sviluppare software
collaborativamente, utilizzando le funzionalità fornite da
Git (vedi \S\ref{git}). \\
Ulteriori informazioni sono disponibili su:
\href{https://github.com/}{github.com} (ultimo accesso 21/11/2023).

\newpage
