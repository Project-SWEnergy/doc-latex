\subsec{O}

\subsubsection*{Ordinazione}
Insieme di prodotti che un ristoratore riceve da un utente base (vedi
\S\ref{utenteBase}).

\subsubsection*{Ordine}
\label{ordine}
Un ordine è l'insieme di prodotti che un Utente base ha ordinato.

\subsubsection*{ORM}
Acronimo di \textit{Object-Relational Mapping}, si riferisce a una tecnica di programmazione 
utilizzata per convertire i dati tra sistemi incompatibili usando la programmazione orientata agli oggetti. 
Gli \textit{ORM} permettono agli sviluppatori di interagire con un database relazionale utilizzando il proprio linguaggio 
di programmazione orientato agli oggetti, evitando di scrivere codice SQL direttamente.

\newpage
