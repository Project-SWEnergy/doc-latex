\section{Cliente}

\subsection{Header}

\begin{figure}[htbp]
    \centering
	\includegraphics[width=0.8\textwidth]{PB/manuale-utente/header-cliente.png}
    \caption{Barra di Navigazione per il cliente}
\end{figure}

La barra di navigazione in alto ti permette di accedere facilmente alle varie 
sezioni della piattaforma:
\begin{itemize}
	\item \texttt{Logo e "Easy Meal"}: cliccando sul logo o sul nome "Easy Meal"
		sei reindirizzato alla Home Page della piattaforma;

	\item \texttt{Esplora}: cliccando su questo riferimento sei reindirizzato
		alla Home Page della piattaforma;

	\item \texttt{Prenotazioni}: cliccando su questo riferimento sei 
		reindirizzato alla pagina di visualizzazione della lista di
		prenotazioni;

	\item \texttt{Notifiche}: cliccando su questo riferimento sei reindirizzato
		alla pagina di visualizzazione della lista di notifiche. Nota che se ci
		sono notifiche non lette, il riferimento sarà seguito da un badge che
		indica il numero di notifiche non lette;

	\item \texttt{Logout}: cliccando su questo riferimento effettuerai il logout
		dalla piattaforma e sarai reindirizzato alla Home Page della 
		piattaforma.
\end{itemize}

\subsection{Home Page}

La Home Page del cliente è analoga a quella dell'utente non autenticato, cambia
solamente l'header.

\subsection{Visualizza in dettaglio un ristorante}

\begin{figure}[htbp]
    \centering
	\includegraphics[width=0.8\textwidth]{PB/manuale-utente/dettagli-ristorante-cliente.png}
    \caption{Pagina di visualizzazione in dettaglio di un ristorante per il
	cliente}
\end{figure}

Questa pagina è analoga a quella dell'utente non autenticato, ci sono tuttavia
due aggiunte:
\begin{itemize}
	\item \texttt{Inserisci recensione}: questo bottone compare in mezzo tra le
		gli orari di apertura di un ristorante e le sue recensioni, accanto al
		pulsante qui sotto. Cliccando su questo bottone si viene reindirizzati
		ad una pagina in cui è possibile inserire una recensione per il
		ristorante. Questo bottone compare solo se si è conclusa una
		prenotazione presso il ristorante e non si è ancora inserita una
		recensione;

	\item \texttt{Prenota}: questo bottone si trova sotto gli orari del
		ristorante. Cliccando su questo bottone si viene reindirizzati alla
		pagina di prenotazione di un tavolo presso il ristorante che si sta
		visualizzando.
\end{itemize}

\subsection{Prenotazione di un tavolo}

\begin{figure}[htbp]
    \centering
	\includegraphics[width=0.8\textwidth]{PB/manuale-utente/prenotazione.png}
    \caption{Pagina di richiesta di una prenotazione per il cliente}
\end{figure}

Si accede a questa pagina cliccando sul bottone \texttt{Prenota} della pagina di
dettaglio di un ristorante. Dopo aver compilato il form con le informazioni
richieste:
\begin{itemize}
	\item Data della prenotazione: viene richiesto di inserire la data in cui si
		vuole prenotare il tavolo nel formato \texttt{mm/gg/aaaa}. Accanto al
		form di inserimento è presente il bottone di un calendario che apre un
		calendario con i giorni. Cliccando su un giorno del calendario, il form
		viene automaticamente compilato con la data selezionata;

	\item Ora della prenotazione: viene richiesto di inserire l'orario in cui si
		vuole prenotare il tavolo nel formato \texttt{hh:mm}. Sono accettati
		solamente orari compresi all'interno degli orari di apertura del
		ristorante presso cui si desidera prenotare;

	\item Numero partecipanti: con un numero di partecipanti maggiore di 1
		appaiono dei campi aggiuntivi per inserire le email dei partecipanti
		supplementari, in questo modo è possibile condividere la prenotazione
		con altri utenti;

	\item Metodo di pagamento: si può scegliere di pagare in modo equidiviso
		(alla romana) oppure di pagare in modo individuale, e quindi ciascun
		partecipante paga solo per se stesso.
\end{itemize}

Dopo aver compilato il form, cliccando sul bottone \texttt{Invia prenotazione} 
si richiede la prenotazione del tavolo. Se la richiesta è andata a buon fine
viene visualizzato un messaggio di successo e si è reindirizzati alla pagina di
visualizzazione della lista di prenotazioni; altrimenti viene visualizzato un
messaggio di errore.

\subsection{Lista delle prenotazioni}
\begin{figure}[htbp]
    \centering
	\includegraphics[width=0.8\textwidth]{PB/manuale-utente/lista-prenotazioni-cliente.png}
    \caption{Pagina di visualizzazione delle prenotazioni per il cliente}
\end{figure}

Si può accedere a questa pagina cliccando sul bottone \texttt{Prenotazioni} della
barra di navigazione. In questa pagina sono visualizzate tutte le prenotazioni
di un cliente. In particolare a seconda dello stato della prenotazione sono
possibili diverse operazioni sulle prenotazioni:
\begin{itemize}
	\item \textbf{"In Attesa"} oppure \textbf{"Approvata"}: 
		\begin{itemize}
			\item \texttt{Cancella}: la prenotazione è annullata per te
				(se una prenotazione è condivisa, la cancellazione della
				prenotazione avviene solo quando tutti i partecipanti hanno cancellato
				la prenotazione). Questa operazione cancella la prenotazione
				dalla lista delle prenotazioni del cliente;

			\item \texttt{Ordina collaborativamente}: cliccando su questo
				bottone si viene reindirizzati alla pagina di gestione delle
				ordinazioni collegate alla prenotazione selezionate.
		\end{itemize}

	\item \textbf{"Pagamento"}: 
		\begin{itemize}
			\item \texttt{Paga}: cliccando su questo bottone si viene
				reindirizzati alla pagina di pagamento della prenotazione
				selezionata.
		\end{itemize}

	\item \textbf{"Conclusa"} oppure \textbf{"Rifiutata"}: 
		\begin{itemize}
			\item \texttt{Cancella}: la prenotazione viene non sarà più visibile
				nella lista delle prenotazioni del cliente.
		\end{itemize}

	\item \textbf{Accettato: No}: si tratta di un invito a partecipare ad una
		prenotazione condivisa.
		\begin{itemize}
			\item \texttt{Conferma}: cliccando su questo bottone si accetta la
				prenotazione selezionata;

			\item \texttt{Cancella}: cliccando su questo bottone si rifiuta la
				prenotazione selezionata.
		\end{itemize}
\end{itemize}

\subsection{Ordinazione collaborativa}
\begin{figure}[htbp]
    \centering
	\includegraphics[width=0.8\textwidth]{PB/manuale-utente/ordinazione-collaborativa.png}
    \caption{Pagina di visualizzazione di un'ordinazione collaborativa}
\end{figure}

\subsection{Pagamento di una prenotazione}
\begin{figure}[htbp]
    \centering
	\includegraphics[width=0.8\textwidth]{PB/manuale-utente/pagamento.png}
    \caption{Pagina di pagamento di una prenotazione}
\end{figure}

Si accede a questa pagina cliccando sul bottone \texttt{Paga} della pagina di
visualizzazione delle prenotazioni, se la prenotazione è in stato "Pagamento".
In questa pagina è mostrato il riepilogo degli ordini effettuati, il totale da
pagare del tavoro e la quota da pagare per te. Per effettuare il pagamento è
sufficiente cliccare sul bottone \texttt{Paga}. A questo punto, è visualizzata
la notifica dell'esito dell'operazione. Se il pagamento va a buon fine, la
pagina viene aggiornata, viene mostrato il riepilogo della prenotazione e il
messaggio "Hai pagato tutto". Altrimenti la pagina non si aggiorna e viene
mostrato un messaggio di errore in rosso in basso a destra.

\begin{figure}[htbp]
    \centering
	\includegraphics[width=0.8\textwidth]{PB/manuale-utente/pagamento-concluso.png}
    \caption{Pagina di pagamento concluso di una prenotazione}
\end{figure}

\subsection{Notifiche}

\begin{figure}[htbp]
    \centering
	\includegraphics[width=0.8\textwidth]{PB/manuale-utente/notifiche.png}
    \caption{Pagina di pagamento concluso di una prenotazione}
\end{figure}

Si accede a questa pagina cliccando sul bottone \texttt{Notifiche} della barra di
navigazione. In questa pagina sono visualizzate tutte le notifiche di un
cliente. In cima alla lista delle notifiche sono presenti due bottoni:
\texttt{Nuove} e \texttt{Lette}. Accedendo alla pagina, è selezionato il filtro
delle notifiche nuove. Cliccando su \texttt{Lette} si visualizzano le notifiche
già lette. Passando sopra una notifica con il mouse, compare un bottone per
segnare la notifica come letta.

\subsection{Inserimento di una recensione}

\begin{figure}[htbp]
    \centering
	\includegraphics[width=0.8\textwidth]{PB/manuale-utente/recensione.png}
    \caption{Pagina di pagamento concluso di una prenotazione}
\end{figure}

Si accede a questa pagina cliccando sul bottone \texttt{Inserisci recensione} della
pagina di visualizzazione in dettaglio di un ristorante. Si può accedere a
questa pagina solo se si è conclusa una prenotazione presso il ristorante e non
si è ancora inserita una recensione. Dopo aver compilato il form con le seguenti
informazioni:
\begin{itemize}
	\item Punteggio: da 0 a 10;
	\item Descrizione: una descrizione della recensione.
\end{itemize}

Cliccando sul bottone \texttt{Invia recensione} si invia la recensione. Se la
richiesta è andata a buon fine viene visualizzato un messaggio di successo. Non
è possibile modificare o eliminare una recensione una volta inviata.
