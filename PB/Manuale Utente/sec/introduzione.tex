\section{Introduzione}
\subsection{Scopo del documento}
Questo manuale è stato creato per fornire un supporto completo agli utenti di EasyMeal, sia dal lato dei clienti che dei ristoratori. 
L'obiettivo principale è consentire agli utenti di sfruttare appieno tutte le funzionalità dell'applicazione al fine di garantire un'
esperienza ottimale. Il manuale include istruzioni dettagliate per l'utilizzo del \textit{software}, fornendo una panoramica generale dei requisiti 
e dei passaggi necessari per l'installazione corretta, oltre ai dettagli dell'utilizzo dell'applicazione una volta installata.



\subsection{Scopo del prodotto}
EasyMeal è una \textit{web app}\g progettata per semplificare la gestione delle prenotazioni nei ristoranti, offrendo funzionalità 
a due parti distinte: 
\begin{itemize}
	\item Cliente\g: La parte del cliente consente di prenotare un tavolo, visualizzare il menù e effettuare 
	ordini\g ;
	\item Ristoratore\g : La parte del ristoratore permette di gestire le prenotazioni, gli ordini dei clienti e la lista degli ingredienti 
	necessari per preparare i piatti ordinati.
\end{itemize}

L'obiettivo del progetto è migliorare l'esperienza nei ristoranti, consentendo agli utenti di effettuare prenotazioni 
in modo intuitivo, personalizzare gli ordini in base alle proprie preferenze alimentari, agevolare la divisione del conto e favorire 
l'interazione con il personale del ristorante. Inoltre, l'applicazione promuoverà la scrittura di recensioni.



\subsection{Accesso alla Piattaforma}
EasyMeal è accessibile come \textit{web app}\g agli utenti autorizzati. L'accesso avviene tramite un browser web, senza richiedere 
l'installazione di software aggiuntivo.
Al fine di garantire la sicurezza e la riservatezza dei dati, l'accesso è limitato agli utenti 
registrati in possesso delle proprie credenziali. Inoltre, la modifica dei dati è consentita esclusivamente ai ristoratori, i quali possono 
aggiornare le informazioni pertinenti al proprio ristorante, come il menù, le immagini e altri dettagli. Le credenziali di accesso vengono 
fornite dal team amministrativo o da personale autorizzato. Una volta ottenuto il link e le credenziali, gli utenti possono accedere alla 
piattaforma da qualsiasi dispositivo connesso a Internet, offrendo un'esperienza flessibile e accessibile ovunque.


\subsection{Installazione del Software}
Inoltre, sarà inclusa una sezione dedicata che fornirà istruzioni dettagliate su come installare correttamente il software EasyMeal sul sistema 
dell'utente nel documento Specifiche tecniche.


\subsection{Glossario}
Al fine di evitare ambiguità nei termini utilizzati nel documento, è disponibile un Glossario che definisce chiaramente tutti i 
termini tecnici utilizzati nel contesto del progetto. Nel testo, i termini presenti nel Glossario sono identificati in corsivo 
con una 'G' ad apice (ad esempio, parola\g ). 
Questa pratica agevola la consultazione del Glossario per una comprensione approfondita dei termini utilizzati.

\subsection{Riferimenti}
\subsubsection{Normativi}
\begin{itemize}
	\item Norme di progetto v2.0.0;
	\item 	\href{https://www.math.unipd.it/~tullio/IS-1/2023/Progetto/C3.pdf}
	      {Documento del capitolato d'appalto C3 - \textit{Easy Meal}} (ultimo accesso 20/03/2024);
	\item \href{https://www.math.unipd.it/~tullio/IS-1/2023/Dispense/PD2.pdf}
	      {Regolamento del progetto} (ultimo accesso 25/03/2024); 
\end{itemize}

\subsubsection{Informativi}
\begin{itemize}
	\item Glossario v2.0.0;
	\item Analisi dei Requisiti v2.0.0;
	\item \href{https://www.math.unipd.it/~rcardin/swea/2023/Diagrammi\%20delle\%20Classi.pdf}
	      {Linguaggio UML - Diagramma delle classi} (ultimo accesso 4/03/2024); 
\end{itemize}

\subsubsection{Tecnologici}
\begin{itemize}
	\item \href{https://www.typescriptlang.org/}
	      {Typescript} (ultimo accesso 12/05/2024);

	\item \href{https://angular.io/}
	      {Angular} (ultimo accesso 12/05/2024);

	\item \href{https://axios-http.com/}
	      {Axios} (ultimo accesso 12/05/2024);

	\item \href{https://nestjs.com/}
	      {NestJS} (ultimo accesso 12/05/2024);

	\item \href{https://orm.drizzle.team/}
	      {Drizzle} (ultimo accesso 12/05/2024);

	\item \href{https://www.docker.com/}
	      {Docker} (ultimo accesso 12/05/2024);

	\item \href{https://insomnia.rest/}
	      {Insomnia} (ultimo accesso 12/05/2024);

	\item \href{https://tailwindcss.com/}
	      {Tailwind CSS} (ultimo accesso 12/05/2024);
\end{itemize}