\section{Introduzione}
\subsection{Scopo del documento}
Il documento qui presente ha come obiettivo principale quello di fornire una panoramica dettagliata sulle funzionalità dell'applicazione EasyMeal e offrire 
istruzioni chiare e concise per garantire un utilizzo efficace del \textit{software}. Gli utenti potranno acquisire familiarità 
con i requisiti minimi necessari per far funzionare correttamente l'applicazione e verranno guidati relativamente alle funzionalità del prodotto.
Si fornirà inoltre una guida su come installare l'applicazione in locale.

\subsection{Scopo del prodotto}
"Easy Meal" è una web app\g progettata per gestire le prenotazioni presso i ristoranti, sia
dal lato dei clienti che dei ristoratori. Il prodotto è composto da due parti distinte:
\begin{itemize}
	\item Cliente\g: consente ai clienti di prenotare un tavolo presso un ristorante, visualizzare
	il menù e effettuare un ordine\g ;
	\item Ristoratore: consente ai ristoratori di gestire le prenotazioni e gli ordini dei clienti,
	oltre a visualizzare la lista degli ingredienti necessari per preparare i piatti ordinati.
\end{itemize}
Il progetto mira a migliorare l'esperienza nei ristoranti sia per i clienti che per i gestori.
L'app consentirà agli utenti di effettuare prenotazioni in modo intuitivo, personalizzare gli ordini in base alle proprie preferenze
alimentari e favorire l'interazione tra i clienti e il personale del ristorante.
Inoltre, agevolerà la divisione del conto e promuoverà la scrittura di recensioni.

\subsection{Glossario}
Al fine di prevenire ambiguità linguistiche e garantire una coerenza nell'utilizzo delle terminologie attraverso i documenti, il team ha compilato un documento interno 
denominato "Glossario". Questo documento fornisce definizioni chiare e precise per i termini che potrebbero risultare ambigui o generare incomprensioni nel testo 
principale. I termini inclusi nel Glossario sono facilmente identificabili grazie a un apice 'G' (ad esempio, parola\g ).
Questa pratica agevola la consultazione del Glossario per una comprensione approfondita dei termini tecnici o specifici utilizzati nel contesto del progetto

\subsection{Riferimenti}
\subsubsection{Normativi}
\begin{itemize}
	\item Norme di progetto v2.0.0;
	\item 	\href{https://www.math.unipd.it/~tullio/IS-1/2023/Progetto/C3.pdf}
	      {Documento del capitolato d'appalto C3 - \textit{Easy Meal}} (ultimo accesso 20/03/2024);
	\item \href{https://www.math.unipd.it/~tullio/IS-1/2023/Dispense/PD2.pdf}
	      {Regolamento del progetto} (ultimo accesso 25/03/2024); 
\end{itemize}

\subsubsection{Informativi}
\begin{itemize}
	\item Glossario v2.0.0;
	\item Analisi dei Requisiti v2.0.0;
	\item \href{https://www.math.unipd.it/~rcardin/swea/2023/Diagrammi\%20delle\%20Classi.pdf}
	      {Linguaggio UML - Diagramma delle classi} (ultimo accesso 4/03/2024); 
\end{itemize}

\subsubsection{Tecnologici}
\begin{itemize}
	\item \href{https://www.typescriptlang.org/}
	      {Typescript} (ultimo accesso 03/04/2024);

	\item \href{https://angular.io/}
	      {Angular} (ultimo accesso 03/04/2024);

	\item \href{https://axios-http.com/}
	      {Axios} (ultimo accesso 03/04/2024);

	\item \href{https://nestjs.com/}
	      {NestJS} (ultimo accesso 03/04/2024);

	\item \href{https://orm.drizzle.team/}
	      {Drizzle} (ultimo accesso 03/04/2024);

	\item \href{https://www.docker.com/}
	      {Docker} (ultimo accesso 03/04/2024);

	\item \href{https://insomnia.rest/}
	      {Insomnia} (ultimo accesso 03/04/2024);

	\item \href{https://tailwindcss.com/}
	      {Tailwind CSS} (ultimo accesso 03/04/2024);
\end{itemize}