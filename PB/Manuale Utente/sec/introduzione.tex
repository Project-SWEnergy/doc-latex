\section{Introduzione}
\subsection{Scopo del documento}
Questo manuale è stato creato per fornire un supporto completo agli utenti di 
EasyMeal, sia dal lato dei clienti che dei ristoratori. 
L'obiettivo principale è consentire agli utenti di sfruttare appieno tutte le
funzionalità dell'applicazione al fine di permettere una buona esperienza d'uso. 
Il manuale include istruzioni dettagliate per l'utilizzo del \textit{software}, 
fornendo i dettagli dell'utilizzo dell'applicazione una volta effettuata la
connessione alla piattaforma.



\subsection{Scopo del prodotto}
EasyMeal è una \textit{web app}\g progettata per semplificare la gestione delle 
prenotazioni nei ristoranti, in effetti si divide in due parti principali:
\begin{itemize}
	\item Cliente\g: La parte del cliente consente di prenotare un tavolo, 
		visualizzare il menù e effettuare prenotazioni e ordini\g ;

	\item Ristoratore\g : La parte del ristoratore permette di gestire le 
		prenotazioni, gli ordini dei clienti e la lista degli ingredienti 
		necessari per preparare i piatti ordinati.
\end{itemize}

L'obiettivo del progetto è migliorare l'esperienza nei ristoranti, consentendo 
agli utenti di effettuare prenotazioni in modo intuitivo, personalizzare gli 
ordini in base alle proprie preferenze alimentari e agevolare la divisione del 
conto. Inoltre, l'applicazione promuove la scrittura di recensioni.

\subsection{Accesso alla Piattaforma}
EasyMeal è accessibile come \textit{web app}\g agli utenti autorizzati. 
L'accesso avviene tramite un browser web, senza richiedere l'installazione di 
software aggiuntivo.
Al fine di garantire la sicurezza e la riservatezza dei dati oltre che a evitare
situazioni scomode per i ristoratori, le prenotazioni e le ordinazioni sono rese
possibili solo dopo aver effettuato l'accesso alla piattaforma.
Una volta ottenuto il link e le credenziali, gli utenti possono accedere alla 
piattaforma da qualsiasi dispositivo connesso a Internet, offrendo un'esperienza 
flessibile e accessibile ovunque.

\subsection{Glossario}
Al fine di evitare ambiguità nei termini utilizzati nel documento, è disponibile 
un Glossario che definisce chiaramente tutti i termini tecnici utilizzati nel 
contesto del progetto. Nel testo, i termini presenti nel Glossario sono 
identificati in corsivo con una 'G' ad apice (ad esempio, parola\g ). 
Questa pratica agevola la consultazione del Glossario per una comprensione 
approfondita dei termini utilizzati.

\subsection{Riferimenti}
\subsubsection{Normativi}
\begin{itemize}
	\item Norme di progetto v2.0.0;
	\item 	\href{https://www.math.unipd.it/~tullio/IS-1/2023/Progetto/C3.pdf}
	      {Documento del capitolato d'appalto C3 - \textit{Easy Meal}} (ultimo accesso 20/03/2024);
	\item \href{https://www.math.unipd.it/~tullio/IS-1/2023/Dispense/PD2.pdf}
	      {Regolamento del progetto} (ultimo accesso 25/03/2024); 
\end{itemize}

\subsubsection{Informativi}
\begin{itemize}
	\item Glossario v2.0.0;
	\item Analisi dei Requisiti v2.0.0;
\end{itemize}
