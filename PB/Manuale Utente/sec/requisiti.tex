\section{Requisiti}
Questa sezione del manuale dettaglia i requisiti minimi necessari per garantire 
un corretto funzionamento dell'applicazione EasyMeal.

\subsection{Requisiti \textit{software}}
L'applicazione EasyMeal è stata testata e confermata come compatibile con i 
seguenti \textit{browser} e relative versioni: 

\begin{longtable}{|c|c|c|}
	\hline
	\textbf{Browser}       & \textbf{ Versione}    \\
	\hline
    Google Chrome             & 123                    \\
    \hline
    Arc                       & 1.26                    \\
    \hline
    Opera GX                       & 124                    \\
    \hline
    Safari                        & 17.3                    \\
    \hline
    Microsfot Edge                 & 123                      \\
    \hline

    \caption{Tabella dei requisiti \textit{software}.}
\end{longtable}

Si sottolinea che l'applicazione potrebbe funzionare su altri \textit{browser} o versioni 
precedenti, ma il pieno supporto non è garantito.

Inoltre, essendo EasyMeal una \textit{web app}, è possibile utilizzarla anche
da dispositivi mobili, come \textit{smartphone} e \textit{tablet}. Infatti il gruppo di sviluppo ha usato gli strumenti forniti dai \textit{browser} per simulare l'esperienza utente 
su dispositivi mobili, verificando che l'applicazione fosse pienamente funzionante.


\subsection{Requisiti hardware}
Poiché EasyMeal è una \textit{web app}, non sono richiesti requisiti \textit{hardware} specifici. 
Tuttavia, per un'esperienza ottimale, si consiglia di avere almeno quanto segue:

\begin{longtable}{|l|p{0.8\textwidth}|}
	\hline
	\textbf{Componente}       & \textbf{ Requisito minimo}   \\
	\hline
     Processore             &  \textit{Quad-core} con \textit{clock rate} di almeno 2.0 GHz      \\
    \hline
     Memoria RAM            &  4GB DDR4       \\
    \hline
    Spazio su disco         &  Almeno $ \geq  20 GB$ di spazio disponibili        \\
    \hline
    Connessione Internet         & Connessione \textit{Internet} stabile e veloce, in grado di supportare le esigenze di traffico dell'applicazione         \\
    \hline

    \caption{Tabella dei requisiti \textit{hardware}.}
\end{longtable}
