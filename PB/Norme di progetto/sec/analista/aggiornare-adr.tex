\subsection{Aggiornamento della "Analisi dei Requisiti"}
\label{aggiornare-adr}

\subsubsection{\textit{Trigger}}
\begin{itemize}
	\item Sono presenti dei dubbi o delle lacune in merito a qualcosa;

	\item Risulta necessario formalizzare qualche concetto o qualche argomento.
\end{itemize}

\subsubsection{Scopo}
\begin{itemize}
	\item Risolvere i dubbi e le lacune riguardo a un argomento, o almeno
	      formalizzare i dubbi e le lacune;

	\item Formalizzare la definizione di un concetto o di un argomento, per
	      renderlo chiaro ed inequivoco.
\end{itemize}

\subsubsection{Svolgimento}
\begin{itemize}
	\item \textbf{Identificazione dei casi d'uso}: in quale modo l'analista ed
	      il gruppo possono individuare i casi d'uso da includere nel documento.
	      Di seguito sono riportati i passi da seguire:
	      \begin{enumerate}
		      \item \textbf{Ipotesi}: l'analista impotizza il flusso di azioni
		            da svolgere per portare a termine l'azione dell'utente;

		      \item \textbf{Dubbi}: l'analista si confronta con il gruppo e
		            formalizza i dubbi e le lacune all'interno delle Discussion di
		            Github\g;

		      \item \textbf{Sperimentazione}: viene implementato il caso d'uso
		            in modo da verifcare il flusso di azioni ipotizzato;

		      \item \textbf{Formalizzazione}: l'analista aggiorna la "Analisi
		            dei Requisiti" rispetto alle informazioni raccolte, potrebbe
		            dover modificare anche i requisiti per tenerli aggiornati
		            rispetto ai casi d'uso. \textit{Nota: viene modificato un
			            documento, quindi si rimanda alla sottosezione che
			            illustra come redigere un documento (vedi
			            \cref{redazione-documento})}.
	      \end{enumerate}
\end{itemize}

\subsubsection{Strumenti}
Lo strumento utilizzato per l'aggiornamento del documento è:
\begin{itemize}
	\item \textbf{Discussion di GitHub\g}: per mantenere e formalizzare i dubbi
	      e le lacune riscontrate.
\end{itemize}
