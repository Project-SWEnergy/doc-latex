\subsection{Codifica}
\label{codifica}

\subsubsection{Descrizione}

Il programmatore scrive il codice sorgente che compone l'applicativo. Il codice
sorgente è scritto in linguaggio TypeScript.

\subsubsection{\textit{Trigger}}
\begin{itemize}
	\item Viene completata la progettazione di una \textit{feature};
\end{itemize}

\subsubsection{Scopo}
\begin{itemize}
	\item Implementare le funzionalità richieste dal proponente;

	\item Soddisfare qualche requisito;
\end{itemize}

\subsubsection{Svolgimento}
Di seguito sono elencate le attività che il programmatore deve svolgere per la
codifica del prodotto software:
\begin{itemize}
	\item \textbf{Progettazione:} il programmatore deve produrre dei commenti o
	      degli appunti che descrivano la struttura del codice che andrà a
	      scrivere nella prossima attività. Questi commenti devono poi essere
	      riorganizzati e riportati nella \textit{issue} corrispondente;

	\item \textbf{Test:} il programmatore deve scrivere dei test per verificare
	      il corretto funzionamento del codice che andrà a scrivere;

	\item \textbf{Codifica di una funzione o metodo:} di seguito sono elencati i
	      passi che il programmatore deve seguire per la codifica del prodotto
	      software:
	      \begin{enumerate}
		      \item \textbf{Pull:} il programmatore esegue un \textit{pull} del
		            codice sorgente dal \textit{repository} remoto;

		      \item \textbf{Branch:} il programmatore crea un nuovo branch di
		            lavoro a partire dal branch \texttt{dev};

		      \item \textbf{Commenti:} il programmatore scrive lo scopo della
		            funzione o del metodo che andrà a codificare e ne descrive
		            la firma;

		      \item \textbf{Codifica:} il programmatore scrive il codice che
		            compone il corpo della funzione o del metodo;

		      \item \textbf{Test:} il programmatore esegue i test di verifica.
		            In caso di fallimento, il programmatore deve correggere il
		            codice e ripetere la verifica;

		      \item \textbf{Iterazione:} se il programmatore vuole scrivere
		            altre funzioni torna al punto 3, altrimenti prosegue
		            con il punto successivo;

		      \item \textbf{Push:} il programmatore esegue un \textit{push}
		            del codice sorgente sul \textit{repository} remoto.

		      \item \textbf{Verifica:} il programmatore segnala al
		            verificatore che il codice è pronto per essere verificato.

		      \item \textbf{Correzione:} se il verificatore segnala degli
		            errori, il programmatore deve correggere il codice e torna
		            al passo precendente. Altrimenti, il programmatore può
		            procedere al passo successivo.

		      \item \textbf{Chiusura:} il programmatore effettua il merge del
		            branch di lavoro con il branch \texttt{dev} e chiude il
		            ticket di \textit{GitHub} corrispondente.
	      \end{enumerate}
\end{itemize}
