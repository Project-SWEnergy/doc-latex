\subsection{Formazione del Personale}

La formazione del personale è un processo organizzativo critico che mira a
sviluppare le competenze e le conoscenze dei membri del \textit{team}, garantendo che
siano adeguatamente equipaggiati per contribuire efficacemente al progetto.

\subsubsection{Scopo}
Incrementare le competenze tecniche e metodologiche del \textit{team}, promuovere
l'innovazione e migliorare la qualità del lavoro svolto, attraverso un approccio
di apprendimento continuo e adattivo.

\subsubsection{Attività}
\subsubsubsection{Analisi dei Bisogni Formativi} 
Valutare le esigenze di
	  formazione del \textit{team}, identificando le lacune nelle competenze e nelle
	  conoscenze.

\subsubsubsection{Pianificazione della Formazione} 
Sviluppare un piano di
	  formazione che includa obiettivi di apprendimento, metodi formativi,
	  risorse necessarie e calendario delle attività formative.

\subsection{Organizzare un \textit{workshop}}
\label{organizzare-workshop}

\subsubsection{Descrizione}

I progettisti sono tenuti a sperimentare con nuove tecnologie, per produrre le
PoC. SWEnergy non norma il processo di sperimentazione e di producezione delle
PoC, d'altra parte, ritiene che sia importante spiegare i risultati ottenuti
dalle PoC al resto del \textit{team}. I \textit{workshop} sono un'insieme di
appunti, di presentazioni e di codice per illustrare i risultati ottenuti dalle
PoC.

\subsubsection{\textit{Trigger}}
\begin{itemize}
	\item Qualche membro del gruppo non conosce qualche tecnologia da
	      implemetare.
\end{itemize}

\subsubsection{Scopo}
\begin{itemize}
	\item Condividere le conoscenze tecniche tra i membri del gruppo;

	\item Documentare le conoscenze tecniche acquisite;

	\item Imparare ad usare la nuova tecnologia;

	\item Provvedere affinché tutti i membri di SWEnergy abbiano una conoscenza
	      di base e sufficient per adottare la tecnologia all'interno del
	      progetto.
\end{itemize}

\subsubsection{Svolgimento}
Di seguito sono elencate le attività che i progettisti devono svolgere per
organizzare un \textit{workshop}:
\begin{itemize}
	\item \textbf{Bozza di appunti:} i progettisti sono tenuti a produrre dei
	      \textit{markdown} per spiegare e riassumere i contenuti delle PoC. I
	      \textit{file} così prodotti sono organizzati come il progettista
	      meglio crede, all'interno del \textit{repository}
	      \texttt{appunti-swe}.

	\item \textbf{Appunti web:} a partire dagli appunti sopra prodotti, sono
	      organizzati i \textit{workshop}. Di seguito sono elencati i passi da
	      seguire per pubblicare gli appunti di un \textit{workshop}:
	      \begin{enumerate}
		      \item Creare una cartella all'interno del \textit{repository}
		            \texttt{Project-SWEnergy.github.io} con il nome del
		            \textit{workshop} da organizzare;

		      \item Creare un \texttt{readme.md} all'interno della
		            cartella appena creata, che collega gli appunti all'interno
		            della cartella tra loro;

		      \item Effettuare il \textit{push} delle modifiche sul
		            \textit{repository} remoto.
	      \end{enumerate}

	\item \textbf{Presentazione:} i progettisti sono tenuti a produrre una
	      presentazione per presentare gli appunti prodotti e le PoC realizzate.
	      Si noti che la presentazione non ha una descrizione prescrittiva
	      perché, a seconda del contenuto e delle conoscenze tecnologiche del
	      progettista, può essere realizzata con diversi strumenti. Viene
	      consigliato l'uso di \href{https://obsidian.md/}{\textit{Obsidian}} e
	      del \textit{plugin} \textit{Advanced Slides}.
\end{itemize}


\subsubsubsection{Valutazione dell'Impatto} 
Misurare l'efficacia della
	  formazione attraverso \textit{feedback}\g, valutazioni e analisi delle
	  prestazioni, per garantire che gli obiettivi di apprendimento siano
	  stati raggiunti.

\subsubsubsection{Miglioramento Continuo} 
Utilizzare i \textit{feedback}\g e i
	  risultati delle valutazioni per perfezionare continuamente le
	  iniziative formative, assicurando che restino rilevanti e utili.

\subsubsection{Risorse}
L'accesso a risorse formative come piattaforme di \textit{e-learning}, libri,
articoli, e la partecipazione a conferenze e \textit{workshop} esterni o interni
sono incoraggiati e supportati dall'organizzazione.

\subsubsection{Cultura dell'Apprendimento}
Promuovere una cultura dell'apprendimento all'interno del \textit{team}, incoraggiando
la condivisione delle conoscenze, la curiosità e l'iniziativa personale
nell'esplorazione di nuove competenze e tecnologie.
