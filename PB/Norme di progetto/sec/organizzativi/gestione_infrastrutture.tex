\subsection{Gestione delle Infrastrutture}

La gestione delle infrastrutture si occupa dell'organizzazione e della
manutenzione delle infrastrutture tecniche necessarie per supportare lo
svolgimento efficace dei processi di ciclo di vita del \textit{software}.

\subsubsection{Scopo}
Assicurare che l'ambiente tecnologico sia adeguatamente configurato, gestito e
manutenuto per supportare le attività di sviluppo, \textit{testing},
\textit{deployment} e operatività del \textit{software}.

\subsubsection{Attività}
\begin{enumerate}
	\item \textbf{Valutazione delle Necessità:} Identificare i requisiti
	      infrastrutturali basati sulle esigenze del progetto, incluse le
	      piattaforme di sviluppo, gli ambienti di \textit{testing} e i sistemi
	      di produzione.
	\item \textbf{Configurazione e Implementazione:} Configurare e implementare
	      le infrastrutture tecniche necessarie, inclusi \textit{hardware},
	      reti, sistemi operativi e servizi.
	\item \textbf{Manutenzione e Aggiornamento:} Eseguire la manutenzione
	      regolare delle infrastrutture per assicurare prestazioni ottimali e
	      applicare aggiornamenti di sicurezza e funzionalità.
	\item \textbf{Monitoraggio e \textit{Troubleshooting}:} Monitorare le
	      infrastrutture per identificare e risolvere tempestivamente eventuali
	      problemi o malfunzionamenti.
	\item \textbf{Gestione della Sicurezza:} Implementare misure di sicurezza
	      appropriate per proteggere le infrastrutture e i dati da accessi non
	      autorizzati e da altre minacce.
\end{enumerate}

\subsubsection{Strumenti}
Sono utilizzati programmi autoprodotti e le \textit{GitHub Actions} per
l'automazione di tutte le attività che lo permettono.

\subsubsection{Documentazione}
Mantenere una documentazione dettagliata sulle configurazioni delle
infrastrutture, sulle procedure operative \textit{standard} per garantire trasparenza e
facilitare la gestione.
