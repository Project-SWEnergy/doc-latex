\subsection{Gestione dei Processi}

La gestione dei processi comprende le attività di pianificazione, monitoraggio
e controllo dei processi di ciclo di vita del \textit{software} all'interno del
progetto, assicurando che siano condotti in modo efficace ed efficiente.

\subsubsection{Scopo}
Il principale obiettivo della gestione dei processi è migliorare la qualità
del \textit{software} prodotto e l'efficienza dello sviluppo, attraverso la
standardizzazione dei processi e l'implementazione delle migliori pratiche.

\subsubsection{Attività}
\subsubsubsection{Pianificazione dei Processi} 
Definire gli obiettivi, le
	  procedure e i piani per l'esecuzione e il controllo dei processi di
	  ciclo di vita del \textit{software} (vedi
	  \cref{pianificazione-attivia} 

\subsection{Aggiornamento del "Piano di progetto"}
\label{aggiornare-pdp}

\subsubsection{Descrizione}

Il responsabile deve aggiornare il documento "Piano di progetto".

\subsubsection{\textit{Trigger}}
\begin{itemize}
	\item Inizio di uno sprint\g;
	\item Fine di uno sprint\g;
\end{itemize}

\subsubsection{Scopo}
\begin{itemize}
	\item Formalizzare la pianificazione delle attività da svolgere durante
	      lo sprint\g;

	\item Disambiguare la pianificazione;

	\item Aggiornare le informazioni relative ai rischi e al modello di
	      sviluppo;

	\item Aggiornare le informazioni utili alla verifica dello stato di
	      avanzamento del progetto;
\end{itemize}

\subsubsection{Svolgimento}
Di seguito sono riportate le attività da completare per aggiornare il documento
"Piano di progetto":
\begin{itemize}
	\item \textbf{Rischi e modello di sviluppo}: per quanto riguarda le sezioni
	      relative ai rischi e al modello di sviluppo, il responsabile aggiorna
	      le informazioni in esse contenute in base all'esperienza maturata
	      durante il periodo da responsabile;

	\item \textbf{Pianificazione}: il responsabile aggiorna la sezione di
	      pianificazione rispettando la struttura già definita nel documento.
	      Eventualmente può proporre modifiche alla struttura di pianificazione
	      di perido. Queste sono discusse nelle riunioni interne.
	      Di seguito sono riportati i passi da seguire per aggiornare la sezione
	      di pianificazione:
	      \begin{enumerate}
		      \item \textbf{Creazione}: nella cartella \texttt{preventivi} viene
		            aggiunto un nuovo file \texttt{MM\_GG-P.tex} dove
		            \texttt{MM} e \texttt{GG} indicano rispettivamente il mese e
		            il giorno di inizio del periodo di riferimento;

		      \item \textbf{Stesura}: seguendo la struttura definita nei
		            preventivi precedenti, il responsabile stila la
		            sotto-sezione, riportando le informazioni di pianificazione
		            relative al periodo di riferimento presenti sul progetto di
		            \textit{GitHub}.
	      \end{enumerate}

	\item \textbf{Consuntivo}: medesimo procedimento della sezione di
	      pianificazione. Di seguito sono riportati i passi da seguire per
	      aggiornare la sezione di consuntivo:
	      \begin{enumerate}
		      \item \textbf{Creazione} nella cartella \texttt{consuntivi} viene
		            aggiunto un nuovo file \texttt{MM\_GG-C.tex} dove
		            \texttt{MM} e \texttt{GG} indicano rispettivamente il mese e
		            il giorno di inizio del periodo di riferimento;

		      \item \textbf{Stesura}: seguendo la struttura definita nei
		            consuntivi precedenti, il responsabile stila la
		            sotto-sezione, riportando le informazioni di consuntivo
		            relative al periodo di riferimento. Nota bene: le
		            \textit{issue} di \textit{GitHub} usate per tenere traccia
		            del consuntivo sono quelle generate e chiuse dai membri del
		            gruppo e non quelle create dal responsabile (che
		            sono invece usate per stilare il preventivo).
	      \end{enumerate}

	\item \textbf{Modifica di un documento}: dal momento che
	      l'aggiornamento del documento "Piano di progetto" rientra nella
	      casistica di modifica di un documento, si rimanda alla sezione
	      che illustra come redigere un documento (vedi
	      \ref{redazione-documento}).
\end{itemize}


\subsubsubsection{Monitoraggio e Controllo} 
Tenere traccia dei progressi
	  rispetto ai piani stabiliti e intervenire in caso di deviazioni, per
	  assicurare l'allineamento con gli obiettivi di progetto.

\subsubsubsection{Valutazione dei Processi} 
Analizzare periodicamente
	  l'efficacia e l'efficienza dei processi attuati, identificando aree di
	  miglioramento.

\subsubsubsection{Miglioramento dei Processi} 
Implementare azioni correttive e
	  miglioramenti basati sui risultati delle valutazioni, per ottimizzare
	  i processi di ciclo di vita del \textit{software}.

\subsection{Organizzare un \textit{meeting} interno}
\label{organizzare-meeting-interno}

\subsubsection{Descrizione}
Il responsabile è tenuto ad organizzare i \textit{meeting} interni, ovvero le
\textit{stand-up}. Le \textit{stand-up} sono riunioni brevi, della durata di
circa 30 minuti, che si svolgono su \textit{Discord}. In esse sono trattati i
seguenti argomenti:
\begin{itemize}
	\item \textbf{\textit{Brainstorming}:} i membri del gruppo riassumono
	      brevemente il lavoro svolto nella settimana;

	\item \textbf{Problemi riscontrati:} i membri del gruppo espongono i
	      problemi riscontrati durante la settimana;

	\item \textbf{\textit{To-do list}:} sono discussi i compiti da svolgere nella
	      settimana successiva;

	\item \textbf{\textit{Dubbi}:} i membri del gruppo espongono i dubbi
	      riguardo alle attività da svolgere;

	\item \textbf{Restrospettiva:} i membri del gruppo espongono i problemi,
	      non inerenti alle attività, riscontrati durante la settimana e le
	      possibili soluzioni. I problemi possono, per esempio, riguardare
	      l'organizzazione del lavoro o la comunicazione tra i membri del
	      gruppo o con il proponente.
\end{itemize}

\subsubsection{\textit{Trigger}}
\begin{itemize}
	\item Risulta necessario organizzare un \textit{meeting} interno;
\end{itemize}

\subsubsection{Scopo}
\begin{itemize}
	\item Rendere la comunicazione tra i membri del gruppo più efficace ed
	      efficiente;

	\item Creare della documentazione usufruibile in caso di dubbi o
	      problematiche;

	\item Formalizzare le decisioni prese durante la riunione.
\end{itemize}

\subsubsection{Svolgimento}
Di seguito sono riportate le attività da completare per organizzare una
\textit{stand-up}:

\begin{itemize}

	\item \textbf{Pianificazione:} il responsabile deve decidere quando
	      svolgere la \textit{stand-up}. Di seguito i passi:
	      \begin{enumerate}
		      \item \textbf{Anticipare la data:} nella \textit{stand-up}
		            precedente il responsabile si informa sulle disponibilità
		            dei membri del gruppo per la prossima \textit{stand-up};

		      \item \textbf{Pianificare la data:} il responsabile propone delle
		            date e degli orari per la prossima \textit{stand-up} e le
		            propone sul gruppo \textit{Telegram} del gruppo. I membri
		            del gruppo esprimono la loro preferenza attraverlo un
		            sondaggio.
	      \end{enumerate}

	\item \textbf{Ordine del giorno:} il responsabile deve stilare un ordine del
	      giorno, ovvero una lista degli argomenti da trattare durante la
	      riunione. Di seguito i passi:
	      \begin{enumerate}
		      \item \textbf{Template:} il responsabile utilizza il template
		            delle \textit{stand-up} situato nella \textit{repository}
		            \texttt{appunti-swe};

		      \item \textbf{Brainstorming:} il responsabile si informa con i
		            membri del gruppo attraverso \textit{Telegram} in merito ai
		            punti che ciascun componente di SWEnergy intende trattare
		            durante la riunione;

		      \item \textbf{\textit{To-do list}:} il responsabile stila la lista
		            delle attività da svolgere nella settimana successiva. La
		            lista viene poi discussa e approvata durante la riunione.
	      \end{enumerate}

	\item \textbf{Verbale della riunione:} il responsabile deve redigere il
	      verbale della riunione, in cui vengono riportati gli argomenti
	      trattati e le decisioni prese. Di seguito i passi per redigere il
	      verbale interno:
	      \begin{enumerate}
		      \item \textbf{Appunti:} l'ordine del giorno (il punto precedente)
		            viene utilizzato come base per stilare il verbale interno;

		      \item \textbf{Template:} viene copiata la cartella della
		            riunione precedente e viene rinominata seguendo il formato:
		            \texttt{YYYY-MM-DD\_I};

		      \item \textbf{Stesura:} poichè si tratta di un documento, si
		            rimanda alla sottosezione che illustra come redigere un
		            documento (vedi \autoref{redazione-documento}).
	      \end{enumerate}
\end{itemize}


\subsubsection{Strumenti}
\begin{itemize}
	\item \textbf{GitHub Projects}: Strumento per la pianificazione e il
	      monitoraggio delle attività di progetto;

	\item \textbf{Git}: Sistema di controllo versione per la gestione degli
	      artefatti;

	\item \textbf{Discord e Telegram}: Strumenti di comunicazione interna;
\end{itemize}
