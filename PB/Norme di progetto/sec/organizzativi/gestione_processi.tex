\subsection{Gestione dei Processi}

La gestione dei processi comprende le attività di pianificazione, monitoraggio
e controllo dei processi di ciclo di vita del \textit{software} all'interno del
progetto, assicurando che siano condotti in modo efficace ed efficiente.\\
Il principale obiettivo della gestione dei processi è migliorare la qualità
del \textit{software} prodotto e l'efficienza dello sviluppo, attraverso la
standardizzazione dei processi e l'implementazione delle migliori pratiche.

\subsubsection{Pianificazione dei Processi} 
Definire gli obiettivi, le
	  procedure e i piani per l'esecuzione e il controllo dei processi di
	  ciclo di vita del \textit{software} (vedi
	  \cref{pianificazione-attivia}) 




\subsubsection{Aggiornamento del "Piano di progetto"}
\label{aggiornare-pdp}

\subsubsection*{Trigger}
\begin{itemize}
	\item Inizio di uno \textit{sprint};
	\item Fine di uno \textit{sprint};
\end{itemize}

\subsubsection*{Scopo}
\begin{itemize}
	\item Formalizzare la pianificazione delle attività da svolgere durante
		lo \textit{sprint};

	\item Disambiguare la pianificazione;

	\item Aggiornare le informazioni relative ai rischi e al modello di
		sviluppo;

	\item Aggiornare le informazioni utili alla verifica dello stato di
		avanzamento del progetto;
\end{itemize}

\subsubsection*{Svolgimento}
\begin{itemize}
	\item \textbf{Rischi e modello di sviluppo}: il responsabile aggiorna
		le informazioni in esse contenute in base all'esperienza maturata
		durante il periodo da responsabile;

	\item \textbf{Pianificazione}: il responsabile aggiorna la sezione di
		pianificazione rispettando la struttura già definita nel documento.
		Eventualmente può proporre modifiche alla struttura di pianificazione
		di perido. Queste sono discusse nelle riunioni interne.
		Di seguito sono riportati i passi da seguire per aggiornare la sezione
		di pianificazione:
		\begin{enumerate}
			\item \textbf{Creazione}: nella cartella \texttt{sprint} viene
					aggiunto un nuovo file \\ \texttt{<numero\_dello\_sprint>.tex};

			\item \textbf{Diagramma di Gantt}: il responsabile copia il
					diagramma di Gantt sviluppato nel \textit{project} di
					GitHub\g;

			\item \textbf{Spiegazione del diagramma}: per ciascuna attività
					riportata nel diagramma di Gantt, il responsabile riporta
					chi se ne occupa e la durata prevista in ore;

			\item \textbf{Preventivo}: il responsabile riporta in forma
					tabellare le ore preventivate per ciascuna persona divisa
					per ruolo e calcola le ore ed il costo totali per il
					periodo;
		\end{enumerate}

	\item \textbf{Consuntivo}:
		\begin{enumerate}
			\item \textbf{Riassunto delle attività svolte}: il responsabile
					legge i \textit{commit} e le \textit{issue\g} chiuse durante
					lo \textit{sprint} e ne riporta un riassunto nel documento;

			\item \textbf{Consuntivo}: il responsabile riporta in forma
					tabellare le ore effettivamente impiegate per ciascuna
					persona divise per ruolo e calcola le ore ed il costo
					totali per il periodo. Le ore effettive si trovano sul foglio di calcolo condiviso su Google Drive.

			\item \textbf{Gestione dei ruoli}: il responsabile riporta in
					un diagramma a torta la distribuzione delle ore per ruolo
					effettivamente impiegate durante lo \textit{sprint}.
		\end{enumerate}

	\item \textbf{Modifica di un documento}: dal momento che
		l'aggiornamento del documento "Piano di progetto" rientra nella
		casistica di modifica di un documento, si rimanda alla sezione
		che illustra come redigere un documento (vedi
		\ref{redazione-documento}).
\end{itemize}

\subsubsection*{Strumenti}
\begin{itemize}
	\item \textbf{GitHub\g:} per la gestione del codice sorgente e altro
		materiale di progetto;

	\item \textbf{Google Drive:} per la gestione dei fogli di calcolo;

	\item \textbf{Preventivi:} si tratta di un programma autoprodotto che
		permette di calcolare in modo automatico le ore e i costi totali per
		il periodo oltre a comporre i grafici del documento.
\end{itemize}
	  





\subsubsection{Monitoraggio e Controllo} 
Tenere traccia dei progressi
	  rispetto ai piani stabiliti e intervenire in caso di deviazioni, per
	  assicurare l'allineamento con gli obiettivi di progetto.

\subsubsection{Valutazione dei Processi} 
Analizzare periodicamente
	  l'efficacia e l'efficienza dei processi attuati, identificando aree di
	  miglioramento.

\subsubsection{Miglioramento dei Processi} 
Implementare azioni correttive e
	  miglioramenti basati sui risultati delle valutazioni, per ottimizzare
	  i processi di ciclo di vita del \textit{software}.




\subsubsection{Organizzare un \textit{meeting} interno}
\label{organizzare-meeting-interno}

Il responsabile coordina gli incontri interni, noti anche come \textit{stand-up}.
Queste riunioni, della durata approssimativa di 30 minuti, si tengono sulla piattaforma Discord e affrontano i seguenti punti chiave:
\begin{itemize}
	\item \textbf{\textit{Brainstorming}:} breve riassunto delle attività svolte durante la settimana;
	\item \textbf{Problemi riscontrati:} vengono presentati e discussi i problemi emersi nel corso della settimana;
	\item \textbf{\textit{To-do list}:} si discutono i compiti previsti per la settimana successiva;
	\item \textbf{Dubbi:} si chiariscono eventuali incertezze relative alle attività imminenti;
	\item \textbf{Restrospettiva:} i membri del gruppo condividono riflessioni sui successi e sulle difficoltà incontrate durante la settimana, esplorando insieme possibili soluzioni.
		Questi problemi possono riguardare, ad esempio, l'organizzazione del lavoro o la comunicazione all'interno del \textit{team} o con il proponente.
\end{itemize}

\subsubsection*{Trigger}
\begin{itemize}
	\item Ogni venerdì, per dare tempo al responsabile di preparare il materiale
		per la \textit{stand-up}.
\end{itemize}

\subsubsection*{Scopo}
\begin{itemize}
	\item Rendere la comunicazione tra i membri del gruppo più efficace ed
		efficiente;

	\item Creare della documentazione usufruibile in caso di dubbi o
		problematiche;

	\item Formalizzare le decisioni prese durante la riunione.
\end{itemize}

\subsubsection*{Svolgimento}
\begin{itemize}

	\item \textbf{Pianificazione:} il responsabile deve decidere quando
		svolgere la \textit{stand-up}. Di seguito i passi:
		\begin{enumerate}
			\item \textbf{Anticipare la data:} nella \textit{stand-up}
					precedente il responsabile si informa sulle disponibilità
					dei membri del gruppo rispetto alla prossima
					\textit{stand-up};

			\item \textbf{Pianificare la data:} il responsabile propone delle
					date e degli orari per la prossima \textit{stand-up} sul
					gruppo Telegram\g del gruppo. I membri di SWEnergy esprimono
					la loro preferenza attraverlo un sondaggio.
		\end{enumerate}

	\item \textbf{Ordine del giorno:} il responsabile stila l'ordine del
		giorno, ovvero una lista degli argomenti da trattare durante la
		riunione. Di seguito i passi:
		\begin{enumerate}
			\item \textbf{\textit{Template}:} il responsabile utilizza il \textit{template}
					delle \textit{stand-up} situato nella \textit{repository\g}
					\texttt{appunti-swe/stand-up/template-stand-up.md};

			\item \textbf{Brainstorming:} il responsabile si informa con i
					membri del gruppo attraverso Telegram\g in merito ai
					punti che bisogna trattare durante la riunione;

			\item \textbf{\textit{To-do list}:} il responsabile stila la lista
					delle attività da svolgere nella settimana successiva. La
					lista viene poi discussa e approvata durante la riunione.
		\end{enumerate}

	\item \textbf{Verbale della riunione:} il responsabile redige il
		verbale della riunione, in cui vengono riportati gli argomenti
		trattati e le decisioni prese. Di seguito i passi per redigere il
		verbale interno:
		\begin{enumerate}
			\item \textbf{Appunti:} l'ordine del giorno (il punto precedente)
					viene utilizzato come base per stilare il verbale interno;

			\item \textbf{\textit{Template}:} viene copiata la cartella di \textit{template} dei
					verbali interni e viene rinominata
					seguendo il formato: \texttt{YYYY-MM-DD\_I};

			\item \textbf{Stesura:} poichè si tratta di un documento, si
					rimanda alla sottosezione che illustra come redigere un
					documento (vedi \cref{redazione-documento}).
		\end{enumerate}
\end{itemize}

\subsubsection*{Strumenti}
\begin{itemize}
	\item \textbf{Discord\g:} per svolgere la riunione;
	\item \textbf{Telegram\g:} per comunicare con i membri del gruppo;
	\item \textbf{GitHub\g:} per la gestione del codice sorgente e altro
		materiale di progetto.
\end{itemize}