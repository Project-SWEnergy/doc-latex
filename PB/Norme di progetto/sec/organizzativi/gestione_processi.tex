\subsection{Gestione dei Processi}

La gestione dei processi comprende le attività di pianificazione, monitoraggio
e controllo dei processi di ciclo di vita del \textit{software} all'interno del
progetto, assicurando che siano condotti in modo efficace ed efficiente.

\subsubsection{Scopo}
Il principale obiettivo della gestione dei processi è migliorare la qualità
del \textit{software} prodotto e l'efficienza dello sviluppo, attraverso la
standardizzazione dei processi e l'implementazione delle migliori pratiche.

\subsubsection{Attività}
\begin{enumerate}
	\item \textbf{Pianificazione dei Processi:} Definire gli obiettivi, le
	      procedure e i piani per l'esecuzione e il controllo dei processi di
	      ciclo di vita del \textit{software} (vedi
	      \cref{pianificazione-attivia} e \cref{aggiornare-pdp}).
	\item \textbf{Monitoraggio e Controllo:} Tenere traccia dei progressi
	      rispetto ai piani stabiliti e intervenire in caso di deviazioni, per
	      assicurare l'allineamento con gli obiettivi di progetto.
	\item \textbf{Valutazione dei Processi:} Analizzare periodicamente
	      l'efficacia e l'efficienza dei processi attuati, identificando aree di
	      miglioramento.
	\item \textbf{Miglioramento dei Processi:} Implementare azioni correttive e
	      miglioramenti basati sui risultati delle valutazioni, per ottimizzare
	      i processi di ciclo di vita del \textit{software}.
	\item \textbf{Formazione e Sviluppo del \textit{Team}:} Assicurare che tutti i membri
	      del \textit{team} abbiano le competenze e le conoscenze necessarie per attuare
	      efficacemente i processi definiti.
\end{enumerate}

\subsubsection{Strumenti}
\begin{itemize}
	\item \textbf{GitHub Projects}: Strumento per la pianificazione e il
	      monitoraggio delle attività di progetto;

	\item \textbf{Git}: Sistema di controllo versione per la gestione degli
	      artefatti;

	\item \textbf{Discord e Telegram}: Strumenti di comunicazione interna;
\end{itemize}
