\subsection{Miglioramento del Processo}

Il miglioramento del processo si basa sul Ciclo di Miglioramento Continuo PDCA.

\subsubsection{Scopo}
Lo scopo del processo consiste nell'ottimizzare i processi organizzativi e
incrementare l'efficacia e l'efficienza nel ciclo di vita del software.

\subsubsection{Attività}
\subsubsubsection{\textit{Plan}}
	  Definire gli obiettivi specifici di miglioramento, identificare le
	  attività necessarie per raggiungerli, stabilire le scadenze e
	  assegnare le responsabilità. Questo include la selezione di metriche
	  di processo per misurare l'efficacia delle azioni di miglioramento

\subsubsubsection{Aggiornamento delle "Norme di progetto"}
\label{aggiornare-ndp}

\subsubsubsubsection{\textit{Trigger}}
\begin{itemize}
	\item Si discute di qualche processo da aggiungere o modificare durante un
	      \textit{meeting}.
\end{itemize}

\subsubsubsubsection{Scopo}
\begin{itemize}
	\item Mantenere il documento coerente rispetto al modello di sviluppo e ai
	      processi adottati da SWEnergy;

	\item Formalizzare i processi adottati da SWEnergy, per chiarire eventuali
	      dubbi e per facilitare l'individuazione di attività e processi da
	      svolgere;

	\item Mostrare l'evoluzione dell'organizzazione del lavoro di SWEnergy;

	\item Evidenziare i dubbi e le lacune intestini ai processi di sviluppo.
\end{itemize}

\subsubsubsubsection{Svolgimento}
\begin{itemize}
	\item \textbf{Identificazione delle attività}: in quale modo
	      l'amministratore ed il gruppo possono individuare le attività da
	      includere nel documento. Di seguito sono riportati i passi da
	      seguire:
	      \begin{enumerate}
		      \item \textbf{Nuova attività}: durante gli incontri,
		            SWEnergy si rende conto che alcune attività si
		            presentano di frequente;

		      \item \textbf{Ipotesi}: SWEnergy ipotizza il flusso di lavoro da
		            svolgere per completare l'attività. Sono stesi degli appunti
		            che verranno poi inseriti nel documento "Norme di progetto";

		      \item \textbf{Sperimentazione}: i componenti del gruppo che
		            svolgono l'attività, sperimentano diverse tecniche per
		            il suo completare, partendo dall'ipotesi iniziale;

		      \item \textbf{Perfezionamento}: i componenti che hanno
		            svolto l'attività, spiegano al gruppo il processo
		            seguito. SWEnergy lo discute e lo valuta;

		      \item \textbf{Formalizzazione}: l'amministratore inserisce
		            l'attività nel documento "Norme di progetto". \textit{Nota:
		            viene modificato un documento, quindi si rimanda alla
		            sottosezione che illustra come redigere un documento
		            (vedi \cref{redazione-documento})}.
	      \end{enumerate}

	\item \textbf{Aggiornamento delle attività}: in seguito ad una discussione
	      organica a SWEnergy, l'amministratore modifica l'attivtà
	      nel documento "Norme di progetto". \textit{Nota:
	      viene modificato un documento, quindi si rimanda alla
	      sottosezione che illustra come redigere un documento
	      (vedi \cref{redazione-documento})}.
\end{itemize}


\subsubsubsection{\textit{Do}}
	  Implementare le attività pianificate, seguendo i piani stabiliti.
	  Questo può includere la formazione del personale, l'aggiornamento
	  delle procedure o l'introduzione di nuovi strumenti e tecnologie.

\subsubsubsection{\textit{Check}}
	  Monitorare e valutare l'esito delle azioni di miglioramento rispetto
	  agli obiettivi prefissati, utilizzando le metriche di processo
	  definite nella fase di pianificazione. Analizzare i dati raccolti per
	  identificare le tendenze, le deviazioni e le aree che necessitano di
	  ulteriori miglioramenti.

\subsubsubsection{\textit{Act}}
	  Sulla base dei risultati ottenuti nella fase di valutazione,
	  intraprendere azioni correttive per consolidare i miglioramenti
	  ottenuti e indirizzare le aree che non hanno raggiunto gli obiettivi
	  prefissati. Questa fase può anche includere la standardizzazione di
	  nuove pratiche di successo e la modifica dei piani di miglioramento
	  per i cicli futuri.

\subsubsubsection{Ciclicità del Processo}
	  Ripetere il ciclo PDCA per garantire un miglioramento continuo dei
	  processi, adattando gli obiettivi e le strategie in base ai risultati
	  ottenuti e alle nuove priorità identificate.
