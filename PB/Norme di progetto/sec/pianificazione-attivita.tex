\subsection{Pianificazione delle attività}
\label{pianificazione-attivia}

\subsubsection{Descrizione}
Il responsabile è tenuto a pianificare le attività da svolgere durante lo sprint
in cui si trova il gruppo e a suddividerle tra i membri del gruppo. Inoltre deve
aggiornare il programma di lavoro in base alle attività svolte e a quelle da
svolgere. La pianificazione avviene tramite l'uso dei diagramma di Gantt
disponibili su \textit{GitHub}.

\subsubsection{\textit{Trigger}}
\begin{itemize}
	\item Comincia una nuova iterazione, che sia uno sprint\g od un
	      mini-sptrint\g.
\end{itemize}

\subsubsection{Scopo}
\begin{itemize}
	\item Aggiornare il programma di lavoro in base alle attività svolte e a
	      quelle da svolgere.

	\item Pianificare le attività da svolgere durante l'iterazione corrente;

	\item Guidare lo svolgimento delle attività;

	\item Produrre la documentazione che permette di tenere traccia delle
	      attività svolte e di quelle da svolgere.
\end{itemize}

\subsubsection{Svolgimento}
Di seguito sono riportate le attività da completare per pianificare le attività
da svolgere:

\begin{itemize}
	\item \textbf{Creazione delle \textit{issue}}: il responsabile deve creare
	      delle issue che descrivono le attività da svolgere e guidano i
	      lavoratori nella loro esecuzione. Di seguito sono riportati i passi
	      per definire le issue:
	      \begin{enumerate}
		      \item \textbf{Identificazione}: il responsabile identifica le
		            attività da svolgere e le aggiunge su \textit{GitHub};

		      \item \textbf{Priorità}: il responsabile assegna una priorità
		            alle issue in base all'urgenza e all'importanza;

		      \item \textbf{Scadenza}: il responsabile assegna una data di
		            scadenza alle issue in base alla priorità e alla durata
		            dell'attività. L'attività viene quindi inserita nel
		            \textit{project} di \textit{Github} corrispondente alla
		            milestone di riferimento. In questo modo viene aggiornato il
		            diagramma di Gantt;

		      \item \textbf{Perfezionamento}: il responsabile guida la
		            discussione in merito alle \textit{issue} durante le
		            riunioni. In questo modo sono aggiornate la priorità, la
		            scadenza e la descrizione;

		      \item \textbf{Assegnazione}: il responsabile assegna le issue
		            ai membri del gruppo in base alle loro competenze e
		            disponibilità.
	      \end{enumerate}
\end{itemize}
