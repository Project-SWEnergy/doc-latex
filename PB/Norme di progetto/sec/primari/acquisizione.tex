\subsection{Acquisizione}

Il processo di acquisizione coinvolge la definizione dei requisiti di sistema e
\textit{software}, la valutazione e selezione dei potenziali fornitori, e la gestione del
contratto con il fornitore selezionato.

\subsubsection{Scopo}
Garantire che il \textit{software} acquisito soddisfi i requisiti stabiliti, rispetti i
vincoli di budget e di tempo, e sia conforme agli \textit{standard} di qualità previsti.

\subsubsection{Attività}
\begin{enumerate}
	\item \textbf{Definizione dei requisiti}: identificazione delle necessità e
	      delle aspettative degli \textit{stakeholder};
	\item \textbf{Selezione del fornitore}: valutazione delle offerte e scelta
	      del fornitore più adatto;
	\item \textbf{Gestione del contratto}: definizione degli accordi
	      contrattuali, monitoraggio della conformità e gestione delle
	      modifiche;
	\item \textbf{Accettazione del \textit{software}}: verifica e approvazione
	      del \textit{software} consegnato rispetto ai requisiti concordati.
\end{enumerate}

\subsubsection{Strumenti}
\begin{itemize}
	\item \textbf{Zoom}: strumento di videoconferenza utilizzato per le comunicazioni a distanza con il committente;
	\item \textbf{Microsoft Teams}: strumento di videoconferenza utilizzato per le comunicazioni a distanza con il proponente;
	\item \textbf{Presentazioni di Google}: strumento per la creazione di presentazioni utilizzato per la comunicazione con il cliente;
	\item \textbf{Advanced Slides}: strumento interno ad Obsidian per la creazione di presentazioni, utilizzato per la comunicazione con il proponente.
\end{itemize}
