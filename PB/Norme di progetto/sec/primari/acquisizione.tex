\subsection{Acquisizione}

Il processo di acquisizione coinvolge la definizione dei requisiti di sistema e
\textit{software}, la valutazione e selezione dei potenziali fornitori, e la gestione del
contratto con il fornitore selezionato.\\
Lo scopo è garantire che il \textit{software} acquisito soddisfi i requisiti stabiliti, rispetti i
vincoli di budget e di tempo, e sia conforme agli \textit{standard} di qualità previsti.

\subsubsection{Definizione dei requisiti} 
Identificazione delle necessità e delle aspettative degli \textit{stakeholder};

\subsubsection{Selezione del fornitore} 
Valutazione delle offerte e scelta del fornitore più adatto;

\subsubsection{Gestione del contratto} 
Definizione degli accordi contrattuali, monitoraggio della conformità e gestione 
delle modifiche;

\subsubsection{Accettazione del \textit{software}} 
Verifica e approvazione del \textit{software} consegnato rispetto ai requisiti 
concordati.

\subsubsection{Strumenti}
\begin{itemize}
	\item \textbf{Zoom}: strumento di videoconferenza utilizzato per le comunicazioni a distanza con il committente;
	\item \textbf{Microsoft Teams}: strumento di videoconferenza utilizzato per le comunicazioni a distanza con il proponente;
	\item \textbf{Presentazioni di Google}: strumento per la creazione di presentazioni utilizzato per la comunicazione con il cliente;
	\item \textbf{Advanced Slides}: strumento interno ad Obsidian per la creazione di presentazioni, utilizzato per la comunicazione con il proponente.
\end{itemize}
