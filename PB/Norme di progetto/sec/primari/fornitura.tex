\subsection{Fornitura}
Il processo di fornitura copre tutte le attività essenziali per la consegna del \textit{software} sviluppato al committente.
In questo contesto, il committente è rappresentato dal corpo docente o dai
revisori del progetto universitario, nonché da un rappresentante di Imola
Informatica, ovvero il proponente.
Questo processo si focalizza sulla preparazione e presentazione del \textit{software} e della relativa documentazione, assicurandosi che siano conformi ai requisiti del corso e alle aspettative degli \textit{stakeholder}.

\subsubsection{Scopo}
L'obiettivo principale è garantire che il \textit{software} e tutti i materiali
di supporto siano pronti per la valutazione finale, rispettando i criteri di
accettazione definiti.

\subsubsection{Attività}
\begin{enumerate}
	\item \textbf{Preparazione finale:} Completamento di tutte le attività di
	      sviluppo, \textit{testing} e documentazione.
	\item \textbf{Revisione della documentazione:} Assicurare che tutta la
	      documentazione sia completa, accurata e pronta per la revisione
	      (vedi \cref{subsec:approvazione}).
	\item \textbf{Presentazione:} Organizzare e condurre una presentazione del
	      progetto, dimostrando le funzionalità del \textit{software} e
	      discutendo la documentazione.
	\item \textbf{Consegna:} Fornire il \textit{software} e tutta la
	      documentazione correlata ai revisori o ai docenti.
\end{enumerate}

\subsubsection{Strumenti}
Gli strumenti utilizzati in questo processo includono sistemi di
\textit{versioning} come Git e strumenti per presentazioni come Presentazioni di Google o LaTeX.

\textit{Nota: Poiché questo progetto si inserisce in un contesto universitario,
	non sono previste attività di supporto o assistenza post-vendita una volta
	consegnato il \textit{software}.}
