\subsection{Fornitura}
Il processo di fornitura copre tutte le attività essenziali per la consegna del \textit{software} sviluppato al committente.
In questo contesto, il committente è rappresentato dal corpo docente o dai revisori del progetto universitario, nonché da un rappresentante di Imola Informatica, ovvero il proponente.
Questo processo si focalizza sulla preparazione e presentazione del \textit{software} e della relativa documentazione, assicurandosi che siano conformi ai requisiti del corso e alle aspettative degli \textit{stakeholder}. \\
L'obiettivo principale è garantire che il \textit{software} e tutti i materiali di supporto siano pronti per la valutazione finale, rispettando i criteri di accettazione definiti.


\subsubsection{Preparazione finale} 
Completamento di tutte le attività di:
\begin{itemize}
	\item sviluppo: \cref{sviluppo};
	\item \textit{testing}: \cref{verifica-codice};
	\item documentazione: \cref{redazione-documento};
\end{itemize}


\subsubsection{Revisione della documentazione} 
Assicurare che tutta la documentazione sia completa, accurata e pronta per la revisione (vedi \cref{subsec:approvazione}).

\subsubsection{Presentazione} 
Organizzare e condurre una presentazione del progetto, dimostrando le 
funzionalità del \textit{software} e discutendo la documentazione.
Vedere \cref{organizzare-meeting-esterno} per ulteriori informazioni.

\subsubsection{Consegna} 
Fornire il \textit{software} e tutta la documentazione correlata ai revisori o ai docenti. \\
Il proponente necessiterà di:
\begin{itemize}
	\item \textit{repository} contenente il codice del progetto completo;
	\item manuale utente;
	\item specifiche tecniche;
\end{itemize}
I docenti necessiteranno di:
\begin{itemize}
	\item norme di progetto;
	\item analisi dei requisiti;
	\item piano di qualifica;
	\item piano di progetto;
	\item verbali interni;
	\item verbali esterni;
	\item manuale utente;
	\item specifiche tecniche;
	\item \textit{repository} contenente il codice del progetto completo;
\end{itemize}

\subsubsection{Strumenti}
Gli strumenti utilizzati in questo processo includono sistemi di
\textit{versioning} come Git e strumenti per presentazioni come Presentazioni di Google o LaTeX.

\textit{Nota: Poiché questo progetto si inserisce in un contesto universitario,
	non sono previste attività di supporto o assistenza post-vendita una volta
	consegnato il \textit{software}.}
