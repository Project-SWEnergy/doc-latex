\subsection{Sviluppo}
\label{sviluppo}
Questo processo comprende tutte le attività necessarie per trasformare i requisiti in un \textit{software} funzionante e conforme alle aspettative degli \textit{stakeholder}. \\
Assicurare la creazione di un \textit{software} che risponda pienamente ai bisogni degli utenti, sia tecnicamente valido, mantenibile e scalabile.


\subsubsection{Aggiornamento della "Analisi dei Requisiti"}
\label{aggiornare-adr}
\subsubsection*{Trigger}
\begin{itemize}
	\item Sono presenti dei dubbi o delle lacune in merito a qualcosa;
	\item Risulta necessario formalizzare qualche concetto o qualche argomento.
\end{itemize}

\subsubsection*{Scopo}
\begin{itemize}
	\item Risolvere i dubbi e le lacune riguardo a un argomento, o almeno formalizzare i dubbi e le lacune;
	\item Formalizzare la definizione di un concetto o di un argomento, per renderlo chiaro ed inequivoco.
\end{itemize}

\subsubsection*{Svolgimento}
\begin{itemize}
	\item \textbf{Identificazione dei casi d'uso}: in quale modo l'analista ed il gruppo possono individuare i casi d'uso da includere nel documento.
		Di seguito sono riportati i passi da seguire:
		\begin{enumerate}
			\item \textbf{Ipotesi}: l'analista impotizza il flusso di azioni da svolgere per portare a termine l'azione dell'utente;
			\item \textbf{Dubbi}: l'analista si confronta con il gruppo e formalizza i dubbi e le lacune all'interno delle Discussion di Github\g;
			\item \textbf{Sperimentazione}: viene implementato il caso d'uso in modo da verifcare il flusso di azioni ipotizzato;
			\item \textbf{Formalizzazione}: l'analista aggiorna la "Analisi dei Requisiti" rispetto alle informazioni raccolte, potrebbe dover modificare anche i requisiti per tenerli aggiornati rispetto ai casi d'uso. 
			\textit{Nota: viene modificato un documento, quindi si rimanda alla sottosezione che illustra come redigere un documento (vedi \cref{redazione-documento})}.
		\end{enumerate}
\end{itemize}

\subsubsection*{Strumenti}
Lo strumento utilizzato per l'aggiornamento del documento è:
\begin{itemize}
	\item \textbf{Discussion di GitHub\g}: per mantenere e formalizzare i dubbi e le lacune riscontrate.
\end{itemize}





\subsubsection{Progettazione}
\label{progettazione}
Nel processo di progettazione del \textit{software}, i progettisti sono incaricati di definire l'architettura del sistema, i moduli e le interazioni tra essi. 
Devono inoltre elaborare i \textit{\textit{test}} di unità e di integrazione, assicurando così la corretta funzionalità dell'intero sistema.

\subsubsection*{Trigger}
\begin{itemize}
	\item Dopo l'RTB avviene la fase di progettazione più vasta, ma non in dettaglio;
	\item Ogni volta che deve essere implementata una nuova funzionalità, viene progettata la struttura del \textit{software} che la implementa in modo dettagliato.
\end{itemize}

\subsubsection*{Scopo}
\begin{itemize}
	\item Definire l'architettura del sistema;
	\item Definire i moduli e le interfacce tra di essi;
	\item Definire i \textit{test} di unità e di integrazione.
\end{itemize}

\subsubsection*{Svolgimento}
\begin{itemize}
	\item \textbf{Progettazione ad alto livello:} il progettista definisce l'architettura del sistema, i moduli e le interfacce tra di essi. 
		Di seguito i passi che vengono seguiti:
		\begin{enumerate}
			\item \textbf{Ripasso dei requisiti:} il progettista studia i requisiti e le specifiche del sistema;
			\item \textbf{Studio delle PoC\g:} il progettista studia le PoC\g per individuare i problemi e le soluzioni adottate;
			\item \textbf{Descrizione:} partendo dalle PoC\g, il progettista crea degli appunti che evidenzino la struttura da realizzare;
			\item \textbf{Definizione dell'architettura:} a partire dalla descrizione del sistema, il progettista crea i diagrammi delle classi, per guidare lo sviluppo del sistema;
			\item \textbf{Appunti integrativi:} il progettista crea degli appunti per motivare le scelte fatte e per supplire alle mancanze dei diagrammi delle classi;
			\item \textbf{Test di integrazione:} il progettista definisce i \textit{test} di integrazione, in modo da verificare che il sistema funzioni correttamente.
		\end{enumerate}
	\item \textbf{Progettazione di dettaglio:} il progettista definisce i dettagli di implementazione di una nuova funzionalità. 
		Di seguito i passi da seguire:
		\begin{enumerate}
			\item \textbf{Scelta della funzionalità:} il progettista sceglie la funzionalità da implementare;
			\item \textbf{Studio dell'architettura:} il progettista studia l'architettura del sistema, per capire come la nuova funzionalità si inserisce nel sistema;
			\item \textbf{Definizione delle interfacce:} il progettista definisce le interfacce tra i moduli;
			\item \textbf{Descrizione:} il progettista crea degli appunti integrativi, per guidare lo sviluppo del programmatore e per motivare le scelte fatte;
			\item \textbf{Definizione dei \textit{test} di unità:} il progettista definisce i \textit{test} di unità, in modo da verificare che la nuova funzionalità sia implementata correttamente.
		\end{enumerate}
\end{itemize}

\subsubsection*{Strumenti}
\begin{itemize}
	\item \textbf{StarUML:} per la creazione dei diagrammi delle classi;
	\item \textbf{GitHub\g:} per la condivisione dei diagrammi delle classi e degli appunti;
\end{itemize}





\subsubsection{Codifica}
\label{codifica}
Il programmatore scrive il codice sorgente che compone l'applicativo. 
Il codice sorgente è scritto in linguaggio TypeScript.

\subsubsection*{Trigger}
\begin{itemize}
	\item Viene completata la progettazione di una \textit{feature};
\end{itemize}

\subsubsection*{Scopo}
\begin{itemize}
	\item Implementare le funzionalità richieste dal proponente;
	\item Soddisfare qualche requisito;
\end{itemize}

\subsubsection*{Svolgimento}
\begin{itemize}
	\item \textbf{Progettazione:} il programmatore deve produrre dei commenti o degli appunti che descrivano la struttura del codice che andrà a scrivere nella prossima attività. 
		Questi commenti devono poi essere riorganizzati e riportati nella \textit{issue\g} corrispondente;
	\item \textbf{Test:} il programmatore implementa i \textit{test} per verificare il corretto funzionamento del codice che andrà a scrivere.
		Per maggiori informazioni vedere \cref{testing};
	\item \textbf{Codifica di una funzione o metodo:} di seguito sono elencati i passi che il programmatore deve seguire per la codifica del prodotto software:
		\begin{enumerate}
			\item \textbf{Pull:} il programmatore esegue un \textit{pull} del codice sorgente dal \textit{repository\g} remoto;
			\item \textbf{Branch:} il programmatore crea un nuovo branch di lavoro a partire dal branch \texttt{dev};
			\item \textbf{Commenti:} il programmatore scrive lo scopo della funzione o del metodo che andrà a codificare e ne descrive la firma;
			\item \textbf{Codifica:} il programmatore scrive il codice che compone il corpo della funzione o del metodo;
			\item \textbf{Test:} il programmatore esegue i \textit{test} di verifica.
			In caso di fallimento, il programmatore deve correggere il codice e ripetere la verifica.
			Per maggiori informazioni vedere \cref{testing};
			\item \textbf{Iterazione:} se il programmatore vuole scrivere altre funzioni torna al punto 3, altrimenti prosegue con il punto successivo;
			\item \textbf{Push:} il programmatore esegue un \textit{push} del codice sorgente sul \textit{repository\g} remoto.
			\item \textbf{Verifica:} il programmatore segnala al verificatore che il codice è pronto per essere verificato.
			\item \textbf{Correzione:} se il verificatore segnala degli errori, il programmatore deve correggere il codice e torna al passo precendente. 
			Altrimenti, il programmatore può procedere al passo successivo.
			\item \textbf{Chiusura:} il programmatore effettua il \textit{merge} del \textit{branch} di lavoro con il \textit{branch} \texttt{dev} e chiude il \textit{ticket} di \textit{GitHub\g} corrispondente.
		\end{enumerate}
\end{itemize}

\subsubsection*{Strumenti}
Per il processo di sviluppo sono utilizzati gli IDE \textit{VSCode} oppure \textit{NeoVim}, il sistema di \textit{versioning} Git e l'organizzazione {GitHub\g} per la gestione del codice sorgente e altro materiale di progetto. 
Sono adottate le \textit{GitHub Actions} per l'automazione di \textit{test} e \textit{deployment}.




\subsubsection{Testing} 
\label{testing}
Verifica della correttezza del \textit{software} attraverso \textit{test} funzionali, di integrazione e di sistema.

\subsubsection*{Trigger}
\begin{itemize}
	\item Viene completata la codifica di una \textit{feature};
\end{itemize}

\subsubsection*{Scopo}
\begin{itemize}
	\item Verificare la corretta implementazione delle funzionalità richieste dal proponente;
	\item Verificare che le funzionalità implementate restituiscano risultati corretti;
\end{itemize}

\subsubsection*{Svolgimento}
Di seguito sono elencati i passaggi che il team di sviluppo deve seguire per condurre il processo di \textit{test} utilizzando il framework Jest:
\begin{itemize}
	\item \textbf{Configurazione dell'Ambiente:} assicurarsi che l'installazione delle dipendenze necessarie sia avvenuta con successo.
	\item \textbf{Scrittura dei Test:} scrivere i \textit{test} per verificare il corretto funzionamento di \textit{controller} e servizi.
	\item \textbf{Organizzazione dei Test:} i \textit{test} devono essere organizzati in \textit{file} appropriati all'interno della struttura del progetto.
		Ogni modulo realizzato è contenuto in una cartella e dispone di un \textit{file} per modulo, \textit{controller} e servizi.
		Vi devono essere anche i corrispondenti \textit{file} con estensione \texttt{.spec.ts} per \textit{controller} e servizi.
	\item \textbf{Esecuzione dei Test:} utilizzando i comandi appropriati forniti da Jest, si eseguono i \textit{test} per verificare il comportamento dell'applicazione.
	\item \textbf{Analisi dei Risultati:} dopo l'esecuzione dei \textit{test}, si analizzano i risultati per identificare eventuali problemi o malfunzionamenti nell'applicazione. 
		Questo può includere la correzione di \textit{bug}, il miglioramento della copertura dei \textit{test} o altre azioni correttive necessarie.
\end{itemize}

\subsubsection*{Strumenti}
Per il processo di sviluppo sono utilizzati gli IDE \textit{VSCode} oppure  \textit{NeoVim}, il sistema di \textit{versioning} Git e l'organizzazione {GitHub\g} per la gestione del codice
sorgente e altro materiale di progetto. 
Sono adottate le \textit{GitHub Actions} per l'automazione di \textit{test} e \textit{deployment}.
Per il \textit{back-end} si è utilizzato il \textit{framework} Jest.
