\subsection{Progettazione}
\label{progettazione}
Nel processo di progettazione del \textit{software}, i progettisti sono incaricati di definire l'architettura del sistema, i moduli e le interazioni tra essi. 
Devono inoltre elaborare i \textit{test} di unità e di integrazione, assicurando così la corretta funzionalità dell'intero sistema.

\subsubsection{\textit{Trigger}}

\begin{itemize}
	\item Dopo l'RTB avviene la fase di progettazione più vasta, ma non in
	      dettaglio;

	\item Ogni volta che viene implementata una nuova funzionalità, viene
	      progettata la struttura del \textit{software} che la implementa in
	      modo dettagliato.
\end{itemize}

\subsubsection{Scopo}
\begin{itemize}
	\item Definire l'architettura del sistema;
	\item Definire i moduli e le interfacce tra di essi;
	\item Definire i test di unità e di integrazione.
\end{itemize}

\subsubsection{Svolgimento}
\begin{itemize}
	\item \textbf{Progettazione ad alto livello:} il progettista definisce
	      l'architettura del sistema, i moduli e le interfacce tra di essi. Di
	      seguito i passi che vengono seguiti:
	      \begin{enumerate}
		      \item \textbf{Ripasso dei requisiti:} il progettista studia i
		            requisiti e le specifiche del sistema;

		      \item \textbf{Studio delle PoC\g:} il progettista studia le PoC\g per
		            individuare i problemi e le soluzioni adottate;

		      \item \textbf{Descrizione:} partendo dalle PoC\g, il progettista
		            crea degli appunti che evidenzino la struttura da
		            realizzare;

		      \item \textbf{Definizione dell'architettura:} a partire dalla
		            descrizione del sistema, il progettista crea i diagrammi
		            delle classi, per guidare lo sviluppo del sistema;

		      \item \textbf{Appunti integrativi:} il progettista
		            crea degli appunti per motivare le scelte fatte e per
		            supplire alle mancanze dei diagrammi delle classi;

		      \item \textbf{Test di integrazione:} il progettista definisce i
		            test di integrazione, in modo da verificare che il sistema
		            funzioni correttamente.
	      \end{enumerate}

	\item \textbf{Progettazione di dettaglio:} il progettista definisce i
	      dettagli di implementazione di una nuova funzionalità. Di seguito i
	      passi da seguire:
	      \begin{enumerate}
		      \item \textbf{Scelta della funzionalità:} il progettista
		            sceglie la funzionalità da implementare;

		      \item \textbf{Studio dell'architettura:} il progettista
		            studia l'architettura del sistema, per capire come
		            la nuova funzionalità si inserisce nel sistema;

		      \item \textbf{Definizione delle interfacce:} il
		            progettista definisce le interfacce tra i moduli;

		      \item \textbf{Descrizione:} il progettista crea degli appunti
		            integrativi, per guidare lo sviluppo del programmatore e
		            per motivare le scelte fatte;

		      \item \textbf{Definizione dei test di unità:} il progettista
		            definisce i test di unità, in modo da verificare che la
		            nuova funzionalità sia implementata correttamente.
	      \end{enumerate}
\end{itemize}

\subsubsection{Strumenti}
\begin{itemize}
	\item \textbf{StarUML:} per la creazione dei diagrammi delle classi;
	\item \textbf{GitHub\g:} per la condivisione dei diagrammi delle classi e
	      degli appunti;
\end{itemize}

