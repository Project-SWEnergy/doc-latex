\subsection{Redazione di un documento}
\label{redazione-documento}

\subsection{Descrizione}

L'analista redige i documenti, in particolare l'analista redige l'"Analisi dei
requisiti".

\subsubsection{\textit{Trigger}}
\begin{itemize}
	\item Sono presenti dei dubbi o delle lacune in merito a qualcosa;

	\item Risulta necessario formalizzare qualche concetto o qualche argomento.
\end{itemize}

\subsubsection{Scopo}
\begin{itemize}
	\item Risolvere i dubbi e le lacune riguardo a un argomento, o almeno
	      formalizzare i dubbi e le lacune;

	\item Formalizzare la definizione di un concetto o di un argomento, per
	      renderlo chiaro ed inequivoco.
\end{itemize}

\subsubsection{Strumenti}
Gli strumenti utilizzati per la creazione dei documenti sono:
\begin{itemize}
	\item \textbf{LaTeX}: linguaggio di \textit{markup} per la creazione di documenti \\
	      \href{https://www.latex-project.org/}{(www.latex-project.org)};
	\item \textbf{VisualStudio Code}: GUI con integrazioni per la creazione di documenti scritti in LaTeX e per la gestione delle repository git \\
	      \href{https://code.visualstudio.com/}{(code.visualstudio.com)}
	      \begin{itemize}
		      \item \textbf{LaTeX Workshop}: estensione utilizzata in VisualStudio Code per la compilazione e la scrittura dei documenti.
	      \end{itemize}
\end{itemize}

\subsubsection{Struttura del documento}

A ciascun documento corrisponde un'omonica cartella che viene creta all'interno
della cartella che rappresenta la fase in cui si trova il progetto quando viene
prodotto il documento. La cartella della fase si trova nella \textit{repository}
\href{https://github.com/Project-SWEnergy/doc-latex}{\texttt{doc-latex}}
dell'organizzazione GitHub del gruppo.
Il nome della cartella è il nome del documento in deve avere la prima lettera
maiuscola; sono previsti gli spazi tra le parole e le parole successive alla
prima sono in minuscolo. Di seguito la struttura della cartella:

\vspace{0.5cm}

\dirtree{%
	.1 / (Nome del documento).
	.2 main.tex.
	.2 sec.
	.3 registro\_modifiche.tex.
	.3 introduzione.tex.
	.3 le\_altre\_sezioni.tex.
}

\subsubsection{\texttt{main.tex}}
Di seguito la struttura del file \texttt{main.tex}:
\begin{itemize}
	\item \textbf{Import dei template}: sono importati i template per la
	      creazione del documento. I template sono: \texttt{copertina.tex},
	      \texttt{header\_footer.tex} e \texttt{variable.tex}. In aggiunta, sono
	      importati i modelli specifici per il documento che si sta redigendo;

	\item \textbf{Inizializzazione delle variabili:} sono inizializzate le
	      variabili che verranno utilizzate nel documento;

	\item \textbf{Struttura del documento:} attraverso l'uso degli
	      \texttt{input} viene gestita la struttura del documento.
\end{itemize}

\subsubsection{Svolgimento}
Di seguito sono elencate le attività che l'analista deve svolgere per la
redazione di un documento:
\begin{itemize}
	\item \textbf{Modifica di un documento}: l'analista aggiorna il documento in
	      base alle modifiche richieste dal verificatore e in base alle
	      informazioni necessarie per la redazione del documento. Di seguito sono
	      elencati i passi per completare l'attività:
	      \begin{enumerate}
		      \item \textbf{\textit{Pull}}: l'analista effettua il \textit{pull}
		            della \textit{repository} \texttt{doc-latex} per avere
		            l'ultima versione della \textit{repository};

		      \item \textbf{\textit{Checkout:}} l'analista effettua il
		            \textit{checkout} del \textit{branch} verso il
		            \textit{branch} chiamato come il documento che si sta
		            redigendo;

		      \item \textbf{Struttura:} l'analista modifica il
		            \texttt{main.tex} in base alle modifiche necessarie;

		      \item \textbf{Gestione dei \textit{file}:} l'analista crea,
		            elimina o rinomina i \textit{file} nella cartella
		            \texttt{sec} in modo tale che siano rispecchiate le
		            modifiche apportate al \texttt{main.tex};

		      \item \textbf{Contenuto:} l'analista modifica i \textit{file}
		            nella cartella \texttt{sec} in base alle modifiche
		            necessarie;

		      \item \textbf{\textit{Push}:} l'analista effettua un
		            \textit{commit} e il \textit{push};

		      \item \textbf{\textit{Pull request}:} l'analista può creare
		            una \textit{pull request} verso il \texttt{main},
		            per chiedere al verificatore, la verifica del documento;

		      \item \textbf{Verifica:} l'analista informa il verificatore
		            che il documento è pronto per la verifica;

		      \item \textbf{Correzione:} l'analista corregge il documento
		            in base alle segnalazioni del verificatore;

		      \item \textbf{Chiusura:} l'analista effettua un push del
		            branch inserendo nel messaggio di \textit{commit} la parola
		            \texttt{close} seguita dal numero della issue che si sta
		            risolvendo;

		      \item \textbf{Secondo merge:} l'analista può concludere la
		            \textit{pull request} con il \textit{main}.
	      \end{enumerate}
\end{itemize}
