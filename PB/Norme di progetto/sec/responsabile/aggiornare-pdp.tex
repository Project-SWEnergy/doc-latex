\subsection{Aggiornamento del "Piano di progetto"}
\label{aggiornare-pdp}

\subsubsection{\textit{Trigger}}
\begin{itemize}
	\item Inizio di uno \textit{sprint};
	\item Fine di uno \textit{sprint};
\end{itemize}

\subsubsection{Scopo}
\begin{itemize}
	\item Formalizzare la pianificazione delle attività da svolgere durante
	      lo \textit{sprint};

	\item Disambiguare la pianificazione;

	\item Aggiornare le informazioni relative ai rischi e al modello di
	      sviluppo;

	\item Aggiornare le informazioni utili alla verifica dello stato di
	      avanzamento del progetto;
\end{itemize}

\subsubsection{Svolgimento}
\begin{itemize}
	\item \textbf{Rischi e modello di sviluppo}: il responsabile aggiorna
	      le informazioni in esse contenute in base all'esperienza maturata
	      durante il periodo da responsabile;

	\item \textbf{Pianificazione}: il responsabile aggiorna la sezione di
	      pianificazione rispettando la struttura già definita nel documento.
	      Eventualmente può proporre modifiche alla struttura di pianificazione
	      di perido. Queste sono discusse nelle riunioni interne.
	      Di seguito sono riportati i passi da seguire per aggiornare la sezione
	      di pianificazione:
	      \begin{enumerate}
		      \item \textbf{Creazione}: nella cartella \texttt{sprint} viene
		            aggiunto un nuovo file \\ \texttt{<numero\_dello\_sprint>.tex};

		      \item \textbf{Diagramma di Gantt}: il responsabile copia il
		            diagramma di Gantt sviluppato nel \textit{project} di
		            GitHub\g;

		      \item \textbf{Spiegazione del diagramma}: per ciascuna attività
		            riportata nel diagramma di Gantt, il responsabile riporta
		            chi se ne occupa e la durata prevista in ore;

		      \item \textbf{Preventivo}: il responsabile riporta in forma
		            tabellare le ore preventivate per ciascuna persona divisa
		            per ruolo e calcola le ore ed il costo totali per il
		            periodo;
	      \end{enumerate}

	\item \textbf{Consuntivo}:
	      \begin{enumerate}
		      \item \textbf{Riassunto delle attività svolte}: il responsabile
		            legge i \textit{commit} e le \textit{issue\g} chiuse durante
		            lo \textit{sprint} e ne riporta un riassunto nel documento;

		      \item \textbf{Consuntivo}: il responsabile riporta in forma
		            tabellare le ore effettivamente impiegate per ciascuna
		            persona divise per ruolo e calcola le ore ed il costo
		            totali per il periodo. Le ore effettive si trovano sul foglio di calcolo condiviso su Google Drive.

		      \item \textbf{Gestione dei ruoli}: il responsabile riporta in
		            un diagramma a torta la distribuzione delle ore per ruolo
		            effettivamente impiegate durante lo \textit{sprint}.
	      \end{enumerate}

	\item \textbf{Modifica di un documento}: dal momento che
	      l'aggiornamento del documento "Piano di progetto" rientra nella
	      casistica di modifica di un documento, si rimanda alla sezione
	      che illustra come redigere un documento (vedi
	      \ref{redazione-documento}).
\end{itemize}

\subsubsection{Strumenti}
\begin{itemize}
	\item \textbf{GitHub\g:} per la gestione del codice sorgente e altro
	      materiale di progetto;

	\item \textbf{Google Drive:} per la gestione dei fogli di calcolo;

	\item \textbf{Preventivi:} si tratta di un programma autoprodotto che
	      permette di calcolare in modo automatico le ore e i costi totali per
	      il periodo oltre a comporre i grafici del documento.
\end{itemize}
