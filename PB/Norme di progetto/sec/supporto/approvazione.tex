\subsection{Approvazione}
\label{subsec:approvazione}
Il processo di approvazione si concentra sulla conferma che i requisiti e il
sistema \textit{software} o prodotto finito soddisfino il loro uso inteso
specifico. L'approvazione può essere condotta in fasi precedenti dello sviluppo.\\
Lo scopo è assicurare che il sistema \textit{software} o il prodotto finito
siano adeguatamente validati rispetto al loro uso previsto, contribuendo
significativamente all'affidabilità e alla soddisfazione dell'utente finale.

\subsubsection{Identificazione} 
Valutare se il progetto richieda uno sforzo
	  di approvazione e il grado di indipendenza organizzativa di tale
	  sforzo.

\subsubsection{Pianificazione} 
Stabilire un processo di approvazione per
	  validare il sistema o il prodotto \textit{software} se il progetto lo
	  richiede. Selezionare i compiti di approvazione, inclusi i metodi, le
	  tecniche e gli strumenti associati.

\subsubsection{Approvare un documento}
\label{approvazione-documento}

\subsubsection*{Trigger}
\begin{itemize}
	\item Un documento viene completato rispetto alla fase attuale del progetto.
\end{itemize}

\subsubsection*{Scopo}
\begin{itemize}
	\item Assicurarsi che il documento soddisfi i requisiti ad esso associati;

	\item Convalidare il contenuto ed il completameto del documento.
\end{itemize}

\subsubsection*{Svolgimento}
\begin{itemize}
	\item \textbf{Approvazione:}
		\begin{itemize}
			\item \textbf{Lettura}: il responsabile legge il
					documento (vedi \cref{verifica-documento});

			\item \textbf{Convalida}: il responsabile verifica che il
					documento soddisfi i requisiti ad esso associati;

			\item \textbf{Aggiornamento della versione}: dopo che il documento
					viene corretto dall'autore, il responsabile aggiorna la
					sua versione ed il suo stato;

			\item \textbf{Versione}: sia $X.Y.Z$ la versione del documento,
					dopo l'approvazione, il valore di $X$ viene incrementato di
					$1$, mentre $Y$ e $Z$ vengono azzerati.
		\end{itemize}
\end{itemize}
	  

\subsubsection{Rapporti} 
Inoltrare i rapporti di approvazione al committente e al proponente.
