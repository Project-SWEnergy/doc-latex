\subsection{Gestione della Configurazione}

La gestione della configurazione è un processo di supporto che assicura il
controllo delle versioni e la tracciabilità dei componenti \textit{software}
durante tutto il ciclo di vita del progetto.
Questo processo si occupa di mantenere la coerenza delle prestazioni, dei dati
funzionali e delle informazioni fisiche di un sistema e dei suoi componenti.
Si focalizza sulla gestione di modifiche e configurazioni per prevenire
disordine e confusione.

\subsubsection{Scopo}
Questo processo ha lo scopo di assicurare che tutti i componenti del
\textit{software} siano identificati, versionati e tracciati nel corso del
tempo, facilitando così la gestione delle modifiche e migliorando la qualità
del prodotto \textit{software}.

\subsubsection{Attività}
\begin{enumerate}
	\item \textbf{Identificazione della configurazione:} Definire e documentare
	      le caratteristiche funzionali e fisiche dei componenti
	      \textit{software}.
	\item \textbf{Controllo della configurazione:} Gestire le modifiche
	      attraverso un processo formale di valutazione, approvazione e
	      implementazione.
	\item \textbf{Registrazione e rapporto dello stato di configurazione:}
	      Tenere traccia di tutte le modifiche apportate ai componenti
	      \textit{software} e documentare lo stato corrente di configurazione.
	\item \textbf{Verifica:} Assicurare che i componenti \textit{software} siano
	      conformi ai requisiti e che le modifiche siano implementate
	      correttamente.
\end{enumerate}

\subsubsection{Strumenti}
\begin{itemize}
	\item \textbf{Git}: Sistema di controllo versione distribuito utilizzato per
	      il tracciamento delle modifiche al codice sorgente;
	\item \textbf{GitHub\g}: Piattaforma di \textit{hosting} per progetti \textit{software}
	      che fornisce strumenti di collaborazione e controllo versione e
	      traccia delle \textit{issue\g};
	\item \textbf{GitHub Actions}: Strumento di automazione per l'esecuzione di
	      \textit{workflow} personalizzati.
\end{itemize}
