\subsection{Revisioni Congiunte con il Cliente}

Le revisioni congiunte con il cliente sono incontri strutturati tra il \textit{team} di
sviluppo e gli \textit{stakeholder} o i clienti per esaminare il progresso del
prodotto \textit{software}, discutere problemi e trovare soluzioni congiunte.

\subsubsection{Scopo}
L'obiettivo di queste revisioni è assicurare che il prodotto \textit{software}
in sviluppo rispecchi fedelmente i requisiti e le aspettative del committente e
del proponente, e che eventuali discrepanze o incomprensioni siano risolte
tempestivamente.

\subsubsection{Attività}
\begin{enumerate}
	\item \textbf{Preparazione della Revisione:} Organizzare l'incontro,
	      definire l'agenda e preparare il materiale da presentare al
	      committente o al proponente (vedi
	      \cref{organizzare-meeting-esterno}).
	\item \textbf{Conduzione della Revisione:} Presentare il lavoro svolto,
	      discutere i progressi e raccogliere \textit{feedback\g} dagli
	      \textit{stakeholder}.
	\item \textbf{Riscontro ai feedback\g:} Analizzare e discutere i
	      \textit{feedback\g} ricevuti per determinare le azioni correttive
	      necessarie.
	\item \textbf{Pianificazione delle Azioni Correttive:} Definire un piano per
	      implementare le modifiche richieste o per risolvere problemi
	      identificati durante la revisione (vedi
	      \cref{pianificazione-attivia}).
	\item \textbf{\textit{Follow-up}:} Monitorare l'attuazione delle azioni
	      correttive e organizzare revisioni successive se necessario.
\end{enumerate}

\subsubsection{Partecipanti}
Includono membri del \textit{team} di sviluppo, rappresentanti del cliente o degli
\textit{stakeholder}, e possono includere anche esperti di dominio o utenti
finali.

\subsubsection{Documentazione}
Tutti gli aspetti salienti della revisione, compresi i \textit{feedback\g}, le
decisioni prese e le azioni correttive pianificate, devono essere documentati
all'interno dei verbali esterni e resi disponibili a tutti i partecipanti per
riferimento futuro.
