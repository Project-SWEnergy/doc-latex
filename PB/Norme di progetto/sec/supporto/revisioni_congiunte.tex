\subsection{Revisioni Congiunte con il Cliente}

Le revisioni congiunte con il cliente sono incontri strutturati tra il \textit{team} di
sviluppo e gli \textit{stakeholder} o i clienti per esaminare il progresso del
prodotto \textit{software}, discutere problemi e trovare soluzioni congiunte.

\subsubsection{Scopo}
L'obiettivo di queste revisioni è assicurare che il prodotto \textit{software}
in sviluppo rispecchi fedelmente i requisiti e le aspettative del committente e
del proponente, e che eventuali discrepanze o incomprensioni siano risolte
tempestivamente.

\subsubsection{Attività}
\subsubsubsection{Conduzione della Revisione} 
Presentare il lavoro svolto,
	  discutere i progressi e raccogliere \textit{feedback\g} dagli
	  \textit{stakeholder}.

\subsubsubsection{Organizzare un \textit{meeting} esterno}
\label{organizzare-meeting-esterno}

Il responsabile organizza i \textit{meeting} esterni, ovvero i SAL\g
tenuti tra il proponente e SWEnergy.
I SAL\g sono riunioni brevi, della durata di circa 30 minuti, che hanno
luogo su \textit{Teams}. In esse
sono trattati i seguenti argomenti:
\begin{itemize}
	\item \textbf{Riassunto:} il responsabile riassume le attività svolte dal
	      gruppo durante lo \textit{sprint};

	\item \textbf{Problemi riscontrati:} il responsabile espone i problemi
	      riscontrati durante lo \textit{sprint};

	\item \textbf{\textit{To-do list}:} sono discussi i compiti da svolgere nella
	      settimana successiva tra il gruppo e il proponente;

	\item \textbf{\textit{Dubbi}:} il responsabile esponge i dubbi riguardo alle
	      attività da svolgere;

	\item \textbf{Restrospettiva:} il responsabile guida la discussione sulla
	      qualità del prodotto e soprattutto del processo.
\end{itemize}

\subsubsubsubsection{\textit{Trigger}}
\begin{itemize}
	\item La domanica precedente al SAL\g.
\end{itemize}

\subsubsubsubsection{Scopo}
\begin{itemize}
	\item Rendere la comunicazione tra i membri del gruppo più efficace ed
	      efficiente;

	\item Creare della documentazione usufruibile in caso di dubbi o
	      problematiche;

	\item Formalizzare le decisioni prese durante la riunione.
\end{itemize}

\subsubsubsubsection{Svolgimento}
\begin{itemize}

	\item \textbf{Pianificazione:} il responsabile deve decidere quando svolgere
	      un SAL\g. Di seguito i passi:
	      \begin{enumerate}
		      \item \textbf{Anticipare la data:} nel SAL\g
		            precedente il responsabile e il proponente concordano
		            la data del prossimo SAL\g;

		      \item \textbf{Pianificare l'ora:} il responsabile contatta su
		            Telegram\g il proponente, gli condivide l'ordine del
		            giorno e concorda l'ora del SAL\g. Le due attività
		            sono svolte in concomitanza, siccome si può prevedere la
		            durata del SAL\g solo dopo aver stilato l'ordine del
		            giorno.
	      \end{enumerate}

	\item \textbf{Ordine del giorno:} il responsabile stila l'ordine del
	      giorno, ovvero una lista degli argomenti da trattare durante la
	      riunione. Di seguito i passi:
	      \begin{enumerate}
		      \item \textbf{Template:} il responsabile utilizza il \textit{template}
		            dei SAL\g, situato nella \textit{repository\g}
		            \texttt{appunti-swe/SAL/template-SAL.md};

		      \item \textbf{Brainstorming:} il responsabile si informa con i
		            membri del gruppo attraverso le \textit{stand-up} in merito
		            allo \textit{status quo} del progetto;

		      \item \textbf{\textit{To-do list}:} il responsabile stila la lista
		            delle attività da svolgere nello \textit{sprint} successivo. La
		            lista viene poi discussa e approvata durante la riunione.
	      \end{enumerate}

	\item \textbf{Verbale della riunione:} il responsabile deve redigere il
	      verbale della riunione, in cui vengono riportati gli argomenti
	      trattati e le decisioni prese. Di seguito i passi per redigere il
	      verbale interno:
	      \begin{enumerate}
		      \item \textbf{Appunti:} l'ordine del giorno (il punto precedente)
		            viene utilizzato come base per stilare il verbale esterno;

		      \item \textbf{\textit{Template}:} viene copiata la cartella di \textit{template} dei
		            verbali esterni e viene rinominata seguendo il formato:
		            \texttt{YYYY-MM-DD\_E};

		      \item \textbf{Stesura:} poichè si tratta di un documento, si
		            rimanda alla sottosezione che illustra come redigere un
		            documento (vedi \cref{redazione-documento}).
	      \end{enumerate}
\end{itemize}

\subsubsubsubsection{Strumenti}
\begin{itemize}
	\item \textbf{Telegram\g:} per comunicare con il proponente e con i membri del
	      gruppo;

	\item \textbf{Microsoft Teams:} per svolgere il SAL\g;

	\item \textbf{GitHub\g:} per condividere l'ordine del giorno e il verbale
	      esterno;

	\item \textbf{\textit{Mail}:} per la condivisione di documenti e per la
	      comunicazione con il proponente.
\end{itemize}


\subsubsubsection{Riscontro ai feedback\g} 
Analizzare e discutere i
	  \textit{feedback\g} ricevuti per determinare le azioni correttive
	  necessarie.

\subsection{Pianificazione delle attività}
\label{pianificazione-attivia}

Il responsabile pianifica le attività da svolgere durante lo sprint
e le suddivider tra i membri del gruppo. Inoltre, aggiorna il piano di progetto
in base alle attività svolte e a quelle da svolgere. La pianificazione avviene
tramite l'uso dei diagrammi di Gantt disponibili su \textit{GitHub}.

\subsubsection{\textit{Trigger}}
\begin{itemize}
	\item Comincia una nuova iterazione, che sia uno sprint\g od un
	      mini-sptrint\g.
\end{itemize}

\subsubsection{Scopo}
\begin{itemize}
	\item Aggiornare il piano di progetto in base alle attività svolte e a
	      quelle da svolgere.

	\item Pianificare le attività da svolgere durante l'iterazione corrente;

	\item Guidare lo svolgimento delle attività;

	\item Produrre la documentazione che permette di tenere traccia delle
	      attività svolte e da svolgere.
\end{itemize}

\subsubsection{Svolgimento}
\begin{itemize}
	\item \textbf{Creazione delle \textit{issue}}: il responsabile crea
	      delle issue che descrivono le attività da svolgere e guidano i
	      lavoratori nella loro esecuzione. Di seguito sono riportati i passi
	      per definire le \textit{issue}:
	      \begin{enumerate}
		      \item \textbf{Identificazione}: il responsabile identifica le
		            attività da svolgere e le aggiunge su \textit{GitHub};

		      \item \textbf{Priorità}: il responsabile assegna una priorità
		            alle issue in base all'urgenza e all'importanza;

		      \item \textbf{Scadenza}: il responsabile assegna una data di
		            scadenza alle issue in base alla priorità e alla durata
		            dell'attività. L'attività viene quindi inserita nel
		            \textit{project} di \textit{Github} corrispondente alla
		            milestone di riferimento. In questo modo viene aggiornato il
		            diagramma di Gantt;

		      \item \textbf{Perfezionamento}: il responsabile guida la
		            discussione in merito alle \textit{issue} durante le
		            riunioni. In questo modo sono aggiornate priorità,
		            scadenza e descrizione;

		      \item \textbf{Assegnazione}: il responsabile assegna le issue
		            ai membri del gruppo in base alle loro competenze e
		            disponibilità.
	      \end{enumerate}
\end{itemize}


\subsubsubsection{\textit{Follow-up}} 
Monitorare l'attuazione delle azioni
	  correttive e organizzare revisioni successive se necessario.

\subsubsection{Partecipanti}
Includono membri del \textit{team} di sviluppo, rappresentanti del cliente o degli
\textit{stakeholder}, e possono includere anche esperti di dominio o utenti
finali.

\subsubsection{Documentazione}
Tutti gli aspetti salienti della revisione, compresi i \textit{feedback\g}, le
decisioni prese e le azioni correttive pianificate, devono essere documentati
all'interno dei verbali esterni e resi disponibili a tutti i partecipanti per
riferimento futuro.
