\subsection{Risoluzione dei Problemi}
La risoluzione dei problemi si occupa della gestione sistematica dei problemi
riscontrati nel \textit{software} o nei processi di sviluppo, dalla loro
identificazione alla loro risoluzione e documentazione.\\
Lo scopo è identificare e risolvere i problemi in modo efficiente per minimizzare l'impatto
sul progetto, migliorando la qualità del prodotto e del processo.
Occorre documentare ogni problema riscontrato, le azioni intraprese per risolverlo e i
risultati ottenuti, per mantenere una tracciabilità e fornire un riferimento
per problemi futuri.

\subsubsection{Identificazione del Problema} 
Riconoscere e documentare i
	  problemi o le discrepanze riscontrate nel \textit{software} o nei
	  processi.

\subsubsection{Analisi del Problema} 
Valutare il problema per comprenderne
	  le cause radice e determinare l'impatto sul progetto.

\subsubsection{Pianificazione delle Azioni Correttive} 
Sviluppare un piano
	  di azioni per risolvere il problema, includendo modifiche al
	  \textit{software} o ai processi.

\subsubsection{Implementazione delle Azioni Correttive} 
Applicare le
	  soluzioni identificate per correggere il problema.

\subsubsection{Verifica e Chiusura} 
Verificare che la soluzione abbia
	  risolto efficacemente il problema e documentare l'esito e le lezioni
	  apprese.

\subsubsection{Strumenti}
\begin{itemize}
	\item \textbf{Discussion di GitHub\g}: Strumento per la gestione ed
	      il mantenimento di discussioni su problemi e soluzioni.
\end{itemize}
