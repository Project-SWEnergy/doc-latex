\subsection{Verifica}

Il processo di verifica è essenziale per assicurare che il codice prodotto sia
conforme alle aspettative e agli \textit{standard} definiti. Questo processo coinvolge
una serie di attività dettagliate per valutare la qualità e la correttezza del
codice.

\subsubsection{Scopo}
Questo processo assicura che il codice sia non solo funzionale ma anche
conforme agli \textit{standard} qualitativi stabiliti, contribuendo significativamente
alla qualità generale del prodotto \textit{software} e dei documenti.

\subsubsection{Attività di Verifica}
A seconda che il prodotto da controllare sia un documento o del codice sorgente,
sono previste le seguenti attività di verifica:
\begin{itemize}
	\item \textbf{Verifica del Documento:} valutare la correttezza e la
	      completezza del documento rispetto agli \textit{standard} e alle linee guida
	      stabilite (vedi \cref{verifica-documento}).

	\item \textbf{Verifica del Codice:} valutare la correttezza e la qualità del
	      codice sorgente rispetto agli \textit{standard} e alle linee guida stabilite
	      (vedi \cref{verifica-codice}).
\end{itemize}
