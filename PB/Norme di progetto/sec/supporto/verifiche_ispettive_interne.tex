\subsection{Verifiche Ispettive Interne}

Le verifiche ispettive interne sono processi attraverso i quali il \textit{team} di
progetto esegue revisioni sistematiche e ispezioni dei propri processi e
prodotti \textit{software}, al fine di identificare e correggere gli errori
prima che il prodotto sia rilasciato o passi alla fase successiva.

\subsubsection{Scopo}
L'obiettivo delle verifiche ispettive interne è migliorare la qualità dei
processi e dei prodotti \textit{software}, riducendo gli errori, aumentando
l'efficienza e garantendo la conformità agli \textit{standard} di progetto.

\subsubsection{Attività}
Le attività tipiche coinvolte nelle verifiche ispettive interne includono:

\begin{enumerate}
	\item \textbf{Pianificazione delle Ispezioni:} Definire gli obiettivi, lo
	      scopo, la portata e il programma delle ispezioni.
	\item \textbf{Preparazione:} Raccogliere e rivedere i documenti, il codice e
	      altri artefatti da ispezionare.
	\item \textbf{Conduzione delle Ispezioni:} Eseguire le ispezioni secondo le
	      procedure stabilite, utilizzando \textit{checklist} o linee guida
	      specifiche per identificare gli errori e le aree di miglioramento.
	\item \textbf{Riunione di Ispezione:} Discutere i risultati delle ispezioni
	      con il \textit{team}, identificare le cause degli errori e decidere le azioni
	      correttive.
	\item \textbf{Implementazione delle Azioni Correttive:} Apportare le
	      modifiche necessarie per risolvere gli errori identificati durante le
	      ispezioni.
	\item \textbf{\textit{Follow-up}:} Verificare che tutte le azioni
	      correttive siano state implementate correttamente e che gli errori
	      siano stati risolti.
\end{enumerate}

\subsubsection{Partecipanti}
Le verifiche ispettive interne coinvolgono diversi ruoli all'interno del \textit{team}
di progetto, tra cui verificatori, analisti, progettisti e programmatori,
ciascuno con responsabilità specifiche nel processo di ispezione.

\subsubsection{Documentazione}
Tutti i risultati delle ispezioni, comprese le scoperte, le decisioni prese e
le azioni correttive pianificate, devono essere documentati e archiviati per
future referenze e valutazioni della qualità.
