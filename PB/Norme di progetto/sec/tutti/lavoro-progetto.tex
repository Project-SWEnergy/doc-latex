\subsection{Lavoro sul progetto}
\label{lavoro-sul-progetto}

\subsubsection{Descrizione}
La presente sezione delinea i procedimenti che ogni componente del \textit{team} è tenuto ad adottare al fine di eseguire il compito che gli è stato affidato dal responsabile di progetto.
Il termine "compito" si intende qui come un'incarico preciso, destinato alla realizzazione individuale da parte di un membro del gruppo, il cui completamento è essenziale per il progresso del progetto nel suo complesso.

% si potrebbe richiedere di inserire un'ora da qualche parte, tipo quanto ci si
% mette per completare l'attività

\subsubsection{\textit{Trigger}}
\begin{itemize}
	\item Il responsabile di progetto assegna un compito ad un membro del
	      gruppo.
\end{itemize}

\subsubsection{Scopo}
\begin{itemize}
	\item Svolgere il compito assegnato;
	\item Risulta conclusa un'\textit{issue\g} nella \textit{repository\g}
	      corrispondente;
	\item Il compito è stato verificato e convalidato da una persona diversa
	      da chi lo ha svolto.
\end{itemize}

\subsubsection{Svolgimento}
In questo caso viene descritto il flusso di lavoro da seguire per completare
un compito assegnato:

\begin{enumerate}
	\item \textbf{Analisi}: si crea una \textit{issue\g} nella
	      \textit{repository\g} nella quale verrà svolto il compito. La
	      \textit{issue\g} sarà assegnata a se stessi o ai membri del gruppo che
	      vi parteciperanno. La \textit{issue\g} deve contenere le seguenti
	      informazioni:
	      \begin{itemize}
		      \item Titolo: deve essere chiaro e conciso, in modo da
		            identificare il compito;

		      \item Descrizione: deve contenere una spiegazione dettagliata
		            del compito da svolgere, in più deve essere indicato il
		            tempo stimato per il completamento del compito;

		      \item \textit{Label}: deve essere assegnata una \textit{label}
		            che identifichi il tipo di compito da svolgere. In
		            particolare, se il compito è relativo alla documentazione,
		            la \textit{label} deve essere il nome del documento; se il
		            compito è relativo allo sviluppo del \textit{software},
		            la \textit{label} deve essere il nome del modulo
		            interessato (per esempio: \textit{database},
		            \textit{service}, \textit{controller}, ecc.);

		      \item \textit{Milestone}: deve essere assegnata una
		            \textit{milestone};

		      \item \textit{Project}: deve essere assegnato il progetto
		            corrispondente alla fase corrente;

		      \item \textit{Esecutori}: devono essere assegnati i membri
		            del gruppo che svolgeranno il compito.
	      \end{itemize}


	\item \textbf{Creazione appunti}: si inseriscono i file degli
	      appunti nel proprio \textit{branch} personale all'interno
	      della \textit{repository\g} \texttt{appunti-swe}.
	      Nella \textit{repository\g}, deve essere presente un \texttt{README.md}
	      contenente l'organizzazione della cartella per permettere agli altri
	      membri di orientarsi;

	\item \textbf{Svolgimento}: si svolge il compito assegnato, al meglio
	      delle proprie capacità e cercando di rispettare le scadenze;

	\item \textbf{Integrazione appunti}: si modificano i file degli
	      appunti precedentemente generati;

	\item \textbf{Verifica}: si chiede ad un membro del gruppo,
	      tendenzialmente al verificatore, di controllare la conformità del
	      lavoro svolto.
\end{enumerate}
