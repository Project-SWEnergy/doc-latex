\subsubsubsection{Redazione di un documento}
\label{redazione-documento}

Tutti redigono qualche documento.

\subsubsubsubsection{\textit{Trigger}}
\begin{itemize}
	\item Bisogna aggiungere qualche contenuto all'interno di un documento;
\end{itemize}

\subsubsubsubsection{Scopo}
\begin{itemize}
	\item Completare il contenuto di un documento;
\end{itemize}

\subsubsubsubsection{Struttura del documento}
Ogni documento è associato a una cartella omonima situata all'interno di una
\textit{directory} più ampia che riflette la fase corrente del progetto in cui
il documento è stato creato.
Questa cartella di fase è localizzabile nel \textit{repository\g}
\texttt{doc-latex} sul GitHub\g dell'organizzazione del gruppo.
Il nome della cartella corrisponde al nome del documento e deve seguire le
regole specificate di seguito:
\begin{itemize}
	\item deve avere la prima lettera maiuscola;
	\item sono previsti gli spazi tra le parole e le parole successive alla
	      prima sono in minuscolo.
\end{itemize}

Di seguito la struttura della cartella:

\vspace{0.5cm}

\dirtree{%
	.1 / (Nome Del Documento).
	.2 main.tex.
	.2 sec.
	.3 registro\_modifiche.tex.
	.3 introduzione.tex.
	.3 le\_altre\_sezioni.tex.
}

\subsubsubsubsection{\texttt{main.tex}}
Di seguito la struttura del file \texttt{main.tex}:
\begin{itemize}
	\item \textbf{\textit{Import} dei \textit{template}}: sono importati i \textit{template} per la
	      creazione del documento. I \textit{template} sono: \texttt{copertina.tex},
	      \texttt{header\_footer.tex} e \texttt{variable.tex}. In aggiunta, sono
	      importati i modelli specifici per il documento che si sta redigendo;

	\item \textbf{Inizializzazione delle variabili:} sono inizializzate le
	      variabili che verranno utilizzate nel documento;

	\item \textbf{Struttura del documento:} attraverso l'uso degli
	      \texttt{input} viene gestita la struttura del documento.
\end{itemize}

\subsubsubsubsection{Svolgimento}
\begin{itemize}
	\item \textbf{Modifica di un documento}: il lavoratore aggiorna il documento
	      in base alle modifiche richieste dal verificatore e in base alle
	      informazioni necessarie per la redazione del documento. Di seguito sono
	      elencati i passi per completare l'attività:
	      \begin{enumerate}
		      \item \textbf{\textit{Pull}}: si effettua il \textit{pull}
		            della \textit{repository\g} \texttt{doc-latex} per avere
		            l'ultima versione della \textit{repository\g};

		      \item \textbf{\textit{Checkout:}} si effettua il
		            \textit{checkout} del \textit{branch} verso il
		            \textit{branch} chiamato come il documento che si sta
		            redigendo;

		      \item \textbf{Struttura:} si modifica il
		            \texttt{main.tex} in base alle modifiche necessarie;

		      \item \textbf{Gestione dei \textit{file}:} si crea,
		            elimina o rinomina i \textit{file} nella cartella
		            \texttt{sec} in modo tale che siano rispecchiate le
		            modifiche apportate al \texttt{main.tex};

		      \item \textbf{Contenuto:} si modifica i \textit{file}
		            nella cartella \texttt{sec} in base alle modifiche
		            necessarie;

		      \item \textbf{\textit{Push}:} si effettua un
		            \textit{commit} e il \textit{push};

		      \item \textbf{\textit{Pull request}:} si può creare
		            una \textit{pull request} verso il \texttt{main},
		            per chiedere al verificatore, la verifica del documento;

		      \item \textbf{Verifica:} si informa il verificatore
		            che il documento è pronto per la verifica;

		      \item \textbf{Correzione:} si corregge il documento
		            in base alle segnalazioni del verificatore;

		      \item \textbf{Chiusura:} si effettua il \textit{push} del
		            branch inserendo nel messaggio di \textit{commit} la parola
		            \texttt{close} seguita dal numero della \textit{issue}\g che si sta
		            risolvendo;

		      \item \textbf{Secondo \textit{merge}:} si può concludere la
		            \textit{pull request} con il \textit{main}.
	      \end{enumerate}
\end{itemize}

\subsubsubsubsection{Strumenti}
Gli strumenti utilizzati per la creazione dei documenti sono:
\begin{itemize}
	\item \textbf{LaTeX}: linguaggio di \textit{markup} per la creazione di documenti \\
	      \href{https://www.latex-project.org/}{(www.latex-project.org)} (ultimo accesso 15/11/2023);
	\item \textbf{VisualStudio Code}: GUI con integrazioni per la creazione di documenti scritti in LaTeX e per la gestione delle \textit{repository}\g git\g \\
	      \href{https://code.visualstudio.com/}{(code.visualstudio.com)} (ultimo accesso 5/12/2023)
	      \begin{itemize}
		      \item \textbf{LaTeX Workshop}: estensione utilizzata in VisualStudio Code per la compilazione e la scrittura dei documenti.
	      \end{itemize}
\end{itemize}
