\subsection{Verificare la correttezza di un documento}
\label{verifica-documento}

Il verificatore deve verificare che i documenti prodotti mentre svolge il suo
ruolo siano conformi alle norme stabilite in questa sotto-sezione.

\subsubsection{\textit{Trigger}}
\begin{itemize}
	\item Viene prodotto un incremento su di un documento;

	\item Un componente di SWEnergy segnala la necessità di una verifica.
\end{itemize}

\subsubsection{Scopo}
\begin{itemize}
	\item Evidenziare gli errori in un documento e segnalarli all'autore del
	      documento;

	\item Assicurarsi che il documento soddisfi le norme qui sotto descritte;

	\item Convalidare l'incremento di un documento per garantirne l'integrità
	      agli altri componenti di SWEnergy.
\end{itemize}

\subsubsection{Norme}
\begin{itemize}
	\item \textbf{Correttezza grammaticale:} il testo deve essere privo di
	      errori grammaticali;

	\item \textbf{Correttezza lessicale:} il testo deve essere privo di errori
	      lessicali;

	\item \textbf{Correttezza ortografica:} il testo deve essere privo di errori
	      ortografici;

	\item \textbf{Correttezza sintattica:} il testo deve essere sintatticamente
	      corretto;

	\item \textbf{Correttezza di contenuto:} il testo deve essere privo di
	      errori di contenuto;

	\item \textbf{Correttezza della struttura:} in ogni documento che contiene
	      il registro delle modifiche, deve essere anche presente
	      un'introduzione che spiega la struttura del documento medesimo,
	      coerente con la struttura del documento;

	\item \textbf{Completezza}: il documento deve essere completo di tutte le
	      sezioni opportune;

	\item \textbf{Coerenza}: il contenuto del documento deve essere
	      coerente con il suo scopo, con le norme qui descritte e con il
	      contenuto di eventuali documenti correlati;

	\item \textbf{Chiarezza espositiva}: il documento deve essere
	      scritto in modo chiaro e comprensibile;
\end{itemize}

\subsubsection{Svolgimento}

Per verificare la correttezza di un documento, il verificatore deve
completare le seguenti attività:
\begin{itemize}
	\item \textbf{Correzione dei refusi:} il verificatore deve correggere i
	      refusi presenti nel documento. Sono considerati refusi gli errori
	      della tipologia grammaticale, lessicale, ortografica e sintattica;

	\item \textbf{Verifica del contenuto:} il verificatore deve verificare
	      che il contenuto del documento sia corretto e coerente con il suo
	      scopo. Di seguito sono riportati i passi da seguire:
	      \begin{enumerate}
		      \item \textbf{Lettura del documento:} il verificatore deve
		            leggere il documento per comprendere il contenuto del
		            documento;

		      \item \textbf{Appunti degli errori}: durante la lettura il
		            verificatore prende nota di eventuali errori;

		      \item \textbf{Ricerca delle soluzioni}: il verificatore deve
		            trovare una soluzione per ogni errore trovato;

		      \item \textbf{Spiegazione degli errori}: il verificatore deve
		            segnalare all'autore del documento gli errori trovati e le
		            relative soluzioni;

		      \item \textbf{Aggiornamento della versione}: dopo che il documento
		            viene corretto dall'autore, il verificatore deve aggiornare
		            la versione del documento;

		      \item \textbf{Versione}: sia $X.Y.Z$ la versione del documento,
		            dopo la verifica, il valore di $Z$ viene incrementato di
		            $1$, se le modifiche apportate al documento si limitano al
		            contenuto e non modificano la struttura del documento,
		            ovvero l'indice non viene modificato; altrimenti il valore
		            di $Y$ viene incrementato di $1$ e $Z$ viene azzerato.
	      \end{enumerate}
\end{itemize}
