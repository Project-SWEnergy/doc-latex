\section{Consuntivo a finire}

Nella seguente sezione sono riportati i dati in forma tabellare
delle ore totali preventivate per la fase RTB rispetto alle ore effettive
utilizzate dal gruppo SWEnergy.

\subsection{Analisi RTB}

\subsubsection{Riepilogo preventivo per la fase RTB}
\begin{table}[H]
	\centering
	\begin{tabular}{l|c|c|c|c|c|c|c}
		\textbf{Nome} & \textbf{Re} & \textbf{Am} & \textbf{An} & \textbf{Pt} & \textbf{Pr} & \textbf{Ve} & \textbf{Totale ore} \\
		\hline
		Alessandro    & 5           & -           & 5           & 15          & 10          & -           & 35                  \\
		Carlo         & 5           & 5           & 10          & 10          & 5           & -           & 35                  \\
		Davide        & 5           & -           & 20          & -           & 5           & 5           & 35                  \\
		Giacomo       & 5           & -           & 15          & -           & -           & 5           & 25                  \\
		Matteo        & -           & 5           & 10          & -           & 10          & 10          & 35                  \\
		Niccolò       & 5           & -           & 10          & 10          & -           & 10          & 35                  \\
		\hline
		%\textbf{Ore totali}   & 25          & 10           & 70          & 35           & 30           & 30           & 60              \\
		%\textbf{Costo totale} & 300         & 100         & 1000        & 125         & -           & -           & 1525
	\end{tabular}
	\caption{Re: Responsabile, Am: Amministratore, An: Analista, Pt: Progettista,
		Pr: Programmatore, Ve: Verificatore, Totale: Totale per persona; valori espressi in ore; Costo totale espresso in euro.}
\end{table}


\subsubsection{Riepilogo consuntivo per la fase RTB}
\begin{table}[H]
	\centering
	\begin{tabular}{l|c|c|c|c|c|c|c}
		\textbf{Nome} & \textbf{Re} & \textbf{Am} & \textbf{An} & \textbf{Pt} & \textbf{Pr} & \textbf{Ve} & \textbf{Totale ore} \\
		\hline
		Alessandro    & 5           & 5           & 5           & 10          & 10          & -           & 35                  \\
		Carlo         & 10          & 10          & 10          & -           & 5           & -           & 35                  \\
		Davide        & 5           & -           & 20          & -           & 5           & 5           & 35                  \\
		Giacomo       & 5           & 5           & 5           & 10          & -           & 10          & 35                  \\
		Matteo        & -           & 5           & 10          & 5           & 10          & 5           & 35                  \\
		Niccolò       & 5           & -           & 10          & 10          & -           & 10          & 35                  \\
		\hline
		%\textbf{Ore totali}   & 30          & 25           & 60          & 35          & 30           & 30           & 60              \\
		%\textbf{Costo totale} & 300         & 100         & 750         & 375         & -           & -           & 1525
	\end{tabular}
	\caption{Re: Responsabile, Am: Amministratore, An: Analista, Pt: Progettista,
		Pr: Programmatore, Ve: Verificatore, Totale: Totale per persona; valori espressi in ore; Costo totale espresso in euro.}
\end{table}


\subsubsection{Riepilogo finale RTB}
Di seguito vengono indicate le spese effettive nelle fase RTB del progetto.
\begin{table}[H]
	\centering
	\begin{tabular}{l|c|c|c|c|c|c}
		\textbf{Ruolo}    & \textbf{Ore P} & \textbf{Ore E} & \textbf{Diff. ore} & \textbf{Costo P} & \textbf{Costo E} & \textbf{Diff. costo} \\
		\hline
		Responsabile      & 25             & 30             & +5                 & 750              & 900              & +150                 \\
		Amministratore    & 10             & 25             & +15                & 200              & 500              & +300                 \\
		Analista          & 70             & 60             & -10                & 1750             & 1500             & -250                 \\
		Progettista       & 35             & 35             & 0                  & 875              & 875              & 0                    \\
		Programmatore     & 30             & 30             & 0                  & 450              & 450              & 0                    \\
		Verificatore      & 30             & 30             & 0                  & 450              & 450              & 0                    \\
		\hline
		Totale Consuntivo & 200            & 210            & +10                & 4475             & 4675             & +200                 \\
		\hline
		%\textbf{Ore totali}   & 10          & 5           & 30          & 15          & -           & -           & 60              \\
		%\textbf{Costo totale} & 300         & 100         & 750         & 375         & -           & -           & 1525
	\end{tabular}
	\caption{Consuntivo periodo RTB, Ore P: Totale Ore preventivate per un singolo ruolo, Ore e: Totale Ore effettive per un singolo ruolo,
		Diff. ore: Differenza tra ore preventivate e ore effettive, Costo P: totale costo preventivato per un singolo ruolo,
		Costo E: totale costo effettivo per un singolo ruolo,  Diff. costo: Differenza tra costo preventivato e costo effettivo;
		Costo orario: espresso in (\euro/h), Costo P, Costo E, Diff. costo: espressi in (\euro)}
\end{table}


\subsubsection{Riepilogo ripartizione percentuale oraria}
\begin{figure}[h]
	\centering
	\begin{tikzpicture}
		\pie[text=legend]{
			14/Responsabile,
			12/Amministratore,
			29/Analista,
			17/Progettista,
			14/Programmatore,
			14/Verificatore
		}
	\end{tikzpicture}
	\caption{Riepilogo ripartizione percentuale oraria}
\end{figure}

L'analisi dei dati relativi ai ruoli svolti nei quattro \textit{sprint} mostra una distribuzione equilibrata delle responsabilità all'interno del \textit{team}, riflettendo una gestione oculata delle risorse.
\begin{itemize}
	\item \textbf{Responsabile}: La percentuale assegnata al Responsabile mostra un coinvolgimento costante nella gestione e nella direzione del progetto. Il fatto che sia una percentuale significativa
	      suggerisce un'impegno considerevole nella supervisione e nel coordinamento delle attività.
	\item \textbf{Amministratore}: La percentuale di tempo dedicato all'Amministratore indica una gestione efficiente delle attività amministrative. Anche se è inferiore rispetto ad alcuni altri ruoli,
	      ha permesso di avere una buona organizzazione nella gestione della documentazione e delle risorse.
	\item \textbf{Analista}: L'alta percentuale assegnata all'Analista è dovuta ad una particolare attenzione dedicata all'analisi dei requisiti e alla definizione chiara degli obiettivi del progetto.
	      Questo impegno iniziale è cruciale per il successo complessivo del progetto.
	\item \textbf{Progettista} : La percentuale dedicata al Progettista mette in luce un focus significativo sulla fase di progettazione del progetto. Questo è coerente con una pratica sviluppo del \textit{software}
	      che presta attenzione alla pianificazione e alla progettazione prima della codifica effettiva.
	\item \textbf{Programmatore}: La percentuale assegnata al Programmatore indica una distribuzione ragionevole del tempo per l'implementazione pratica del progetto. Anche se non è la percentuale più alta,
	      indica un impegno costante nella fase di sviluppo effettivo.
	\item \textbf{Verificatore}: La percentuale dedicata al Verificatore è coerente con gli \textit{standard} di qualità del processo. Un impegno costante nella verifica contribuisce alla riduzione degli errori e
	      alla produzione di un prodotto di alta qualità.
\end{itemize}

Complessivamente, la distribuzione dei ruoli riflette una strategia ben ponderata, con un'attenzione particolare alle fasi iniziali di analisi e progettazione, seguite da uno
sforzo equilibrato nell'implementazione e nella verifica.



\subsubsection{Suddivisione ruoli per persona}
\begin{flushleft}
	\begin{figure}[h]
		\centering
		\begin{tikzpicture}
			\begin{axis}[
					ybar stacked,
					bar width=0.6cm,
					xlabel={}, % Rimuovo l'etichetta dell'asse x
					ylabel={Ore totali},
					symbolic x coords={Alessandro, Carlo, Davide, Giacomo, Matteo, Niccolò},
					xtick=data,
					x tick label style={rotate=45,anchor=east}, % Nomi obliqui di 45 gradi
					ytick={0,5,...,50}, % Suddivisione asse y ogni 5 da 0 a 50
					nodes near coords={},
					nodes near coords align={vertical},
					legend style={at={(0.5,-0.3)},anchor=north,legend columns=-1,draw=none}, % Maggiore distanza verticale
					every node near coord/.append style={font=\tiny}, % Dimensione del font delle etichette (vuoto per rimuoverle)
					every axis plot/.append style={draw=none}, % Rimuovo i bordi delle barre
					ymax=50, % Estendo l'asse y fino al valore 50
				]

				% Dati per l'asse delle x
				\addplot[fill=colore1] coordinates {(Alessandro, 5) (Carlo, 10) (Davide, 5) (Giacomo, 5) (Matteo, 0) (Niccolò, 5)};
				\addplot[fill=colore2] coordinates {(Alessandro, 5) (Carlo, 10) (Davide, 0) (Giacomo, 5) (Matteo, 5) (Niccolò, 0)};
				\addplot[fill=colore3] coordinates {(Alessandro, 5) (Carlo, 10) (Davide, 20) (Giacomo, 5) (Matteo, 10) (Niccolò, 10)};
				\addplot[fill=colore4] coordinates {(Alessandro, 10) (Carlo, 0) (Davide, 0) (Giacomo, 10) (Matteo, 5) (Niccolò, 10)};
				\addplot[fill=colore5] coordinates {(Alessandro, 10) (Carlo, 5) (Davide, 5) (Giacomo, 0) (Matteo, 10) (Niccolò, 0)};
				\addplot[fill=colore6] coordinates {(Alessandro, 0) (Carlo, 0) (Davide, 5) (Giacomo, 10) (Matteo, 5) (Niccolò, 10)};

				\legend{Responsabile, Amministratore, Analista, Progettista, Programmatore, Verificatore}

			\end{axis}
		\end{tikzpicture}
		\caption*{Rappresentazione ore assegnate ad una persona che ha eseguito un determinato ruolo}
	\end{figure}
\end{flushleft}



\subsection{Considerazioni}
Di seguito viene riportata la tabella delle assegnazione delle ore che il gruppo SWEnergy si impegna a rispettare.

\begin{table}[H]
	\centering
	\begin{tabular}{l|r|r|r|r|r|r|r}
		\textbf{Membro}        & \textbf{Re} & \textbf{Am} & \textbf{An}  & \textbf{Pt}
		                       & \textbf{Pr} & \textbf{Ve} & \textbf{Tot}                                 \\
		\hline
		Alessandro Tigani Sava & 16          & 7           & 12           & 15          & 27  & 17  & 94  \\
		Carlo Rosso            & 16          & 7           & 13           & 15          & 26  & 17  & 94  \\
		Davide Maffei          & 16          & 7           & 13           & 14          & 27  & 17  & 94  \\
		Giacomo Gualato        & 16          & 8           & 12           & 15          & 26  & 17  & 94  \\
		Matteo Bando           & 15          & 8           & 13           & 15          & 27  & 16  & 94  \\
		Niccolò Carlesso       & 16          & 8           & 12           & 15          & 27  & 16  & 94  \\
		\hline
		\textbf{Totale}        & 95          & 45          & 75           & 89          & 160 & 100 & 564 \\
	\end{tabular}

	\caption{Assegnamento delle ore per persona; valori espressi in ore.
		Re: Responsabile, Am: Amministratore, An: Analista, Pt:
		Progettista, Pr: Programmatore, Ve: Verificatore, Tot: Totale ore per
		membro; valori espressi in ore.}
\end{table}

Come evidenziato, il team SWEnergy ha scelto di allocare circa 94 ore per ciascun membro. Come si può osservare nella sezione precedente, dopo questi quattro \textit{sprint}, ciascun
individuo ha utilizzato 35 ore, lasciando una rimanente disponibilità di 59 ore per la fase di PB e CA.


\begin{table}[H]
	\centering
	\begin{tabular}{l|c}
		\textbf{Ruolo} & \textbf{Ore utilizzate} \\
		\hline
		Responsabile   & 30                      \\
		Amministratore & 25                      \\
		Analista       & 60                      \\
		Progettista    & 35                      \\
		Programmatore  & 30                      \\
		Verificatore   & 30                      \\
		\hline
		Totale         & 210                     \\
	\end{tabular}
	\caption{Ore utilizzate per ruolo; valori espressi in ore.}
\end{table}

Come si evidenzia, il \textit{team} SWEnergy ha impiegato 210 ore, disponendo quindi di un residuo di 354 ore per la fase PB e CA. Inoltre, rispetto alla pianificazione iniziale,
si è sottostimata leggermente il numero di ore per il ruolo del Responsabile, attribuendo meno rilevanza all'Amministratore. Tuttavia, la suddivisione oraria prevista per gli altri ruoli,
formulata all'inizio, è stata attentamente ponderata e sarà mantenuta.

Di conseguenza, si prevede l'utilizzo di circa 13 ore per ciascun membro nel ruolo di Responsabile e di circa 10 ore ciascuno per il ruolo di Amministratore. Va notato che,
sebbene avessimo leggermente sovrastimato il ruolo di analista, è stata lasciata un'ora residua per eventuali modifiche o integrazioni ai requisiti. Tuttavia, si sta considerando
la possibilità di ridurre tali ore residue per riallocarle verso ruoli che potrebbero necessitare di maggiori risorse.

Di seguito viene riportata la tabella con la nuova ripartizione oraria per la prossima fase PB:

\begin{table}[H]
	\renewcommand{\arraystretch}{1.5}
	\centering
	\begin{tabular}{l|r|r|r|r|r|r|r}
		\textbf{Membro}        & \textbf{Re} & \textbf{Am} & \textbf{An}  & \textbf{Pt}
		                       & \textbf{Pr} & \textbf{Ve} & \textbf{Tot}                                 \\
		\hline
		Alessandro Tigani Sava & 8           & 5           & 4            & 8           & 17  & 17  & 59  \\
		Carlo Rosso            & 5           & 0           & 3            & 15          & 21  & 15  & 59  \\
		Davide Maffei          & 8           & 11          & 0            & 12          & 18  & 10  & 59  \\
		Giacomo Gualato        & 8           & 7           & 2            & 9           & 26  & 7   & 59  \\
		Matteo Bando           & 13          & 5           & 3            & 10          & 17  & 11  & 59  \\
		Niccolò Carlesso       & 8           & 10          & 2            & 5           & 27  & 7   & 59  \\
		\hline
		\textbf{Totale}        & 95          & 45          & 75           & 89          & 160 & 100 & 564 \\
	\end{tabular}

	\caption{Preventivo e assegnazione ore per la fase PB; valori espressi in ore.
		Re: Responsabile, Am: Amministratore, An: Analista, Pt:
		Progettista, Pr: Programmatore, Ve: Verificatore, Tot: Totale ore per
		membro; valori espressi in ore.}
\end{table}



\begin{table}[H]
	\centering
	\begin{tabular}{l|c|c}
		\textbf{Ruolo} & \textbf{Ore utilizzate} & \textbf{Costo} \\
		\hline
		Responsabile   & 50                      & 1500           \\
		Amministratore & 38                      & 760            \\
		Analista       & 14                      & 350            \\
		Progettista    & 59                      & 1475           \\
		Programmatore  & 126                     & 1890           \\
		Verificatore   & 66                      & 990            \\
		\hline
		Costo totale   &                         & 6965           \\
	\end{tabular}
	\caption{Suddivisione oraria per ruolo per la fase PB}
\end{table}

Pertanto, la somma del costo totale della fase RTB (4675 euro) al costo totale preventivato della fase PB (6965 euro) risulta in un totale di 11.160 euro,
in perfetta conformità con il \textit{budget} inizialmente preventivato di 11.750 euro.


Il conclusione il gruppo SWEnergy conferma nuovamente il \textit{budget} iniziale stimato e anche la data di consegna prevista al 10/05/2024



\subsection{Analisi PB}
\subsubsection{Riepilogo preventivo per la fase PB}
\begin{table}[H]
	\centering
	\begin{tabular}{l|c|c|c|c|c|c|c}
		\textbf{Nome}         & \textbf{Re} & \textbf{Am} & \textbf{An} & \textbf{Pt} & \textbf{Pr} & \textbf{Ve} & \textbf{Totale ore} \\
		\hline
		Alessandro            & 5           & -           & -           & 15          & 25          & 10          & 55                  \\
		Carlo                 & 5           & -           & -           & 15          & 20          & 15          & 55                  \\
		Davide                & 5           & 5           & -           & 15          & 20          & 10          & 55                  \\
		Giacomo               & -           & -           & -           & 10          & 40          & 5           & 55                  \\
		Matteo                & 5           & 10          & -           & 5           & 30          & 5           & 55                  \\
		Niccolò               & 5           & 5           & -           & 10          & 10          & 10          & 40                  \\
		\hline
		\textbf{Ore totali}   & 25          & 20          & 0           & 70          & 145         & 55          & 315                 \\
		\textbf{Costo totale} & 750         & 400         & 0           & 1750        & 2175        & 825         & 5900
	\end{tabular}
	\caption{Re: Responsabile, Am: Amministratore, An: Analista, Pt: Progettista, Pr: Programmatore, Ve: Verificatore, Totale: Totale per persona; valori espressi in ore; Costo totale espresso in euro.}
\end{table}


\subsubsection{Riepilogo consuntivo per la fase PB}
\begin{table}[H]
	\centering
	\begin{tabular}{l|c|c|c|c|c|c|c}
		\textbf{Nome}         & \textbf{Re} & \textbf{Am} & \textbf{An} & \textbf{Pt} & \textbf{Pr} & \textbf{Ve} & \textbf{Totale ore} \\
		\hline
		Alessandro            & 5           & -           & -           & 15          & 30          & 10          & 60                  \\
		Carlo                 & 5           & -           & -           & 15          & 30          & 10          & 60                  \\
		Davide                & 5           & 5           & -           & 15          & 30          & 5           & 60                  \\
		Giacomo               & -           & -           & -           & 10          & 35          & 15          & 60                  \\
		Matteo                & 5           & 10          & -           & 5           & 35          & 5           & 60                  \\
		Niccolò               & 5           & 5           & -           & 10          & 5           & 20          & 45                  \\
		\hline
		\textbf{Ore totali}   & 25          & 20          & 0           & 70          & 165         & 65          & 345                 \\
		\textbf{Costo totale} & 750         & 400         & 0           & 1750        & 2475        & 975         & 6350
	\end{tabular}
	\caption{Re: Responsabile, Am: Amministratore, An: Analista, Pt: Progettista, Pr: Programmatore, Ve: Verificatore, Totale: Totale per persona; valori espressi in ore; Costo totale espresso in euro.}
\end{table}


\subsubsection{Riepilogo finale RTB}
Di seguito vengono indicate le spese effettive nelle fase RTB del progetto.
\begin{table}[H]
	\centering
	\begin{tabular}{l|c|c|c|c|c|c}
		\textbf{Ruolo}    & \textbf{Ore P} & \textbf{Ore E} & \textbf{Diff. ore} & \textbf{Costo P} & \textbf{Costo E} & \textbf{Diff. costo} \\
		\hline
		Responsabile      & 25             & 25             & 0                  & 750              & 750              & 0                    \\
		Amministratore    & 20             & 20             & 0                  & 400              & 400              & 0                    \\
		Analista          & 0              & 0              & 0                  & 0                & 0                & 0                    \\
		Progettista       & 70             & 70             & 0                  & 1750             & 1750             & 0                    \\
		Programmatore     & 145            & 165            & +20                & 2175             & 2475             & +300                 \\
		Verificatore      & 55             & 65             & +10                & 825              & 975              & +150                 \\
		\hline
		Totale Consuntivo & 315            & 345            & +30                & 5900             & 6350             & +450                 \\
		\hline
	\end{tabular}
	\caption{Consuntivo periodo RTB, Ore P: Totale Ore preventivate per un singolo ruolo, Ore e: Totale Ore effettive per un singolo ruolo,
		Diff. ore: Differenza tra ore preventivate e ore effettive, Costo P: totale costo preventivato per un singolo ruolo,
		Costo E: totale costo effettivo per un singolo ruolo,  Diff. costo: Differenza tra costo preventivato e costo effettivo,
		Costo P, Costo E, Diff. costo: espressi in (\euro)}
\end{table}


\subsubsection{Riepilogo ripartizione percentuale oraria}
\begin{figure}[h]
	\centering
	\begin{tikzpicture}
		\pie[text=legend]{
			7/Responsabile,
			6/Amministratore,
			0/Analista,
			20/Progettista,
			48/Programmatore,
			19/Verificatore
		}
	\end{tikzpicture}
	\caption{Riepilogo ripartizione percentuale oraria}
\end{figure}

L'analisi dei dati relativi ai ruoli svolti nei quattro \textit{sprint} mostra una distribuzione equilibrata delle responsabilità all'interno del \textit{team}, riflettendo una gestione oculata delle risorse.
\begin{itemize}
	\item \textbf{Responsabile}: La percentuale assegnata al Responsabile mostra un coinvolgimento costante nella gestione e nella direzione del progetto. Rispetto alla fase RTB ha richiesto un numero lievemente minore di ore siccome il gruppo aveva già maturato esperienza nel ruolo.
	\item \textbf{Amministratore}: La percentuale di tempo dedicato all'Amministratore indica una gestione efficiente delle attività amministrative. Anche se è inferiore rispetto ad alcuni altri ruoli, ha permesso di avere una buona organizzazione nella gestione della documentazione e delle risorse.
	\item \textbf{Analista}: Il ruolo di analista non ha richiesto l'assegnazione di ore.
	\item \textbf{Progettista} : La percentuale dedicata al Progettista mette in luce un \textit{focus} significativo sulla fase di progettazione. Questo è coerente con una pratica sviluppo del \textit{software} che presta attenzione alla pianificazione e alla progettazione prima della codifica effettiva. La quantità di ore assegnate a tale incarico è rimasta coerente con quella definita in fase RTB.
	\item \textbf{Programmatore}: La maggiore quantità di ore disponibili è stata assegnata al ruolo di Programmatore, nonostante la riduzione dei requisiti obbligatori. Il gruppo aveva inizialmente sottostimato il tempo necessario per la corretta implementazione di tutte le funzionalità richieste.
	\item \textbf{Verificatore}: La percentuale dedicata al Verificatore è coerente con gli \textit{standard} di qualità del processo. Così come sono state aumentate le ore di codifica, anche la distribuzione delle ore assegnate alle attività di verifica hanno subito un incremento rispetto alla fase precedente.
\end{itemize}

\newpage

\subsubsection{Suddivisione ruoli per persona}
\begin{flushleft}
	\begin{figure}[h]
		\centering
		\begin{tikzpicture}
			\begin{axis}[
					ybar stacked,
					bar width=0.6cm,
					xlabel={}, % Rimuovo l'etichetta dell'asse x
					ylabel={Ore totali},
					symbolic x coords={Alessandro, Carlo, Davide, Giacomo, Matteo, Niccolò},
					xtick=data,
					x tick label style={rotate=45,anchor=east}, % Nomi obliqui di 45 gradi
					ytick={0,10,...,100}, % Suddivisione asse y ogni 5 da 0 a 50
					nodes near coords={},
					nodes near coords align={vertical},
					legend style={at={(0.5,-0.3)},anchor=north,legend columns=-1,draw=none}, % Maggiore distanza verticale
					every node near coord/.append style={font=\tiny}, % Dimensione del font delle etichette (vuoto per rimuoverle)
					every axis plot/.append style={draw=none}, % Rimuovo i bordi delle barre
					ymax=80, % Estendo l'asse y fino al valore 50
				]

				% Dati per l'asse delle x
				\addplot[fill=colore1] coordinates {(Alessandro, 5)  (Carlo, 5)  (Davide, 5)  (Giacomo, 0)  (Matteo, 5) (Niccolò, 5)};
				\addplot[fill=colore2] coordinates {(Alessandro, 0)  (Carlo, 0)  (Davide, 5)  (Giacomo, 0)  (Matteo, 10) (Niccolò, 5)};
				\addplot[fill=colore3] coordinates {(Alessandro, 0)  (Carlo, 0)  (Davide, 0)  (Giacomo, 0)  (Matteo, 0) (Niccolò, 0)};
				\addplot[fill=colore4] coordinates {(Alessandro, 15) (Carlo, 15) (Davide, 15) (Giacomo, 10) (Matteo, 5) (Niccolò, 10)};
				\addplot[fill=colore5] coordinates {(Alessandro, 30) (Carlo, 30) (Davide, 30) (Giacomo, 35) (Matteo, 35) (Niccolò, 5)};
				\addplot[fill=colore6] coordinates {(Alessandro, 10) (Carlo, 10) (Davide, 5)  (Giacomo, 15) (Matteo, 5) (Niccolò, 20)};

				\legend{Responsabile, Amministratore, Analista, Progettista, Programmatore, Verificatore}

			\end{axis}
		\end{tikzpicture}
		\caption*{Rappresentazione ore assegnate ad una persona che ha eseguito un determinato ruolo}
	\end{figure}
\end{flushleft}




\subsection{Considerazioni finali}
Come mostrato nella Tabella \ref{t:orario}, quasi tutti i membri del gruppo hanno mantenuto un impegno orario in linea con le aspettative, passando dalle 94 ore totali previste ad inizio progetto a 95 ore complessive.
In generale, il totale delle ore di lavoro previste è passato dalle 564 previste alle 555 effettive.
Come emerge dalla differenza con il preventivo effettuato al termine della fase RTB, il gruppo aveva sovrastimato le ore da dedicare alle attività di Responsabile, mentre aveva sottostimato quelle da assegnare alla codifica dell'applicativo.

Di seguito viene riportata la tabella che illustra gli impegni orari assunti da ogni membro del gruppo durante l'intero progetto.
\begin{table}[H]
	\centering
	\begin{tabular}{l|r|r|r|r|r|r|r}
		\textbf{Membro}         & \textbf{Re} & \textbf{Am} & \textbf{An} & \textbf{Pt} & \textbf{Pr} & \textbf{Ve} & \textbf{Tot} \\
		\hline
		Alessandro Tigani Sava  & 10          & 5           & 5           & 25          & 40          & 10          & 95           \\
		Carlo Rosso             & 15          & 10          & 10          & 15          & 35          & 10          & 95           \\
		Davide Maffei           & 10          & 5           & 20          & 15          & 35          & 10          & 95           \\
		Giacomo Gualato         & 5           & 5           & 5           & 20          & 35          & 25          & 95           \\
		Matteo Bando            & 5           & 15          & 10          & 10          & 45          & 10          & 95           \\
		Niccolò Carlesso        & 10          & 5           & 10          & 20          & 5           & 30          & 80           \\
		\hline
		\textbf{Totale}         & 55          & 45          & 60          & 105         & 195         & 95          & 555          \\
		\textbf{Differenza RTB} & -40         & 0           & -15         & +16         & +35         & -5          & -9           \\
		\hline
	\end{tabular}
	\caption{Assegnamento delle ore per persona; valori espressi in ore.
		Re: Responsabile, Am: Amministratore, An: Analista, Pt:
		Progettista, Pr: Programmatore, Ve: Verificatore, Tot: Totale ore per
		membro; valori espressi in ore.}
	\label{t:orario}
\end{table}

Come mostrato in Tabella \ref{t:differenza}, gli errori effettuati durante le attività di pianificazione si sono compensati, hanno portato ad una variazione trascurabile (inferiore al 10\%) del costo preventivato per la fase PB.
Inoltre tale variazione ha portato ad un risparmio, essendo le ore del ruolo di Responsabile più costose di quelle del ruolo Programmatore.
Notiamo come, in generale, il gruppo ha sovrastimato le ore dedicate alle attività di analisi e gestione del progetto, mentre ha sottostimato quelle inerenti alla progettazione e codifica. \\
Questa variazione porta il costo complessivo delle attività alla cifra di 11.025 \euro, inferiore alla cifra preventivata in fase di candidatura di 11.750 \euro, ma con una data di consegna variata al DATA\_DA\_DEFINIRE.

\begin{table}[H]
	\centering
	\begin{tabular}{l|c|c|c|c|c|c}
		\textbf{Ruolo} & \textbf{Ore P} & \textbf{Ore E} & \textbf{Diff. ore} & \textbf{Costo P} & \textbf{Costo E} & \textbf{Diff. costo} \\
		\hline
		Responsabile   & 50             & 25             & -25                & 1500             & 750              & -750                 \\
		Amministratore & 38             & 20             & -18                & 760              & 400              & -360                 \\
		Analista       & 14             & 0              & -14                & 350              & 0                & -350                 \\
		Progettista    & 59             & 70             & +11                & 1475             & 1750             & +275                 \\
		Programmatore  & 126            & 165            & +39                & 1890             & 2475             & +585                 \\
		Verificatore   & 66             & 65             & -1                 & 990              & 975              & -15                  \\
		\hline
		Totale         & 353            & 345            & -8                 & 6965             & 6350             & -615                 \\
		\hline
	\end{tabular}
	\caption{Consuntivo periodo PB, Ore P: Totale Ore preventivate per un singolo ruolo, Ore e: Totale Ore effettive per un singolo ruolo,
		Diff. ore: Differenza tra ore preventivate e ore effettive, Costo P: totale costo preventivato per un singolo ruolo,
		Costo E: totale costo effettivo per un singolo ruolo,  Diff. costo: Differenza tra costo preventivato e costo effettivo,
		Costo P, Costo E, Diff. costo: espressi in (\euro)}
	\label{t:differenza}
\end{table}
