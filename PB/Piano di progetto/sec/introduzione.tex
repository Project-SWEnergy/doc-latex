\section{Introduzione}

Il presente documento, intitolato "Piano di Progetto", descrive e spiega le
decisioni organizzative adottate dal gruppo SWEnergy per lo sviluppo del
progetto "\textit{Easy Meal}", proposto dall'azienda
\href{https://imolainformatica.it/}{Imola Informatica} (ultimo accesso 25/03/2024).

\subsection{Scopo del documento}

Il "Piano di Progetto" è suddiviso nelle seguenti sezioni:

\begin{itemize}
	\item \textbf{Analisi dei rischi}: Questa sezione identifica i potenziali
	      rischi individuati dal gruppo e le relative strategie di mitigazione.
	      Un approccio proattivo alla gestione dei rischi è fondamentale
	      per garantire la riuscita del progetto.

	\item \textbf{Modello di sviluppo}: In questa sezione, si descrive
	      l'organizzazione temporale del \textit{team} di SWEnergy.
	      Il modello di sviluppo adottato fornisce una visione chiara delle fasi
	      e delle attività coinvolte nel progetto, consentendo una gestione
	      efficace delle risorse e dei tempi.

	\item \textbf{Pianificazione}: Questa sezione offre una visione dettagliata
	      del piano di lavoro del \textit{team} di progetto.
	      La pianificazione è strutturata in \textit{sprint}, ognuno dei quali
	      rappresenta una unità di tempo focalizzata su obiettivi specifici.
	      Ogni \textit{sprint} è accompagnato da un preventivo delle ore di lavoro stimate
	      e un consuntivo delle ore di lavoro effettivamente impiegate.

	      La pianificazione si basa su una suddivisione chiara delle attività, delle risorse
	      e dei tempi necessari per il completo sviluppo del progetto.
	      L'approccio a \textit{sprint} consente una gestione agile, con la possibilità di
	      adattare la pianificazione in risposta alle dinamiche emergenti.

	\item \textbf{Preventivo a finire}: Attraverso un calendario dettagliato e con l'ultimo
	      \textit{sprint} della fase RTB, questa sezione stabilisce le basi per la successiva fase: PB.
	      Il Preventivo a Finire rappresenta uno strumento cruciale per valutare
	      la sostenibilità delle date di scadenza e dei costi preventivati.

	      Il calendario riflette chiaramente le attività pianificate, le risorse assegnate
	      e i tempi stimati.
	      Inoltre, attraverso l'analisi delle performance durante gli \textit{sprint} precedenti,
	      si ottengono dati concreti per valutare la coerenza tra le previsioni iniziali
	      e la realtà del lavoro svolto.

	      L'ultima fase di ogni \textit{sprint} offre l'opportunità di riflettere sulle strategie
	      adottate per mitigare i rischi e di implementare eventuali correzioni necessarie.
	      Si pone particolare enfasi sulla verifica se le date di scadenza e i costi possono
	      essere mantenuti conformemente ai piani iniziali, o se sia necessario apportare modifiche.

	      La sezione prevede una dettagliata analisi delle variabili che potrebbero influenzare
	      il \textit{budget} e i tempi, permettendo al \textit{team} di anticipare potenziali deviazioni e di
	      implementare azioni correttive in modo tempestivo. La trasparenza nella comunicazione
	      delle modifiche, se necessarie, è prioritaria per mantenere la fiducia degli \textit{stakeholder}
	      e garantire il successo complessivo del progetto.
\end{itemize}
Questo documento ha l'obiettivo primario di aggregare in maniera organica, coesa e
omogenea tutte le informazioni relative alla pianificazione del progetto.
La sua creazione mira a fornire un solido punto di riferimento per la gestione complessiva del progetto,
garantendo che tutti gli elementi chiave siano documentati in modo chiaro e accessibile.
\newline
Al termine della prima fase del progetto (RTB), questo documento sarà impiegato per una valutazione
approfondita dell'andamento del lavoro svolto. Inoltre, servirà come strumento chiave per spiegare
in dettaglio le decisioni prese durante la fase di pianificazione.
La documentazione accurata delle scelte effettuate fornirà un contesto essenziale per comprendere
le ragioni dietro le azioni intraprese, facilitando una comunicazione trasparente e una
valutazione critica delle performance del \textit{team} di progetto.

\subsection{Scopo del prodotto}

"\textit{Easy Meal}" è una \textit{web app}\g progettata per gestire le
prenotazioni presso i ristoranti, sia dal lato dei clienti che dei ristoratori.
Il prodotto finale sarà composto da due parti:

\begin{itemize}
	\item \textbf{Cliente\g}: consente ai clienti di prenotare un tavolo presso un
	      ristorante, visualizzare il menù e effettuare un ordine\g;

	\item \textbf{Ristoratore}: consente ai ristoratori di gestire le
	      prenotazioni e gli ordini dei clienti, oltre a visualizzare la lista
	      degli ingredienti necessari per preparare i piatti ordinati.
\end{itemize}

\subsection{Glossario}

Al fine di prevenire ambiguità linguistiche e garantire una coerenza nell'utilizzo
delle terminologie attraverso i documenti, il \textit{team} ha compilato un documento
interno denominato "Glossario".
Questo documento fornisce definizioni chiare e precise per i termini che potrebbero
risultare ambigui o generare incomprensioni nel testo principale.
I termini inclusi nel Glossario sono facilmente identificabili grazie a un apice 'G'
(ad esempio, parola\g).
Questa pratica agevola la consultazione del Glossario per una comprensione approfondita
dei termini tecnici o specifici utilizzati nel contesto del progetto.

\subsection{Riferimenti}

\subsubsection{Normativi}
\begin{itemize}
	\item "Norme di progetto";
	\item 	\href{https://www.math.unipd.it/~tullio/IS-1/2023/Progetto/C3.pdf}
	      {Documento del capitolato d'appalto C3 - \textit{Easy Meal}} (ultimo accesso 20/03/2024);
	\item \href{https://www.math.unipd.it/~tullio/IS-1/2023/Dispense/PD2.pdf}
	      {Regolamento del progetto} (ultimo accesso 25/03/2024);
\end{itemize}

\subsubsection{Informativi}

Slide dell'insegnamento di Ingegneria del \textit{Software}:
\begin{itemize}
	\item \href{https://www.math.unipd.it/~tullio/IS-1/2023/Dispense/T3.pdf}
	      {Modelli di sviluppo del \textit{software}} (ultimo accesso 25/03/2024);
	\item \href{https://www.math.unipd.it/~tullio/IS-1/2023/Dispense/T4.pdf}
	      {Gestione di progetto} (ultimo accesso 25/03/2024);
	\item \href{https://www.math.unipd.it/~tullio/IS-1/2023/Dispense/T5.pdf}
	      {Analisi dei requisiti} (ultimo accesso 25/03/2024);
	\item Glossario;
\end{itemize}

\subsection{Scadenze}
Il \textit{team} di SWEnergy si impegna a rispettare le seguenti scadenze per il
completamento del progetto:
\begin{itemize}
	\item \textbf{Prima revisione (avanzamento RTB)}: 16 gennaio 2024;
	\item \textbf{Seconda revisione (avanzamento PB)}: da definire;
	\item \textbf{Terza revisione (avanzamento CA)}: da definire;
\end{itemize}

L'indicazione di "Da definire" per le revisioni successive evidenzia l'attenzione
del \textit{team} nei confronti di una pianificazione flessibile, che può essere
adattata in base all'andamento e ai risultati delle fasi precedenti del progetto.
Tale approccio consente al \textit{team} di rispondere in modo dinamico alle esigenze emergenti,
assicurando un percorso di sviluppo efficiente e adeguato.
