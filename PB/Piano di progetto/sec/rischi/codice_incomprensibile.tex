\risktech{Codice incomprensibile}
\label{risk:codice incomprensibile}
\begin{itemize}
	\item \textbf{Descrizione}: questo rischio riguarda la produzione di codice
	      da parte di alcuni membri del gruppo che risulta
	      difficile da comprendere per gli altri membri del \textit{team}.
	\item \textbf{Identificazione}:
	      \begin{itemize}
		      \item \textbf{\textit{Code review}}: durante la fase di verifica del codice,
		            i verificatori potrebbero riscontrare difficoltà
		            nella comprensione del codice, evidenziando
		            potenziali problemi di chiarezza e leggibilità.
	      \end{itemize}

	\item \textbf{Mitigazione}:
	      \begin{itemize}
		      \item \textbf{"Norme di progetto"}: il gruppo ha definito delle linee guida dettagliate
		            per la stesura del codice, al fine di uniformare lo stile di scrittura e facilitare
		            la comprensione. Le norme sono disponibili nel documento "Norme di progetto"
		            nella sotto-sezione "Codifica" sotto il ruolo di programmatore;

		      \item \textbf{\textit{Testing}}: il codice deve essere sottoposto a un processo di
		            \textit{testing} approfondito. Questo non solo aiuta a individuare eventuali errori o \textit{bug},
		            ma contribuisce anche a facilitare la comprensione del codice, illustrando
		            chiaramente i casi d'uso. Si rimanda alla sotto-sezione "Verifica del codice"
		            del documento "Norme di progetto" sotto il ruolo di verificatore.
	      \end{itemize}
\end{itemize}
