\riskplan{Comprensione dei requisiti carente}
\label{risk:comprensione dei requisiti carente}
\begin{itemize}
	\item \textbf{Descrizione}:
	      Il gruppo o qualche suo membro potrebbe non essere in grado di
	      comprendere i requisiti del progetto, oppure potrebbe riscontrare
	      delle difficoltà a causa di una cattiva comprensione dei requisiti.
	\item \textbf{Identificazione}:
	      \begin{itemize}
		      \item \textbf{Dubbi}: i membri del gruppo hanno dei dubbi in merito ai
		            requisiti;

		      \item \textbf{Dibattiti sui requisiti}: i membri del gruppo
		            discutono tra loro in merito ai requisiti;

		      \item \textbf{Discrepanza nella progettazione}: i membri del
		            gruppo progettano in modo diverso, a causa di una cattiva
		            comprensione dei requisiti.
	      \end{itemize}

	\item \textbf{Mitigazione}:
	      \begin{itemize}
		      \item \textbf{Dibattito interno}: SWEnergy si è diviso in coppie
		            per approfondire i casi d'uso e i requisiti del progetto.
		            Successivamente, si è tenuta una riunione interna in cui ciascuna coppia
		            ha esposto  le proprie considerazioni e i propri dubbi. In
		            questo modo, si è cercato di chiarire i dubbi e di
		            uniformare la comprensione dei requisiti;

		      \item \textbf{"Analisi dei requisiti"}: il metodo più formale per
		            ovviare a questa situazione risulta essere
		            l'"Analisi dei requisiti".
		            I requisiti devono essere chiari e completi. Inoltre, il documento 
					include i casi d’uso, che facilitano una migliore comprensione 
					dei requisiti concordati con il proponente;

		      \item \textbf{Dialogo con il proponente}: si instaura un dialogo attivo 
			  con il proponente per discutere dei requisiti, chiarire eventuali dubbi 
			  e definire in maggior dettaglio le funzionalità del prodotto;

		      \item \textbf{Messaggi tempestivi con il proponente}: in caso di dubbi
		            semplici e veloci da risolvere, si inviano dei messaggi al
		            proponente per ottenere una risposta tempestiva, riducendo così 
					eventuali incertezze e ritardi nella comprensione dei requisiti.
	      \end{itemize}

	\item \textbf{Riscontro}: Non sono emerse conseguenze significative. Conformemente al processo di mitigazione si è deciso di svolgere un dibattito interno tra i vari membri di SWEnergy
	e di effettuare un dialogo con il proponente per poter chiarire i dubbi e le incertezze emerse: questi due metodi di mititgazioni sono risultati efficaci e hanno permesso di risolvere 
	le problematiche velocemente, infatti continuano ad essere adottati regolarmente. I messaggi tempestivi con il proponente sono stati utilizzati in maniera limitata, in quanto non si sono
	presentate molte situazioni in cui fossero necessari, ma tutte le volte in cui sono stati usati hanno permesso di ottenere risposte rapide e risolvere i dubbi in tempi brevi.
	L'unica criticità riscontrata è relativa alla mitigazione "Analisi dei requisiti", che nonostante sia stata attuata non si è rivelata del tutto efficace, in quanto diverse volte
	i dubbi dei vari componenti del gruppi non venivano risolti in maniera chiara e completa, per questo motivo il gruppo ha deciso di utilizzarla in maniera limitata, accompagnata
	da un dibattito interno e un dialogo con il proponente.
	
\end{itemize}
