\riskcom{Comunicazione esterna carente}
\label{risk:comunicazione esterna carente}
\begin{itemize}
	\item \textbf{Descrizione}:
	      Le comunicazioni con il proponente o con il committente non sono
	      efficaci ed efficienti, causando riunioni esterne più lunghe del
	      previsto e rallentando le attività; oppure rallentando le attività
	      del gruppo a causa di risposte tardive o mancanti.

	\item \textbf{Identificazione}:
	      \begin{itemize}
		      \item \textbf{Dubbi ripetuti}: durante le riunioni esterne, i
		            membri del gruppo possono porre domande già presentate in
		            precedenza;

		      \item \textbf{Riunioni esterne lunghe}: le riunioni esterne
		            possono protrarsi oltre il tempo previsto;

		      \item \textbf{Risposte tardive o mancanti}: il proponente o il
		            committente può rispondere in ritardo o non rispondere
		            affatto alle comunicazioni del gruppo.
	      \end{itemize}

	\item \textbf{Mitigazione}:
	      \begin{itemize}
		      \item \textbf{Ordine del giorno}: il responsabile si impegna a
		            stilare l'ordine del giorno delle riunioni esterne in anticipo, 
					discutendone la struttura con il gruppo e condividendolo con il 
					proponente e il committente per tempo;

		      \item \textbf{\SAL{ (SAL)}}: il gruppo si impegna a mantenere il
		            proponente aggiornato sullo stato di avanzamento del
		            progetto, riducendo così la necessità di riunioni esterne 
					prolungate e migliorando la qualità del supporto del proponente;

		      \item \textbf{Retrospettive}: si pianificano delle retrospettive all'interno 
			  		dei SAL con il proponente, durante le quali si discute la qualità 
					delle comunicazioni e si propongono soluzioni \textit{ad hoc} per migliorare 
					la comunicazione esterna;

		      \item \textbf{Comunicazioni frequenti}: il proponente viene tenuto
		            aggiornato frequentemente sullo stato di avanzamento del
		            progetto mediante gli appositi canali di comunicazione:
		            \textit{Telegram}\g e \textit{email};

		      \item \textbf{Diario di bordo}: il gruppo si impegna a a mantenere 
			  		diari di bordo quando richiesti dal committente, aggiornandolo 
					così sullo stato di avanzamento del progetto;

		      \item \textbf{\textit{Meeting} supplementari}: in caso di dubbi o 
			  		incertezze, il gruppo può richiedere \textit{meeting} supplementari 
					con il proponente o il committente per una chiara comprensione e 
					risoluzione dei problemi;

		      \item \textbf{Documentazione}: il responsabile aggiorna la documentazione 
			  		correlata agli argomenti delle riunioni esterne, fornendo ai membri 
					del gruppo un riferimento utile in caso di dubbi o incertezze.
	      \end{itemize}

	\item \textbf{Riscontro}: "Ordine del giorno", "\SAL{ (SAL)}", "Retrospettive", "Comunicazioni frequenti", "Diario di bordo" e "Documentazione" 
	si sono rivelate delle mitigazioni efficaci e hanno permesso di risolvere il problema della comunicazione esterna carente. 
	L'unica criticità riscontrata è che in alcuni casi le risposte del proponente o del committente non si sono rivelate del tutto chiare per i cmomponenti del gruppo e per questo motivo tali
	tecniche di mitigazione sono state accompagnate da \textit{meeting} supplementari che si sono rivelati utili per sopperire ai problemi dovuti a risposte poco chiare.
	La documentazione si è rivelata utile per facilitare la comunicazione esterna, ma in alcuni casi non è stata sufficiente a risolvere i problemi di comunicazione e per questo motivo viene accompagnata da altri metodi di mitigazione.
\end{itemize}
