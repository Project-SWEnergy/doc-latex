\riskcom{Comunicazione interna carente}
\label{risk:comunicazione interna carente}
\begin{itemize}
	\item \textbf{Descrizione}:
	      La comunicazione interna non è efficace ed efficiente, causando riunioni
	      interne più lunghe del previsto e rallentando le attività.
	\item \textbf{Identificazione}:
	      \begin{itemize}
		      \item \textbf{Dubbi ripetuti}: durante le riunioni interne, i
		            membri del gruppo possono porre domande già presentate in
		            precedenza;

		      \item \textbf{Riunioni interne lunghe}: le riunioni interne
		            possono protrarsi oltre il tempo previsto;

		      \item \textbf{Fraintendimenti frequenti}: i membri del gruppo
		            possono fraintendersi frequentemente.

	      \end{itemize}
	\item \textbf{Mitigazione}:
	      \begin{itemize}
		      \item \textbf{Documentazione}: il gruppo si impegna a redigere 
			  		documentazione adeguata per facilitare la comunicazione interna. 
					La documentazione può assumere forme diverse a seconda dell'argomento;

		      \item \textbf{\textit{Meeting} frequenti}: il gruppo stabilisce incontri interni 
			  		frequenti per ridurre la durata delle riunioni e migliorare la comunicazione 
					interna. Questo permette un flusso costante di informazioni e la risoluzione 
					tempestiva di eventuali dubbi;

		      \item \textbf{Ordine del giorno}: ogni riunione viene pianificata con un ordine 
			  		del giorno ben definito, garantendo la discussione di tutti gli argomenti 
					rilevanti per lo sviluppo del progetto e definendo il tempo dedicato a 
					ciascun punto;

		      \item \textbf{Retrospettiva}: il gruppo riflette sulle sfide riscontrate nella 
			  		comunicazione interna e sviluppa soluzioni \textit{ad hoc} per migliorare 
					il flusso delle informazioni e prevenire futuri fraintendimenti.
	      \end{itemize}

	\item \textbf{Riscontro}: Le tecniche di mitigazione "\textit{meeting} frequenti", "Ordine del giorno" e "Retrospettiva" sono state efficaci e hanno permesso di ridurre la durata delle riunioni interne e migliorare la comunicazione interna.
	permettendo di risolvere il problema di comunicazione interna carente. L'unica criticità riscontrata è la durata dei \textit{meeting} interni, che in alcuni casi si sono protratti oltre il tempo previsto. Il gruppo si impegnerà 
	a ridurre la durata di tali incontri per evitare rallentamenti nello sviluppo del progetto. La documentazione si è rivelata utile per facilitare la comunicazione interna, ma in alcuni casi non è stata sufficiente a risolvere 
	i problemi di comunicazione e per questo motivo viene accompagnata da altri metodi di mitigazione.
\end{itemize}
