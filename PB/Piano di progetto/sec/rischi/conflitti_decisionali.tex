\riskcom{Conflitti decisionali}
\label{risk:conflitti decisionali}
\begin{itemize}
	\item \textbf{Descrizione}:
	      Il gruppo potrebbe dilungarsi nella discussione di una sola idea, senza
	      raggiungere una decisione finale.
	\item \textbf{Identificazione}:
	      \begin{itemize}
		      \item un punto dell'ordine del giorno subisce un ritardo grave;
	      \end{itemize}
	\item \textbf{Mitigazione}:
	      \begin{itemize}

		      \item \textbf{Dibattito}: i membri del gruppo si impegnano in una 
			  		discussione riguardo all'importanza del punto dell'ordine del 
					giorno per determinare se è necessario approfondire ulteriormente 
					la discussione o meno.

		      \item \textbf{Approfondimento}: se il punto dell'ordine del giorno è 
			  ritenuto importante, almeno due membri del gruppo si dedicano a uno studio 
			  approfondito dei pro e contro delle varie soluzioni possibili. 
			  Possono richiedere supporto al proponente o al committente per chiarire i dubbi.

		      \item \textbf{Votazione}: alla fine del dibattito, i membri del gruppo 
			  votano per la soluzione che ritengono più opportuna. 
			  La votazione è considerata conclusa quando la maggioranza dei membri 
			  del gruppo ha espresso la propria preferenza e il risultato non è un pareggio.

		      \item \textbf{Arbitro imparziale}: il responsabile del progetto ha il compito 
			  di vigilare sul corretto svolgimento del dibattito e della votazione, 
			  intervenendo se la discussione si dilunga eccessivamente. 
			  Il suo ruolo è quello di garantire l'efficienza e l'imparzialità del processo decisionale.
	      \end{itemize}
	\item \textbf{Riscontro}: Rischio non ancora verificatosi.
\end{itemize}
