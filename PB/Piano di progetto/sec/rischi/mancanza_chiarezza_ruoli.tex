\riskcom{Mancanza di chiarezza nei ruoli e responsabilità}
\label{risk:mancanza di chiarezza nei ruoli e responsabilità}
\begin{itemize}
	\item \textbf{Descrizione}:
			La mancanza di chiarezza riguardo ai ruoli e alle responsabilità 
			all'interno del \textit{team} può generare confusione, conflitti e ritardi 
			nelle attività.

	\item \textbf{Identificazione}:
	      \begin{itemize}
		      \item comunicazioni ambigue o incomplete riguardo ai compiti e alle responsabilità;

		      \item incontri regolari per chiarire eventuali dubbi e garantire che tutti 
			  		i membri siano consapevoli dei propri compiti;
	      \end{itemize}

	\item \textbf{Mitigazione}:
	      \begin{itemize}
		      \item stesura e aggiornamento costante di una chiara matrice dei ruoli e responsabilità;

		      \item aggiornare e consultare per ciascun ruolo i compiti principali e responsabilità.
			  		Si rimanda alla sezione del ruolo specifico nel documento "Norme di progetto".
			\item \textbf{Organizzazione di incontri}: organizzare incontri regolari per chiarire eventuali dubbi e garantire che tutti i membri siano consapevoli 
					dei propri compiti e responsabilità.	
	      \end{itemize}

	\item \textbf{Riscontro}: Le due mitigazioni proposte sono state adottate e si sono rivelate efficaci. 
	La stesura e l'aggiornamento costante della matrice dei ruoli e responsabilità ha permesso di chiarire i compiti e le responsabilità di ciascun membro del gruppo, 
	evitando conflitti e ritardi nelle attività. L'aggiornamento e la consultazione dei compiti principali e delle responsabilità per ciascun 
	ruolo sono stati utili per garantire che tutti i membri fossero consapevoli dei propri compiti e responsabilità. L'unica criticità riscontrata è stata la poca 
	interazione tra i membri del gruppo per chiarire eventuali dubbi riguardo ai compiti e alle responsabilità, siccome tutte le mitigazioni sono prettamente di consultazione. 
	Per questo motivo, si è deciso di aggiungere un terzo metodo di mitigazione, ovvero l'organizzazione di incontri per chiarire eventuali dubbi e garantire che tutti i membri.
\end{itemize}
