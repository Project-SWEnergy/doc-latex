\subsection{Sprint - 3}
\textbf{Inizio}: 30-12-2023 \\
\textbf{Fine}: 12-01-2024

\subsubsection{Gestione dei rischi}
\textbf{Rischi attesi verificati}:

\begin{itemize}
	\item RP-3 Interfacce incoerenti
	      \begin{itemize}
		      \item \textbf{Esito mitigazione}: A seguito di un tantativo di integrazione tra la parte \textit{front-end} e \textit{back-end}, il gruppo ha riscontrato delle incoerenze tra le interfacce.
		            Attraverso un dialogo interno i membri del gruppo hanno trovato una soluzione al problema. L'esito è risultato soddisfacente poiché alla fine si è riusciti a integrare le due
		            parti con successo, producendo un PoC funzionante e corretto.

		      \item \textbf{Impatto}: Come conseguenza di questo rischio atteso, il gruppo ha dovuto dedicare più tempo del previsto per la realizzazione del PoC, dovendo risolvere
		            questo problema e andando a sovraccaricare di lavoro alcuni membri del gruppo.
	      \end{itemize}
\end{itemize}
\subsubsection{Diagramma di Gantt}

\begin{ganttchart}[
		x unit=0.6cm, % Adjust the width of each day
		y unit chart=0.6cm,
		bar/.style={fill=blue!50},
		bar height=0.5,
		time slot format=isodate,
		time slot unit=day,
		vgrid,
		today=2024-01-8,
		today rule/.style={draw=red, ultra thick}
	]{2023-12-30}{2024-01-12}
	\gantttitlecalendar{day} \\
	\ganttbar{Analisi dei requisiti}{2023-12-30}{2024-01-05} \\
	\ganttbar{Piano di qualifica}{2023-12-30}{2024-01-05} \\
	\ganttbar{Piano di progetto}{2024-01-07}{2024-01-12} \\
	\ganttbar{PoC \textit{front-end}}{2023-12-30}{2024-01-12} \\
	\ganttbar{PoC \textit{back-end}}{2023-12-30}{2024-01-12}
\end{ganttchart}

Dove:
\begin{itemize}
	\item \textbf{"Analisi dei requisiti"}: questa \textit{issue} è eseguita da
	      Carlo Rosso. Per svolgere questa attività, il gruppo ha deciso di
	      dedicare 5 ore;

	\item \textbf{"Piano di progetto"}: questa \textit{issue} è eseguita da
	      Giacomo Gualato. Per svolgere questa attività, il gruppo ha deciso
	      di dedicare 5 ore;

	\item \textbf{"PoC \textit{front-end}"}: questa \textit{issue} è eseguita
	      da Alessandro Tigani Sava e Matteo Bando. Per svolgere questa
	      attività, il gruppo ha deciso di dedicare 15 ore.

	\item \textbf{"PoC \textit{back-end}"}: questa \textit{issue} è eseguita
	      da Carlo Rosso, Davide Maffei, Niccolò Carlesso. Per svolgere questa
	      attività, il gruppo ha deciso di dedicare 20 ore.

	\item \textbf{Verifica dei documenti}: questo compito è eseguito da
	      Matteo Bando. Per svolgere questa attività, il gruppo ha deciso
	      di dedicare 5 ore.
\end{itemize}

\subsubsection{Preventivo}

\begin{table}[H]
	\centering
	\begin{tabular}{l|r|r|r|r|r|r|r}
		\textbf{Nome}         & \textbf{Re} & \textbf{Am} & \textbf{An} & \textbf{Pt} & \textbf{Pr} & \textbf{Ve} & \textbf{Totale} \\
		\hline
		Alessandro            & -           & -           & -           & -           & 10          & -           & 10              \\
		Carlo                 & -           & -           & 5           & -           & 5           & -           & 10              \\
		Davide                & 5           & -           & -           & -           & 5           & -           & 10              \\
		Giacomo               & -           & -           & 5           & -           & -           & -           & 5               \\
		Matteo                & -           & -           & -           & -           & -           & 10          & 10              \\
		Niccolò               & -           & -           & -           & 10          & -           & -           & 10              \\
		\hline
		\textbf{Ore totali}   & 5           & -           & 10          & 10          & 20          & 10          & 55              \\
		\textbf{Costo totale} & 150         & -           & 250         & 250         & 300         & 150         & 1100
	\end{tabular}
	\caption{Re: Responsabile, Am: Amministratore, An: Analista, Pt: Progettista,
		Pr: Programmatore, Ve: Verificatore, Totale: Totale per persona; valori espressi in ore; Costo totale espresso in euro.}
\end{table}

\subsubsection{Riassunto delle attività svolte}

\begin{itemize}
	\item \textbf{Verifica documenti}: verifica delle "Norme di progetto" e
	      "Piano di progetto";

	\item \textbf{Glossario}: aggiornamento del documento;

	\item \textbf{Norme di progetto}: aggiornamento del documento;

	\item \textbf{Analisi dei requisiti}: inserimento degli UML;

	\item \textbf{PoC front-end}: realizzazione di un PoC in TypeScript, secondo
	      i requisiti concordati con il proponente;

	\item \textbf{PoC back-end}: realizzazione di un PoC in TypeScript, secondo
	      i requisiti concordati con il proponente;
\end{itemize}

\subsubsection{Consuntivo}

\begin{table}[H]
	\centering
	\begin{tabular}{l|r|r|r|r|r|r|r}
		\textbf{Nome}         & \textbf{Re} & \textbf{Am} & \textbf{An} & \textbf{Pt} & \textbf{Pr} & \textbf{Ve} & \textbf{Totale} \\
		\hline
		Alessandro            & -           & -           & -           & -           & 10          & -           & 10              \\
		Carlo                 & -           & -           & 5           & -           & 5           & -           & 10              \\
		Davide                & 5           & -           & -           & -           & 5           & -           & 10              \\
		Giacomo               & -           & 5           & 5           & -           & -           & -           & 10              \\
		Matteo                & -           & -           & -           & -           & 5           & 5           & 10              \\
		Niccolò               & -           & -           & -           & 10          & -           & -           & 10              \\
		\hline
		\textbf{Ore totali}   & 5           & 5           & 10          & 10          & 25          & 5           & 60              \\
		\textbf{Costo totale} & 150         & 100         & 250         & 250         & 375         & 75          & 1200
	\end{tabular}
	\caption{Re: Responsabile, Am: Amministratore, An: Analista, Pt: Progettista,
		Pr: Programmatore, Ve: Verificatore, Totale: Totale per persona; valori espressi in ore; Costo totale espresso in euro.}
\end{table}

\subsubsection{Gestione dei ruoli}

\begin{figure}[h]
	\centering
	\begin{tikzpicture}
		\pie[text=legend]{
			11/Responsabile,
			8/Amministratore,
			16/Analista,
			16/Progettista,
			41/Programmatore,
			8/Verificatore
		}
	\end{tikzpicture}
	\caption{Grafico delle proporzioni dei ruoli ricoperti dai membri del gruppo}
\end{figure}

L'11\% delle risorse è stato dedicato al ruolo di Responsabile, indicando un impegno continuo nella gestione e coordinamento durante il terzo \textit{sprint}.
L'8\% delle risorse è stato assegnato al ruolo di Amministratore, sottolineando la gestione attenta delle attività amministrative e di supporto necessarie per la conclusione del PoC.
Il 16\% delle risorse è stato dedicato al ruolo di Analista, riflettendo un'attenzione costante all'analisi dei requisiti.
Altro 16\% è stato destinato al ruolo di Progettista, indicando un \textit{focus} continuo sulla progettazione delle soluzioni.
Un significativo 41\% delle risorse è stato assegnato al ruolo di Programmatore, evidenziando un aumento considerevole rispetto agli \textit{sprint} precedenti.
Questo è coerente con il contesto del PoC, sottolineando un maggiore sforzo nella fase di codifica durante la realizzazione del progetto.
L'8\% delle risorse è stato dedicato al ruolo di Verificatore, indicando che, nonostante l'attenzione sulla codifica, si è mantenuta una proporzione di risorse per le attività di verifica, contribuendo alla qualità del PoC.

\subsubsection{Analisi retrospettiva}

Durante il terzo \textit{sprint}, il gruppo ha concentrato i suoi sforzi sull'avanzamento del progetto, con particolare attenzione alla realizzazione del \textit{Proof of Concept}
(PoC) sia per il \textit{front-end} che per il \textit{back-end}.

Esaminando le ore impiegate nel terzo \textit{sprint} in relazione allo stato di avanzamento del progetto, il periodo precedente si presenta sostanzialmente positivo, ma
non del tutto soddisfacente, in quanto saranno necessarie alcune modifiche ai documenti per poterli concludere definitivamente. In particolare, l'alto impegno
orario nel ruolo del programmatore ha portato al completamento della Proof of
Concept (PoC), che ha subito notevoli miglioramenti grazie all'aggiunta di nuove
funzionalità. Il secondo ruolo più significativo è stato quello dell'analista,
poiché durante questa fase è stato concluso il documento di Analisi dei
Requisiti.
Questa attenzione particolare a tali ruoli è dovuta al fatto che il gruppo ha
sostenuto l'incontro di revisione dell'RTB con il professore Riccardo Cardin,
ottenendo il semaforo verde.
Con la conclusione di questo \textit{sprint}, il gruppo ha preso una pausa a
seguito della sessione d'esami.

\textbf{Obiettivi raggiunti}:
\begin{itemize}
	\item Conclusione del PoC e del documento Analisi dei Requisiti.
	\item Il gruppo è avanzato nella stesura delle Norme di Progetto, Piano di Progetto e del Piano di Qualifica.
	\item Inserimento dei diagrammi di Gantt e dei grafici a torta all'interno del Piano di Progetto.
	\item Aggiornamento del Glossario.
\end{itemize}

\textbf{Obiettivi mancati}:
\begin{itemize}
	\item Conclusione della stesura delle Norme di Progetto, Piano di Progetto, Piano di Qualifica e del Glossario.
\end{itemize}

\textbf{Problematiche riscontrate}:
\begin{itemize}
	\item Carico di lavoro eccessivo per alcuni membri del gruppo.
	\item Ripartizione scorretta delle attività da svolgere.
	\item Il \textit{team} ha incontrato delle scadenze di consegna e requisiti documentali inizialmente poco chiari, generando un aumento del carico di lavoro necessario per rispettarle.
\end{itemize}

\textbf{Soluzioni attuate}:
\begin{itemize}
	\item Impegnarsi a fissare delle scadenze chiare così da ridurre il carico di lavoro.
\end{itemize}



