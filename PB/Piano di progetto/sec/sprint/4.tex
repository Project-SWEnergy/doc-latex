\subsection{Sprint - 4}

\textbf{Inizio}: 19-02-2024 \\
\textbf{Fine}: 03-03-2024

\subsubsection{Gestione dei rischi}
\textbf{Rischi attesi verificati}:

\begin{itemize}
	\item RC-1 Comunicazione interna carente
	      \begin{itemize}
		      \item \textbf{Esito mitigazione}: La comunicazione interna tra i vari componenti del gruppo a volte è risultata inefficace ed inefficiente causando un rallentamento delle attività.
		            In alcune occasioni venivano svolte delle attività o del lavoro in modo non coordinato, o senza averlo dichiarato in precedenza causando problematiche.
		            L'azione di mitigazione adottata si è dimostrata efficace.
		      \item \textbf{Impatto}: Nessuna conseguenza significativa è stata
		            riscontrata, se non un rallentamento dell'avanzamento del progetto poco significativo, in quanto
		            i componenti del gruppo si sono riuniti al più presto per poter risolvere la problematica.
	      \end{itemize}
\end{itemize}

\subsubsection{Diagramma di Gantt}

\begin{ganttchart}[
		x unit=0.6cm, % Adjust the width of each day
		y unit chart=0.6cm,
		bar/.style={fill=blue!50},
		bar height=0.5,
		time slot format=isodate,
		time slot unit=day,
		vgrid,
		today=2024-02-19,
		today rule/.style={draw=red, ultra thick}
	]{2024-02-19}{2024-03-03}
	\gantttitlecalendar{day} \\
	\ganttbar{Analisi dei requisiti}{2024-02-19}{2024-02-22} \\
	\ganttbar{Piano di progetto}{2024-02-21}{2024-02-27} \\
	\ganttbar{Piano di qualifica}{2024-02-21}{2024-02-27} \\
	\ganttbar{Norme di progetto}{2024-02-19}{2024-02-27} \\
	\ganttbar{Presentazione RTB}{2024-02-22}{2024-02-29}
\end{ganttchart}

Dove:
\begin{itemize}
	\item \textbf{"Analisi dei requisiti"}: questa \textit{issue} è eseguita da
	      Davide Maffei. Per
	      svolgere questa attività, il gruppo ha deciso di dedicare 3 ore;

	\item \textbf{"Piano di progetto"}: questa \textit{issue} è eseguita da
	      Giacomo Gualato e Davide Maffei. Per svolgere questa attività, il gruppo ha deciso
	      di dedicare 8 ore;

	\item \textbf{"Piano di qualifica"}: questa \textit{issue} è eseguita da
	      Alessandro Tigani Sava. Per svolgere questa attività, il gruppo ha
	      deciso di dedicare 5 ore;

	\item \textbf{"Norme di progetto"}: questa \textit{issue} è eseguita da Carlo Rosso.
	      Per svolgere questa attività, il gruppo ha deciso di
	      dedicare 5 ore;

	\item \textbf{"Presentazione RTB"}: questa \textit{issue} è eseguita
	      da Matteo Bando e Niccolò Carlesso. Per svolgere questa
	      attività, il gruppo ha deciso di dedicare 9 ore.
\end{itemize}

\subsubsection{Preventivo}

\begin{table}[H]
	\centering
	\begin{tabular}{l|r|r|r|r|r|r|r}
		\textbf{Nome}         & \textbf{Re} & \textbf{Am} & \textbf{An} & \textbf{Pt} & \textbf{Pr} & \textbf{Ve} & \textbf{Totale} \\
		\hline
		Alessandro            & -           & -           & 5           & -           & -           & -           & 5               \\
		Carlo                 & -           & -           & 5           & -           & -           & -           & 5               \\
		Davide                & -           & -           & -           & -           & -           & 5           & 5               \\
		Giacomo               & -           & -           & -           & -           & -           & 5           & 5               \\
		Matteo                & -           & 5           & -           & -           & -           & -           & 5               \\
		Niccolò               & 5           & -           & -           & -           & -           & -           & 5               \\
		\hline
		\textbf{Ore totali}   & 5           & 5           & 10          & -           & -           & 10          & 30              \\
		\textbf{Costo totale} & 150         & 100         & 250         & -           & -           & 150         & 650
	\end{tabular}
	\caption{Re: Responsabile, Am: Amministratore, An: Analista, Pt: Progettista,
		Pr: Programmatore, Ve: Verificatore, Totale: Totale per persona; valori espressi in ore; Costo totale espresso in euro.}
\end{table}

\subsubsection{Riassunto delle attività svolte}
\begin{enumerate}
	\item \textbf{Verbali interni}: redazione e verifica dei verbali interni, uno datato 12/02/2024 e l'altro 25/02/2024;

	\item \textit{\textbf{Retrospective}}: \textit{Retrospective}  dei documenti in vista della presentazione per la prima fase del RTB;

	\item \textbf{Piano di progetto}: stesura e verifica del quarto \textit{sprint}, comprensivo del consuntivo a finire e di alcune piccole modifiche,
	      come il controllo dei riferimenti al glossario alla prima occorrenza della parola;

	\item \textbf{Glossario}: aggiornamento continuo;

	\item \textbf{Norme di progetto}: apportate varie modifiche, inclusi riferimenti al glossario in ogni occorrenza della parola,
	      riferimenti di sezione con simbolo, revisione e aggiornamento costante;

	\item \textbf{Piano di qualifica}: progressivo sviluppo con l'inserimento di ulteriori grafici relativi alle metriche adottate;

	\item \textbf{Analisi dei requisiti}: controllo dei riferimenti al glossario alla prima occorrenza della parola, inserimento di una sezione che indichi la presenza del glossario e revisione e modifiche generali;

	\item \textbf{Presentazione RTB}: elaborazione della presentazione destinata all'incontro con il professor Tullio Vardanega.
\end{enumerate}

\subsubsection{Consuntivo}

\begin{table}[H]
	\centering
	\begin{tabular}{l|r|r|r|r|r|r|r}
		\textbf{Nome}         & \textbf{Re} & \textbf{Am} & \textbf{An} & \textbf{Pt} & \textbf{Pr} & \textbf{Ve} & \textbf{Totale} \\
		\hline
		Alessandro            & -           & -           & 5           & -           & -           & -           & 5               \\
		Carlo                 & -           & -           & 5           & -           & -           & -           & 5               \\
		Davide                & -           & -           & -           & -           & -           & 5           & 5               \\
		Giacomo               & -           & -           & -           & -           & -           & 5           & 5               \\
		Matteo                & -           & 5           & -           & -           & -           & -           & 5               \\
		Niccolò               & 5           & -           & -           & -           & -           & -           & 5               \\
		\hline
		\textbf{Ore totali}   & 5           & 5           & 10          & -           & -           & 10          & 30              \\
		\textbf{Costo totale} & 150         & 100         & 250         & -           & -           & 150         & 650
	\end{tabular}
	\caption{Re: Responsabile, Am: Amministratore, An: Analista, Pt: Progettista,
		Pr: Programmatore, Ve: Verificatore, Totale: Totale per persona; valori espressi in ore; Costo totale espresso in euro.}
\end{table}

\newpage
\subsubsection{Gestione dei ruoli}

\begin{figure}[h]
	\centering
	\begin{tikzpicture}
		\pie[text=legend]{
			18/Responsabile,
			16/Amministratore,
			33/Analista,
			0/Progettista,
			0/Programmatore,
			33/Verificatore
		}
	\end{tikzpicture}
	\caption{Grafico delle proporzioni dei ruoli ricoperti dai membri del gruppo}
\end{figure}

Il 18\% delle risorse è stato dedicato al ruolo di Responsabile, indicando una maggiore attenzione nella gestione e coordinamento
durante la fase successiva alla conclusione del PoC.
Il 16\% delle risorse è stato assegnato al ruolo di Amministratore, sottolineando la continuità nella gestione delle attività amministrative e di supporto, necessarie ora, ad esempio, per la creazione di documentazione e la preparazione della consegna.
Il 33\% delle risorse è stato dedicato al ruolo di Analista, riflettendo una significativa concentrazione sulla documentazione.
Nessuna risorsa è stata assegnata per quanto riguarda i ruoli di Progettista e Programmatore, questo è dovuto al fatto che le attività sono state orientate verso la documentazione piuttosto che verso la progettazione e la codifica, coerente con il completamento del PoC.
Il 33\% delle risorse è stato assegnato al ruolo di Verificatore, indicando un'impegno considerevole nella verifica della documentazione prodotta, contribuendo alla qualità e all'accuratezza del materiale presentato.

\subsubsection{Analisi retrospettiva}

Esaminando le ore impiegate in questo \textit{sprint} e considerando lo stato di avanzamento del progetto, il periodo appena concluso presenta un bilancio complessivamente positivo,
nonostante il calo di produttività verificatosi nella prima parte dello \textit{sprint}. Questo rallentamento, causato principalmente dalla sessione d'esami, era stato previsto e gestito evitando
di sovraccaricare la pianificazione con troppe attività nella prima metà dello \textit{sprint}.
Durante il quarto \textit{sprint}, tutte le documentazioni destinate alla seconda parte della revisione RTB sono state completate. Inoltre, è stato dedicato sforzo alla preparazione della
presentazione per il Professor Tullio Vardanega. Il gruppo ha preso la decisione di apportare correzioni al documento di Analisi dei Requisiti in
risposta agli errori segnalati dal Professor Riccardo Cardin.
Il Piano di Progetto, le Norme di Progetto, il Piano di Qualifica, così come il Glossario, sono stati aggiornati e sono in procinto di essere approvati.

\textbf{Obiettivi raggiunti}:
\begin{itemize}
	\item Correzione del documento Analisi dei Requisiti.
	\item Preparazione della presentazione per la seconda parte della revisione RTB.
	\item Conclusione della redazione di tutta la documentazione: Piano di Progetto, Norme di Progetto, Piano di Qualifica e Glossario.
\end{itemize}

\textbf{Obiettivi mancati}:
\begin{itemize}
	\item Nessuno.
\end{itemize}

\textbf{Problematiche riscontrate}:
\begin{itemize}
	\item Nessuno.
\end{itemize}

\textbf{Soluzioni attuate}:
\begin{itemize}
	\item Nessuna.
\end{itemize}

