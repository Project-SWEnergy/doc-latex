\subsection{Sprint - 5}
\textbf{Inizio}: 11-03-2024 \\
\textbf{Fine}: 24-03-2024

\subsubsection{Gestione dei rischi}
\textbf{Rischi attesi verificati}:

\begin{itemize}
	\item RC-3 Comunicazione esterna carente
	      \begin{itemize}
		      \item \textbf{Esito mitigazione}: La comunicazione esterna è
		            risultata carente, infatti non è stato organizzato il SAL
		            con il proponente prima della fine dello sprint.

		      \item \textbf{Impatto}: Il ritardo accumulato potrebbe portare
		            ad una lesione del rapporto con il proponente.
	      \end{itemize}

	\item RC-4 Mancanza di chiarezza nei ruoli e responsabilità
	      \begin{itemize}
		      \item \textbf{Esito mitigazione}: La mancanza di chiarezza nei
		            ruoli e responsabilità ha portato ad un ritardo nel cambio
		            dei ruoli, rallentando a cascata il lavoro di tutto il
		            gruppo.

		      \item \textbf{Impatto}: Non è stata organizzata il SAL con il
		            proponente e il cambio dei ruoli è stato effettuato in
		            ritardo.
	      \end{itemize}
\end{itemize}

\subsubsection{Diagramma di Gantt}

\begin{ganttchart}[
		x unit=0.6cm, % Adjust the width of each day
		y unit chart=0.6cm,
		bar/.style={fill=blue!50},
		bar height=0.5,
		time slot format=isodate,
		time slot unit=day,
		vgrid,
		today=2024-03-12,
		today rule/.style={draw=red, ultra thick}
	]{2024-03-11}{2024-03-24}
	\gantttitlecalendar{day} \\
	\ganttbar{Studio Tecnologie}{2024-03-11}{2024-03-17} \\
	\ganttbar{Documentazione}{2024-03-18}{2024-03-24} \\
	\ganttbar{Pianificazione}{2024-03-18}{2024-02-24} \\
	\ganttbar{Database}{2024-03-18}{2024-03-24} \\
\end{ganttchart}

Dove:
\begin{itemize}
	\item \textbf{"Studio tecnologie"}: questa \textit{issue} è eseguita da
	      ciascun componente di SWEnergy;

	\item \textbf{"Documentazione"}: questa \textit{issue} è eseguita da
	      Davide Maffei e Niccolò Carlesso. Per svolgere questa attività, il
	      gruppo ha deciso di dedicare 10 ore;

	\item \textbf{"Pianificazione"}: questa \textit{issue} è eseguita da
	      Alessandro Tigani Sava e Carlo Rosso. Per svolgere questa attività,
	      il gruppo ha deciso di dedicare 10 ore;

	\item \textbf{"Database"}: questa \textit{issue} è eseguita da
	      Giacomo Gualato e Matteo Bando. Per svolgere questa attività, il
	      gruppo ha deciso di dedicare 10 ore;
\end{itemize}

\subsubsection{Preventivo}

\begin{table}[H]
	\centering
	\begin{tabular}{l|r|r|r|r|r|r|r}
		\textbf{Nome}         & \textbf{Re} & \textbf{Am} & \textbf{An} & \textbf{Pt} & \textbf{Pr} & \textbf{Ve} & \textbf{Totale} \\
		\hline
		Alessandro            & -           & -           & -           & 5           & -           & 5           & 10              \\
		Carlo                 & -           & -           & -           & 5           & -           & 5           & 10              \\
		Davide                & -           & -           & -           & 10          & -           & -           & 10              \\
		Giacomo               & -           & -           & -           & -           & 5           & -           & 5               \\
		Matteo                & 5           & -           & -           & -           & 5           & -           & 10              \\
		Niccolò               & -           & -           & -           & 5           & -           & -           & 5               \\
		\hline
		\textbf{Ore totali}   & 5           & -           & -           & 25          & 10          & 10          & 50              \\
		\textbf{Costo totale} & 150         & -           & -           & 625         & 150         & 150         & 1075
	\end{tabular}
	\caption{Re: Responsabile, Am: Amministratore, An: Analista, Pt: Progettista,
		Pr: Programmatore, Ve: Verificatore, Totale: Totale per persona; valori espressi in ore; Costo totale espresso in euro.}
\end{table}

\subsubsection{Riassunto delle attività svolte}
\begin{enumerate}
	\item \textbf{Studio tecnologie}: Alessandro Tigani Sava e Carlo Rosso hanno
	      tenuto due presentazioni per spiegare le tecnologie utilizzate nei PoC
	      agli altri componenti di SWEnergy;

	\item \textbf{Correzione della documentazione}: sono stati corretti i
	      documenti rispetto all'esito RTB, per quanto riguarda gli errori
	      commessi e compresi;

	\item \textbf{Approfondimento della correzione}: è stata stilata una lista
	      di argomenti e domande da porre ai committenti per chiarire i dubbi
	      inerenti all'esito della revisione RTB;

	\item \textbf{Pianificazione della documentazione}: sono stati indagati i
	      documenti da consegnare per la revisione PB. Inoltre, è stato
	      approfondito in quale modo deve evolvere il contenuto di tutti i
	      documenti;

	\item \textbf{Database}: è stato ristrutturato il database, in modo da
	      renderlo adatto a soddisfare tutti i requisiti;
\end{enumerate}

\subsubsection{Consuntivo}

\begin{table}[H]
	\centering
	\begin{tabular}{l|r|r|r|r|r|r|r}
		\textbf{Nome}         & \textbf{Re} & \textbf{Am} & \textbf{An} & \textbf{Pt} & \textbf{Pr} & \textbf{Ve} & \textbf{Totale} \\
		\hline
		Alessandro            & -           & -           & -           & 5           & 5           & -           & 10              \\
		Carlo                 & -           & -           & -           & 5           & -           & 5           & 10              \\
		Davide                & -           & -           & -           & 10          & -           & -           & 10              \\
		Giacomo               & -           & -           & -           & -           & 5           & -           & 5               \\
		Matteo                & 5           & -           & -           & -           & 5           & -           & 10              \\
		Niccolò               & -           & -           & -           & 5           & -           & -           & 5               \\
		\hline
		\textbf{Ore totali}   & 5           & -           & -           & 25          & 15          & 5           & 50              \\
		\textbf{Costo totale} & 150         & -           & -           & 625         & 225         & 75          & 1075
	\end{tabular}
	\caption{Re: Responsabile, Am: Amministratore, An: Analista, Pt: Progettista,
		Pr: Programmatore, Ve: Verificatore, Totale: Totale per persona; valori espressi in ore; Costo totale espresso in euro.}
\end{table}

\newpage
\subsubsection{Gestione dei ruoli}

\begin{figure}[h]
	\centering
	\begin{tikzpicture}
		\pie[text=legend]{
			10/Responsabile,
			0/Amministratore,
			0/Analista,
			50/Progettista,
			30/Programmatore,
			10/Verificatore
		}
	\end{tikzpicture}
	\caption{Grafico delle proporzioni dei ruoli ricoperti dai membri del gruppo}
\end{figure}

\subsubsection{Analisi retrospettiva}

\textbf{Obiettivi raggiunti}:

\textbf{Obiettivi mancati}:

\textbf{Problematiche riscontrate}:

\textbf{Soluzioni attuate}:
