\subsection{Sprint - 5 PB}
\textbf{Inizio}: 11-03-2024 \\
\textbf{Fine}: 24-03-2024

\subsubsection{Gestione dei rischi}
\textbf{Rischi attesi verificati}:

\begin{itemize}
	\item RC-3 Comunicazione esterna carente
	      \begin{itemize}
		      \item \textbf{Esito mitigazione}: La comunicazione esterna è
		            risultata carente, infatti non è stato organizzato il SAL
		            con il proponente prima della fine dello \textit{sprint}.

		      \item \textbf{Impatto}: Il ritardo accumulato potrebbe portare
		            ad una lesione del rapporto con il proponente.
	      \end{itemize}

	\item RC-4 Mancanza di chiarezza nei ruoli e responsabilità
	      \begin{itemize}
		      \item \textbf{Esito mitigazione}: La mancanza di chiarezza nei
		            ruoli e responsabilità ha portato ad un ritardo nel cambio
		            dei ruoli, rallentando a cascata il lavoro di tutto il
		            gruppo.

		      \item \textbf{Impatto}: Non è stato organizzato il SAL con il
		            proponente e il cambio dei ruoli è stato effettuato in
		            ritardo.
	      \end{itemize}
\end{itemize}

\subsubsection{Diagramma di Gantt}

\begin{ganttchart}[
		x unit=0.6cm, % Adjust the width of each day
		y unit chart=0.6cm,
		bar/.style={fill=blue!50},
		bar height=0.5,
		time slot format=isodate,
		time slot unit=day,
		vgrid,
		today=2024-03-12,
		today rule/.style={draw=red, ultra thick}
	]{2024-03-11}{2024-03-24}
	\gantttitlecalendar{day} \\
	\ganttbar{Studio Tecnologie}{2024-03-11}{2024-03-17} \\
	\ganttbar{Documentazione}{2024-03-18}{2024-03-24} \\
	\ganttbar{Pianificazione}{2024-03-18}{2024-03-24} \\
	\ganttbar{Database}{2024-03-18}{2024-03-24} \\
\end{ganttchart}

Dove:
\begin{itemize}
	\item \textbf{"Studio tecnologie"}: questa \textit{issue} è eseguita da
	      ciascun componente di SWEnergy;

	\item \textbf{"Documentazione"}: questa \textit{issue} è eseguita da
	      Davide Maffei e Niccolò Carlesso. Per svolgere questa attività, il
	      gruppo ha deciso di dedicare 10 ore;

	\item \textbf{"Pianificazione"}: questa \textit{issue} è eseguita da
	      Alessandro Tigani Sava e Carlo Rosso. Per svolgere questa attività,
	      il gruppo ha deciso di dedicare 10 ore;

	\item \textbf{"Database"}: questa \textit{issue} è eseguita da
	      Giacomo Gualato e Matteo Bando. Per svolgere questa attività, il
	      gruppo ha deciso di dedicare 10 ore;
\end{itemize}

\subsubsection{Preventivo}

\begin{table}[H]
	\centering
	\begin{tabular}{l|r|r|r|r|r|r|r}
		\textbf{Nome}         & \textbf{Re} & \textbf{Am} & \textbf{An} & \textbf{Pt} & \textbf{Pr} & \textbf{Ve} & \textbf{Totale} \\
		\hline
		Alessandro            & -           & -           & -           & 5           & -           & 5           & 10              \\
		Carlo                 & -           & -           & -           & 5           & -           & 5           & 10              \\
		Davide                & -           & -           & -           & 10          & -           & -           & 10              \\
		Giacomo               & -           & -           & -           & -           & 10          & -           & 10              \\
		Matteo                & 5           & -           & -           & -           & 5           & -           & 10              \\
		Niccolò               & -           & -           & -           & 5           & -           & -           & 5               \\
		\hline
		\textbf{Ore totali}   & 5           & -           & -           & 25          & 15          & 10          & 55              \\
		\textbf{Costo totale} & 150         & -           & -           & 625         & 225         & 150         & 1150
	\end{tabular}
	\caption{Re: Responsabile, Am: Amministratore, An: Analista, Pt: Progettista,
		Pr: Programmatore, Ve: Verificatore, Totale: Totale per persona; valori espressi in ore; Costo totale espresso in euro.}
\end{table}

\subsubsection{Riassunto delle attività svolte}
\begin{enumerate}
	\item \textbf{Studio tecnologie}: Alessandro Tigani Sava e Carlo Rosso hanno
	      tenuto due presentazioni per spiegare le tecnologie utilizzate nei PoC
	      agli altri componenti di SWEnergy;

	\item \textbf{Correzione della documentazione}: sono stati corretti i
	      documenti rispetto all'esito RTB, per quanto riguarda gli errori
	      commessi e compresi;

	\item \textbf{Approfondimento della correzione}: è stata stilata una lista
	      di argomenti e domande da porre ai committenti per chiarire i dubbi
	      inerenti all'esito della revisione RTB;

	\item \textbf{Pianificazione della documentazione}: sono stati indagati i
	      documenti da consegnare per la revisione PB. Inoltre, è stato
	      approfondito in quale modo deve evolvere il contenuto di tutti i
	      documenti;

	\item \textbf{Database}: è stato ristrutturato il \textit{database}, in modo da
	      renderlo adatto a soddisfare tutti i requisiti;
\end{enumerate}

\subsubsection{Consuntivo}

\begin{table}[H]
	\centering
	\begin{tabular}{l|r|r|r|r|r|r|r}
		\textbf{Nome}         & \textbf{Re} & \textbf{Am} & \textbf{An} & \textbf{Pt} & \textbf{Pr} & \textbf{Ve} & \textbf{Totale} \\
		\hline
		Alessandro            & -           & -           & -           & 5           & -           & 5           & 10              \\
		Carlo                 & -           & -           & -           & 5           & -           & 5           & 10              \\
		Davide                & -           & -           & -           & 10          & -           & -           & 10              \\
		Giacomo               & -           & -           & -           & -           & 10          & -           & 10              \\
		Matteo                & 5           & -           & -           & -           & 5           & -           & 10              \\
		Niccolò               & -           & -           & -           & 5           & -           & 5           & 10              \\
		\hline
		\textbf{Ore totali}   & 5           & -           & -           & 25          & 15          & 15          & 60              \\
		\textbf{Costo totale} & 150         & -           & -           & 625         & 225         & 225         & 1225
	\end{tabular}
	\caption{Re: Responsabile, Am: Amministratore, An: Analista, Pt: Progettista,
		Pr: Programmatore, Ve: Verificatore, Totale: Totale per persona; valori espressi in ore; Costo totale espresso in euro.}
\end{table}

\newpage
\subsubsection{Gestione dei ruoli}

\begin{figure}[h]
	\centering
	\begin{tikzpicture}
		\pie[text=legend]{
			8/Responsabile,
			0/Amministratore,
			0/Analista,
			42/Progettista,
			25/Programmatore,
			25/Verificatore
		}
	\end{tikzpicture}
	\caption{Grafico delle proporzioni dei ruoli ricoperti dai membri del gruppo}
\end{figure}

L' 8\% delle risorse sono state dedicate al ruolo di Responsabile, il quale ha avuto una riduzione rispetto allo \textit{sprint} precedente. Nonostante sia
cruciale per la pianificazione e la definizione degli obiettivi dello \textit{sprint} si è deciso di ridurre il tempo dedicato a questo ruolo per dare più spazio agli altri, ma anche perchè il gruppo ha acquisito una maggiore esperienza e autonomia. Il 42\% delle risorse sono state dedicate al ruolo di Progettista, il quale ha avuto un aumento rispetto allo \textit{sprint} precedente dettato dalla necessittà in fase di PB di effettuare una progettazione più dettagliata per realizzare un MVP funzionante e solido.
Il 25\% delle risorse sono state dedicate al ruolo di Programmatore, ruolo fondamentale per la realizzazione del \textit{software}. La percentuale è ancora bassa poichè
il gruppo ha iniziato a programmare solo nella seconda settimana dello \textit{sprint}, ma è destinata ad aumentare nei prossimi \textit{sprint}.
Il 25\% delle risorse sono state dedicate al ruolo di Verificatore, determinando una leggera diminuzione dallo \textit{sprint} precedente, derivata dal passaggio da RTB a PB che ha spinto il gruppo a concentrarsi maggiormente sulle attività di progettazione.
Amministratore e Analista non hanno avuto risorse dedicate in quanto non sono stati necessari per lo svolgimento delle attività.


\subsubsection{Analisi retrospettiva}
Esaminando le ore impiegate in questo \textit{sprint} e considerando lo stato di avanzamento del progetto, il periodo appena concluso presenta un bilancio complessivamente positivo.
Si è dovuto ristrutturare il \textit{database} per la realizzazione del MVP ed effettuare uno studio delle tecnologie da utilizzare per la realizzazione del \textit{software} da parte di tutto il gruppo.\\
Per quanto riguarda la documentazione, nella seconda settimana è avvenuto un approfondimento della correzione che il gruppo ha ricevuto a seguito della valutazione della fase RTB.
A seguito di ciò tutta la documentazione è stata corretta secondo le indicazioni ricevute ed è avvenuta una pianificazione della documentazione da consegnare per la revisione PB.

\textbf{Obiettivi raggiunti}:
\begin{itemize}
	\item Studio delle tecnologie.
	\item Ristrutturazione del \textit{database}.
	\item Correzione della documentazione e inizio stesura della documentazione per la revisione PB.
	\item Stesura iniziale del codice.
\end{itemize}


\textbf{Obiettivi mancati}:
\begin{itemize}
	\item Nessuno.
\end{itemize}

\textbf{Problematiche riscontrate}:
\begin{itemize}
	\item Ripresa e organizzazione del lavoro.
\end{itemize}

\textbf{Soluzioni attuate}:
\begin{itemize}
	\item Pianificazione delle attività da svolgere.
	\item Riunione per la ripartizione dei compiti e organizzazione del lavoro rimasto.
\end{itemize}
