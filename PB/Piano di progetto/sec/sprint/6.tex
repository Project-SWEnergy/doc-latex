\subsection{Sprint - 6 PB}
\textbf{Inizio}: 25-03-2024 \\
\textbf{Fine}: 7-04-2024

\subsubsection{Gestione dei rischi}
\textbf{Rischi attesi verificati}:

\begin{itemize}
	\item RT-1 Conoscenza delle tecnologie carente
	      \begin{itemize}
		      \item \textbf{Esito mitigazione}: La conoscenza delle tecnologie
		            è risultata carente siccome per alcuni membri queste tecnologie risultato nuove. 
					Per mitigare questo rischio, il gruppo ha deciso di dedicare più tempo allo studio autonomo delle tecnologie, combinato a degli 
					\textit{workshop} interni e dividendosi in due gruppi (\textit{frontend} e \textit{backend}) così da dimuire l'\textit{hoverhead}.

		      \item \textbf{Impatto}: L'impatto di questo rischio non è stato significativo, se non per un lieve rallentamento generale del lavoro, 
			  in quanto i membri del gruppo hanno dovuto dedicare più tempo allo studio delle tecnologie. Le mitigazioni applicate si sono rilevate efficaci ed efficienti in 
			  quanto in poco tempo il team è riuscito a colmare le lacune e a proseguire con lo sviluppo del progetto.
	      \end{itemize}
\end{itemize}

\subsubsection{Diagramma di Gantt}

\begin{ganttchart}[
		x unit=0.6cm, % Adjust the width of each day
		y unit chart=0.6cm,
		bar/.style={fill=blue!50},
		bar height=0.5,
		time slot format=isodate,
		time slot unit=day,
		vgrid,
		today=2024-03-26,
		today rule/.style={draw=red, ultra thick}
	]{2024-03-25}{2024-04-07}
	\gantttitlecalendar{day} \\
	\ganttbar{Documentazione}{2024-03-25}{2024-03-31} \\
	\ganttbar{Sperimentazione frontend}{2024-03-25}{2024-03-31} \\
	\ganttbar{Implementazione frontend}{2024-04-01}{2024-04-07} \\
	\ganttbar{Backend}{2024-03-25}{2024-04-07} \\
\end{ganttchart}

Dove:
\begin{itemize}
	\item \textbf{"Documentazione"}: questa \textit{issue} è eseguita da
	      Carlo e Matteo Bando. Per svolgere questa attività, il gruppo ha deciso di
	      dedicare 20 ore.

	\item \textbf{"Sperimentazione frontend"}: questa \textit{issue} è eseguita
	      da Davide Maffei e Giacomo Gualato. Per svolgere questa attività, il
	      gruppo ha deciso di dedicare 10 ore.

	\item \textbf{"Implementazione frontend"}: questa \textit{issue} è eseguita
	      da Davide Maffei e Giacomo Gualato. Per svolgere questa attività, il
	      gruppo ha deciso di dedicare 10 ore.

	\item \textbf{"Backend"}: questa \textit{issue} è eseguita da Alessandro
	      Tigani Sava e Niccolò Carlesso. Per svolgere questa attività, il
	      gruppo ha deciso di dedicare 20 ore.
\end{itemize}

\subsubsection{Preventivo}

\begin{table}[H]
	\centering
	\begin{tabular}{l|r|r|r|r|r|r|r}
		\textbf{Nome}         & \textbf{Re} & \textbf{Am} & \textbf{An} & \textbf{Pt} & \textbf{Pr} & \textbf{Ve} & \textbf{Totale} \\
		\hline
		Alessandro            & -           & -           & -           & -           & 5           & 5           & 10              \\
		Carlo                 & -           & -           & -           & 5           & -           & 5           & 10              \\
		Davide                & 5           & -           & -           & -           & 5           & -           & 10              \\
		Giacomo               & -           & -           & -           & 5           & 5           & -           & 10              \\
		Matteo                & -           & 5           & -           & -           & 5           & -           & 10              \\
		Niccolò               & -           & -           & -           & 5           & 5           & -           & 10              \\
		\hline
		\textbf{Ore totali}   & 5           & 5           & -           & 15          & 25          & 10          & 60              \\
		\textbf{Costo totale} & 150         & 100         & -           & 375         & 375         & 150         & 1150
	\end{tabular}
	\caption{Re: Responsabile, Am: Amministratore, An: Analista, Pt: Progettista,
		Pr: Programmatore, Ve: Verificatore, Totale: Totale per persona; valori espressi in ore; Costo totale espresso in euro.}
\end{table}

\subsubsection{Riassunto delle attività svolte}

\begin{enumerate}
	\item \textbf{Stesura documentazione}: In questo \textit{sprint} si sono concluse le correzioni dei documenti rispetto all'esito RTB,
	ed è stata iniziata la stesura dei documenti nuovi richiesti per la revisione PB: "Manuale Utente" e "Specifica Tecnica".

	\item \textbf{Frontend}: Le tecnologie per il \textit{frontend} sono state sperimentate e studiante nella prima settimana e implementate durante la seconda settimana
	dello \textit{sprint}. Durante questa fase sono state realizzate le pagine di login e di registrazione, e sono stati implementati i primi servizi per la comunicazione con il \textit{backend}. 

	\item \textbf{Backend}: Il lavoro del \textit{backend} si è concentrato sulla realizzazione dei servizi per la comunicazione con il \textit{frontend} attraverso le API.
\end{enumerate}

\subsubsection{Consuntivo}
\begin{table}[H]
	\centering
	\begin{tabular}{l|r|r|r|r|r|r|r}
		\textbf{Nome}         & \textbf{Re} & \textbf{Am} & \textbf{An} & \textbf{Pt} & \textbf{Pr} & \textbf{Ve} & \textbf{Totale} \\
		\hline
		Alessandro            & -           & -           & -           & -           & 5           & 5           & 10              \\
		Carlo                 & -           & -           & -           & 5           & -           & 5           & 10              \\
		Davide                & 5           & -           & -           & -           & 5           & -           & 10              \\
		Giacomo               & -           & -           & -           & 5           & 5           & -           & 10              \\
		Matteo                & -           & 5           & -           & -           & 5           & -           & 10              \\
		Niccolò               & -           & -           & -           & -           & 10           & -           & 10              \\
		\hline
		\textbf{Ore totali}   & 5           & 5           & -           & 10          & 30          & 10          & 60              \\
		\textbf{Costo totale} & 150         & 100         & -           & 350         & 400         & 150         & 1150
	\end{tabular}
	\caption{Re: Responsabile, Am: Amministratore, An: Analista, Pt: Progettista,
		Pr: Programmatore, Ve: Verificatore, Totale: Totale per persona; valori espressi in ore; Costo totale espresso in euro.}
\end{table}

\subsubsection{Gestione dei ruoli}
\begin{figure}[h]
	\centering
	\begin{tikzpicture}
		\pie[text=legend]{
            10/Responsabile,
            10/Amministratore,
            0/Analista,
            15/Progettista,
            50/Programmatore,
            15/Verificatore
        }
	\end{tikzpicture}
	\caption{Grafico delle proporzioni dei ruoli ricoperti dai membri del gruppo}
\end{figure}

Per quanto riguarda il ruolo del Programmatore il gruppo ha visto un aumento del tempo dedicato alla programmazione, che rappresenta ora il 50\% del totale delle ore, 
rispetto al 30\% dello sprint precedente. Questo è dovuto ad una maggiore enfasi sull'implementazione pratica del lavoro, con il \textit{team} che si è concentrato maggiormente sulla scrittura del codice.
D'altra parte, c'è stato un leggero aumento nel tempo dedicato alla progettazione (15\% rispetto al 10\% dello \textit{sprint} precedente), il che 
è dovuto a degli aggiornamenti all'architettura o al design del sistema. Inoltre, abbiamo mantenuto una distribuzione equilibrata del tempo tra il Responsabile e il Verificatore
(entrambi al 15\% e 10\% in relazione allo \textit{sprint} passato), il che suggerisce che la gestione del progetto e il controllo di qualità sono rimasti focali e importanti anche durante questo \textit{sprint}.
La mancanza di ore assegnate al ruolo di Analista durante questo \textit{sprint} è dovuto al fatto che non c'erano attività specifiche richieste per tale ruolo.

\subsubsection{Analisi retrospettiva}
Esaminando le ore impiegate in questo \textit{sprint}, il periodo appena concluso presenta un bilancio complessivamente positivo anche se il gruppo del \textit{frontend} 
ha impiegato più tempo del previsto per la realizzazione delle pagine, trovandosi indietro rispetto al gruppo del \textit{backend}, questo è dovuto dal fatto che
la fase di stesura del codice ha subito uno slittamento temporale siccome prima di iniziare a programmare è stato necessario approfondire e impratichirsi con le tecnologie da utilizzare.
La documentazione è in una buona posizione, con la stesura dei documenti in corso e in linea con la pianificazione. Sia il "Manuale Utente" che la "Specifica Tecnica" 
sono stati iniziati e sono in fase di completamento.

\textbf{Obiettivi raggiunti}:
\begin{itemize}
	\item Implementazione delle pagine di login e registrazione.
	\item Implementazione dei servizi per la comunicazione con il \textit{backend}.
	\item Stesura del "Manuale Utente".
	\item Stesura della "Specifica Tecnica".
\end{itemize}

\textbf{Obiettivi mancati}: Nessuno.

\textbf{Problematiche riscontrate}: Nessuno.

\textbf{Soluzioni attuate}: Nessuno.