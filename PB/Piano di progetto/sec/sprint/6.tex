\subsection{Sprint - 6 PB}
\textbf{Inizio}: 25-03-2024 \\
\textbf{Fine}: 07-04-2024

\subsubsection{Gestione dei rischi}
\textbf{Rischi attesi verificati}:

\begin{itemize}
	\item RT-1 Conoscenza delle tecnologie carente
	      \begin{itemize}
		      \item \textbf{Esito mitigazione}: La conoscenza delle tecnologie è risultata carente siccome per alcuni membri queste tecnologie risultato nuove.
		            Per mitigare questo rischio, il gruppo ha deciso di dedicare più tempo allo studio autonomo delle tecnologie, combinato a degli \textit{workshop} interni e dividendosi in due gruppi (\textit{frontend} e \textit{backend}) così da dimuire l'\textit{hoverhead}.
		      \item \textbf{Impatto}: L'impatto di questo rischio non è stato significativo, se non per un lieve rallentamento generale del lavoro, in quanto i membri del gruppo hanno dovuto dedicare più tempo allo studio delle tecnologie.
		            Le mitigazioni applicate si sono rilevate efficaci ed efficienti in quanto in poco tempo il team è riuscito a colmare le lacune e a proseguire con lo sviluppo del progetto.
	      \end{itemize}
	\item RP-4 Costi e tempi imprevisti
	      \begin{itemize}
		      \item \textbf{Esito mitigazione}: La tempistica prevista per la codifica è risultata troppo ottimistica. Il gruppo ha provveduto a ridistribuire il carico di lavoro per migliorare la pianificazione degli \textit{sprint} successivi. L'esito della mitigazione sarà visibile nell'esito dei consuntivi relativi ai successivi \textit{sprint}.
		      \item  \textbf{Impatto}: L'impatto è stato significativo ed ha provocato una allocazione non prevista di ore di lavoro nelle attività di programmazione. 
	      \end{itemize}
\end{itemize}

\subsubsection{Diagramma di Gantt}

\begin{ganttchart}[
		x unit=0.6cm, % Adjust the width of each day
		y unit chart=0.6cm,
		bar/.style={fill=blue!50},
		bar height=0.5,
		time slot format=isodate,
		time slot unit=day,
		vgrid,
		today=2024-03-26,
		today rule/.style={draw=red, ultra thick}
	]{2024-03-25}{2024-04-07}
	\gantttitlecalendar{day} \\
	\ganttbar{Documentazione}{2024-03-25}{2024-03-31} \\
	\ganttbar{Sperimentazione frontend}{2024-03-25}{2024-03-31} \\
	\ganttbar{Implementazione frontend}{2024-04-01}{2024-04-07} \\
	\ganttbar{Backend}{2024-03-25}{2024-04-07} \\
\end{ganttchart}

Dove:
\begin{itemize}
	\item \textbf{"Documentazione"}: questa \textit{issue} è eseguita da
	      Carlo Rosso e Matteo Bando. Per svolgere questa attività, il gruppo ha deciso di
	      dedicare 20 ore.

	\item \textbf{"Sperimentazione frontend"}: questa \textit{issue} è eseguita
	      da Davide Maffei e Giacomo Gualato. Per svolgere questa attività, il
	      gruppo ha deciso di dedicare 10 ore.

	\item \textbf{"Implementazione frontend"}: questa \textit{issue} è eseguita
	      da Davide Maffei e Giacomo Gualato. Per svolgere questa attività, il
	      gruppo ha deciso di dedicare 10 ore.

	\item \textbf{"Backend"}: questa \textit{issue} è eseguita da Alessandro
	      Tigani Sava e Niccolò Carlesso. Per svolgere questa attività, il
	      gruppo ha deciso di dedicare 20 ore.
\end{itemize}

\subsubsection{Preventivo}

\begin{table}[H]
	\centering
	\begin{tabular}{l|r|r|r|r|r|r|r}
		\textbf{Nome}         & \textbf{Re} & \textbf{Am} & \textbf{An} & \textbf{Pt} & \textbf{Pr} & \textbf{Ve} & \textbf{Totale} \\
		\hline
		Alessandro            & -           & -           & -           & -           & 5           & 5           & 10              \\
		Carlo                 & -           & -           & -           & 5           & -           & 5           & 10              \\
		Davide                & 5           & -           & -           & -           & 5           & -           & 10              \\
		Giacomo               & -           & -           & -           & 5           & 5           & -           & 10              \\
		Matteo                & -           & 5           & -           & -           & 5           & -           & 10              \\
		Niccolò               & -           & -           & -           & 5           & 5           & -           & 10              \\
		\hline
		\textbf{Ore totali}   & 5           & 5           & -           & 15          & 25          & 10          & 60              \\
		\textbf{Costo totale} & 150         & 100         & -           & 375         & 375         & 150         & 1150
	\end{tabular}
	\caption{Re: Responsabile, Am: Amministratore, An: Analista, Pt: Progettista,
		Pr: Programmatore, Ve: Verificatore, Totale: Totale per persona; valori espressi in ore; Costo totale espresso in euro.}
\end{table}

\subsubsection{Riassunto delle attività svolte}

\begin{enumerate}
	\item \textbf{Stesura documentazione}: In questo \textit{sprint} si sono concluse le correzioni dei documenti rispetto all'esito RTB,
	      ed è stata iniziata la stesura dei documenti nuovi richiesti per la revisione PB: "Manuale Utente" e "Specifica Tecnica".

	\item \textbf{Frontend}: Le tecnologie per il \textit{frontend} sono state sperimentate e studiante nella prima settimana e implementate durante la seconda settimana
	      dello \textit{sprint}. Durante questa fase sono state realizzate le pagine di login e di registrazione, e sono stati implementati i primi servizi per la comunicazione con il \textit{backend}.

	\item \textbf{Backend}: Il lavoro del \textit{backend} si è concentrato sulla realizzazione dei servizi per la comunicazione con il \textit{frontend} attraverso le API.
\end{enumerate}

\subsubsection{Consuntivo}
\begin{table}[H]
	\centering
	\begin{tabular}{l|r|r|r|r|r|r|r}
		\textbf{Nome}         & \textbf{Re} & \textbf{Am} & \textbf{An} & \textbf{Pt} & \textbf{Pr} & \textbf{Ve} & \textbf{Totale} \\
		\hline
		Alessandro            & -           & -           & -           & -           & 10          & 5           & 15              \\
		Carlo                 & -           & -           & -           & 5           & 5           & 5           & 15              \\
		Davide                & 5           & -           & -           & -           & 10          & -           & 15              \\
		Giacomo               & -           & -           & -           & 5           & 10          & -           & 15              \\
		Matteo                & -           & 5           & -           & -           & 10          & -           & 15              \\
		Niccolò               & -           & -           & -           & 5           & -           & 5           & 10              \\
		\hline
		\textbf{Ore totali}   & 5           & 5           & -           & 15          & 45          & 15          & 85              \\
		\textbf{Costo totale} & 150         & 100         & -           & 375         & 675         & 225         & 1525
	\end{tabular}
	\caption{Re: Responsabile, Am: Amministratore, An: Analista, Pt: Progettista,
		Pr: Programmatore, Ve: Verificatore, Totale: Totale per persona; valori espressi in ore; Costo totale espresso in euro.}
\end{table}

\subsubsection{Gestione dei ruoli}
\begin{figure}[h]
	\centering
	\begin{tikzpicture}
		\pie[text=legend]{
			8/Responsabile,
			8/Amministratore,
			0/Analista,
			25/Progettista,
			42/Programmatore,
			17/Verificatore
		}
	\end{tikzpicture}
	\caption{Grafico delle proporzioni dei ruoli ricoperti dai membri del gruppo}
\end{figure}

Per quanto riguarda il ruolo del Programmatore il gruppo ha visto un aumento del tempo dedicato alla codifica, che rappresenta ora il 42\% del totale delle ore,
rispetto al 25\% dello \textit{sprint} precedente. 
Questo incremento è il risultato di un maggior impegno nell'attuare concretamente il lavoro, con il \textit{team} che ha focalizzato la sua attenzione sulla scrittura del codice.\\
D'altra parte, si è registrata una diminuzione nel tempo dedicato alla fase di progettazione (ora pari al 25\%, rispetto al 42\% dello \textit{sprint} precedente). Questa diminuzione è il risultato del fatto che le attività di progettazione si sono concentrate esclusivamente sugli eventuali aggiornamenti all'architettura o al design del sistema, i quali hanno richiesto meno tempo da parte del gruppo. \\
Si è mantenuta una distribuzione equilibrata del tempo del Responsabile, fissandolo all'8\% sia in questo \textit{sprint} che in quello precedente, mantenendo così costante il livello di gestione del progetto.\\
In questo \textit{sprint} sono state invece introdotte alcune ore per l'Amministratore, poiché questa fase ha richiesto una maggiore quantità di codifica rispetto agli \textit{sprint} precedenti, rendendo necessario l'inserimento di questo ruolo.\\
Infine, il ruolo del Verificatore è rimasto costante intorno al 20\% (17\% in questo \textit{sprint} e 25\% nel precedente), per garantire un livello costante di qualità.\\
La mancanza di ore assegnate al ruolo di Analista durante questo \textit{sprint} è dovuta al fatto che non sono state richieste attività specifiche per tale ruolo.

\subsubsection{Analisi retrospettiva}
Esaminando le ore impiegate in questo \textit{sprint}, il periodo appena concluso evidenzia delle problematiche relative alla pianificazione del lavoro.\\
Sia il gruppo del \textit{frontend} che quello del \textit{backend} hanno impiegato più tempo del necessario nella codifica del progetto.
Il \textit{team} ritiene che tale problematica sia sorta dalla scarsa conoscenza delle tecnologie utilizzate al momento della pianificazione del lavoro, che ha portato ad una sottostima del tempo necessario per la codifica.
Si registra anche un leggero disallineamento nella velocità di sviluppo dei due gruppi.
L'aumento delle ore di lavoro è avvenuto senza però determinare un eccessivo aumento dei costi dello \textit{sprint}.\\
Per quanto riguarda la documentazione, si è in una buona posizione, con la stesura dei documenti in corso e in linea con la pianificazione. 
Entrambi il "Manuale Utente" e la "Specifica Tecnica" sono stati avviati e si trovano in fase di completamento.

\textbf{Obiettivi raggiunti}:
\begin{itemize}
	\item Implementazione delle pagine di \textit{login} e registrazione.
	\item Implementazione dei servizi per la comunicazione con il \textit{backend}.
	\item Stesura del "Manuale Utente".
	\item Stesura della "Specifica Tecnica".
\end{itemize}

\textbf{Obiettivi mancati}: Nessuno.

\textbf{Problematiche riscontrate}: \begin{itemize}
	\item Sottostima delle ore assegnate alla codifica.
\end{itemize}

\textbf{Soluzioni attuate}: \begin{itemize}
	\item Aumento delle ore assegnate alla codifica.
\end{itemize}