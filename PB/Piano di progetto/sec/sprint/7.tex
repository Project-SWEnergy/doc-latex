\subsection{Sprint - 7 PB}
\textbf{Inizio}: 08-04-2024 \\
\textbf{Fine}: 21-04-2024

\subsubsection{Gestione dei rischi}
\textbf{Rischi attesi verificati}:

\begin{itemize}
	\item Nessuno.
\end{itemize}

\subsubsection{Diagramma di Gantt}

\begin{ganttchart}[
		x unit=0.6cm, % Adjust the width of each day
		y unit chart=0.6cm,
		bar/.style={fill=blue!50},
		bar height=0.5,
		time slot format=isodate,
		time slot unit=day,
		vgrid,
		today=2024-04-08,
		today rule/.style={draw=red, ultra thick}
	]{2024-04-08}{2024-04-21}
	\gantttitlecalendar{day} \\
	\ganttbar{Documentazione}{2024-04-08}{2024-04-14} \\
	\ganttbar{Implementazione frontend}{2024-04-08}{2024-04-21} \\
	\ganttbar{Implementazione backend}{2024-04-08}{2024-04-21} \\
\end{ganttchart}

Dove:
\begin{itemize}
	\item \textbf{"Documentazione"}: questa \textit{issue} è eseguita da
	      Alessandro Tigani Sava e Giacomo Gualato. Per svolgere questa attività, il gruppo ha deciso di
	      dedicare 20 ore.
	\item \textbf{"Implementazione frontend"}: questa \textit{issue} è eseguita
	      da Davide Maffei e Carlo Rosso. Per svolgere questa attività, il
	      gruppo ha deciso di dedicare 20 ore.
	\item \textbf{"Implementazione backend"}: questa \textit{issue} è eseguita da Matteo Bando e Niccolò Carlesso. Per svolgere questa attività, il
	      gruppo ha deciso di dedicare 20 ore.
\end{itemize}

\subsubsection{Preventivo}

\begin{table}[H]
	\centering
	\begin{tabular}{l|r|r|r|r|r|r|r}
		\textbf{Nome}         & \textbf{Re} & \textbf{Am} & \textbf{An} & \textbf{Pt} & \textbf{Pr} & \textbf{Ve} & \textbf{Totale} \\
		\hline
		Alessandro            & 5           & -           & -           & 5           & 5           & -           & 15              \\
		Carlo                 & -           & -           & -           & -           & 10           & 5           & 15              \\
		Davide                & -           & -           & -           & 5           & 5           & 5           & 15              \\
		Giacomo               & -           & -           & -           & -           & 15          	& -           & 15              \\
		Matteo                & -           & 5           & -           & 5           & 5           & -           & 15              \\
		Niccolò               & -           & -           & -           & -           & 5           & 5           & 10              \\
		\hline
		\textbf{Ore totali}   & 5           & 5           & -           & 15          & 45          & 15          & 85              \\
		\textbf{Costo totale} & 150         & 100         & -           & 375         & 675         & 225         & 1525
	\end{tabular}
	\caption{Re: Responsabile, Am: Amministratore, An: Analista, Pt: Progettista,
		Pr: Programmatore, Ve: Verificatore, Totale: Totale per persona; valori espressi in ore; Costo totale espresso in euro.}
\end{table}

\subsubsection{Riassunto delle attività svolte}

\begin{enumerate}
	\item \textbf{Stesura documentazione}: In questo \textit{sprint} si è continuata la stesura dei documenti richiesti per la revisione PB: "Manuale Utente", "Specifica Tecnica", "Piano di qualifica", "Piano di progetto". Sono state aggiornate le Norme di Progetto.

	\item \textbf{Frontend}: Durante questa fase sono state svolte le attività relative alla registrazione, all'autenticazione, alla visualizzazione della \textit{homepage} e dei ristoranti per un utente generico$^G$, nonché alla gestione dei piatti per tutte le categorie di utenti previste.

	\item \textbf{Backend}: Il lavoro del \textit{backend} si è concentrato sulla realizzazione di tutte le API di base per la gestione delle tabelle in \textit{database}, in particolare la gestione dei piatti presenti nel menu e la registrazione completa dei ristoratori.
\end{enumerate}

\subsubsection{Consuntivo}
\begin{table}[H]
	\centering
	\begin{tabular}{l|r|r|r|r|r|r|r}
		\textbf{Nome}         & \textbf{Re} & \textbf{Am} & \textbf{An} & \textbf{Pt} & \textbf{Pr} & \textbf{Ve} & \textbf{Totale} \\
		\hline
		Alessandro            & 5           & -           & -           & 5           & 5           & -           & 15              \\
		Carlo                 & -           & -           & -           & -           & 10           & 5           & 15              \\
		Davide                & -           & -           & -           & 5           & 5           & 5           & 15              \\
		Giacomo               & -           & -           & -           & -           & 15          	& -           & 15              \\
		Matteo                & -           & 5           & -           & 5           & 5           & -           & 15              \\
		Niccolò               & -           & -           & -           & -           & 5           & 5           & 10              \\
		\hline
		\textbf{Ore totali}   & 5           & 5           & -           & 15          & 45          & 15          & 85              \\
		\textbf{Costo totale} & 150         & 100         & -           & 375         & 675         & 225         & 1525
	\end{tabular}
	\caption{Re: Responsabile, Am: Amministratore, An: Analista, Pt: Progettista,
		Pr: Programmatore, Ve: Verificatore, Totale: Totale per persona; valori espressi in ore; Costo totale espresso in euro.}
\end{table}

\subsubsection{Gestione dei ruoli}
\begin{figure}[h]
	\centering
	\begin{tikzpicture}
		\pie[text=legend]{
            6/Responsabile,
            6/Amministratore,
            0/Analista,
            17/Progettista,
            53/Programmatore,
            18/Verificatore
        }
	\end{tikzpicture}
	\caption{Grafico delle proporzioni dei ruoli ricoperti dai membri del gruppo}
\end{figure}

In questo \textit{sprint} il focus è stato maggiormente incentrato sull'implementazione del codice, con il 53\% del tempo del \textit{team} dedicato alla programmazione. Questo rappresenta un leggero aumento rispetto allo \textit{sprint} precedente, in cui si attestava al 43\%.
D'altra parte, c'è stata una leggera riduzione nel numero di ore dedicate al ruolo di Progettista, passando dal 25\% al 17\%, poichè la fase di progettazione ha richiesto solo qualche modifica a quanto già effettuato nello \textit{sprint} precedente.
Gli altri ruoli nel team sono rimasti pressoché stabili, con le ore di Amministratore e Responsabile al 6\% ciascuno. 
La percentuale delle ore dedicate al ruolo di Verificatore è leggermente aumentata rispetto allo sprint precedente, dettata sia dall'implementazione dei \textit{test} sul codice, sia dalla verifica dei vari docummenti elaborati.

\subsubsection{Analisi retrospettiva}
Esaminando le ore impiegate in questo \textit{sprint}, il periodo appena concluso presenta un bilancio complessivamente positivo. Durante il sesto \textit{sprint} il gruppo si è reso conto di aver sottostimato le ore necessarie per svolgere le attività di programmazione, perciò nel preventivo del seguente \textit{sprint} si è deciso di incrementare le ore totali dei componenti del gruppo da 10 a 15 basandosi sull'esperienza acquisita dallo \textit{sprint} precedente.\\
Il divario tra il gruppo assegnato allo sviluppo del \textit{backend} e quello assegnato al \textit{frontend} è stato colmato, tuttavia è stato registrato un lieve rallentamento nei lavori rispetto alle previsioni, poiché sono emerse problematiche non individuate durante la fase di creazione del Proof of Concept (PoC). Queste problematiche hanno richiesto una revisione di porzioni di codice, comportando un adattamento delle attività in corso.\\
Per quanto riguarda la documentazione, ci troviamo in una situazione soddisfacente: la redazione dei documenti è in corso ed è allineata con la pianificazione. Entrambi il "Manuale Utente" e la "Specifica Tecnica" sono stati avviati e si trovano in fase di completamento.

\textbf{Obiettivi raggiunti}:
\begin{itemize}
	\item Completamento delle pagine di \textit{login} e registrazione.
	\item Visualizzazione di \textit{homepage} ed elenco ristoranti per un utente generico$^G$.
	\item Gestione dei piatti presenti in menu.
	\item Stesura del "Manuale Utente".
	\item Stesura della "Specifica Tecnica".
\end{itemize}

\textbf{Obiettivi mancati}: Nessuno.

\textbf{Problematiche riscontrate}: Nessuno.

\textbf{Soluzioni attuate}: Nessuno.