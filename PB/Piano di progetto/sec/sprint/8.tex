\subsection{Sprint - 8 PB}
\textbf{Inizio}: 22-04-2024 \\
\textbf{Fine}: 5-05-2024

\subsubsection{Gestione dei rischi}
\textbf{Rischi attesi verificati}:

\begin{itemize}
	\item Nessuno.
\end{itemize}

\subsubsection{Diagramma di Gantt}

\begin{ganttchart}[
		x unit=0.6cm, % Adjust the width of each day
		y unit chart=0.6cm,
		bar/.style={fill=blue!50},
		bar height=0.5,
		time slot format=isodate,
		time slot unit=day,
		vgrid,
		today=2024-04-22,
		today rule/.style={draw=red, ultra thick}
	]{2024-04-22}{2024-05-05}
	\gantttitlecalendar{day} \\
	\ganttbar{Documentazione}{2024-05-01}{2024-05-05} \\
	\ganttbar{Implementazione frontend}{2024-04-22}{2024-05-05} \\
	\ganttbar{Implementazione backend}{2024-04-22}{2024-05-05} \\
\end{ganttchart}

Dove:
\begin{itemize}
	\item \textbf{"Documentazione"}: questa \textit{issue} è eseguita da Niccolò Carlesso e Davide Maffei. Per svolgere questa attività, il gruppo ha deciso di dedicare 20 ore.
	\item \textbf{"Implementazione frontend"}: questa \textit{issue} è eseguita da Giacomo Gualato e Carlo Rosso. Per svolgere questa attività, il gruppo ha deciso di dedicare 20 ore.
	\item \textbf{"Implementazione backend"}: questa \textit{issue} è eseguita da Matteo Bando e Alessandro Tigani Sava. Per svolgere questa attività, il gruppo ha deciso di dedicare 20 ore.
\end{itemize}

\subsubsection{Preventivo}

\begin{table}[H]
	\centering
	\begin{tabular}{l|r|r|r|r|r|r|r}
		\textbf{Nome}         & \textbf{Re} & \textbf{Am} & \textbf{An} & \textbf{Pt} & \textbf{Pr} & \textbf{Ve} & \textbf{Totale} \\
		\hline
		Alessandro            & -           & -           & -           & 5           & 5           & -           & 10              \\
		Carlo                 & -           & -           & -           & 5           & 5           & -           & 10              \\
		Davide                & -           & 5           & -           & -           & 5           & -           & 10              \\
		Giacomo               & -           & -           & -           & -           & 5          	& 5           & 10              \\
		Matteo                & -           & -           & -           & -           & 5           & 5           & 10              \\
		Niccolò               & 5           & -           & -           & 5           & -           & -           & 10              \\
		\hline
		\textbf{Ore totali}   & 5           & 5           & -           & 15          & 25          & 10          & 60              \\
		\textbf{Costo totale} & 150         & 100         & -           & 375         & 375         & 150         & 1150
	\end{tabular}
	\caption{Re: Responsabile, Am: Amministratore, An: Analista, Pt: Progettista,
		Pr: Programmatore, Ve: Verificatore, Totale: Totale per persona; valori espressi in ore; Costo totale espresso in euro.}
\end{table}

\subsubsection{Riassunto delle attività svolte}

\begin{enumerate}
	\item \textbf{Stesura documentazione}: In questo \textit{sprint} si è continuata la stesura dei documenti richiesti per la revisione PB: "Manuale Utente", "Specifica Tecnica", "Piano di qualifica", "Piano di progetto".

	\item \textbf{Codifica}: È stata implementata la gestione delle prenotazioni, la gestione dell'ordinazione collaborativa dei pasti, il caricamento delle immagini da parte del ristoratore e la gestione degli ingredienti e dei piatti lato ristoratore.
\end{enumerate}

\subsubsection{Consuntivo}
\begin{table}[H]
	\centering
	\begin{tabular}{l|r|r|r|r|r|r|r}
		\textbf{Nome}         & \textbf{Re} & \textbf{Am} & \textbf{An} & \textbf{Pt} & \textbf{Pr} & \textbf{Ve} & \textbf{Totale} \\
		\hline
		Alessandro            & -           & -           & -           & 5           & 5           & -           & 10              \\
		Carlo                 & -           & -           & -           & 5           & 5           & -           & 10              \\
		Davide                & -           & 5           & -           & -           & 5           & -           & 10              \\
		Giacomo               & -           & -           & -           & -           & 5          	& 5           & 10              \\
		Matteo                & -           & -           & -           & -           & 5           & 5           & 10              \\
		Niccolò               & 5           & -           & -           & 5           & -           & -           & 10              \\
		\hline
		\textbf{Ore totali}   & 5           & 5           & -           & 15          & 25          & 10          & 60              \\
		\textbf{Costo totale} & 150         & 100         & -           & 375         & 375         & 150         & 1150
	\end{tabular}
	\caption{Re: Responsabile, Am: Amministratore, An: Analista, Pt: Progettista,
		Pr: Programmatore, Ve: Verificatore, Totale: Totale per persona; valori espressi in ore; Costo totale espresso in euro.}
\end{table}

\subsubsection{Gestione dei ruoli}
\begin{figure}[h]
	\centering
	\begin{tikzpicture}
		\pie[text=legend]{
            10/Responsabile,
            10/Amministratore,
            0/Analista,
            30/Progettista,
            45/Programmatore,
            15/Verificatore
        }
	\end{tikzpicture}
	\caption{Grafico delle proporzioni dei ruoli ricoperti dai membri del gruppo}
\end{figure}

Per quanto riguarda il ruolo del Programmatore il gruppo ha utilizzato un numero di ore pari al 45\% di quelle disponibili, in linea con lo \textit{sprint} precedente.
Siccome in questo \textit{sprint} sono state seguite le stesse procedure adottate in quello precedente, si è scelto di non variare il numero di ore assegnate alle attività di progettazione.
Inoltre, è stata mantenuta una distribuzione equilibrata del tempo tra il Responsabile e il Verificatore (entrambi al 10\% e 15\% in relazione allo \textit{sprint} passato), il che suggerisce che la gestione del progetto e il controllo di qualità sono rimasti focali e importanti anche durante questo \textit{sprint}.
Anche in questa fase si ha una mancanza di ore assegnate al ruolo di Analista, dovuto al fatto che non c'erano attività specifiche richieste per tale ruolo.


\subsubsection{Analisi retrospettiva}
Esaminando le ore impiegate in questo \textit{sprint}, il periodo appena concluso presenta un bilancio complessivamente positivo, anche considerando il lieve ritardo nello sviluppo.
Le componenti chiave del progetto stanno venendo correttamente implementate, nonostante alcune incomprensioni con il proponente riguardo al metodo di impementazione di alcune caratteristiche inerenti le prenotazioni condivise.
A tal proposito la soluzione presentata dal gruppo è stata valutata accettabile e quindi mantenuta.\\
Il gruppo ha compreso che l'implementazione di alcune funzionalità, tra cui la \textit{chat}, richiederebbero di ritardare la data di consegna del progetto.
Non essendo possibile prolungare l'attività di sviluppo, il gruppo ha presentato la problematica al proponente, che ha accettato di ridurre il numero dei requisiti obbligatori. \\
La documentazione è in una buona posizione, con la stesura dei documenti in corso e in linea con la pianificazione. Sia il "Manuale Utente" che la "Specifica Tecnica" 
sono stati iniziati e sono in fase di completamento.

\textbf{Obiettivi raggiunti}:
\begin{itemize}
	\item Completamento gestione delle prenotazioni.
	\item Completamento gestione dell'ordinazione collaborativa dei pasti.
	\item Caricamento delle immagini da parte del ristoratore.
	\item Gestione degli ingredienti.
	\item Gestione dei piatti lato ristoratore.
	\item Stesura del "Manuale Utente".
	\item Stesura della "Specifica Tecnica".
\end{itemize}


\textbf{Obiettivi mancati}: Nessuno.

\textbf{Problematiche riscontrate}: Il tempo rimanente non permette lo sviluppo di tutti i requisiti obbligatori del progetto.

\textbf{Soluzioni attuate}: Dialogo con il proponente per valutare una riduzione dei requisiti obbligatori.