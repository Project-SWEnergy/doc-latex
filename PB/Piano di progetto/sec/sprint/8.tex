\subsection{Sprint - 8 PB}
\textbf{Inizio}: 22-04-2024 \\
\textbf{Fine}: 05-05-2024

\subsubsection{Gestione dei rischi}
\textbf{Rischi attesi verificati}:

\begin{itemize}
	\item  RP-2 Comprensione dei requisiti carente.
		\begin{itemize}
			\item \textbf{Esito mitigazione}: Il metodo di implementazione in merito ad un requisito è risultato diverso da quanto desiderato dal proponente.
				Questa problematica è stata risolta con un colloquio avvenuto con il referente aziendale il quale, dopo aver visionato lo stato del progetto, ha approvato l'implementazione proposta dal gruppo.
			\item \textbf{Impatto}: Questa problematica non ha avuto un impatto sullo sviluppo del progetto, in quanto la soluzione proposta dal gruppo è stata comunque ritenuta valida per il soddisfacimento del requisito.
		\end{itemize}
	\item RP-4 Costi e tempi imprevisti
		\begin{itemize}
			\item \textbf{Esito mitigazione}: Le attività di codifica sono proseguite in modo più telto del previsto, questo ha comportato un consumo eccessivo di ore di lavoro e all'impossibilità di soddisfare tutti i requisiti richiesti dal proponente.
			Il gruppo ha avviato un confronto con il proponente, il quale si è dichiarato favorevole ad una diminuzione del numero di requisiti.
			\item \textbf{Impatto}: Questa modifica al numero di requisiti obbligatori ha alleggerito notevolmente il carico di lavoro, permettendo al gruppo di terminare il progetto nei tempi previsti.
		\end{itemize}
\end{itemize}

\subsubsection{Diagramma di Gantt}

\begin{ganttchart}[
		x unit=0.6cm, % Adjust the width of each day
		y unit chart=0.6cm,
		bar/.style={fill=blue!50},
		bar height=0.5,
		time slot format=isodate,
		time slot unit=day,
		vgrid,
		today=2024-04-22,
		today rule/.style={draw=red, ultra thick}
	]{2024-04-22}{2024-05-05}
	\gantttitlecalendar{day} \\
	\ganttbar{Documentazione}{2024-05-01}{2024-05-05} \\
	\ganttbar{Implementazione frontend}{2024-04-22}{2024-05-05} \\
	\ganttbar{Implementazione backend}{2024-04-22}{2024-05-05} \\
\end{ganttchart}

Dove:
\begin{itemize}
	\item \textbf{"Documentazione"}: questa \textit{issue} è eseguita da Niccolò Carlesso e Davide Maffei. Per svolgere questa attività, il gruppo ha deciso di dedicare 20 ore.
	\item \textbf{"Implementazione frontend"}: questa \textit{issue} è eseguita da Giacomo Gualato e Carlo Rosso. Per svolgere questa attività, il gruppo ha deciso di dedicare 20 ore.
	\item \textbf{"Implementazione backend"}: questa \textit{issue} è eseguita da Matteo Bando e Alessandro Tigani Sava. Per svolgere questa attività, il gruppo ha deciso di dedicare 20 ore.
\end{itemize}

\subsubsection{Preventivo}

\begin{table}[H]
	\centering
	\begin{tabular}{l|r|r|r|r|r|r|r}
		\textbf{Nome}         & \textbf{Re} & \textbf{Am} & \textbf{An} & \textbf{Pt} & \textbf{Pr} & \textbf{Ve} & \textbf{Totale} \\
		\hline
		Alessandro            & -           & -           & -           & 5           & 10           & -           & 15              \\
		Carlo                 & -           & -           & -           & 5           & 10           & -           & 15              \\
		Davide                & -           & 5           & -           & -           & 10           & -           & 15              \\
		Giacomo               & -           & -           & -           & 5           & 10          	& -           & 15              \\
		Matteo                & -           & -           & -           & -           & 15           & -           & 15              \\
		Niccolò               & 5           & -           & -           & -           & -           & 5           & 10              \\
		\hline
		\textbf{Ore totali}   & 5           & 5           & -           & 15          & 55          & 5          & 85              \\
		\textbf{Costo totale} & 150         & 100         & -           & 375         & 825         & 75         & 1525
	\end{tabular}
	\caption{Re: Responsabile, Am: Amministratore, An: Analista, Pt: Progettista,
		Pr: Programmatore, Ve: Verificatore, Totale: Totale per persona; valori espressi in ore; Costo totale espresso in euro.}
\end{table}

\subsubsection{Riassunto delle attività svolte}

\begin{enumerate}
	\item \textbf{Stesura documentazione}: In questo \textit{sprint} si è continuata la stesura dei documenti richiesti per la revisione PB: "Manuale Utente", "Specifica Tecnica", "Piano di qualifica", "Piano di progetto".

	\item \textbf{Codifica}: È stata implementata la gestione delle prenotazioni, la gestione dell'ordinazione collaborativa dei pasti, il caricamento delle immagini da parte del ristoratore e la gestione degli ingredienti e dei piatti lato ristoratore.
\end{enumerate}

\subsubsection{Consuntivo}
\begin{table}[H]
	\centering
	\begin{tabular}{l|r|r|r|r|r|r|r}
		\textbf{Nome}         & \textbf{Re} & \textbf{Am} & \textbf{An} & \textbf{Pt} & \textbf{Pr} & \textbf{Ve} & \textbf{Totale} \\
		\hline
		Alessandro            & -           & -           & -           & 5           & 10           & -           & 15              \\
		Carlo                 & -           & -           & -           & 5           & 10           & -           & 15              \\
		Davide                & -           & 5           & -           & -           & 10           & -           & 15              \\
		Giacomo               & -           & -           & -           & 5           & 10          	& -           & 15              \\
		Matteo                & -           & -           & -           & -           & 15           & -           & 15              \\
		Niccolò               & 5           & -           & -           & -           & -           & 5           & 10              \\
		\hline
		\textbf{Ore totali}   & 5           & 5           & -           & 15          & 55          & 5          & 85              \\
		\textbf{Costo totale} & 150         & 100         & -           & 375         & 825         & 75         & 1525
	\end{tabular}
	\caption{Re: Responsabile, Am: Amministratore, An: Analista, Pt: Progettista,
		Pr: Programmatore, Ve: Verificatore, Totale: Totale per persona; valori espressi in ore; Costo totale espresso in euro.}
\end{table}

\subsubsection{Gestione dei ruoli}
\begin{figure}[h]
	\centering
	\begin{tikzpicture}
		\pie[text=legend]{
            6/Responsabile,
            6/Amministratore,
            0/Analista,
            17/Progettista,
            65/Programmatore,
            6/Verificatore
        }
	\end{tikzpicture}
	\caption{Grafico delle proporzioni dei ruoli ricoperti dai membri del gruppo}
\end{figure}

Durante questo \textit{sprint} le ore del ruolo Programmatore sono lievemente aumentate, essendo la codifica ancora l'attività principale di questo periodo, ma nel complesso rimangono coerenti con quelle dello \textit{sprint} precedente e con le attività che il gruppo si era prefissato per questo periodo.\\ 
Per quanto riguarda invece Responsabile, Amministratore e Progettista non ci sono state variazioni con i periodi precedenti, le ore si attestano anche in questo \textit{sprint} intorno al 6\% per Responsabile e Amministratore e al 17\% per il ruolo di Progettista. 
Il ruolo del Verificatore invece è lievemente calato in questo \textit{sprint} portandosi al 6\%, questo per necessità di concentrarsi maggiormente sulla codifica. Ci aspettiamo comunque che nello \textit{sprint} successivo le ore per tale ruolo saliranno nuovamente per la conseguente verifica delle attività svolte in questo periodo.


\subsubsection{Analisi retrospettiva}
Esaminando le ore impiegate in questo \textit{sprint}, il periodo appena concluso presenta un bilancio complessivamente positivo, anche considerando il lieve ritardo nello sviluppo.
Le componenti chiave del progetto stanno venendo correttamente implementate, nonostante alcune incomprensioni con il proponente riguardo al metodo di impementazione di alcune caratteristiche inerenti le prenotazioni condivise.
A tal proposito la soluzione presentata dal gruppo è stata valutata accettabile e quindi mantenuta.\\
Il gruppo ha compreso che l'implementazione di alcune funzionalità, tra cui la \textit{chat}, richiederebbero di ritardare la data di consegna del progetto.
Non essendo possibile prolungare l'attività di sviluppo, il gruppo ha presentato la problematica al proponente, che ha accettato di ridurre il numero dei requisiti obbligatori. \\
La documentazione è in una buona posizione, con la stesura dei documenti in corso e in linea con la pianificazione. Sia il "Manuale Utente" che la "Specifica Tecnica" 
sono stati iniziati e sono in fase di completamento.

\textbf{Obiettivi raggiunti}:
\begin{itemize}
	\item Completamento gestione delle prenotazioni.
	\item Completamento gestione dell'ordinazione collaborativa dei pasti.
	\item Caricamento delle immagini da parte del ristoratore.
	\item Gestione degli ingredienti.
	\item Gestione dei piatti lato ristoratore.
	\item Stesura del "Manuale Utente".
	\item Stesura della "Specifica Tecnica".
\end{itemize}


\textbf{Obiettivi mancati}: Nessuno.

\textbf{Problematiche riscontrate}: 
\begin{itemize}
	\item Il tempo rimanente non permette lo sviluppo di tutti i requisiti obbligatori del progetto.
\end{itemize}

\textbf{Soluzioni attuate}: \begin{itemize}
	\item Dialogo con il proponente per valutare una riduzione dei requisiti obbligatori.
\end{itemize}