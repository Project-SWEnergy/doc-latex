\subsection{Sprint - 9 PB}
\textbf{Inizio}: 06-05-2024 \\
\textbf{Fine}: 19-05-2024 \\

\subsubsection{Gestione dei rischi}
\textbf{Rischi attesi verificati}:

\begin{itemize}
	\item Nessuno.
\end{itemize}

\subsubsection{Diagramma di Gantt}

\begin{ganttchart}[
		x unit=0.6cm, % Adjust the width of each day
		y unit chart=0.6cm,
		bar/.style={fill=blue!50},
		bar height=0.5,
		time slot format=isodate,
		time slot unit=day,
		vgrid,
		today=2024-04-22,
		today rule/.style={draw=red, ultra thick}
	]{2024-04-22}{2024-05-05}
	\gantttitlecalendar{day} \\
	\ganttbar{Documentazione}{2024-05-01}{2024-05-05} \\
	\ganttbar{Verifica frontend}{2024-04-22}{2024-05-05} \\
	\ganttbar{Verifica backend}{2024-04-22}{2024-05-05} \\
\end{ganttchart}

Dove:
\begin{itemize}
	\item \textbf{"Documentazione"}: questa \textit{issue} è eseguita da Niccolò Carlesso e Davide Maffei. Per svolgere questa attività, il gruppo ha deciso di dedicare 15 ore.
	\item \textbf{"Verifica frontend"}: questa \textit{issue} è eseguita da Giacomo Gualato e Carlo Rosso. Per svolgere questa attività, il gruppo ha deciso di dedicare 10 ore.
	\item \textbf{"Verifica backend"}: questa \textit{issue} è eseguita da Matteo Bando e Alessandro Tigani Sava. Per svolgere questa attività, il gruppo ha deciso di dedicare 10 ore.
\end{itemize}

\subsubsection{Preventivo}

\begin{table}[H]
	\centering
	\begin{tabular}{l|r|r|r|r|r|r|r}
		\textbf{Nome}         & \textbf{Re} & \textbf{Am} & \textbf{An} & \textbf{Pt} & \textbf{Pr} & \textbf{Ve} & \textbf{Totale} \\
		\hline
		Alessandro            & -           & -           & -           & -           & -           & 5           & 5               \\
		Carlo                 & 5           & -           & -           & -           & -           & -           & 5               \\
		Davide                & -           & -           & -           & -           & -           & 5           & 5               \\
		Giacomo               & -           & 5           & -           & -           & 5           & -           & 10              \\
		Matteo                & -           & -           & -           & -           & -           & 5           & 5               \\
		Niccolò               & -           & -           & -           & -           & -           & 5           & 5               \\
		\hline
		\textbf{Ore totali}   & 5           & 5           & -           & -           & 5           & 20          & 35              \\
		\textbf{Costo totale} & 150         & 100         & -           & -           & 75          & 300         & 625
	\end{tabular}
	\caption{Re: Responsabile, Am: Amministratore, An: Analista, Pt: Progettista,
		Pr: Programmatore, Ve: Verificatore, Totale: Totale per persona; valori espressi in ore; Costo totale espresso in euro.}
\end{table}

\subsubsection{Riassunto delle attività svolte}

\begin{enumerate}
	\item \textbf{Stesura documentazione}: In questo \textit{sprint} si è continuata la stesura dei documenti richiesti per la revisione PB: "Manuale Utente", "Specifica Tecnica", "Piano di qualifica", "Piano di progetto".

	\item \textbf{Codifica}: Si è completata l'implementazione delle ultime funzionalità dell'applicativo, sono state poi completate le attività di verifica del codice tramite la realizzazione di test automatizzati.
\end{enumerate}

\subsubsection{Consuntivo}
\begin{table}[H]
	\centering
	\begin{tabular}{l|r|r|r|r|r|r|r}
		\textbf{Nome}         & \textbf{Re} & \textbf{Am} & \textbf{An} & \textbf{Pt} & \textbf{Pr} & \textbf{Ve} & \textbf{Totale} \\
		\hline
		Alessandro            & -           & -           & -           & -           & -           & 5           & 5               \\
		Carlo                 & 5           & -           & -           & -           & -           & -           & 5               \\
		Davide                & -           & -           & -           & -           & -           & 5           & 5               \\
		Giacomo               & -           & 5           & -           & -           & 5           & -           & 10              \\
		Matteo                & -           & -           & -           & -           & -           & 5           & 5               \\
		Niccolò               & -           & -           & -           & -           & -           & 5           & 5               \\
		\hline
		\textbf{Ore totali}   & 5           & 5           & -           & -           & 5           & 20          & 35              \\
		\textbf{Costo totale} & 150         & 100         & -           & -           & 75          & 300         & 625
	\end{tabular}
	\caption{Re: Responsabile, Am: Amministratore, An: Analista, Pt: Progettista,
		Pr: Programmatore, Ve: Verificatore, Totale: Totale per persona; valori espressi in ore; Costo totale espresso in euro.}
\end{table}

\subsubsection{Gestione dei ruoli}
\begin{figure}[h]
	\centering
	\begin{tikzpicture}
		\pie[text=legend]{
			14/Responsabile,
			14/Amministratore,
			0/Analista,
			0/Progettista,
			14/Programmatore,
			58/Verificatore
		}
	\end{tikzpicture}
	\caption{Grafico delle proporzioni dei ruoli ricoperti dai membri del gruppo}
\end{figure}

Durante questo \textit{sprint} le ore di lavoro sono calate, essendo arrivati al limite previsto di 95 ore complessive.\\
Le attività di Responsabile ed Amministratore hanno ricevuto lo stesso quantitativo di ore rispetto agli \textit{sprint} precedenti.\\
Sono state ridotte le ore per il ruolo di Programmatore fino ad essere il 14\% del totale, questo perchè i requisiti concordati con il proponente sono stati soddisfatti.\\
Le ore assegnate al ruolo di Verificatore sono aumentate fino a raggiungere il 58\% del totale.
Le attività si sono concentrate sulla verifica del codice tramite creazione di test di unità automatizzati e test di sistema, oltre alla verifica della documentazione prodotta.
Non è stato necessario assegnare ore ai ruoli di Analista e Progettista.

\subsubsection{Analisi retrospettiva}
Esaminando le ore impiegate in questo \textit{sprint}, il periodo appena concluso presenta un bilancio positivo.
Le componenti del progetto mirate al soddisfacimento dei requisiti obbligatori sono state correttamente implementate e sono state eseguite le attività di verifica necessarie.
Il gruppo rimane ora in attesa della presentazione del MVP al referente aziendale e, in caso di esito positivo, della presentazione con i docenti per temrinare la fase PB.
Sono stati completati, e sono in procinto di essere approvati, i documenti:
\begin{itemize}
	\item Manuale Utente;
	\item Specifiche Tecniche;
	\item Analisi dei requisiti;
	\item Norme di progetto;
\end{itemize}


\textbf{Obiettivi raggiunti}:
\begin{itemize}
	\item Soddisfacimento dei requisiti obbligatori;
	\item Verifica del codice;
	\item Stesura del documento "Manuale Utente".
	\item Stesura del documento "Specifiche Tecniche".
	\item Approvazione del documento "Analisi dei requisiti".
	\item Approvazione del documento "Norme di progetto".
\end{itemize}


\textbf{Obiettivi mancati}: Nessuno.

\textbf{Problematiche riscontrate}: Nessuno.

\textbf{Soluzioni attuate}: nessuno.