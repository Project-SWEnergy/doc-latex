\section{Qualità di prodotto}
La qualità di prodotto viene garantita dall'adozione dello standard \textbf{ISO/IEC 9126} che propone caratteristiche di qualità misurabili attraverso metriche ben definite.

\subsection{Obiettivi}
Si effettua una suddivisione in:
\begin{itemize}
    \item \textbf{Qualità interna}: fa riferimento alla qualità della documentazione interna prodotta.
    \item \textbf{Qualità esterna}: fa riferimento alla qualità del software rilasciato.
\end{itemize}

\subsubsection{Qualità interna}
Volge al raggiungimento dei seguenti obiettivi:
\begin{itemize}
    \item \textbf{Usabilità}: comprensibilità e leggibilità della documentazione prodotta.
\end{itemize}
\begin{table}[H]
    \centering
    \begin{tabularx}{\textwidth}{p{3.5cm}|X|l|l}
        \hline
        \textbf{Metrica} & \textbf{Codice}   & \textbf{Valore ottimale}  & \textbf{Valore accettabile}   \\
        \hline
        \multicolumn{4}{c}{\textbf{Usabilità}} \\
        \hline
        Indice di Gulpease & MPD1 &  $\ge 70\%$ & $\ge 60\%$    \\
        \hline
    \end{tabularx}
    \caption{Valori di riferimento per le metriche di qualità interna}
\end{table}

\subsubsection{Qualità esterna}
Volge al raggiungimento dei seguenti obiettivi:
\begin{itemize}
    \item \textbf{Funzionalità}: il prodotto deve soddisfare i requisiti stabiliti in fase di analisi.
    \item \textbf{Affidabilità}: sotto determinate condizioni il prodotto deve mantenere un determinato livello di prestazioni per un periodo di tempo prestabilito.
    \item \textbf{Efficienza}: il prodotto deve raggiungere l'obiettivo minimizzando l'utilizzo delle risorse .
    \item \textbf{Usabilità}: l'utente deve essere in grado di utilizzare il prodotto e comprenderne le sue funzioni.
    \item \textbf{Manutenibilità}: il prodotto deve poter essere modificato nel tempo senza che gli aggiornamenti ne compromettano le funzionalità.
    \item \textbf{Portabilità}: il prodotto deve essere fruibile in ambienti diversi
\end{itemize}
\begin{table}[H]
    \centering
    \begin{tabularx}{\textwidth}{p{5.5cm}|X|l|l}
        \hline
        \textbf{Metrica} & \textbf{Codice}   & \textbf{Valore ottimale}  & \textbf{Valore accettabile}   \\
        \hline
        \multicolumn{4}{c}{\textbf{Funzionalità}} \\
        \hline
        Copertura dei requisiti & MPD2 &  $100\%$ & $80\%$    \\
        \hline
        \multicolumn{4}{c}{\textbf{Affidabilità}} \\
        \hline
        Densità di failure & MPD3 & $\le 10\%$ & $\le 20\% $ \\
        \hline
        \multicolumn{4}{c}{\textbf{Efficienza}} \\
        \hline
        Tempo medio di risposta & MPD4 & 2 secondi & 3 secondi \\
        \hline
        \multicolumn{4}{c}{\textbf{Usabilità}} \\
        \hline
        Average cyclomatic complexity & MPD5 & 10 & 20 \\
        Facilità di utilizzo & MPD6 & 5 click & $\le 10$ click \\
        Facilità di apprendimento delle funzionalità & MPD7 & $\le 5$ minuti & $\le 15$ minuti \\
        \hline
        \multicolumn{4}{c}{\textbf{Manutenibilità}} \\
        \hline
        Comprensibilità del codice & MPD8 & $\ge 80\%$ & $\ge 70\%$ \\
        \hline
        \multicolumn{4}{c}{\textbf{Portabilità}} \\
        \hline
        Browser supportati & MPD9 & $100\%$ & $\ge 80\%$ \\
        \hline
    \end{tabularx}
    \caption{Valori di riferimento per le metriche di qualità esterna}
\end{table}