\section{Architettura di Deployment}

\subsection{\textit{Frontend}}

\subsection{\textit{Backend}}

Per quanto riguarda il \textit{deployment} del \textit{backend}, il \textit{team}
SWEnergy ha optato per un'architettura monolitica. Questa scelta è stata
determinata dalle dimensioni contenute del sistema e dalla mancanza di esigenze
di scalabilità e manutenibilità che richiederebbero l'adozione di
un'architettura a microservizi. Nonostante l'architettura monolitica possa
aumentare le dipendenze del \textit{backend} e rendere più complessa
l'individuazione dei problemi, l'uso della \textit{dependency injection} fornita
da Nest.js consente di mantenere il codice ben strutturato e modulare,
facilitando l'individuazione di eventuali dipendenze indesiderate attraverso
l'uso di \textit{software} di analisi statica.

Per garantire una maggiore manutenibilità e scalabilità, il \textit{team} ha deciso di
sviluppare una libreria interna contenente i moduli condivisi tra i vari
componenti del \textit{backend}, mentre ogni funzionalità del sistema è
sviluppata in moduli separati. Questo approccio permette di mantenere il codice
più organizzato e facilita eventuali future estensioni o modifiche.

L'architettura monolitica è stata preferita poiché il sistema richiede
principalmente l'implementazione di API REST per l'interazione col \textit{database}.
Questo pone il \textit{database} come componente fondamentale del sistema, evidenziando
la dipendenza tra il \textit{database} e il \textit{backend}. Inoltre, l'interfaccia
fornita dal \textit{backend} semplifica l'implementazione del \textit{frontend},
fornendo le operazioni di interazione con il \textit{database} e i dati necessari al
\textit{frontend} senza esporre la struttura del \textit{database} o le operazioni di
implementazione.

L'architettura scelta contribuisce a ridurre i tempi e i costi di
sviluppo, poiché riduce il numero di interfacce da implementare e testare,
riducendo così la complessità complessiva del sistema.

Infine, SWEnergy utilizza Docker per il \textit{deployment} del
\textit{backend}, in questo modo è possibile garantire la portabilità del
sistema, ma soprattutto la scalabilità orizzontale, in quanto è possibile creare
più istanze del \textit{backend} e bilanciare il carico tra di esse.
