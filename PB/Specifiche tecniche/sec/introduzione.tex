\section{Introduzione}

\subsection{Scopo del documento}

Il presente documento, intitolato "Specifiche Tecniche", espone e approfondisce
le scelte architetturali e di design adottate dal gruppo SWEnergy per lo
sviluppo del progetto "Easy Meal", proposto dall'azienda \href{https://imolainformatica.it}{Imola Informatica} 
(ultimo accesso 25/03/2024). In particolare, il documento mira a giustificare le
scelte implementate al fine di consentire agli sviluppatori di comprendere e
mantenere il codice in modo efficace ed efficiente. Inoltre, è inclusa una
sezione dedicata ai requisiti implementati per garantire che il prodotto
soddisfi le richieste del proponente.

\subsection{Organizzazione del documento}

Il documento è strutturato in 4 sezioni principali:

\begin{itemize}
	\item \textbf{Tecnologie adottate}: si compone di una descrizione di
	      ciascuna tecnolgia adottata per lo sviluppo del progetto.
	      Le tecnologie sono divise in tre categorie: di codifica,
	      di analisi statica e di analisi dinamica. Alla fine della sezione è
	      presente una tabella riassuntiva;

	\item \textbf{Architettura di deployment}: illustra l'architettura di
	      \textit{deployment} e ne motiva la scelta;

	\item \textbf{Architettura implementativa}: espone le scelte implementative
	      adottate per soddisfare i requisiti richiesti. Inoltre, sono spiegati
	      i \textit{desing pattern} utilizzati e le motivazioni che hanno
	      portato alla loro adozione;

	\item \textbf{Requisiti}: sono riportati i requisiti concordati con il
	      proponente e descritti in dettaglio nel documento "Analisi dei
	      Requisiti", nella versione 2.0.0. Per ciascun requisito è indicato lo
	      stato di implementazione. Infine, è presente un grafico riassuntivo
	      dello stato di avanzamento dei requisiti.
\end{itemize}


\subsection{Scopo del prodotto}

"\textit{Easy Meal}" è una \textit{web app}\g progettata per gestire le
prenotazioni presso i ristoranti, sia dal lato dei clienti che dei ristoratori.
Il prodotto finale sarà composto da due parti:

\begin{itemize}
	\item \textbf{Cliente\g}: consente ai clienti di prenotare un tavolo presso un
	      ristorante, visualizzare il menù e effettuare un ordine\g;

	\item \textbf{Ristoratore}: consente ai ristoratori di gestire le
	      prenotazioni e gli ordini dei clienti, oltre a visualizzare la lista
	      degli ingredienti necessari per preparare i piatti ordinati.
\end{itemize}

\subsection{Glossario}

Al fine di prevenire ambiguità linguistiche e garantire una coerenza nell'utilizzo
delle terminologie attraverso i documenti, il \textit{team} ha compilato un documento
interno denominato "Glossario".
Questo documento fornisce definizioni chiare e precise per i termini che potrebbero
risultare ambigui o generare incomprensioni nel testo principale.
I termini inclusi nel Glossario sono facilmente identificabili grazie a un apice 'G'
(ad esempio, parola\g).
Questa pratica agevola la consultazione del Glossario per una comprensione approfondita
dei termini tecnici o specifici utilizzati nel contesto del progetto.

\subsection{Riferimenti}

\subsubsection{Normativi}
\begin{itemize}
	\item "Norme di progetto";
	\item 	\href{https://www.math.unipd.it/~tullio/IS-1/2023/Progetto/C3.pdf}
	      {Documento del capitolato d'appalto C3 - \textit{Easy Meal}} (ultimo accesso 20/03/2024);
	\item \href{https://www.math.unipd.it/~tullio/IS-1/2023/Dispense/PD2.pdf}
	      {Regolamento del progetto} (ultimo accesso 25/03/2024);
\end{itemize}

\subsubsection{Informativi}
\begin{itemize}
	\item Glossario v2.0.0;
	\item Analisi dei Requisiti v2.0.0;
	\item \href{https://www.math.unipd.it/~rcardin/swea/2023/Diagrammi\%20delle\%20Classi.pdf}
	      {Linguaggio UML - Diagramma delle classi} (ultimo accesso 4/03/2024);
\end{itemize}

\subsubsection{Tecnologici}
\begin{itemize}
	\item \href{https://www.typescriptlang.org/}
	      {Typescript} (ultimo accesso 03/04/2024);

	\item \href{https://angular.io/}
	      {Angular} (ultimo accesso 03/04/2024);

	\item \href{https://axios-http.com/}
	      {Axios} (ultimo accesso 03/04/2024);

	\item \href{https://nestjs.com/}
	      {NestJS} (ultimo accesso 03/04/2024);

	\item \href{https://orm.drizzle.team/}
	      {Drizzle} (ultimo accesso 03/04/2024);

	\item \href{https://www.docker.com/}
	      {Docker} (ultimo accesso 03/04/2024);

	\item \href{https://insomnia.rest/}
	      {Insomnia} (ultimo accesso 03/04/2024);

	\item \href{https://tailwindcss.com/}
	      {Tailwind CSS} (ultimo accesso 03/04/2024);
\end{itemize}

