\section{Requisiti}

\subsection{Funzionali}

Di seguito viene riportata la specifica relativa ai requisiti funzionali, che delineano le funzionalità del Sistema, le azioni eseguibili
da parte del Sistema e le informazioni che il Sistema può fornire. La presenza di ogni requisito viene giustificata riportando la fonte, che può essere un UC oppure presente
nel testo del capitolato d'appalto. Mentre i codici univoci sottostanti indicano:
\begin{enumerate}
	\item RFO: Requisito Funzionale Obbligatorio;
	\item RFF: Requisito Funzionale Facoltativo;
	\item RFD: Requisito Funzionale Desiderabile.
\end{enumerate}


\begin{longtable}{|l|p{0.65\textwidth}|p{3cm}|}
	\hline
	\textbf{ID} & \textbf{Descrizione}                                                                                                    & \textbf{Stato}  \\
	\endfirsthead
	\hline
	\textbf{ID} & \textbf{Descrizione}                                                                                                    & \textbf{Stato}  \\
	\hline
	\endhead
	\hline
	RFO1        & L'Utente generico e L'Utente base devono poter visualizzare l'elenco dei ristoranti disponibili.                        & Soddisfatto \\
	\hline
	RFO2        & L'Utente generico e L'Utente base devono poter ricercare un ristorante attraverso il nome.               				  & Soddisfatto \\
	\hline
	RFO3        & L'Utente generico e L'Utente base devono poter visualizzare un ristorante.                                              & Soddisfatto \\
	\hline
	RFD4        & L'Utente generico e L'Utente base devono poter condividere un \textit{link} di un ristorante.                           & Non soddisfatto \\
	\hline
	RFD5        & L'Utente generico e L'Utente base devono poter visualizzare la pagina delle  \textit{FAQ\g}.                            & Non soddisfatto \\
	\hline
	RFO6        & L'Utente generico deve poter effettuare l'accesso al Sistema.                                                           & Soddisfatto \\
	\hline
	RFO7        & L'Utente generico deve poter effettuare la registrazione al Sistema come Utente base o Utente ristoratore.              & Soddisfatto \\
	\hline
	RFO8        & L'Utente generico deve visualizzare un messaggio d'errore se l'accesso fallisce.                                        & Non soddisfatto \\
	\hline
	RFO9        & L'Utente generico deve visualizzare un messaggio d'errore se la registrazione fallisce.                                 & Non soddisfatto \\
	\hline
	RFD10       & L'Utente base deve poter visualizzare i suoi dati utente.                                                               & Non soddisfatto \\
	\hline
	RFD11       & L'Utente base deve poter modificare i suoi dati utente.                                                                 & Non soddisfatto \\
	\hline
	RFD12       & L'Utente base deve poter visualizzare lo storico dei suoi ordini.                                                       & Non soddisfatto \\
	\hline
	RFO13       & L'Utente base deve poter visualizzare la lista delle sue prenotazioni, ed in caso andare in dettaglio.                  & Non soddisfatto \\
	\hline
	RFO14       & L'Utente base deve poter visualizzare la notifica dello stato della sua prenotazione.                                   & Non soddisfatto \\
	\hline
	RFD15       & L'Utente base deve poter elimare il proprio \textit{account}.                                                           & Non soddisfatto \\
	\hline
	RFO16       & L'Utente base deve poter prenotare un tavolo.                                                                           & Non soddisfatto \\
	\hline
	RFO17       & L'Utente base deve poter condividere la prenotazione.                                                                   & Non soddisfatto \\
	\hline
	RFO18       & L'Utente base deve poter annullare la prenotazione.                                                                     & Non soddisfatto \\
	\hline
	RFO19       & L'Utente base deve poter accedere ad una prenotazione mediante \textit{link} di condivisione                            & Non soddisfatto \\
	\hline
	RFO20       & L'Utente base deve poter annullare il proprio ordine.                                                                   & Non soddisfatto \\
	\hline
	RFO21       & L'Utente base deve poter creare un ordinazione collaborativa dei pasti.                                                 & Non soddisfatto \\
	\hline
	RFO22       & L'Utente base deve poter annullare la propria ordinazione.                                                              & Non soddisfatto \\
	\hline
	RFO23       & L'Utente base deve poter dividere il conto in maniera equa oppure proporzionale.                                        & Non soddisfatto \\
	\hline
	RFO24       & L'Utente base deve poter visualizzare il messaggio d'errore che la divisione del conto è stata già effettuata.          & Non soddisfatto \\
	\hline
	RFO25       & L'Utente base deve poter pagare il conto.                                                                               & Non soddisfatto \\
	\hline
	RFO26       & L'Utente base deve poter visualizzare l'errore relativo al pagamento fallito.                                           & Non soddisfatto \\
	\hline
	RFO27       & L'Utente base deve poter inserire \textit{feedback} e recensioni.                                                       & Non soddisfatto \\
	\hline
	RFO28       & L'Utente base deve poter visualizzare la notifica di richiesta di inserimento \textit{feedback}.                        & Non soddisfatto \\
	\hline
	RFO29       & L'Utente base deve poter visualizzare la notifica relativa alla modifica della sua ordinazione.                         & Non soddisfatto \\
	\hline
	RFD30       & L'Utente base deve poter visualizzare la notifica relativa al suo \textit{feedback} che ha ricevuto una risposta.       & Non soddisfatto \\
	\hline
	RFD31       & L'Utente base deve poter inserire e modificare le proprie allergie.                                                     & Non soddisfatto \\
	\hline
	RFD32       & L'Utente base deve poter visualizzare un messaggio se seleziona un piatto di cui è allergico.                        & Non soddisfatto \\
	\hline
	RFO33       & L'Utente base deve poter visualizzare il menù di un ristorante.                                                      & Non soddisfatto \\
	\hline
	RFD34       & L'Utente autenticato deve poter effettuare il \textit{logout}.                                                       & Non soddisfatto \\
	\hline
	RFO35       & L'Utente autenticato deve poter comunicare attraverso la \textit{chat}.                                              & Non soddisfatto \\
	\hline
	RFD36       & L'Utente autenticato deve poter visualizzare la notifica relativa all'arrivo di un nuovo messaggio in \textit{chat}. & Non soddisfatto \\
	\hline
	RFO37       & L'Utente ristoratore deve poter visualizzare la notifica relativa ad una nuova prenotazione.                         & Non soddisfatto \\
	\hline
	RFD38       & L'Utente ristoratore deve poter visualizzare la notifica relativa ad un nuovo ordine.                                & Non soddisfatto \\
	\hline
	RFO39       & L'Utente ristoratore deve poter visualizzare la notifica relativa all'avvenuto pagamento.                            & Non soddisfatto \\
	\hline
	RFD40       & L'Utente ristoratore deve poter visualizzare la notifica relativa all'inserimento di un \textit{feedback}.           & Non soddisfatto \\
	\hline
	RFO41       & L'Utente ristoratore deve poter visualizzare la lista delle prenotazioni in dettaglio e con la lista degli ingredienti. & Non soddisfatto \\
	\hline
	RFO42       & L'Utente ristoratore deve poter accettare una prenotazione.                                                             & Non soddisfatto \\
	\hline
	RFO43       & L'Utente ristoratore deve poter rifiutare una prenotazione.                                                             & Non soddisfatto \\
	\hline
	RFO44       & L'Utente ristoratore deve poter terminare una prenotazione.                                                             & Non soddisfatto \\
	\hline
	RFO45       & L'Utente ristoratore deve poter visualizzare la lista delle ordinazioni.                                                & Non soddisfatto \\
	\hline
	RFD46       & L'Utente ristoratore deve poter modificare un ordinazione.                                                              & Non soddisfatto \\
	\hline
	RFO47       & L'Utente ristoratore deve poter visualizzare lo stato di pagamento di una prenotazione.                                 & Non soddisfatto \\
	\hline
	RFO48       & L'Utente ristoratore deve poter visualizzare la lista dei \textit{feedback}.                                            & Non soddisfatto \\
	\hline
	RFD49       & L'Utente ristoratore deve poter segnalare un \textit{feedback}.                                                         & Non soddisfatto \\
	\hline
	RFD50       & L'Utente ristoratore deve poter rispondere ad un \textit{feedback}.                                                     & Non soddisfatto \\
	\hline
	RFD51       & L'Utente ristoratore deve poter modificare le informazioni del suo ristorante.                                          & Non soddisfatto \\
	\hline
	RFO52       & L'Utente ristoratore deve poter gestire il menù, inserendo, eliminando e modificando dei piatti.                        & Soddisfatto \\
	\hline
	RFO53       & L'Utente ristoratore deve poter gestire gli ingredienti, inserendo e eliminando degli ingredienti.                      & Soddisfatto \\
	\hline
	RFO54       & L'Utente ristoratore deve poter assegnare gli ingredienti ad un piatto.                                                 & Non soddisfatto \\
	\hline
	RFO55       & L'Utente ristoratore deve poter visualizzare la notifica relativa all'annullamento di un ordinazione.                   & Non soddisfatto \\
	\hline
	RFO56       & L'Utente ristoratore deve poter visualizzare la notifica relativa all'annullamento di una prenotazione.                 & Non soddisfatto \\
	\hline
	RFF57       & L'Utente generico deve poter effettuare l'accesso al Sistema attraverso un sistema di terze parti.                 	& Non soddisfatto \\
	\hline
	RFD58	   & L'Utente generico e L'Utente base devono poter ricercare un ristorante attraverso luogo e filtri.  					& Non soddisfatto \\
	\hline
\end{longtable}
