\section{Setup}

In questa sezione viene approfondito il metodo per eseguire l'applicativo
sviluppato. In particolare, si descrive come configurare l'ambiente di
esecuzione e come eseguire il programma; si illustra come eseguire i test e come
configurare l'ambiente di sviluppo.

\subsection{Installazione}

\subsubsection{Pre-requisiti}

Per eseguire l'applicativo è necessario avere installato docker nel proprio
sistema. Per installare docker, si può fare riferimento alla documentazione
ufficiale di docker:
\url{https://www.docker.com/products/docker-desktop/}[16/05/2024].
Infine, è necessario avere installato git per clonare il repository.

\subsubsection{Prima esecuzione}

Di seguito sono riportati i passi da seguire per avviare Easy-Meal:
\begin{enumerate}
	\item Clonare il repository: \\
		\texttt{git clone https://github.com/Project-SWEnergy/Easy-Meal.git}

	\item Entrare nella cartella del progetto: \\
		\texttt{cd Easy-Meal}

	\item Avviare il daemon di docker;

	\item Eseguire la build del progetto e avviare i container: \\
		\texttt{docker-compose up}

	\item Aprire il seguente indirizzo nel browser: \\
		\texttt{http://localhost:8000}

	\item Eseguire il login con le seguenti credenziali:
		\begin{itemize}
			\item \textbf{System}: PostgreSQL
			\item \textbf{Server}: db
			\item \textbf{Username}: postgres
			\item \textbf{Password}: postgres
			\item \textbf{Database}: easymeal
		\end{itemize}

	\item Selezionare la voce di menù \texttt{Import};

	\item Cliccare sul pulsante \texttt{Choose file} e selezionare il file
		\texttt{backend/drizzle/0000\_gorgeous\_natasha\_romanoff.sql} presente 
		nella cartella del progetto;

	\item Cliccare sul pulsante \texttt{Execute} per eseguire la query di
		crezione delle tabelle, viene visualizzato un messaggio di successo;

	\item Cliccare sul pulsante \texttt{Choose file} e selezionare il file
		\texttt{backend/dump/populate.sql} presente nella cartella del 
		progetto;

	\item Cliccare sul pulsante \texttt{Execute} per popolare le tabelle, viene
		visualizzato un messaggio di successo;

	\item Aprire il seguente indirizzo nel browser: \\ 
		\texttt{http://localhost:4200}
\end{enumerate}

Dopo aver seguito i passi sopra descritti, l'applicativo Easy-Meal è pronto per
essere utilizzato. In effetti, l'ultimo passo apre il sito web dell'applicativo.

\subsection{Esecuzione}

Se l'applicativo è già stato avviato in precedenza, è possibile riavviarlo
seguendo i seguenti passi:

\begin{enumerate}
	\item Avviare il daemon di docker;

	\item Eseguire la build del progetto e avviare i container: \\
		\texttt{docker-compose up}

	\item Aprire il seguente indirizzo nel browser: \\
		\texttt{http://localhost:4200}
\end{enumerate}

\subsection{Stop}

Per arrestare l'applicativo, è sufficiente eseguire il comando
\texttt{docker-compose down} nella cartella del progetto.

\subsection{Test}

Sono configurati due sistemi di test indipendenti: i test del \textit{backend} e
i test del \textit{frontend}.

\subsubsection{\textit{Backend}}

!TODO

\subsubsection{\textit{Frontend}} 

Per eseguire i test del \textit{frontend} è necessario avere installato Node.js 
e quindi anche npm. Per installare Node.js, si può fare riferimento alla
documentazione ufficiale di Node.js:
\url{https://nodejs.org/en/download/package-manager}[16/05/2024]. 
Di seguito sono riportati i passi da seguire per avviare i test:

\begin{enumerate}
	\item Entrare nella cartella del progetto: \\
		\texttt{cd Easy-Meal\textbackslash frontend} (su Windows) \\
		\texttt{cd Easy-Meal/frontend} (su Unix)

	\item Eseguire il comando: \\
		\texttt{npm install}

	\item Eseguire il comando: \\
		\texttt{ng test}

	\item Aprire il seguente indirizzo nel browser: \\
		\texttt{http://localhost:9876}

	\item Cliccare sul pulsante \texttt{DEBUG} per eseguire i test.
\end{enumerate}

Non solo, Angular mette a disposizione un servizio per visualizzare la copertura
dei test, il codice e la qualità del codice. Per avviare il servizio di
visualizzazione della copertura dei test, si seguano i seguenti passi:
\begin{enumerate}
	\item Entrare nella cartella del progetto: \\
		\texttt{cd Easy-Meal\textbackslash frontend} (su Windows) \\
		\texttt{cd Easy-Meal/frontend} (su Unix)

	\item Eseguire il comando: \\
		\texttt{npm install}

	\item Eseguire il comando: \\
		\texttt{ng test ----code-coverage=true}

	\item Aprire il seguente indirizzo nel browser: \\
		\texttt{http://localhost:9876}

	\item Cliccare sul pulsante \texttt{DEBUG} per eseguire i test;\\

	\item Aprire il seguente file con il browser: \\
		\texttt{coverage/po-c-frontend/index.html}
\end{enumerate}

\subsection{Configurazione dell'ambiente di sviluppo}


\subsubsection{Backend}

!TODO

\subsubsection{Frontend}

Come per i test, anche in questo caso è necessario avere installato Node.js e
npm.
Di seguito sono riportati i passi da seguire per configurare l'ambiente di
sviluppo del \textit{frontend}:

\begin{enumerate}
	\item Entrare nella cartella del progetto: \\
		\texttt{cd Easy-Meal}

	\item Avviare i container: \\
		\texttt{docker-compose up}

	\item Aprire la \textit{dashboard} di docker e bloccare il container
		\texttt{easymeal-frontend-<N>}.

	\item Installare le dipendenze: \\
		\texttt{npm install}

	\item Avviare il server di sviluppo: \\
		\texttt{npm run start}

	\item Aprire il seguente indirizzo nel browser: \\
		\texttt{http://localhost:4200}
\end{enumerate}

In questo modo è possibile sviluppare il \textit{frontend} senza dover eseguire
la \textit{build} del container del \textit{frontend} ad ogni modifica, invece
Angular ricompila automaticamente il \textit{frontend} ad ogni modifica.
Infatti, ogni modifica di qualche file del \textit{frontend} causa un 
aggiornamento della pagina web.
