\section{Tecnologie adottate}

\subsection{Tecnologie per la codifica}

\subsubsection{TypeScript}

I membri del team SWEnergy, prima di iniziare il progetto, possedevano una
solida conoscenza di C++. Alcuni di loro avevano esperienza anche con Java,
Python e JavaScript. Il committente ha illustrato che Imola Informatica fa un
ampio uso di TypeScript e Java per lo sviluppo software, sottolineando inoltre
che per creare un'applicazione web, specialmente per quanto riguarda il
frontend, è necessario utilizzare JavaScript o TypeScript. Partendo da queste
considerazioni, SWEnergy ha optato per l'uso di JavaScript come linguaggio
principale per lo sviluppo del progetto, poiché consente di scrivere codice sia
per il frontend che per il backend. Infine, è stata scelta TypeScript, un
superset di JavaScript, per la sua capacità di aggiungere tipi statici al
linguaggio, migliorando la robustezza, la manutenibilità e la leggibilità del
codice. Attraverso l'analisi statica del codice, è possibile individuare errori
di programmazione direttamente durante la fase di sviluppo, evitando che si
manifestino durante l'esecuzione, il che migliora la qualità del software e
riduce il tempo di sviluppo.

\subsubsection{Angular}

Per lo sviluppo del frontend, SWEnergy ha scelto di adottare Angular, un
framework open-source per la creazione di applicazioni web. Questa decisione è
stata guidata dalla familiarità di due membri del team con questo framework.
SWEnergy ha inteso capitalizzare sulle competenze pregresse di questi due membri
per ridurre il tempo di apprendimento, organizzando workshop dedicati per il
resto del team al fine di uniformare le conoscenze.

Per il frontend, abbiamo optato per l'utilizzo della versione 17.3.3 di Angular,
la più recente disponibile al momento. Abbiamo scelto questa versione
ritenendola la più adatta per il nostro apprendimento, considerando anche
possibili implementazioni future di Angular. È importante notare che questa
versione è scarsamente supportata da ChatGPT e dispone di risorse online meno
dettagliate rispetto alle versioni precedenti.

Consideriamo questo aspetto come un punto positivo: ci consente di sfruttare le
intelligenze artificiali per comprendere i metodi di soluzione più appropriati,
mentre richiede che il codice venga sempre adattato alle nostre specifiche
esigenze, promuovendo così una migliore comprensione del framework. Inoltre, la
documentazione fornita da Angular è chiara e completa, agevolando il nostro
processo di apprendimento del framework.

\subsubsection{Axios}

Per gestire le chiamate HTTP dal frontend al backend, SWEnergy ha adottato
Axios, un client HTTP basato su promise. La decisione di utilizzare Axios è
stata influenzata dalla raccomandazione del proponente, che ha sottolineato la
facilità e la velocità con cui è possibile effettuare test di carico sulle API
utilizzando questo strumento.

\subsubsection{Tailwind CSS}

Vista l'ampia necessità di utilizzo di CSS nello sviluppo del frontend, il team
ha scelto di adottare Tailwind, un framework che introduce un nuovo approccio
nel linguaggio CSS. L'obiettivo è rendere il codice CSS più riusabile e
manutenibile, semplificando così il processo di sviluppo e migliorando
l'efficienza del team.

\subsubsection{Node.js}

Per la realizzazione del backend, SWEnergy ha optato per Node.js, un runtime
JavaScript open-source che si basa sul motore JavaScript V8 di Google Chrome. La
scelta di Node.js è stata essenzialmente dettata dalla necessità di un runtime
JavaScript per eseguire il codice del backend, considerando che abbiamo scelto
TypeScript come linguaggio principale. Node.js rappresenta la scelta più diffusa
e supportata per questo tipo di scenario.

\subsubsection{Nest.js}

Per la gestione delle API, SWEnergy ha optato per l'utilizzo di Nest.js, un
framework basato su Node.js progettato per la creazione di applicazioni
server-side efficienti, scalabili e facili da mantenere. La decisione di
adottare Nest.js è stata guidata dalla raccomandazione del proponente, che ha
evidenziato la sua similitudine con Angular nella dinamica di utilizzo della
dependency injection. Questo approccio consente la creazione di un'applicazione
modulare, dove i componenti possono essere facilmente sostituiti e testati.

\subsubsection{Drizzle}

Per la gestione del database, SWEnergy ha scelto Drizzle, un ORM (Object
Relational Mapping) progettato per TypeScript. Drizzle consente di definire il
database, le relazioni tra le tabelle e le query direttamente in TypeScript,
fornendo quindi anche le classi necessarie per interagire con il database. La
decisione di adottare Drizzle è stata motivata dalla sua somiglianza con SQL, in
quanto tutti i membri del team sono molto familiari con questo linguaggio.
Inoltre, la similitudine con SQL rende le query estremamente efficienti, poiché
la traduzione è diretta e non richiede una complessa analisi.

\subsubsection{PostgreSQL}

SWEnergy ha optato per l'adozione di PostgreSQL come database per il progetto,
poiché tutti i membri del team possiedono una discreta esperienza con questo
DBMS. La decisione è stata motivata dalla consapevolezza che ci sono già
numerose nuove tecnologie da apprendere nel contesto del progetto; pertanto, per
quanto riguarda il database, è stata preferita una soluzione con cui tutti i
membri del team si sentono già a proprio agio.

\subsubsection{Docker}

L'adozione di Docker per containerizzare l'intero progetto offre numerosi
vantaggi fondamentali. In primo luogo, Docker consente di rendere il prodotto
eseguibile su qualunque hardware, garantendo una maggiore portabilità e facilità
di distribuzione. Grazie alla standardizzazione dell'ambiente di esecuzione
tramite i container Docker, è possibile eliminare i problemi legati alla
differenza di configurazioni hardware e software tra i vari ambienti di
sviluppo, test e produzione.

Inoltre, l'utilizzo di Docker permette di semplificare il processo di gestione
delle dipendenze e delle librerie necessarie per il progetto, fornendo un
ambiente isolato e riproducibile in cui eseguire l'applicazione senza
interferenze esterne. Ciò contribuisce a ridurre i conflitti e i problemi di
compatibilità tra le diverse componenti del sistema, migliorando la coerenza e
l'affidabilità del software.

Infine, Docker offre la possibilità di scalare orizzontalmente il prodotto
utilizzando tecnologie come Kubernetes. Questo significa che è possibile gestire
facilmente un carico di lavoro in crescita distribuendo automaticamente i
container su più nodi, garantendo così una maggiore disponibilità e prestazioni
dell'applicazione anche in presenza di un elevato numero di richieste. In
sintesi, l'adozione di Docker per containerizzare il vostro progetto offre un
modo efficiente e affidabile per gestire l'intero ciclo di vita
dell'applicazione, garantendo flessibilità, portabilità e scalabilità.

\subsection{Tecnologie per l'analisi statica del codice}

\subsubsection{Language Server Protocol}

L'adozione di un Language Server Protocol (LSP) nel nostro processo di sviluppo
rappresenta un passo significativo verso un ambiente di sviluppo più efficiente
e produttivo. Grazie all'utilizzo di un LSP, siamo in grado di integrare
funzionalità avanzate di analisi statica e segnalazione degli errori
direttamente nel nostro editor di codice, garantendo una rapida identificazione
e risoluzione di potenziali problemi di programmazione.

\subsection{Tecnologie per l'analisi dinamica del codice}

\subsubsection{Insomnia}

Per verificare il funzionamento delle API e del backend, SWEnergy ha optato per
l'utilizzo di Insomnia, un'applicazione open-source progettata per testare le
API REST. In effetti, il team ha sviluppato un insieme di test per assicurarsi
che le API rispondano correttamente alle richieste e restituiscano i dati
previsti, garantendo così il corretto funzionamento del sistema.


\subsubsection{Angular}

\subsubsection{Nest.js}

\begin{table}[H]
	\centering
	\begin{tabularx}{\textwidth}{lXl}
		\hline
		\textbf{Nome} & \textbf{Descrizione}                                                  & \textbf{Versione} \\
		\hline
		Typescript    & Linguaggio di programmazione                                          & 5.4               \\
		\hline
		Angular       & Framework per lo sviluppo di applicazioni web                         & 17.3.3            \\
		\hline
		Axios         & Libreria per effettuare richieste HTTP                                & 1.6.8             \\
		\hline
		Tailwind CSS  & Framework per la gestione di file CSS                                 & 3.4.3             \\
		\hline
		Node.js       & Runtime JavaScript                                                    & 20.12.0           \\
		\hline
		NestJS        & Framework per lo sviluppo di applicazioni server-side                 & 10.3.6            \\
		\hline
		Drizzle       & Object Relational Mapping                                             & 0.30.6            \\
		\hline
		PostgreSQL    & Sistema di gestione di basi di dati                                   & 16.2              \\
		\hline
		Docker        & Piattaforma per lo sviluppo, il deploy e l'esecuzione di applicazioni & /                 \\
		\hline
		Insomnia      & Crezione di un batch di test collaborativo                            & 8.6.1             \\
		\hline
	\end{tabularx}
	\caption{Tabella delle tecnologie adottate}
\end{table}
