\section{Tecnologie adottate}

\subsection{Tecnologie per la codifica}

\subsubsection{TypeScript}
I membri del \textit{team} SWEnergy, prima di iniziare il progetto, possedevano una
solida conoscenza di C++. Alcuni di loro avevano esperienza anche con Java,
Python e JavaScript. Il committente ha illustrato che Imola Informatica fa un
ampio uso di TypeScript e Java per lo sviluppo \textit{software}, sottolineando
inoltre che per creare un'applicazione \textit{web}, specialmente per quanto riguarda il
\textit{frontend}, è necessario utilizzare JavaScript o TypeScript. Partendo da
queste considerazioni, SWEnergy ha optato per l'uso di JavaScript come
linguaggio principale per lo sviluppo del progetto, poiché consente di scrivere
codice sia per il \textit{frontend} che per il \textit{backend}. Infine, è stato
scelto TypeScript, un superset di JavaScript, per la sua capacità di aggiungere
tipi statici al linguaggio, migliorando la robustezza, la manutenibilità e la
leggibilità del codice. Attraverso l'analisi statica del codice, è possibile
individuare errori di programmazione direttamente durante la fase di sviluppo,
evitando che si manifestino durante l'esecuzione, il che migliora la qualità del
\textit{software} e riduce il tempo di sviluppo.


\subsubsection{Angular}
Per lo sviluppo del \textit{frontend}, SWEnergy ha scelto di adottare Angular,
un \textit{framework} \textit{open-source} per la creazione di applicazioni \textit{web}. Questa
decisione è stata guidata dalla familiarità di due membri del \textit{team} con questo
\textit{framework}. SWEnergy ha inteso capitalizzare sulle competenze pregresse
di questi due membri per ridurre il tempo di apprendimento, organizzando
\textit{workshop} dedicati per il resto del \textit{team} al fine di uniformare le
conoscenze.

Per il \textit{frontend}, abbiamo optato per l'utilizzo della versione 17.3.3 di
Angular, la più recente disponibile al momento. Abbiamo scelto questa versione
ritenendola la più adatta per il nostro apprendimento, considerando anche
possibili implementazioni future di Angular. È importante notare che questa
versione è scarsamente supportata da ChatGPT e dispone di risorse online meno
dettagliate rispetto alle versioni precedenti.

Consideriamo questo aspetto come un punto positivo: ci consente di sfruttare le
intelligenze artificiali per comprendere i metodi di soluzione più appropriati,
mentre richiede che il codice venga sempre adattato alle nostre specifiche
esigenze, promuovendo così una migliore comprensione del \textit{framework}.
Inoltre, la documentazione fornita da Angular è chiara e completa, agevolando il
nostro processo di apprendimento del \textit{framework}.


\subsubsection{Axios}
Per gestire le chiamate HTTP dal \textit{frontend} al \textit{backend}, SWEnergy
ha adottato Axios, un client HTTP basato su promise. La decisione di utilizzare
Axios è stata influenzata dalla raccomandazione del proponente, che ha
sottolineato la facilità e la velocità con cui è possibile effettuare \textit{test} di
carico sulle API utilizzando questo strumento.


\subsubsection{Tailwind CSS}
Vista l'ampia necessità di utilizzo di CSS nello sviluppo del \textit{frontend},
il \textit{team} ha scelto di adottare Tailwind, un \textit{framework} che introduce un
nuovo approccio nel linguaggio CSS. L'obiettivo è rendere il codice CSS più
riusabile e manutenibile, semplificando così il processo di sviluppo e
migliorando l'efficienza del \textit{team}.


\subsubsection{Node.js}
Per la realizzazione del \textit{backend}, SWEnergy ha optato per Node.js, un
\textit{runtime} JavaScript \textit{open-source} che si basa sul motore JavaScript V8 di
Google Chrome. La scelta di Node.js è stata essenzialmente dettata dalla
necessità di un \textit{runtime} JavaScript per eseguire il codice del
\textit{backend}, considerando la scelta di TypeScript come linguaggio
principale. Node.js rappresenta la scelta più diffusa e supportata per questo
tipo di scenario.


\subsubsection{Nest.js}
Per la gestione delle API, SWEnergy ha optato per l'utilizzo di Nest.js, un
\textit{framework} basato su Node.js progettato per la creazione di applicazioni
\textit{server-side} efficienti, scalabili e facili da mantenere. La decisione di
adottare Nest.js è stata guidata dalla raccomandazione del proponente, che ha
evidenziato la sua similitudine con Angular nella dinamica di utilizzo della
\textit{dependency injection}. Questo approccio consente la creazione di
un'applicazione modulare, dove i componenti possono essere facilmente sostituiti
e testati.


\subsubsection{Drizzle}
Per la gestione del \textit{database}, SWEnergy ha scelto Drizzle, un ORM (\textit{Object
	Relational Mapping}) progettato per TypeScript. Drizzle consente di definire
il \textit{database}, le relazioni tra le tabelle e le \textit{query} direttamente in
TypeScript, fornendo quindi anche le classi necessarie per interagire con il
\textit{database}. La decisione di adottare Drizzle è stata motivata dalla sua
somiglianza con SQL, in quanto tutti i membri del \textit{team} sono molto familiari
con questo linguaggio. Inoltre, la similitudine con SQL rende le
\textit{query} estremamente efficienti, poiché la traduzione è diretta e non
richiede una complessa analisi.


\subsubsection{PostgreSQL}
SWEnergy ha optato per l'adozione di PostgreSQL come \textit{database} per il progetto,
poiché tutti i membri del \textit{team} possiedono una discreta esperienza con questo
DBMS. La decisione è stata motivata dalla consapevolezza che ci sono già
numerose nuove tecnologie da apprendere nel contesto del progetto; pertanto, per
quanto riguarda il \textit{database}, è stata preferita una soluzione con cui tutti i
membri del \textit{team} si sentono già a proprio agio.


\subsubsection{Docker}
L'adozione di Docker per la creazione di \textit{container} offre numerosi vantaggi fondamentali.
In primo luogo, Docker consente di rendere il prodotto eseguibile su qualunque \textit{hardware}, garantendo una maggiore portabilità e facilità di distribuzione.
Grazie alla standardizzazione dell'ambiente di esecuzione tramite i \textit{container} Docker, è possibile eliminare i problemi legati alla differenza di configurazioni \textit{hardware} e \textit{software} tra i vari ambienti di sviluppo, \textit{test} e produzione.

Inoltre, l'utilizzo di Docker permette di semplificare il processo di gestione
delle dipendenze e delle librerie necessarie per il progetto, fornendo un
ambiente isolato e riproducibile in cui eseguire l'applicazione senza
interferenze esterne. Ciò contribuisce a ridurre i conflitti e i problemi di
compatibilità tra le diverse componenti del sistema, migliorando la coerenza e
l'affidabilità del \textit{software}.

Infine, Docker offre la possibilità di scalare orizzontalmente il prodotto
utilizzando tecnologie come Kubernetes. Questo significa che è possibile gestire
facilmente un carico di lavoro in crescita distribuendo automaticamente i
\textit{container} su più nodi, garantendo così una maggiore disponibilità e prestazioni
dell'applicazione anche in presenza di un elevato numero di richieste. In
sintesi, l'adozione di Docker offre un modo efficiente e affidabile per gestire l'intero ciclo di vita dell'applicazione, garantendo flessibilità, portabilità e scalabilità.

\subsection{Tecnologie per l'analisi statica del codice}

\subsubsection{Language Server Protocol}

L'adozione di un \textit{Language Server Protocol} (LSP) nel nostro processo di sviluppo
rappresenta un passo significativo verso un ambiente di sviluppo più efficiente
e produttivo. Grazie all'utilizzo di un LSP, siamo in grado di integrare
funzionalità avanzate di analisi statica e segnalazione degli errori
direttamente nel nostro \textit{editor} di codice, garantendo una rapida
identificazione e risoluzione di potenziali problemi di programmazione.

\subsection{Tecnologie per l'analisi dinamica del codice}

\subsubsection{Insomnia}
Per verificare il funzionamento delle API e del \textit{backend}, SWEnergy ha scelto di utilizzare Insomnia, un'applicazione \textit{open-source} progettata per il testing delle API REST. 
Il \textit{team} ha creato un ambiente condiviso all'interno di Insomnia, consentendo l'esecuzione di tutte le API disponibili e l'osservazione dei risultati. 
Questo ambiente memorizza le diverse tipologie di chiamate e vari \textit{set} di dati di esempio.
Insomnia facilita l'esecuzione di test manuali, permettendo di modificare agevolmente i dati inviati al \textit{backend} e di visualizzare i risultati restituiti. 
Questo consente una rapida verifica della corrispondenza tra i risultati ottenuti e quelli attesi, migliorando l'efficienza e l'accuratezza dei test.


\subsubsection{Angular}
Angular fornisce nativamente il comando \texttt{ng test}, che è uno strumento
potente per eseguire test unitari e di integrazione.

I test unitari sono progettati per verificare l'accuratezza delle singole
unità di codice, come componenti e servizi. Con Angular, \texttt{ng test}
sfrutta il framework Jasmine per definire i test e Karma come test runner per
eseguire i test in un ambiente di browser. Questo permette di simulare il
comportamento dell'applicazione in un contesto realistico, garantendo che ogni
unità funzioni come previsto.

Oltre ai test unitari, \texttt{ng test} supporta anche i test di integrazione,
che verificano l'interazione tra più unità di codice. Questi test sono cruciali
per assicurare che i vari componenti dell'applicazione funzionino
correttamente insieme e che l'integrazione tra loro non introduca bug o
comportamenti inattesi. Anche in questo caso, Jasmine offre la possibilità di
definire i \textit{mock} e le \textit{stub} necessari per simulare le dipendenze
esterne e garantire che i test siano eseguiti in modo isolato.


\subsubsection{Nest.js}
Nest.js fornisce una integrazione di \textit{default} con Jest, un \textit{framework} di testing JavaScript ampiamente utilizzato e noto per la sua velocità e semplicità. 
Si tratta della scelta ideale per eseguire test siccome offre un'ottima integrazione con NestJS, permettendo di sfruttare appieno le caratteristiche del \textit{framework} come i moduli, i servizi e i controller. 
Inoltre, Jest supporta il \textit{mocking} e la simulazione delle dipendenze, facilitando l'isolamento delle unità di codice da testare. 
Le sue capacità di test asincroni e il supporto integrato per TypeScript, che è comunemente utilizzato con NestJS, garantiscono una scrittura di test più fluida e meno soggetta a errori. 
Infine, Jest fornisce un'interfaccia utente intuitiva e \textit{report} di copertura del codice dettagliati, rendendo più semplice identificare e correggere i \textit{bug} e migliorare la qualità complessiva del \textit{software}.


\subsection{Riepilogo}
La seguente tabella fornisce un riepilogo dettagliato delle tecnologie adottate dal gruppo per lo sviluppo dell'applicazione Easy Meal.
\begin{table}[H]
	\centering
	\begin{tabularx}{\textwidth}{lXl}
		\hline
		\textbf{Nome} & \textbf{Descrizione}                                                           & \textbf{Versione} \\
		\hline
		Typescript    & Linguaggio di programmazione                                                   & 5.4               \\
		\hline
		Angular       & \textit{Framework} per lo sviluppo di applicazioni web                         & 17.3.3            \\
		\hline
		Axios         & Libreria per effettuare richieste HTTP                                         & 1.6.8             \\
		\hline
		Tailwind CSS  & \textit{Framework} per la gestione di file CSS                                 & 3.4.3             \\
		\hline
		Node.js       & \textit{Runtime} JavaScript                                                    & 20.12.0           \\
		\hline
		NestJS        & \textit{Framework} per lo sviluppo di applicazioni \textit{server-side}        & 10.3.6            \\
		\hline
		Drizzle       & \textit{Object Relational Mapping}                                             & 0.30.6            \\
		\hline
		PostgreSQL    & Sistema di gestione di basi di dati                                            & 16.2              \\
		\hline
		Docker        & Piattaforma per lo sviluppo, il \textit{deploy} e l'esecuzione di applicazioni & /                 \\
		\hline
		Insomnia      & Crezione di un \textit{batch} di \textit{test} collaborativo                   & 8.6.1             \\
		\hline
	\end{tabularx}
	\caption{Tabella delle tecnologie adottate}
\end{table}
