\section{\textit{Brainstorming}}

Prima dell'incontro, è stato condiviso con Alessandro Staffolani l'ordine del giorno e 
una bozza dell'"Analisi dei requisiti". All'inizio dell'incontro, il 
responsabile del gruppo SWEnergy ha presentato brevemente il lavoro svolto 
fino ad ora dal gruppo e ha spiegato cosa è necessario completare per la prima 
fase del progetto, ovvero l'RTB.

\section{Organizzazione del lavoro}

Attraverso \textit{Telegram}, è stato condiviso uno schema per illustrare lo 
svolgimento delle attività relative alla prima fase del progetto in ordine. 
Durante la riunione, il proponente ha commentato lo schema e ha suggerito lo 
sviluppo parallelo del \textit{front-end} e del \textit{back-end}. In 
particolare, ha apprezzato l'idea del gruppo di creare due team di lavoro per lo 
sviluppo dei due componenti. Tale divisione mira a semplificare la gestione del 
progetto e l'organizzazione del lavoro, consentendo ai membri del gruppo di 
approfondire le tecnologie utilizzate. I due gruppi di lavoro si scambieranno 
i ruoli per la seconda fase del progetto, la PB.

\section{Analisi dei requisiti}

Il proponente ha apprezzato il lavoro svolto dal gruppo e ha fornito alcuni 
suggerimenti generali, consigliando di concentrarsi maggiormente sulle 
funzionalità da implementare.

\section{Retrospettiva}

A partire da questo incontro, il gruppo adotterà una strategia agile. Pertanto, 
ha deciso di tenere una breve retrospettiva per valutare il lavoro svolto fino a 
questo momento e migliorare l'organizzazione del lavoro. Al momento, non sono 
emerse criticità né suggerimenti per migliorare da parte di SWEnergy o del 
proponente. Tuttavia, Alessandro Staffolani ha chiarito il ruolo dell'azienda nel 
progetto: non partecipa attivamente allo sviluppo, ma fornisce indicazioni sui 
requisiti e le tecnologie da utilizzare, oltre a offrire consulenza per la 
gestione e lo sviluppo del progetto.

\section{Analisi delle tecnologie}

SWEnergy ha condiviso delle note all'interno dell'"Analisi dei requisiti" 
riguardo alle tecnologie che è propenso ad adottare. Di conseguenza, il 
proponente ha suggerito alcune tecnologie complementari a quelle proposte dal 
gruppo. Di seguito sono elencate le tecnologie citate dal proponente:

\begin{itemize}
	\item \textbf{\textit{Nest.js}}: un \textit{framework} per lo sviluppo di 
		applicazioni \textit{Node.js} che utilizza \textit{TypeScript}. Il 
		proponente ha consigliato questo \textit{framework} per lo sviluppo del 
		\textit{back-end}, poiché la sua architettura è simile a quella di
		\textit{Angular}: entrambe sfruttano la \textit{dependency injection}.
    
    \item \textbf{\textit{socket}}: il proponente ha consigliato l'utilizzo di 
		questa tecnologia per la comunicazione in tempo reale tra i
		\textit{client} e il \textit{server}, in particolare per le
		conversazioni.
    
    \item \textbf{\textit{OpenAPI}}: il proponente ha consigliato di utilizzare 
		questa tecnologia per facilitare la comunicazione tra il
		\textit{front-end} e il \textit{back-end}, oltre a generare 
		automaticamente la documentazione del \textit{back-end}.
    
    \item \textbf{\textit{Axios}}: il proponente ha suggerito di utilizzare 
		questa tecnologia per testare gli \textit{endpoint} del 
		\textit{back-end} e per analizzare il carico di lavoro del
		\textit{server}.
\end{itemize}

\section{Conclusioni}

Gli obiettivi concordati con il proponente per il primo \textit{sprint} 
includono:

\begin{itemize}
	\item \textbf{"Analisi dei requisiti"}: stesura avanzata, ma non conclusiva.
    
    \item \textbf{"Analisi delle tecnologie"}: studio delle tecnologie 
		fondamentali per lo sviluppo del progetto e approfondimento delle 
		tecnologie consigliate dal proponente.
    
    \item \textbf{"Piano di progetto"}: stesura avanzata, ma non conclusiva.
    
    \item \textbf{"Piano di qualifica"}: stesura avanzata, ma non conclusiva.

    \item \textbf{Database}: progettazione del database, inclusa la 
		definizione dello schema logico e concettuale.
\end{itemize}

Il responsabile del gruppo SWEnergy redigerà il verbale esterno dell'incontro 
e organizzerà una riunione con i membri di SWEnergy per domenica sera, intorno 
alle 20, per pianificare le azioni basate su quanto discusso con il proponente 
in questo incontro.
