\section{Presentazione del \textit{Proof of Concept}}
Durante l'incontro è stato presentato il prodotto realizzato per la fase RTB.
La presentazione si è concentrata sulle operazioni effettuabili dalle diverse tipologie di utenza previste, ossia ristoratori ed utenti base.
Per quanto riguarda la struttura grafica dell'applicativo si è precisato che quella presentata rappresenta una bozza, utile a fornire un'idea dello stile scelto per l'interfaccia utente ma che subirà modifiche per adattarsi alle nuove funzionalità introdotte nelle successive fasi di progetto.\\
Sono state poi presentate le tecnologie utilizzate per la realizzazione del PoC, sia relative al \textit{front-end} che al \textit{back-end}.
La presentazione si è conclusa con una rapida illustrazione del \textit{back-end} e del \textit{database} realizzato.


\section{Considerazioni}
Il referente aziendale ha confermato che i requisiti concordati relativi al PoC sono stati soddisfatti ed ha fornito un parere positivo sul lavoro svolto.
Sono stati forniti consigli utili per migliorare l'applicativo durante la successiva fase di progetto, in particolare:
\begin{itemize}
	\item La procedura di \textit{login} avverrà in modo diverso in base alla tipologia di utente che effettuerà l'accesso, non si effettuerà uno smistamento in base al tipo di account a cui si cercherà di accedere ma si effettuerà l'accesso su una piattaforma distinta in base al proprio tipo di utenza.
	\item Si suggerisce di modificare il metodo utilizzato per catalogare la tipologia di utenza nel \textit{database}, indicandola esplicitamente in associazione ad un determinato identificativo utente.
	\item Si suggerisce l'adozione \textit{JSON Web Token} come \textit{standard} per l'autenticazione dell'utenza nella successiva fase di progetto.
\end{itemize}


\section{Conclusioni}
L'approvazione del PoC da parte del referente aziendale conferma la possibilità di presentare il lavoro svolto per la fase di avanzamento RTB.
Il gruppo effettuerà una riunione in data successiva al presente incontro per formalizzare la decisione ed organizzare il lavoro da svolgere.\\
Visto l'avvicinarsi della sessione di esami relativa al primo semestre, è stata concordata con il referente la possibilità di effettuare una pausa dal lavoro inerente al progetto.
Successive comunicazioni con l'azienda avverranno tramite Telegram, come stabilito.