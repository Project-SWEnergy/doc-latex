\section{\textit{Brainstorming}}

Dopo aver completato l'RTB, il 7 marzo, SWEnergy e Imola Informatica si sono
incotrati per discutere il proseguimento del progetto. In particolare, SWEnergy
ha spiegato come si è svolta la revisione RTB con i committenti e quali sono
stati i \textit{feedback} ricevuti. In questo periodo, SWEnergy non ha
implementato nuove \textit{feature}.

\section{Prossimi passi}

SWEnergy ha discusso l'organizzazione del tempo e delle risorse per lo sprint
futuro. In particolare, è stato deciso di dedicare la settimana dall'11 marzo al
17 merzo allo studio delle tecnologie: i due sottogruppi di SWEnergy si
scambiamo le responsabilità rispetto al \textit{backend} e al \textit{frontend}.
Mentre la settimana successiva dal 18 marzo al 24 marzo sarà dedicata alla
progettazione, in particolare la progettazione si definisce in progettazione dei
documenti da consegnare in seguito alla revisione PB e alla progettazione del
database, il cuore dell'applicazione. L'obiettivo dello sprint a venire è quello
di aver restrutturato il PoC in modo che possa essere lo scheletro 
dell'applicazione finale.

\section{JWT}

In una riunione precedente, Staffolani aveva consigliato l'adozione di JWT
(\textit{JSON Web Token}) per la gestione delle sessioni e quindi
dell'autenticazione. Dunque, SWEnergy ha esposto i propri dubbi rispetto a
questa tecnolgia. In particolare, è stato chiesto di spiegare come funzionassero
e di che cosa si tratta. Se sono in grado di recepire l'utente che sta eseguendo
le richieste e se sono sicuri. Staffolani ha fornito una spiegazione dettagliata
sull'argomento e ha chiarito i dubbi di SWEnergy. In aggiunta, ha fornito delle
fonti per approfondire l'argomento e per avere degli esempi di implementazione
della tecnologia.

\section{Chat}

SWEnergy ha chiesto a Imola Informatica qualche tecnologia per implementare la
chat e un'introduzione di base all'argomento. Staffolani ha consigliato l'uso di
socket.io e ha spiegato che si utilizzano i JWT per autenticare gli utenti. Ha 
spiegato che la
chat deve essere gestita mediante dei socket, ovvero dei canali di comunicazione
bidirezionali tra il client e il server. Inoltre, ha fornito delle fonti per 
approfondire l'argomento.

\section{Test}

SWEnergy ha chiesto a Imola Informatica come svolgere i test. In particolare, è
stato chiesto consiglio in merito all'automatizzazione dei test e dei test in
\textit{batch}. Imola Informatica ha spiegato che sia Angular che NestJS
di default forniscono dei \textit{framework} per i test e ha consigliato di
usare questi strumenti, senza adottare altri \textit{framework} di terze parti.
