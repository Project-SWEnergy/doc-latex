\section{\textit{Cookie}}

L'incontro è stato aperto con la richiesta di un parere riguardante l'utilizzo 
dei \textit{cookie}, attualmente impiegati come mezzo di registrazione del 
\textit{token} generato al momento del \textit{login}. 
Il referente ha espresso il suo favore verso questa scelta del gruppo, tuttavia 
ha evidenziato che ci potrebbero essere delle difficoltà aggiuntive rispetto 
all'utilizzo del local storage.

\section{Manuale Utente}

La seconda questione sollevata ha riguardato la redazione del Manuale Utente. 
In particolare, il gruppo ha chiesto chiarimenti riguardo alla definizione del 
destinatario del progetto al fine di stabilire se debba contenere solo 
informazioni sull'utilizzo dell'applicativo (clienti dei ristoranti e gestori
dei ristoranti) o anche istruzioni per l'azienda su come gestire l'applicazione, 
ad esempio per quanto riguarda il \textit{deploy}. 
Il referente ha chiarito che il manuale dovrà essere destinato agli utenti 
finali dell'applicazione, identificati come utente base$^G$ e utente 
ristoratore$^G$.
Dovrà però essere prodotto anche un documento, separato dal Manuale Utente, che 
includa le informazioni inerenti la gestione dell'applicazione e destinato alla 
sola azienda.

In conclusione al termine del progetto, il gruppo dovrà presentare all'azienda 
tre tipi di documentazione:
\begin{itemize}
    \item \textbf{funzionale}: include l'analisi dei requisiti e degli use case, 
		già prodotta dal gruppo;
    \item \textbf{tecnica}: sulla descrizione del software, dovrà includere le 
		informazioni sulla gestione dell'applicativo;
    \item \textbf{generica}: include il manuale utente.
\end{itemize}

\section{Presentazione del lavoro svolto}

Il gruppo ha presentato il lavoro svolto in questa fase di progetto, sia lato 
\textit{back-end} che \textit{front-end}.
Il referente ha criticato la scelta di utilizzare un \textit{link} ad una 
risorsa esterna per l'inserimento di immagini, sia durante la creazione dei 
piatti che del ristorante.
Ha quindi proceduto a fornire indicazioni utili a comprendere come implementare 
l'\textit{upload} di \textit{file} e come gestirli all'interno dell'applicativo.
