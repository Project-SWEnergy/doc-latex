\section{Richiesta di riduzione dei requisiti obbligatori }

L'incontro è stato aperto con la richiesta da parte del gruppo di ridurre il 
numero di requisiti obbligatori, in particolare quello riguardante la gestione 
della \textit{chat} tra il cliente ed il ristoratore.
La motivazione è dovuta al fatto che il tempo a disposizione per lo sviluppo 
dell'MVP è limitato, siccome la data di consegna prevista per il 10/05/2024 è 
ormai prossima, ed il gruppo ritiene che la gestione della \textit{chat} possa 
far ritardare ulteriormente il completamento del progetto.
Per questo motivo SWEnergy ha esposto tale problematica al proponente Alessandro 
Staffolani, il quale ha accettato la richiesta di togliere il requisito 
obbligatorio riguardante la \textit{chat}.

\section{Domande}
Successivamente, il gruppo ha posto alcune domande al proponente riguardo la 
gestione del pagamento delle ordinazioni. 
Il dubbio riguardava se si dovesse fare utilizzo di un servizio esterno per la 
gestione dei pagamenti, come ad esempio \textit{PayPal}. 
Il proponente ha risposto che non è necessario utilizzare un servizio di terze 
parti, ma che il processo può essere gestito all'interno dell'applicazione in 
maniera tale da simulare un pagamento reale, attraverso l'impiego di "scontrini" 
e "\textit{transcation log}" che permettano di tenere traccia di quanto il 
cliente ha pagato e che quindi il conto sia effettivamente saldato.  


\section{Presentazione del lavoro svolto}
Il gruppo ha presentato il lavoro svolto in questa fase di progetto, sia lato 
\textit{back-end} che \textit{front-end}:
\begin{itemize}
    \item È stata implementata la gestione delle prenotazioni.
    \item È stata implementata la gestione dell'ordinazione collaborativa dei 
		pasti.
    \item È stata implementata la gestione del caricamento delle immagini da 
		parte del ristoratore in maniera corretta come suggerito 
		precedenetemente dal proponente.
    \item È stata implementata la gestione degli ingredienti e dei piatti lato 
		ristoratore.
\end{itemize}

Successivamente alla presentazione del lavoro svolto, il gruppo si è reso conto 
che c'è stata un'incomprensione riguardo la gestione dell'ordinazione 
collaborativa.
La natura di tale fraintendimento non è relativa ad una scarsa comprensione dei 
requisiti, ma piuttosto a come tale requisito dovesse essere implementato.
SWEnergy ha presentato la gestione dell'ordinazione collaborativa in maniera 
asincrona, mentre il proponente aveva in mente una gestione sincrona con 
l'utilizzo dei \textit{socket}.
Consapevole del tempo che tale modifica richiederebbe, il proponente ha ritenuto 
accettabile il lavoro svolto dal gruppo e ha permesso di continuare con la 
gestione asincrona dell'ordinazione collaborativa, ritenendo l'implementazione 
comunque valida e funzionante ed assecondando la necessità del gruppo di non 
variare la data di consegna. \\
Infine si è discusso riguardo all'implementazione delle notifiche: Staffolani è 
d'accordo con l'idea di sviluppo proposta dal gruppo, ovvero quella di mostrare 
le notifiche solo quando l'utente accede all'applicazione tramite una finestra 
d'avviso, andando ad interrogare il database e verificando se ci sono stati 
aggiornamenti. Il proponente ha però suggerito di implementare le notifiche con 
l'utilizzo di \textit{socket} così che il gruppo possa prendere familiarità con 
questa tecnologia, solo se il tempo a disposizione lo permette. SWEnergy è 
d'accordo con tale proposta e si impegna a implementare le notifiche con 
l'utilizzo dei \textit{socket} se non si verificheranno ritardi nello sviluppo.
