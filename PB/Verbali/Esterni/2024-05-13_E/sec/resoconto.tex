\section{\textit{Brainstorming}}

La riunione è cominciata con il riassunto delle attività svolte. In particolare
è stato commentato lo stato del progetto: sono stati svolti 33 requisiti
obbligatori rispetto ai 36 totali. Inoltre sono stati svolti 4 requisiti
facoltativi. Dunque è stato mostrato il prodotto fino ad ora sviluppato. Infatti
sono stati completati i requisiti in merito alla gestione delle notifiche sia
lato cliente che lato ristoratore. Infine è stata completata la gestione delle
prenotazioni lato ristoratore e la visualizzazione delle prenotazioni di
dettaglio.

\section{Riduzione dei requisiti}

SWEnergy ha richiesto al proponente di ridefinire l'MVP, in modo tale da poter
concludere questo progetto il più presto possibile, poiché il tempo a
disposizione è terminato la settimana scorsa. Per questo motivo, guardando
l'Analisi dei Requisiti si è discusso rispetto ai requisiti obbligatori che 
diventano facoltativi. Così il numero di requisiti obbligatori si conferma a 36.

\section{Documentazione}

Infine, SWEnergy ha richiesto se i documenti denominati Manuale Utente e
Specifiche Tecniche fossero di interesse al proponente. Il proponente ha
spiegato che questi documenti sono comodi da scorrere in fase di test di
accettazione, per questo motivo è stato deciso di consegnarli al prossimo
incontro insieme al prodotto finale, così da permettere di convalidare
opportunamente l'MVP.
