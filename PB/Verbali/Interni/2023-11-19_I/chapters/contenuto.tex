\section{\textit{Brainstorming}}

Il relatore ha proposto di cambiare l'approccio alla scrittura della
documentazione, suggerendo di redigere i documenti in markdown per poi
compilarli in \LaTeX{}. Questa opzione sarà approfondita dal relatore e
successivamente discussa con il gruppo. Si impegnerà a elaborare una lista
dettagliata di \textit{issue} da inserire nel \textit{backlog} del progetto, al
fine di facilitare la comprensione dello stato del progetto, agevolare la
suddivisione dei compiti e migliorare la qualità del lavoro asincrono. Le
\textit{issue} saranno raggruppate in \textit{milestone}.

Dopo l'introduzione, il relatore ha delineato gli obiettivi delle prossime due
settimane, i quali verranno discussi nei dettagli nelle sezioni seguenti.

Infine, il gruppo ha esaminato gli appunti riguardanti i casi d'uso (vedi
\textit{issue} \#91) e ha distribuito i compiti fino all'incontro con il
proponente del 24/11/2023. SWEnergy ha deciso di tenere uno \textit{stand-up
	meeting} la sera del 23/11/2023 per effettuare un \textit{brainstorming} in
preparazione all'incontro con il proponente.

\section{Analisi dei requisiti}

Dopo aver raccolto gli appunti sui casi d'uso da parte di tutto il gruppo, tre
analisti si impegnano a standardizzare i casi d'uso con i seguenti compiti:

\begin{enumerate}
	\item Elaborare una lista dei casi d'uso.
	\item Fornire una breve e sintetica descrizione per i casi d'uso che la
	      necessitano.
	\item Redigere un elenco delle domande emerse fino a questo momento.
	\item Realizzare un diagramma dei casi d'uso (preferibilmente in
	      \textit{UML}).
	\item Creare un diagramma dei casi d'uso tramite la piattaforma
	      \href{https://excalidraw.com/}{Excalidraw}.
	\item Identificare i requisiti derivanti dai casi d'uso individuati finora.
	\item Assegnare una priorità a quattro livelli ai requisiti.
	\item Elencare le domande emerse durante l'analisi.
\end{enumerate}

Il diagramma dei casi d'uso da realizzare tramite Excalidraw è pensato per
agevolare la discussione con il proponente in quanto risulta più semplice da
elaborare e modificare rispetto a un diagramma UML. Inoltre, è molto
semplice da condividere in tempo reale. Finora è stato delineato ciò che si
prevede di completare in vista dello \textit{stand-up meeting} del 24/11/2023.

Successivamente, il gruppo ha discusso la struttura del documento "Analisi dei
requisiti" e ha deciso di utilizzare un approccio \textit{top-down} per
semplificare la stesura del documento. In particolare, il documento sarà
strutturato in modo gerarchico, partendo dai casi d'uso e arrivando ai
requisiti. Si prevede di redigere la prima bozza del documento durante la
settimana successiva, entro il 1/12/2023.
Sono emersi alcuni dubbi riguardo alla struttura del documento, i quali saranno
posti ai docenti durante le lezioni.

SWEnergy ha deciso di standardizzare la nomenclatura dei casi d'uso per
agevolare la comprensione e l'organizzazione: il nome di un caso d'uso sarà
composto da \textit{UC} seguito da un numero progressivo a partire da 1, e da
un titolo.

\section{Incontro con il proponente}

%Il relatore ha approfondito i primi incontri con il proponente dei gruppi
%degli anni passati e dell'anno corrente per comprendere lo svolgimento
%dell' incontro, i tipi di appunti da preparare, le domande da porre e come
%preparare il team di SWEnergy. In particolare, si è raccomandato che ciascun
%membro del gruppo sia preparato a rispondere alle domande relative allo
%\textit{stack} tecnologico di sua competenza. 
Sono stati brevemente discussi i principali temi da affrontare durante
l'incontro:

\begin{itemize}
	\item Durata degli sprint.
	\item Obiettivi del primo sprint.
	\item Conoscenza dello \textit{stack} tecnologico del team.
	\item Discussione dei casi d'uso e dei requisiti.
\end{itemize}

Il gruppo ha deciso di utilizzare un \textit{file} Excel condiviso su
\textit{Google Drive} per raccogliere le informazioni relative allo
\textit{stack} tecnologico di ciascun membro del gruppo.

\section{Glossario}

Il gruppo ha deciso di sviluppare il glossario attraverso un \textit{file} in
formato "CSV" (Comma Separated Values), che verrà successivamente convertito
in un documento \LaTeX{}. Questa scelta mira a facilitare la stesura del
glossario e la creazione di un programma che automatizzi i riferimenti al
glossario presenti nel resto della documentazione. L'Amministratore dovrà
inoltre creare il \textit{template} del documento.

\section{"Piano di progetto" e "Piano di qualifica"}

SWEnergy ha deciso di approfondire le tematiche relative al "Piano di progetto
" e al "Piano di qualifica" durante la settimana in corso. L'obiettivo è
comprendere la struttura dei documenti e quali informazioni inserire, con
particolare attenzione a ciò che potrebbe interessare al proponente. Pertanto,
per il \textit{brainstorming} del 23/11/2023, saranno creati i primi \textit{
	template} dei documenti, da discutere in una riunione successiva.

\section{Retrospettiva}

Infine, è stata condotta una breve retrospettiva per valutare come migliorare
la qualità degli incontri e del lavoro asincrono. I risultati sono stati
alquanto limitati e non sono emerse criticità particolari fino a questo momento.
Uno dei motivi potrebbe essere la durata dell'incontro e l'orario di chiusura.
A seguito della retrospettiva, il gruppo ha creato il "Diario di bordo del
20/11/2023".
