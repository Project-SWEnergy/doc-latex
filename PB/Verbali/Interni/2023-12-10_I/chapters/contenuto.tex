\section{\textit{Brainstorming}}
Durante la presente sessione interna è stata condotta una sintesi dettagliata della situazione attuale del progetto. 
Il gruppo ha successivamente intrapreso una discussione inerente le caratteristiche dei documenti richiesti per la fase di avanzamento e ha organizzato le attività di formazione.
Sono stati infine individuati i prossimi obiettivi e fissate le scadenze per l'attuale fase.

\section{Documentazione}
Durante lo \textit{sprint} concluso con la presente riunione sono stati realizzati, in tutto o in parte, alcuni tra i documenti richiesti per l'avanzamento di fase.
Ogni membro del gruppo a cui è stata assegnata una attività di redazione o verifica ha aggiornato i compagni sullo stato del proprio lavoro, ricevendo \textit{feedback} utili al completamento dell'attività, in particolare: 
\begin{itemize}
	\item \textbf{Glossario}: si è accettata l'idea di automatizzare il processo di ricerca dei termini, in esso inseriti, all'interno dei documenti redatti. \'E stata inoltre avanzata la proposta di rendere il glossario una risorsa web più facilmente consultabile, l'idea verrà rivalutata in base  al tempo disponibile al gruppo per la sua implementazione.
	\item \textbf{Piano di progetto}: il gruppo si è confrontato sul contenuto del documento, in particolare sulla porzione relativa alla pianificazione del lavoro.
	\item \textbf{Piano di qualifica}: sono state prese decisioni sugli \textit{standard} da adottare durante il progetto ed è stata ideata la struttura del documento che dovrà descriverli.
\end{itemize}


\section{Formazione e PoC}
Per la realizzazione del \textit{Proof of Concept} si è scelto di dividere il lavoro in \textit{front-end} e \textit{back-end}
affrontando lo studio in gruppi formati da due persone, che successivamente verranno ruotati in modo da permettere all'intero gruppo di svolgere attività in ogni ambito della programmazione.\\
Nell'ambito dell'attuale \textit{sprint} è stato definito un referente per ogni gruppo, il loro compito sarà quello di realizzare materiale utile alla formazione interna in merito alle tecnologie a loro assegnati.
Per quanto riguarda il front-end, Alessandro Tigani si occuperà dello studio e della formazione inerente le tecnologie Angular e TypeScript.
Per quanto riguarda il back-end, Carlo Rosso si occuperà dello studio e della formazione inerente le tecnologie PostgreSQL e Node.js.\\

In merito alla realizzazione del PoC il gruppo si è organizzato per contattare il referente dell'azienda Imola Informatica al fine di avere delucidazioni sulle componenti da inserire al suo interno.


