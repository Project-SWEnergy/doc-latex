\section{\textit{Brainstorming}}

Durante la presente sessione interna, il precedente responsabile, Carlo Rosso, ha condotto
una sintesi dettagliata della situazione attuale del progetto. Ha posto particolare enfasi
sull'incontro avvenuto con il proponente il 15/12/2023 e sul diario di bordo che si è svolto il
giorno successivo (18/12/2023) con il Professore Vardanega. Il gruppo ha successivamente intrapreso
una discussione approfondita relativa alle tecnologie impiegate nell'attuale fase di sviluppo del PoC,
identificando inoltre eventuali dubbi e correzioni da apportare al documento dell'analisi
dei requisiti in previsione dell'imminente incontro con il Professor Cardin.
\\

\subsection*{Diario di bordo:}
Durante il diario di bordo si è fatto più chiarezza e si è parlato di vari aspetti del progetto:

\begin{itemize}
	\item 	\textbf{PdQ}:
	      Il "Piano di qualifica" si riferisce alla valutazione della qualità attraverso
	      l'utilizzo di metriche obiettive.
	      La sua funzione principale consiste nella misurazione della qualità prodotta in confronto
	      agli standard obiettivi.
	      Gli obiettivi primari di un PdQ sono orientati verso la qualità attesa.


	\item 	\textbf{PdP}:
	      Il Piano di Progetto (PdP) svolge un ruolo essenziale nel fornire una visione chiara e dettagliata
	      dello stato e degli obiettivi del progetto.
	      La sua struttura deve comprendere obiettivi di costo, la situazione attuale del progetto e impegni
	      specifici per migliorare le prestazioni.

	      Il PdP implica una pianificazione degli obiettivi, una strategia per affrontarli e un processo continuo di riscontri.
	      Questo processo si alimenta alla conclusione di ogni sprint, basandosi sull'apprendimento derivato
	      dagli esiti ottenuti e orientando la pianificazione futura.
	      In sintesi, il PdP costituisce un documento dinamico, riflettendo l'evoluzione del progetto e
	      offrendo una guida chiara per il miglioramento continuo.

	\item 	\textbf{Quantità di Requisiti}:
	      La determinazione della quantità di requisiti assume un aspetto cruciale nello sviluppo del progetto.
	      È consigliabile che il nostro insieme di requisiti abbia una cardinalità di almeno cinquanta,
	      al fine di garantire una copertura esaustiva e un'adeguata comprensione delle specifiche del sistema.
	      Questo criterio rappresenta un parametro fondamentale nella fase iniziale del processo di ingegneria del software,
	      poiché un numero sufficientemente ampio di requisiti favorisce una definizione più completa e dettagliata del prodotto finale,
	      facilitando inoltre le successive fasi di progettazione e implementazione.

	\item 	\textbf{Metriche}:
	      Per garantire la qualità del prodotto, è essenziale stabilire obiettivi tramite l'implementazione di metriche,
	      selezionate con attenzione per evitare l'inclusione casuale di misurazioni professionali senza chiara rilevanza.
	      Le tipologie di metriche comprendono la valutazione della qualità del prodotto e la misurazione delle attività.

	      Queste metriche forniscono una visione chiara della nostra posizione rispetto agli obiettivi prefissati e del
	      confronto tra intenzioni progettuali ed effettiva implementazione.
	      Utilizzando un cruscotto con grafici temporali, monitoriamo l'andamento nel tempo, raccogliendo dati ad ogni
	      aggiornamento del \textit{repository} per creare una \textit{baseline} di partenza.


	\item 	\textbf{PoC}:
	      Il Proof of Concept (PoC) riveste un ruolo fondamentale nel nostro processo di sviluppo,
	      mirando a valutare l'idoneità delle tecnologie adottate per la realizzazione delle idee progettuali.
	      Le sue funzioni principali si concentrano su due input e due output.

	      Il PoC si propone di: \\
	      \textbf{Input:}
	      \begin{itemize}
		      \item Individuare le tecnologie necessarie.
		      \item Verificare la fattibilità, ovvero se le tecnologie utilizzate riescono a implementare ciò che è stato concepito.
	      \end{itemize}

	      \textbf{Output:}
	      \begin{itemize}
		      \item Confermare se le tecnologie individuate soddisfano le aspettative, fornendo riscontri sulla compatibilità tecnologica.
		      \item Approfondire l'analisi dei requisiti, poiché potrebbe rivelare nuovi requisiti precedentemente non considerati,
		            migliorando così la comprensione del contesto progettuale.
		      \item Emettere \textit{feedback} e rispondere alle domande pre-PoC costituisce una pratica essenziale per valutare le decisioni
		            iniziali e acquisire una comprensione approfondita delle sfide progettuali.
	      \end{itemize}
\end{itemize}

\section{Analisi dei Requisiti}

Nella recente settimana, i precedenti analisti, Davide Maffei, Matteo Bando e Giacomo Gualato,
hanno dedicato tempo ed energie alla continua esplorazione dei casi d'uso e alla ricerca accurata dei requisiti.
Durante la fase di stesura e discussione del documento, sono emersi diversi dubbi,
alcuni dei quali ancora aperti e destinati a essere esaminati nel prossimo incontro con il Professore Cardin,
la cui data è attualmente in via di definizione.

I principali interrogativi che hanno suscitato riflessioni approfondite includono:

\begin{itemize}
	\item La scelta tra l'inclusione di poche immagini con un diagramma esteso o l'adozione di più immagini con diagrammi più contenuti e raggruppati.
	\item L'opportunità di utilizzare più codici per identificare i casi d'uso (UCB/UC/UCE/UCG/UCA).
	\item La validità dell'inclusione di uno o più casi d'uso relativi agli errori, e la necessità di una definizione chiara del concetto di "errore".
	\item Nel contesto dei casi d'uso relativi alle notifiche, la determinazione dell'attore principale: il sistema o l'utente generico.
	\item La valutazione dell'approccio \textit{Layered} come architettura.
	\item La rappresentazione di interazioni di \textit{chat} tra due utenti che eseguono azioni simili.
	\item La verifica della correttezza della struttura utilizzata per gli UC, che attualmente comprende descrizione,
	      attore principale, attore secondario, precondizioni, postcondizioni, scenario principale e scenari secondari, se presenti.
\end{itemize}

Questi interrogativi costituiscono punti critici che richiedono ulteriore discussione al fine di garantire la coerenza
e la completezza del documento di "Analisi dei requisiti".
La prossima interazione con il Professore Cardin fornirà l'opportunità ideale per esaminare dettagliatamente questi aspetti
e definire le direzioni migliori per il proseguimento del lavoro.



\section{PoC}
Perseguendo la realizzazione del \textit{Proof of Concept} (PoC), è stata assunta la decisione di procedere, al momento, con una progettazione dell'architettura
\textit{layered}.
Il proseguimento dell'attività di sviluppo è previsto sia per il backend che per il frontend durante il periodo delle festività natalizie,
confermando l'impegno continuo nella fase di implementazione.
Questo approccio consente di garantire una costruzione robusta e coerente del PoC, preparando il terreno per le fasi successive del progetto.


\section{Norme di progetto}
Le "Norme di progetto" assumono il ruolo di delineare le modalità con cui affrontiamo le attività.
Questa settimana, e eventualmente anche la successiva, vedrà la definizione dettagliata dei processi e delle procedure
implementate nel contesto del progetto.
L'obiettivo è garantire una chiara e coerente esecuzione delle attività, contribuendo alla gestione efficace delle
risorse e alla realizzazione di un lavoro di alta qualità.


\section{Piano di progetto}
Il "Piano di progetto" continuerà a concentrarsi sull'analisi dei preventivi e consuntivi,
nonché sull'organizzazione del tempo attuale. L'attenzione si focalizzerà sull'attuale attività in corso,
inclusi il calcolo e la valutazione di preventivi e consuntivi attraverso un programma dedicato.
Questa fase mira a garantire una gestione finanziaria efficace e una chiara organizzazione temporale,
contribuendo così al successo complessivo del progetto.


\section{Gestione del tempo delle attività}
La gestione del tempo prevede la suddivisione delle \textit{issue} in due gruppi, che operano in modalità di autogestione.
Al fine di ottimizzare l'efficienza, i gruppi hanno la facoltà di aggiungere \textit{task}, strutturandole in maniera atomica.
Inoltre, ogni gruppo dichiara una stima temporale per la risoluzione delle rispettive task,
promuovendo una pianificazione accurata e una gestione tempestiva delle attività.


\section{Cambio dei ruoli}
I ruoli appena designati sono consultabili nella tabella presente nel documento.
Attualmente, ciascun membro concluderà le attività in corso prima di iniziare ufficialmente un nuovo sprint,
integrando così i nuovi ruoli assegnati. Questo approccio è concepito per assicurare una transizione senza intoppi
e una transizione fluida alle responsabilità aggiornate del team.

Durante il periodo delle vacanze, non verranno effettuati ulteriori cambiamenti di ruoli.
Tuttavia, è previsto un meeting tra il 26 e il 30, escludendo la data del 26 e la serata del 29,
per discutere eventuali aggiustamenti e coordinare le attività future.
