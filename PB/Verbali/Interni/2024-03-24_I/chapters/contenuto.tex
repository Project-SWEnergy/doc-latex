\section{\textit{Brainstorming}}

Davide e Niccolò hanno spiegato il lavoro svolto durante la settimana. In
particolare sono state spiegate le correzioni apportate a ciascun documento.

\section{Revisione delle correzioni}

Poi si è dedicato del tempo alla discussione sulle correzioni non
pienamente comprese, ovvero quelle segnalate dai docenti ma che non hanno
generato modifiche nei documenti. Si è riconosciuto che la comprensione del
problema e delle relative richieste è risultata difficile, pertanto sono state
formulate domande da rivolgere ai docenti. È stato concordato di richiedere un
incontro con loro per chiarire tali correzioni, al fine di migliorare la qualità
degli artefatti prodotti dal gruppo.

\section{Frontend}

Per quanto riguarda il frontend, si è discusso in merito alle funzionalità da
implementare in questo sprint. Non solo, è stato deciso in linea di massima la
struttura del lavoro da svolgere, in modo da poter iniziare a sviluppare il
frontend. In particolare, è stato deciso di sviluppare le pagine di
autenticazione e di registrazione per quanto riguarda i clienti e i ristoratori.
Questo prevede l'utilizzo dei JWT per l'autenticazione, per cui bisogna
approfondire anche questo aspetto tecnologico, per implementarlo correttamente.
