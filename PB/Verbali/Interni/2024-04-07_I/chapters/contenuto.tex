\section{\textit{Brainstorming}}
I vari membri del gruppo hanno esposto il lavoro effettuato nel corso dello \textit{sprint} appena concluso.\\
In particolare in merito alla documentazione, Carlo Rosso si è occupato di avviare la redazione del documento "Specifica tecnica". 
Il documento "Analisi dei requisiti" è stato correttamente aggiornato a seguito delle correzioni emerse al termine della fase RTB.
In merito allo sviluppo si riscontra una corretta conclusione delle componenti relative alle operazioni di registrazione ed autenticazione delle due tipologie di utenza previste. \\
La fase iniziale della riunione si è conclusa con la condivisione delle informazioni ottenute durante il diario di bordo avvenuto in data 05/05/2024.

\section{Documentazione}
In merito alla documentazione il gruppo si è confrontato sul contenuto del documento "Specifica tecnica", in particolare in merito alla presenza di una sezione che indichi i requisiti correttamente implementati all'interno del prodotto.\\
Sono state definite le modifiche da apportare ai documenti "Norme di progetto", "Piano di qualifica" e "Piano di progetto".

\section{Sviluppo}
In seguito alla dichiarazione di interruzione del servizio, il gruppo ha affrontato la problematica relativa all'abbandono della piattaforma "FL0" in modo da evitare possibili problematiche a ridosso della consegna conclusiva del progetto.
A tal proposito si è scelto di utilizzare Docker$^G$, come nella prima fase di progetto, per la gestione di un \textit{database} locale.\\

Il gruppo ha poi definito i successivi obiettivi legati allo sviluppo, decidendo di dedicarsi al completamento del profilo inerente il ristorante.
Saranno quindi implementati i componenti necessari al completo inserimento e gestione dei piatti in elenco.
Per quanto riguarda l'utenza base sarà realizzata l'interfaccia di selezione dei ristoranti presenti all'interno del database.