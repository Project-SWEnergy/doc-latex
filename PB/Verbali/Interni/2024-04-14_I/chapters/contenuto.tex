\section{\textit{Brainstorming}}
Durante la riunione, i vari membri del gruppo hanno presentato i progressi compiuti 
durante la prima parte dello \textit{sprint} attuale.

Il gruppo responsabile dello sviluppo \textit{back-end} ha segnalato il progresso conforme al piano 
e le modifiche apportate alla gestione degli errori, ora in linea con gli standard di sviluppo.

Il gruppo dedicato allo sviluppo \textit{front-end} ha annunciato il completamento delle attività relative 
alla registrazione, all'autenticazione, alla visualizzazione della homepage e dei ristoranti per un utente generico$^G$, 
nonché alla gestione dei piatti per tutte le categorie di utenti previste. Inoltre, sono stati eseguiti i test di unità sulle componenti implementate.

È emerso un rallentamento generale rispetto alle aspettative. Durante lo sviluppo, sono state rilevate problematiche 
non individuate durante la fase di creazione del PoC, sia sul \textit{front-end} che sul \textit{back-end}. 
Ciò ha reso necessaria l'implementazione di modifiche per migliorare il codice.


\section{Documentazione}
Il gruppo ha deliberato l'aggiornamento dei seguenti documenti:
\begin{itemize}
    \item \textbf{Manuale utente}: si procederà con la redazione della sezione riguardante le attività di autenticazione e registrazione.
    \item \textbf{Piano di qualifica}: saranno aggiornate le metriche e i test di unità effettuati.
    \item \textbf{Piano di progetto}: Si procederà con l'aggiornamento relativo agli \textit{sprint} conclusi.
\end{itemize}


\section{Sviluppo}
Il \textit{team} incaricato dello sviluppo del back-end si concentrerà sul completamento delle operazioni CRUD 
per garantire una gestione completa del database. Saranno disponibili per affrontare richieste più 
specifiche provenienti dal \textit{team} \textit{front-end}, garantendo una collaborazione sinergica tra le due squadre 
per il successo del progetto.

Il \textit{team front-end} si dedicherà allo sviluppo delle componenti destinate alla gestione delle prenotazioni. 
Questo coinvolge la creazione di un'interfaccia utente intuitiva e funzionale sia per i clienti che per 
i ristoratori. Gli utenti saranno in grado di effettuare prenotazioni, visualizzare le loro prenotazioni 
attive e gestire le prenotazioni esistenti. Inoltre, i ristoratori avranno accesso a strumenti specializzati 
per gestire le prenotazioni, consentendo loro di accettare, rifiutare o modificare le prenotazioni in base 
alle loro esigenze operative.

Questa suddivisione dei compiti e delle responsabilità assicurerà che il sistema sia sviluppato in modo 
completo e coerente, offrendo un'esperienza personale ottimale sia sul \textit{front-end} che sul\textit{back-end}.



\section{Incontro con il referente aziendale}
Durante l'incontro con il referente aziendale, Alessandro Staffolani, il gruppo ha stabilito una serie di 
quesiti da discutere e formalizzare. In particolare, si richiederà il suo parere su diversi argomenti chiave:
\begin{itemize}
    \item Scelta dell'Utilizzo dei \textit{cookie} per la Gestione dei \textit{token} di Autenticazione: 
    Si desidera ottenere un chiarimento sull'opportunità di utilizzare i cookie come metodo per gestire i token di autenticazione nel sistema. 
    
    \item Redazione del Manuale Utente: 
    Si chiederà il suo input riguardo alla redazione del manuale utente. Si vuole assicurare che il manuale sia completo, chiaro e informativo, 
    in modo da guidare gli utenti attraverso tutte le funzionalità e le operazioni del sistema.

    \item Gestione del Sistema di Notifica: 
    Si discuterà della gestione del sistema di notifica e di come garantire che le notifiche siano tempestive, pertinenti e facilmente gestibili dagli utenti. 

L'incontro con il referente aziendale sarà un'opportunità importante per ottenere \textit{feedback} e consulenza su queste questioni cruciali, 
contribuendo così a guidare il progetto nella direzione giusta e assicurando il suo successo complessivo.
