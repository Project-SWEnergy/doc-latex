\section{\textit{Brainstorming}}
I vari membri del gruppo hanno esposto il lavoro effettuato nel corso della prima parte dello \textit{sprint} in corso.\\
Il gruppo che si occupa dello sviluppo \textit{back-end} ha notificato l'avanzamento del lavoro pianificato e l'avvenuta modifica della gestione degli errori, che ora risulta più affine agli \textit{standard} di sviluppo.\\
Il gruppo che si occupa dello sviluppo \textit{front-end} ha dichiarato il completamento delle attività inerenti la registrazione ed autenticazione, la visualizzazione di \textit{homepage} e ristoranti per un utente generico$^G$, la gestione dei piatti per tutte le tipologie di utenti previste.
Sono stati infine eseguiti i test di unità sulle componenti implementate.\\
In generale è stato riscontrato un rallentamento del lavoro rispetto alle aspettative.
Durante lo sviluppo sono state notate problematiche che non erano state riscontrate durante la creazione del PoC, sia lato \textit{back-end} che \textit{front-end}, questo ha portato all'implementazione di modifiche migliorative al codice.


\section{Documentazione}
Il gruppo ha deciso di procedere con l'aggiornamento dei documenti:
\begin{itemize}
    \item \textbf{Manuale utente}: redazione della porzione relativa alle attività di autenticazione e registrazione.
    \item \textit{Piano di qualifica}: aggiornamento delle metriche e dei test di unità effettuati.
    \item \textbf{Piano di progetto}: aggiornamento relativo agli \textit{sprint} conclusi.
\end{itemize}


\section{Sviluppo}
I membri del gruppo assegnati allo sviluppo del \textit{back-end} si dedicheranno al completamento delle operazioni CRUD per la completa gestione del database, rimanendo a disposizione in caso avvengano richieste più specifiche da parte del gruppo assegnato al \textit{front-end}.\\
Quest'ultimo di impegnerà nello sviluppo delle componenti dedicate alla prenotazione, sia lato cliente che ristoratore.

\section{Incontro con il referente aziendale}
In merito all'incontro con il referente aziendale Alessandro Staffolani, il gruppo ha formalizzato alcune delle domande da porre.\\
In particolare sarà richiesto il suo parere in merito a:
\begin{itemize}
    \item scelta dell'utilizzo dei \textit{cookie} per la gestione dei token di autenticazione;
    \item redazione del manuale utente;
    \item gestione del sistema di notifica.
\end{itemize}