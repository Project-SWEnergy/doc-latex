\section{\textit{Brainstorming}}

All'inizio della riunione ciasun membro del gruppo ha presentato il lavoro
svolto. In particolare, sono stati aggiornati i requisiti soddisfatti
all'interno del documento specifiche tecniche.  
Dunque, per quanto riguarda il frontend, è stata implementata la sezione che si
occupa di gestire gli ingredienti per i ristoratori.

\section{Implementazione dell'ordine condiviso}

Si è discusso rispetto al metodo implementativo per la gestione dell'ordine
condiviso. In particolare, sono state discusse le api da implementare lato
backend per la gestione dell'ordine condiviso; non solo, è stata discussa anche
la firma delle api, ovvero, quali parametri sono necessari al backend per poter
gestire in modo corretto l'ordine condiviso e in quale forma dovrebbero essere
passati per mantenere semplice e leggibile il codice.

\section{Rimozione di ingredienti}

Il database del backend non prevede la rimozione di alcuni ingredienti da parte
dei clienti, per quanto essa sia una funzionalità obbligatoria, quindi si è
discusso rispetto al metodo più efficace ed efficiente per implementare tale
\textit{feature}. Si è deciso di aggiungere una tabella, che abbina a ciascun
piatto ordinato una lista di ingredienti da rimuovere, per cui ciascun
abbinamento tra il piatto e l'ingrediente in questa tabella rappresenta un
ingrediente da rimuovere dal piatto. Infine, si sono discusse le implicazioni
che questa scelta comporta rispetto alla gestione degli ingredienti da parte del
ristoratore: infatti il ristoratore dovrebbe essere in grado di visualizzare gli
ingredienti necessari per la preparazione di tutti i piatti in un arco
temporale specificato.

\section{Logout}

Si è notata la mancanza di un'api per eseguire il logout, necessaria poiché i
cookie non sono accessibili da javascript. Inoltre la gestione dell'indirizzo
di un ristorante dovrebbe essere gestita insieme ai dati del ristorante
medesimo.

\section{Pianificazione}

Si è discusso rispetto alla pianificazione del lavoro da svolgere. Il team del
backend ha spiegato che lasciando da parte l'implementazione della chat, il
completamento del lavoro inerente al backend dovrebbe concludersi in circa 3
settimane, associando più o meno una settimana all'avanzamento della
documentazione, una settimana per completare le \textit{feature} rimanenti e una
settimana per il \textit{testing}. Il team del rontend ha evidenziato che le medesime
tempistiche si applicano anche al loro lavoro. Aggiungendo la chat a queste
stime si prevede di completare il lavoro in 5 settimane, a meno di ulteriori
ritardi. Dunque SWEnergy ha discusso rispetto ad eventuali requisiti obbligatori
che potrebbero essere rimossi per rispettare le tempistiche. In particolare, si
prevede di richiedere al proponente di rimuovere la chat come requisito
obbligatorio, in modo da poter completare il lavoro nei tempi previsti.
