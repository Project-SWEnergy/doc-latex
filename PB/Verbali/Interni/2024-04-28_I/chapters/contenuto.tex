\section{\textit{Brainstorming}}

I membri del frontend hanno spiegato le funzionalità che sono state completate
nel mini-sprint concluso. In particolare, è stata completata la gestione delle
immagini e la visualizzazione delle immagini nei modi richiesti dal proponente.
Viene completata la gestione degli ingredienti e l'abbinamento degli ingredienti
ai piatti. Infine, è stata completata la gestione di un'ordinazione condivisa.\\
Per quanto riguarda il backend, nel mini-sprint concluso sono state completate
le CRUD per la gestione del database e sono state implementate le api di
gestione delle immagini.

\section{Preparazione del SAL}

Il gruppo ha discusso rispetto agli argomenti da trattare durante il SAL ed è
stato concordato il giorno, ovvero martedì 30 aprile. Si chiede al responsabile
di fissare l'orario con il proponente.\\
In generale, il gruppo si rende conto che il progetto è in ritardo rispetto al
piano di lavoro, per questo motivo si vuole chiedere al proponente se è
possibile ridefinire i requisiti e le scadenze.

\subsection{Chat}

Il gruppo ha discusso rispetto alla chat e si vuole chiedere al proponente di
rimuoverla dai requisiti obbligatori e quindi dall'MVP. Questo permette di
tagliare drasticamente i tempi, infatti il gruppo ha dedicato poco tempo
all'analisi della chat e non ha ancora iniziato a lavorarci.

\subsection{Pagamento}

Il gruppo ha un dubbio rispetto al pagamento, infatti non è chiaro se il
proponente voglia che sia implementato un sistema di pagamento funzionante,
oppure se sia sufficiente sviluppare un sistema di pagamento fittizio,
modificabile in modo tale da poter essere sostituito con un sistema di pagamento
reale. Si chiede al proponente di chiarire questo punto.

\section{Setup}

Il frontend ha bisogno di un'api che permetta di capire se l'utente è loggato.
Dunque viene richiesta l'implementazione di un'api che ritorni un oggetto così
descritto:

\begin{lstlisting}
{
	result: boolean,
	message: string,
	data: [{
		"id": number,
		"role": "user" | "restaurant"
	}]
}
\end{lstlisting}

Così che, se un utente ha i cookie di autenticazione, possa risultare loggato
anche nel frontend.

\section{Unità di misura}

Gli ingredienti associati ai piatti devono avere una quantità, in modo tale che
il ristoratore avere un'indicazione di quanto acquistarne. Questo vuol dire che
ciascun ingrediente ha bisogno anche di un'unità di misura. Il gruppo ha
discusso rispetto al modo in cui gestire l'unità di misura degli ingredienti ed
è stato concordato di modificare la tabella \texttt{ingredients} aggiungendo il
campo \texttt{unit\_of\_measurement}. In questo modo, l'unità di misura diventa
indipendente dal piatto, viene inserita direttamente nell'ingrediente ed è
possibile sommare i valori degli ingredienti di ciascun piatto.

\section{Inserimento degli orari}

Per quanto riguarda l'inserimento degli orari di apertura del ristorante, il
gruppo ha discusso rispetto all'api che permette di inserire un orario di
apertura. Viene concordato che l'api dovrebbe richiedere un payload del tipo:

\begin{lstlisting}
[
	{
		"id_day": string,
		"start": string,    // formato HH:MM:SS
		"end": string       // formato HH:MM:SS
	},
]
\end{lstlisting}

In questo modo è possibile inserire più orari di apertura effettuando una sola
chiamata al backend.
