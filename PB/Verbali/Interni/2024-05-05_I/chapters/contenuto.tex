\section{\textit{Brainstorming}}

I membri del frontend hanno mostrato le funzionalità implementate, in
particolare è stata mostrata la funzionalità di visualizzazione delle
prenotazioni lato ristoratore, la quale implementa anche la visualizzazione
degli ingredienti necessari per completare le prenotazioni visualizzate.
Inoltre, viene mostrato il funzionamento dell'inserimento dei \textit{feedback}
lato utente base. Viene mostrata l'implementazione della registrazione di un
ristorante, tuttavia viene evidenziato un \textit{bug} che non permette il
reindirizzamento alla pagina home dopo la registrazione. Viene aggiornata la
gestione degli ingredienti con le nuove api fornite dal backend. Infine viene
completata la gestione di un ordine lato utente base.

\section{Notifiche}

Il gruppo ha discusso riguardo alla gestione delle notifiche, in particolare è
stato deciso di evitare di implementare il sistema di notifiche mediante socket
perché si tratta di una tecnologia non ancora toccata e considerando il tempo a
disposizione si è deciso di implementare un sistema di notifiche tramite
\textit{polling}. Infine viene richiesto al backend di implementare un endpoint
univoco per prelevare le notifiche; si richiede al frontend di implementare una
pagina di visualizzazione delle notifiche sia lato utente base che lato
utente ristoratore.

\section{Prenotazione}

Giacomo Gualato ha implementato la funzionalità di prenotazione lato utente,
tuttavia tale implementazione risulta funzionante solamente in locale, per cui
viene richiesto a Davide Maffei di reimplementare tale funzionalità.

\section{Testing}

Si nota che la copertura dei test non è soddisfacente, per cui si richiede uno
sforzo maggiore da parte di tutti i membri del gruppo per aumentare la copertura
dei test. In particolare, SWEnergy vuole raggiungere una copertura almeno
dell'80\% per quanto riguarda gli statement e i branch testati. Non solo, si
nota che è problematico implementare nuove funzionalità senza avere una buona
copertura dei test, per cui si richiede maggiore diligenza da parte di tutti i
membri del gruppo.

\section{Requisiti}

Il gruppo ha discusso rispetto ad un eventuale raggiungimento dell'MVP. Si
ritiene plausibile il raggiungimento dell'MVP entro venerdì 17 maggio. Tuttavia,
per raggiungere tale obiettivo è necessario concordare con il proponente quali
requisiti sono necessari per raggiungere tale obiettivo. Dunque si è deciso di
aggiornare l'Analisi dei Requisiti in modo tale da avere una lista di requisiti
obbligatori soddifacibili entro il 17 maggio. Tale lista sarà poi rivista nel
prossimo \textit{meeting} interno ed infine sarà condivisa con il proponente,
per avere un riscontro.
