\section{\textit{Brainstorming}}

Sono state mostrate le nuove funzionalità implementate. In particolare è stata
corretta la registrazione di un ristorante. Viene mostrata la nuova interfaccia
per la gestione delle prenotazioni lato ristoratore e il dettaglio di una
prenotazione. Viene mostrata la pagina di visualizzazione delle notifiche.
Infine viene mostrata la pagina di gestione delle prenotazioni lato utente base
e la gestione di una prenotazione lato utente base. Viene notato qualche
problema nel dettaglio di una prenotazione, per cui è richiesta una correzione.
Viene riassunto lo stato di copertura dei test: il frontend è coperto oltre il
90\% per quanto riguarda gli statement e supera l'85\% per quanto riguarda i
branch coperti. Il backend arriva a coprire il 75\% dei moduli.

\section{Preparazione del SAL}

Dopo aver riassunto lo \textit{status quo} del progetto, sono stati discussi i
dubbi da esporre al committente. Così si è deciso di chiedere un SAL per
aggiornare il committente sullo stato del progetto e per discutere i dubbi
emersi. In particolare è stata aggiornata l'Analisi dei Requisiti, per questo
motivo si vuole confermare che il proponente sia d'accordo con le modifiche
apportate. Inoltre, SWEnergy ha bisogno di comprendere se il proponente voglia
ricevere i documenti: Manuale Utente e Specifiche Tecniche, prima o dopo il test
di accetazione e dunque con la consegna del prodotto. Infine, il gruppo ha
deciso di condividere la \textit{repository} del progetto \textit{Easy-Meal} con
il proponente, per quanto non sia stata ancora rilasciata una versione stabile.

\section{Repository unica}

Il gruppo ha discusso rispetto alle caratteristiche che dovrebbe avere una
\textit{repository} unica del \textit{software EasyMeal}. In particolare, si è 
deciso che il prodotto debba partire con un unico comando: 
\texttt{docker compose up}. Poi devono essere definiti i comandi per gestire il
database, in particolare per effettuare la prima migrazione e per popolare il
database con dati di esempio. Si è discusso in merito all'implementazione delle
github action per il controllo della qualità del codice e per la formattazione
del codice, in merito a quest'ultimo punto si è deciso di dare una bassa
priorità per il momento. Infine nella \textit{repository} unica deve essere
presente un \textit{README} che illustri lo scopo del progetto, come avviarlo e
come utilizzare il software. Rimane in sospeso la formalizzazione della
visualizzazione dei test perché il \textit{frontend} e il \textit{backend} 
utilizzano due \textit{framework} diversi per i test.
