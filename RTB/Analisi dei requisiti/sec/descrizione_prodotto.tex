\section{Descrizione del prodotto}
Il progetto mira a migliorare l'esperienza nei ristoranti sia per i clienti che per i ristoratori. Si concentrerà sulle difficoltà legate alle prenotazioni e agli ordini, 
semplificando questi processi attraverso un'applicazione web responsiva. L'app consentirà agli utenti di prenotare tavoli in modo intuitivo, personalizzare gli ordini in base alle proprie preferenze alimentari e favorire 
l'interazione tra clienti e personale del ristorante. Inoltre, faciliterà la divisione del conto e promuoverà la scrittura di recensioni

\subsection{Scopo del prodotto}
Non sono sicuro che questa subsection sia necessaria, può essere sufficiente
quella precedente. <----- CREDO BASTI LA PARTE SOPRA

\subsection{Funzionalità}

\subsection{Attori}
Gli attori che interagiscono con il sistema sono i seguenti:
% Servonno anche le immagini?
\begin{itemize}
    \item Utente esterno: si tratta di un utente che non ha eseguito l'autenticazione
    \item Utente base: si tratta di un attore autenticato, rappresenta un possibile cliente di un ristorante
    \item Utente ristoratore: si tratta di un attore autenticato, rappresenta l'amministratore del ristorante
\end{itemize}



\subsection{Requisiti non funzionali}

