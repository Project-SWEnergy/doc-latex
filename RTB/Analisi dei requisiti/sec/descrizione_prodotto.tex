\section{Descrizione del prodotto}
Il progetto mira ad ottimizzare l'esperienza nei ristoranti sia per i clienti che per i gestori.
L'attenzione sarà focalizzata sulle sfide legate alle prenotazioni e agli ordini, semplificando tali 
processi attraverso un'applicazione \textit{web} responsiva.
Questa \textit{app} consentirà agli utenti di effettuare prenotazioni in modo intuitivo, personalizzare 
gli ordini in base alle proprie preferenze alimentari e favorire l'interazione tra i clienti e 
il personale del ristorante. \\
Inoltre, agevolerà la divisione del conto e promuoverà la scrittura di recensioni.


\subsection{Funzionalità}

\begin{itemize}
	\item \textbf{Registrazione di nuovi utenti:} Gli utenti hanno la possibilità di creare un account per accedere a tutte le funzionalità dell'applicazione.
	\item \textbf{Prenotazione di un tavolo:} Gli utenti possono effettuare la prenotazione di un tavolo in un ristorante in base alla disponibilità.
	\item \textbf{Ordinazione collaborativa dei pasti:} Gli utenti possono collaborare per ordinare i pasti, consentendo a ciascuno di aggiungere piatti al carrello.
	\item \textbf{Interazione con lo staff del ristorante:} Gli utenti hanno la facoltà di comunicare con lo staff del ristorante per fare richieste speciali o risolvere eventuali problemi.
	\item \textbf{Divisione del conto:} L'applicazione offre la possibilità di dividere il conto tra gli utenti in modo equo o personalizzato.
	\item \textbf{Consultazione delle prenotazioni da parte di un amministratore del ristorante:} Gli amministratori del ristorante possono consultare le prenotazioni per gestire la disponibilità dei tavoli.
	\item \textbf{Inserimento di \textit{feedback} e recensioni:} Gli utenti possono lasciare \textit{feedback} e recensioni sui ristoranti e sui piatti ordinati.
\end{itemize}

\subsection{Attori}
Gli attori che interagiscono con il sistema sono i seguenti:
% Servonno anche le immagini?
% Manca Utente generico
\begin{itemize}
	\item \textbf{Utente generico:} si tratta di un utente che non ha eseguito l'autenticazione
	\item \textbf{Utente base:} si tratta di un attore autenticato, rappresenta un possibile cliente di un ristorante
	\item \textbf{Utente ristoratore:} si tratta di un attore autenticato, rappresenta l'amministratore del ristorante
\end{itemize}
