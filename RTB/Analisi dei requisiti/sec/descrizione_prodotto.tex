\section{Descrizione del prodotto}
Il progetto mira a migliorare l'esperienza nei ristoranti sia per i clienti che per i ristoratori. 
Si concentrerà sulle difficoltà legate alle prenotazioni e agli ordini, semplificando questi processi attraverso un'applicazione \textit{web} responsiva. 
L'\textit{app} consentirà agli utenti di prenotare tavoli in modo intuitivo, personalizzare gli ordini in base alle proprie preferenze alimentari e favorire l'interazione tra clienti e personale del ristorante. 
Inoltre, faciliterà la divisione del conto e promuoverà la scrittura di recensioni.

\subsection{Funzionalità}
\paragraph*{Ristoratori} Avranno la possibilità di creare una vetrina relativa al proprio ristorante e di presentare ai clienti i loro prodotti.\\
Potranno ricevere e gestire le prenotazioni \textit{online}, eventualmente dialogando con i clienti tramite \textit{chat} per rispondere alle loro necessità.
Per i ristoratori sarà possibile visualizzare in anticipo le ordinazioni effettuate dai clienti, si potranno quindi ottimizzare le risorse necessarie alla preparazione dei piatti, inoltre potranno consultare statistiche utili a comprendere le preferenze dei loro clienti.
\paragraph*{Clienti} Potranno esaminare le offerte dei ristoranti presenti nell'applicativo, prenotare un tavolo ed effettuare una ordinazione preventiva dei piatti.\\
La prenotazione e l'ordinazione potrà avvenire in condivisione con altri utenti dell'applicazione, ogni utente invitato potrà quindi aggiungere piatti all'ordine collettivo. 
Al termine della consumazione sarà possibile visualizzare la suddivisione del conto in base alle ordinazioni effettuate.
Infine ogni cliente che ha effettuato una consumazione avrà la possibilità di recensire il locale.

\subsection{Attori}
Gli attori che interagiscono con il sistema sono i seguenti:
% Servonno anche le immagini?
% Manca Utente generico
\begin{itemize}
    \item Utente esterno: si tratta di un utente che non ha eseguito l'autenticazione
    \item Utente base: si tratta di un attore autenticato, rappresenta un possibile cliente di un ristorante
    \item Utente ristoratore: si tratta di un attore autenticato, rappresenta l'amministratore del ristorante
\end{itemize}



\subsection{Requisiti non funzionali}

