\section{Introduzione}

\subsection{Scopo del documento}
Nel presente documento, viene presentata una descrizione dettagliata del prodotto, basata sull'analisi dei bisogni dell'utente 
emersi durante l'esaminazione del capitolato e attraverso gli incontri con l'azienda proponente. 
La modellazione viene realizzata tramite UML$^G$, identificando in modo approfondito requisiti e attori presenti nel progetto. 
Questo approccio consente di descrivere in dettaglio le varie componenti del prodotto e di indicare la struttura di ciascuna funzionalità.\\

\noindent
I casi d'uso sono strutturati nel seguente modo:
\begin{itemize}
	\item Titolo: fornito di un codice identificativo e di un nome esplicativo per agevolare il tracciamento.
	\item Attori: diverse tipologie di utenti che interagiscono con il sistema, suddivisi in:
	\begin{itemize}
		\item Attore primario: colui che interagisce attivamnete con il sistema dall'esterno.
		\item Attore secondario: posso anche non essere presenti. 
		Sono sempre esterni, ma non interagiscono attivamente con il sistema; invece, interagiscono per il raggiungimento dello scopo dell'attore primario o per aiutarlo.
	\end{itemize}
	\item Precondizioni: stato in cui si deve trovare il sistema affinchè una funzionalità sia disponibile ad un attore.
	\item Postcondizioni: serie di informazioni che rappresentano lo stato del sistema dopo l'esecuzione del caso d'uso.
	\item Scenario principale: sequenza di azioni dettagliata che descrive il \textit{workflow} della funzionalità.
	\item Descrizione: note aggiuntive inserite quando si ritiene utile ampliare la spiegazione per approfondire la comprensione del caso d'uso.
	\item Scenario secondario: scenario che inizialmente ha lo stesso \textit{workflow} di quello principale ma che, ad un tratto, cambia, solitamente relativo a errori.
	\item Trigger: evento scatenante che provoca l'esecuzione automatica del caso d'uso.
\end{itemize}

\subsection{Riferimenti}
\subsubsection{Riferimenti normativi}
\begin{itemize}
    \item \href{https://www.math.unipd.it/~tullio/IS-1/2023/Progetto/C3.pdf}{Capitolato C3 - \textit{Easy Meal}}.
    \item \href{https://project-swenergy.github.io/}{Norme di progetto v2.0.0}.
    \item \href{https://www.math.unipd.it/~tullio/IS-1/2023/Dispense/PD2.pdf}{Regolamento progetto didattico}.
\end{itemize}

\subsubsection{Riferimenti informativi}
\begin{itemize}
    \item \href{https://www.math.unipd.it/~tullio/IS-1/2023/Dispense/T5.pdf}{Lezione T05 - Analisi dei requisiti}.
    \item \href{https://www.math.unipd.it/~rcardin/swea/2023/Diagrammi%20delle%20Classi.pdf}{Diagrammi delle classi}.
    \item \href{https://www.math.unipd.it/~rcardin/swea/2022/Diagrammi%20Use%20Case.pdf}{Diagrammi dei casi d'uso}.
\end{itemize}