\section{Introduzione}

\subsection{Scopo del documento}
Nel seguente documento si fornisce una descrizione del prodotto, analizzando i bisogni dell'utente che sono emersi in seguito all'analisi del capitolato e agli incontri effettuati con l'azienda proponente. \\
La modellazione avviene tramite UML$^G$ individuando nel dettaglio requisiti ed attori presenti nel progetto, in tal modo sarà possibile descrivere le varie componenti del programma ed indicare come strutturare ogni attività. \\

\noindent
I casi d'uso vengono descritti seguendo i seguenti punti:
\begin{itemize}
	\item Titolo: dotato di codice identificativo ed un nome esplicativo.
	\item Attori: diverse tipologie di utenti che interagiscono con il sistema.
	\item Precondizioni: stato del sistema prima dell'esecuzione di un caso d'uso.
	\item Postcondizioni: stato del sistema al termine dello scenario principale, se non vi sono stati errori.
	\item Scenario principale: descrizione del flusso con il quale viene svolta l'attività descritta.
	\item Descrizione: note aggiuntive.
	\item Scenari alternativi: scenari differenti da quello principale e che non soddisfano le postcondizioni, solitamente relativi ad errori.
\end{itemize}

\subsection{Riferimenti}
