\section{Introduzione}

\subsection{Scopo del documento}
Nel seguente documento si fornisce una descrizione del prodotto, analizzando i bisogni dell'utente che sono emersi in seguito all'analisi del capitolato e agli incontri effettuati con l'azienda proponente. \\
La modellazione avviene tramite UML$^G$ individuando nel dettaglio requisiti ed attori presenti nel progetto, in tal modo sarà possibile descrivere le varie componenti del programma ed indicare come strutturare ogni attività. \\

\noindent
I casi d'uso vengono descritti seguendo i seguenti punti:
\begin{itemize}
	\item Titolo: dotato di codice identificativo ed un nome esplicativo.
	\item Attori: diverse tipologie di utenti che interagiscono con il sistema.
	\item Precondizioni: stato del sistema prima dell'esecuzione di un caso d'uso.
	\item Postcondizioni: stato del sistema al termine dello scenario principale, se non vi sono stati errori.
	\item Scenario principale: descrizione del flusso con il quale viene svolta l'attività descritta.
	\item Descrizione: note aggiuntive che vengono inserite quando si ritiene che sia utile amplia la spiegazione per approfondire la comprensione del caso d'uso.
	\item Scenario secondario: scenari differenti da quello principale e che non soddisfano le postcondizioni, solitamente relativi ad errori.
	\item Trigger: evento scatenante del caso d'uso.
\end{itemize}

\subsection{Riferimenti}
\subsubsection{Riferimenti normativi}
\begin{itemize}
    \item \href{https://www.math.unipd.it/~tullio/IS-1/2023/Progetto/C3.pdf}{Capitolato C3 - \textit{Easy Meal}}.
    \item \href{https://project-swenergy.github.io/}{Norme di progetto v2.0.0}.
    \item \href{https://www.math.unipd.it/~tullio/IS-1/2023/Dispense/PD2.pdf}{Regolamento progetto didattico}.
\end{itemize}

\subsubsection{Riferimenti informativi}
\begin{itemize}
    \item \href{https://www.math.unipd.it/~tullio/IS-1/2023/Dispense/T5.pdf}{Lezione T05 - Analisi dei requisiti}.
    \item \href{https://www.math.unipd.it/~rcardin/swea/2023/Diagrammi%20delle%20Classi.pdf}{Diagrammi delle classi}.
    \item \href{https://www.math.unipd.it/~rcardin/swea/2022/Diagrammi%20Use%20Case.pdf}{Diagrammi dei casi d'uso}.
\end{itemize}