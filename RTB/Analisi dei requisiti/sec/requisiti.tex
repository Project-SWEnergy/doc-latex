\section{\textit{Requirements baseline}}

In questa sezione sono presentati i requisiti individuati durante la fase di analisi, derivanti dai casi d'uso, dall'esame del capitolato d'appalto e dagli incontri
sia interni che esterni con il proponente. Ciascun requisito è associato a un codice univoco, seguendo un formalismo stabilito tramite simboli.

Sono state individuate tre principali tipologie di requisiti:
\begin{enumerate}
	\item Funzionali, relativi all'usabilità del prodotto finale;
	\item Di qualità, riguardanti gli strumenti e la documentazione da fornire;
	\item Di vincolo, concernenti le tecnologie da presentare.
\end{enumerate}

\subsection{Funzionali}
\label{sec:funzionali}

Di seguito viene riportata la specifica relativa ai requisiti funzionali, che delineano le funzionalità del Sistema, le azioni eseguibili
da parte del Sistema e le informazioni che il Sistema può fornire. La presenza di ogni requisito viene giustificata riportando la fonte, che può essere un UC oppure presente
nel testo del capitolato d'appalto. Mentre i codici univoci sottostanti indicano:
\begin{enumerate}
	\item RFO: Requisito Funzionale Obbligatorio;
	\item RFF: Requisito Funzionale Facoltativo;
	\item RFD: Requisito Funzionale Desiderabile.
\end{enumerate}


\begin{table}[H]
	\renewcommand{\arraystretch}{1.5}
	\centering
	\begin{tabularx}{\textwidth}{l|X|p{2cm}}
		\textbf{ID} & \textbf{Descrizione}                                                                                       & \textbf{Fonte}                                                                                                                                               \\
		\hline
		RFO1        & L'Utente generico e L'Utente base devono poter visualizzare l'elenco dei ristoranti disponibili.           & \autoref{usecase:Visualizzazione elenco ristoranti}                                                                                                          \\
		\hline
		RFO2        & L'Utente generico e L'Utente base devono poter ricercare un ristorante attraverso il nome,luogo e filtri.  & \autoref{usecase:Ricerca di ristoranti}                                                                                                                      \\
		\hline
		RFO3        & L'Utente generico e L'Utente base devono poter visualizzare un ristorante.                                 & \autoref{usecase:Visualizzazione di un ristorante}                                                                                                           \\
		\hline
		RFD4        & L'Utente generico e L'Utente base devono poter condividere un \textit{link} di un ristorante.              & \autoref{usecase:Condivisione link del ristorante}                                                                                                           \\
		\hline
		RFD5        & L'Utente generico e L'Utente base devono poter visualizzare la pagina delle  \textit{\ac{FAQ}}.            & \autoref{usecase:Visualizzazione FAQ}                                                                                                                        \\
		\hline
		RFO6        & L'Utente generico deve poter effettuare l'accesso al Sistema.                                              & \autoref{usecase:Effettua accesso}, \autoref{usecase:Effettua accesso tradizionale}, \autoref{usecase:Effettua accesso per terze parti}                      \\
		\hline
		RFO7        & L'Utente generico deve poter effettuare la registrazione al Sistema come Utente base o Utente ristoratore. & \autoref{usecase:Effettua registrazione}, \autoref{usecase:Effettua registrazione Utente base} e \autoref{usecase:Effettua registrazione Utente ristoratore} \\
		\hline
		RFO8        & L'Utente generico deve visualizzare un messaggio d'errore se l'accesso fallisce.                           & \autoref{usecase:Visualizzazione errore d'accesso}                                                                                                           \\
		\hline
		RFO9        & L'Utente generico deve visualizzare un messaggio d'errore se la registrazione fallisce.                    & \autoref{usecase:Errore registrazione account esistente} e \autoref{usecase:Errore registrazione recapito occupato}                                          \\
		\hline
		RFD10       & L'Utente base deve poter visualizzare i suoi dati utente.                                                  & \autoref{usecase:Visualizzazione dati utente}                                                                                                                \\
		\hline
		RFD11       & L'Utente base deve poter modificare i suoi dati utente.                                                    & \autoref{usecase:Modifica dati utente}                                                                                                                       \\
		\hline
		RFD12       & L'Utente base deve poter visualizzare lo storico dei suoi ordini.                                          & \autoref{usecase:Storico ordini}                                                                                                                             \\
	\end{tabularx}
	\caption{Tabella dei requisiti funzionali}
\end{table}

\begin{table}[H]
	\renewcommand{\arraystretch}{1.5}
	\centering
	\begin{tabularx}{\textwidth}{l|X|p{2cm}}
		\textbf{ID} & \textbf{Descrizione}                                                                                           & \textbf{Fonte}                                                                                                     \\
		\hline
		RFO13       & L'Utente base deve poter visualizzare la lista delle sue prenotazioni, ed in caso andare in dettaglio.         & \autoref{usecase:Visualizzazione lista prenotazioni}, \autoref{usecase:Visualizzazione del riepilogo prenotazione} \\
		\hline
		RFO14       & L'Utente base deve poter visualizzare la notifica dello stato della sua prenotazione.                          & \autoref{usecase:Visualizzazione notifica stato della prenotazione}                                                \\
		\hline
		RFD15       & L'Utente base deve poter elimare il proprio \textit{account}.                                                  & \autoref{usecase:Eliminazione account}                                                                             \\
		\hline
		RFO16       & L'Utente base deve poter prenotare un tavolo.                                                                  & \autoref{usecase:Prenotazione di un tavolo}                                                                        \\
		\hline
		RFO17       & L'Utente base deve poter condividere la prenotazione.                                                          & \autoref{usecase:Condivisione della prenotazione}                                                                  \\
		\hline
		RFO18       & L'Utente base deve poter annullare la prenotazione.                                                            & \autoref{usecase:Annullamento della prenotazione}                                                                  \\
		\hline
		RFO19       & L'Utente base deve poter accedere ad una prenotazione mediante \textit{link} di condivisione                   & \autoref{usecase:Accesso alla prenotazione}                                                                        \\
		\hline
		RFO20       & L'Utente base deve poter annullare il proprio ordine.                                                          & \autoref{usecase:Annullamento dell'ordinazione}                                                                    \\
		\hline
		RFO21       & L'Utente base deve poter creare un ordinazione collaborativa dei pasti.                                        & \autoref{usecase:Creazione dell'ordinazione collaborativa dei pasti}                                               \\
		\hline
		RFO22       & L'Utente base deve poter annullare la propria ordinazione.                                                     & \autoref{usecase:Annullamento dell'ordinazione}                                                                    \\
		\hline
		RFO23       & L'Utente base deve poter dividere il conto in maniera equa oppure proporzionale.                               & \autoref{usecase:Selezione della modalità di divisione del conto}                                                  \\
		\hline
		RFO24       & L'Utente base deve poter visualizzare il messaggio d'errore che la divisione del conto è stata già effettuata. & \autoref{usecase:Visualizzazione errore divisione del conto già effettuata}                                        \\
		\hline
		RFO25       & L'Utente base deve poter pagare il conto.                                                                      & \autoref{usecase:Pagamento del conto}                                                                              \\
		\hline
		RFO26       & L'Utente base deve poter visualizzare l'errore relativo al pagamento fallito.                                  & \autoref{usecase:Visualizzazione errore pagamento}                                                                 \\
	\end{tabularx}
	\caption{Tabella dei requisiti funzionali}
\end{table}


\begin{table}[H]
	\renewcommand{\arraystretch}{1.5}
	\centering
	\begin{tabularx}{\textwidth}{l|X|p{2cm}}
		\textbf{ID} & \textbf{Descrizione}                                                                                                 & \textbf{Fonte}                                                                                \\
		\hline
		RFO27       & L'Utente base deve poter inserire \textit{feedback}.                                                                 & \autoref{usecase:Inserimento di feedback}                                                     \\
		\hline
		RFO28       & L'Utente base deve poter visualizzare la notifica di richiesta di inserimento \textit{feedback}.                     & \autoref{usecase:Visualizzazione della notifica di richiesta di inserimento feedback}         \\
		\hline
		RFO29       & L'Utente base deve poter visualizzare la notifica relativa alla modifica della sua ordinazione.                      & \autoref{usecase:Visualizzazione notifica modifica ordinazione}                               \\
		\hline
		RFD30       & L'Utente base deve poter visualizzare la notifica relativa al suo \textit{feedback} che ha ricevuto una risposta.    & \autoref{usecase:Visualizzazione notifica risposta feedback}                                  \\
		\hline
		RFD31       & L'Utente base deve poter inserire e modificare le proprie allergie.                                                  & \autoref{usecase:Effettua registrazione Utente base} e \autoref{usecase:Modifica dati utente} \\
		\hline
		RFD32       & L'Utente base deve poter visualizzare un messaggio se seleziona un piatto di cui è allergico.                        & \autoref{usecase:Visualizzazione messaggio di selezione di una pietanza con allergene}        \\
		\hline
		RFO33       & L'Utente base deve poter visualizzare il menù di un ristorante.                                                      & \autoref{usecase:Visualizzazione menù}                                                        \\
		\hline
		RFD34       & L'Utente autenticato deve poter effettuare il \textit{logout}.                                                       & \autoref{usecase:Effettua Logout}                                                             \\
		\hline
		RFO35       & L'Utente autenticato deve poter comunicare attraverso la \textit{chat}.                                              & \autoref{usecase:Comunicazione attraverso chat}                                               \\
		\hline
		RFD36       & L'Utente autenticato deve poter visualizzare la notifica relativa all'arrivo di un nuovo messaggio in \textit{chat}. & \autoref{usecase:Visualizzazione notifica nuovo messaggio in chat}                            \\
		\hline
		RFO37       & L'Utente ristoratore deve poter visualizzare la notifica relativa ad una nuova prenotazione.                         & \autoref{usecase:Visualizzazione notifica nuova prenotazione}                                 \\
		\hline
		RFD38       & L'Utente ristoratore deve poter visualizzare la notifica relativa ad un nuovo ordine.                                & \autoref{usecase:Visualizzazione notifica nuovo ordine}                                       \\
		\hline
		RFO39       & L'Utente ristoratore deve poter visualizzare la notifica relativa all'avvenuto pagamento.                            & \autoref{usecase:Visualizzazione notifica di avvenuto pagamento}                              \\
		\hline
		RFD40       & L'Utente ristoratore deve poter visualizzare la notifica relativa all'inserimento di un \textit{feedback}.           & \autoref{usecase:Visualizzazione notifica di inserimento feedback}                            \\
	\end{tabularx}
	\caption{Tabella dei requisiti funzionali}
\end{table}


\begin{table}[H]
	\renewcommand{\arraystretch}{1.5}
	\centering
	\begin{tabularx}{\textwidth}{l|X|c}
		\textbf{ID} & \textbf{Descrizione}                                                                                                    & \textbf{Fonte}                                                       \\
		\hline
		RFO41       & L'Utente ristoratore deve poter visualizzare la lista delle prenotazioni in dettaglio e con la lista degli ingredienti. & \autoref{usecase:Visualizzazione lista prenotazioni}                 \\
		\hline
		RFO42       & L'Utente ristoratore deve poter accettare una prenotazione.                                                             & \autoref{usecase:Accetta prenotazione}                               \\
		\hline
		RFO43       & L'Utente ristoratore deve poter rifiutare una prenotazione.                                                             & \autoref{usecase:Rifiuta prenotazione}                               \\
		\hline
		RFO44       & L'Utente ristoratore deve poter terminare una prenotazione.                                                             & \autoref{usecase:Termina prenotazione}                               \\
		\hline
		RFO45       & L'Utente ristoratore deve poter visualizzare la lista delle ordinazioni.                                                & \autoref{usecase:Visualizzazione lista ordinazioni}                  \\
		\hline
		RFD46       & L'Utente ristoratore deve poter modificare un ordinazione.                                                              & \autoref{usecase:Modifica ordinazione}                               \\
		\hline
		RFO47       & L'Utente ristoratore deve poter visualizzare lo stato di pagamento di una prenotazione.                                 & \autoref{usecase:Visualizzazione stato di pagamento}                 \\
		\hline
		RFO48       & L'Utente ristoratore deve poter visualizzare la lista dei \textit{feedback}.                                            & \autoref{usecase:Visualizzazione lista feedback}                     \\
		\hline
		RFD49       & L'Utente ristoratore deve poter segnalare un \textit{feedback}.                                                         & \autoref{usecase:Segnalazione di un feedback}                        \\
		\hline
		RFD50       & L'Utente ristoratore deve poter rispondere ad un \textit{feedback}.                                                     & \autoref{usecase:Risposta ad un feedback}                            \\
		\hline
		RFD51       & L'Utente ristoratore deve poter modificare le informazioni del suo ristorante.                                          & \autoref{usecase:Modifica informazioni ristorante}                   \\
		\hline
		RFO52       & L'Utente ristoratore deve poter gestire il menù, inserendo, eliminando e modificando dei piatti.                        & \autoref{usecase:Modifica menù}                                      \\
		\hline
		RFO53       & L'Utente ristoratore deve poter gestire gli ingredienti, inserendo e eliminando degli ingredienti.                      & \autoref{usecase:Modifica lista ingredienti}                         \\
		\hline
		RFO54       & L'Utente ristoratore deve poter assegnare gli ingredienti ad un piatto.                                                 & \autoref{usecase:Assegnamento ingredienti ad un piatto}              \\
		\hline
		RFO55       & L'Utente ristoratore deve poter visualizzare la notifica relativa all'annullamento di un ordinazione.                   & \autoref{usecase:Visualizzazione notifica annullamento ordine}       \\
		\hline
		RFO56       & L'Utente ristoratore deve poter visualizzare la notifica relativa all'annullamento di una prenotazione.                 & \autoref{usecase:Visualizzazione notifica annullamento prenotazione} \\
	\end{tabularx}
	\caption{Tabella dei requisiti funzionali}
\end{table}

\newpage
\subsubsection{Tracciamento}
Il tracciamento "Requisito - Fonti" è presente nella tabella soprastante (vedi \autoref{sec:funzionali}), in cui è possibile vedere la corrispondenza tra i requisiti e le fonti da cui derivano.\\
La seguente tabella è relativa al tracciamento "Fonti - Requisito", in cui è possibile vedere la corrispondenza tra le fonti e i requisiti a cui fanno riferimento.\\

\begin{longtable}{|l|c|}
	\hline
	\textbf{Fonte}                                                                                                                                               & \textbf{ ID requisito} \\
	\hline
	\autoref{usecase:Visualizzazione elenco ristoranti}                                                                                                          & RFO1                   \\
	\hline
	\autoref{usecase:Ricerca di ristoranti}                                                                                                                      & RFO2                   \\
	\hline
	\autoref{usecase:Visualizzazione di un ristorante}                                                                                                           & RFO3                   \\
	\hline
	\autoref{usecase:Condivisione link del ristorante}                                                                                                           & RFD4                   \\
	\hline
	\autoref{usecase:Visualizzazione FAQ}                                                                                                                        & RFD5                   \\
	\hline
	\autoref{usecase:Effettua accesso}, \autoref{usecase:Effettua accesso tradizionale}, \autoref{usecase:Effettua accesso per terze parti}                      & RFO6                   \\
	\hline
	\autoref{usecase:Effettua registrazione}, \autoref{usecase:Effettua registrazione Utente base} e \autoref{usecase:Effettua registrazione Utente ristoratore} & RFO7                   \\
	\hline
	\autoref{usecase:Visualizzazione errore d'accesso}                                                                                                           & RFO8                   \\
	\hline
	\autoref{usecase:Errore registrazione account esistente} e \autoref{usecase:Errore registrazione recapito occupato}                                          & RFO9                   \\
	\hline
	\autoref{usecase:Visualizzazione dati utente}                                                                                                                & RFD10                  \\
	\hline
	\autoref{usecase:Modifica dati utente}                                                                                                                       & RFD11                  \\
	\hline
	\autoref{usecase:Storico ordini}                                                                                                                             & RFD12                  \\
	\hline
	\autoref{usecase:Visualizzazione lista prenotazioni}, \autoref{usecase:Visualizzazione del riepilogo prenotazione}                                           & RFO13                  \\
	\hline
	\autoref{usecase:Visualizzazione notifica stato della prenotazione}                                                                                          & RFO14                  \\
	\hline
	\autoref{usecase:Eliminazione account}                                                                                                                       & RFD15                  \\
	\hline
	\autoref{usecase:Prenotazione di un tavolo}                                                                                                                  & RFO16                  \\
	\hline
	\autoref{usecase:Condivisione della prenotazione}                                                                                                            & RFO17                  \\
	\hline
	\autoref{usecase:Annullamento della prenotazione}                                                                                                            & RFO18                  \\
	\hline
	\autoref{usecase:Accesso alla prenotazione}                                                                                                                  & RFO19                  \\
	\hline
	\autoref{usecase:Annullamento dell'ordinazione}                                                                                                              & RFO20                  \\
	\hline
	\autoref{usecase:Creazione dell'ordinazione collaborativa dei pasti}                                                                                         & RFO21                  \\
	\hline
	\autoref{usecase:Annullamento dell'ordinazione}                                                                                                              & RFO22                  \\
	\hline
	\autoref{usecase:Selezione della modalità di divisione del conto}                                                                                            & RFO23                  \\
	\hline
	\autoref{usecase:Visualizzazione errore divisione del conto già effettuata}                                                                                  & RFO24                  \\
	\hline
	\autoref{usecase:Pagamento del conto}                                                                                                                        & RFO25                  \\
	\hline
	\autoref{usecase:Visualizzazione errore pagamento}                                                                                                           & RFO26                  \\
	\hline
	\autoref{usecase:Inserimento di feedback}                                                                                                                    & RFO27                  \\
	\hline
	\autoref{usecase:Visualizzazione della notifica di richiesta di inserimento feedback}                                                                        & RFO28                  \\
	\hline
	\autoref{usecase:Visualizzazione notifica modifica ordinazione}                                                                                              & RFO29                  \\
	\hline
	\autoref{usecase:Visualizzazione notifica risposta feedback}                                                                                                 & RFD30                  \\
	\hline
	\autoref{usecase:Effettua registrazione Utente base} e \autoref{usecase:Modifica dati utente}                                                                & RFD31                  \\
	\hline
	\autoref{usecase:Visualizzazione messaggio di selezione di una pietanza con allergene}                                                                       & RFD32                  \\
	\hline
	\autoref{usecase:Visualizzazione menù}                                                                                                                       & RFO33                  \\
	\hline
	\autoref{usecase:Effettua Logout}                                                                                                                            & RFD34                  \\
	\hline
	\autoref{usecase:Comunicazione attraverso chat}                                                                                                              & RFO35                  \\
	\hline
	\autoref{usecase:Visualizzazione notifica nuovo messaggio in chat}                                                                                           & RFD36                  \\
	\hline
	\autoref{usecase:Visualizzazione notifica nuova prenotazione}                                                                                                & RFO37                  \\
	\hline
	\autoref{usecase:Visualizzazione notifica nuovo ordine}                                                                                                      & RFD38                  \\
	\hline
	\autoref{usecase:Visualizzazione notifica di avvenuto pagamento}                                                                                             & RFO39                  \\
	\hline
	\autoref{usecase:Visualizzazione notifica di inserimento feedback}                                                                                           & RFD40                  \\
	\hline
	\autoref{usecase:Visualizzazione lista prenotazioni}                                                                                                         & RFO41                  \\
	\hline
	\autoref{usecase:Accetta prenotazione}                                                                                                                       & RFO42                  \\
	\hline
	\autoref{usecase:Rifiuta prenotazione}                                                                                                                       & RFO43                  \\
	\hline
	\autoref{usecase:Termina prenotazione}                                                                                                                       & RFO44                  \\
	\hline
	\autoref{usecase:Visualizzazione lista ordinazioni}                                                                                                          & RFO45                  \\
	\hline
	\autoref{usecase:Modifica ordinazione}                                                                                                                       & RFD46                  \\
	\hline
	\autoref{usecase:Visualizzazione stato di pagamento}                                                                                                         & RFO47                  \\
	\hline
	\autoref{usecase:Visualizzazione lista feedback}                                                                                                             & RFO48                  \\
	\hline
	\autoref{usecase:Segnalazione di un feedback}                                                                                                                & RFD49                  \\
	\hline
	\autoref{usecase:Risposta ad un feedback}                                                                                                                    & RFD50                  \\
	\hline
	\autoref{usecase:Modifica informazioni ristorante}                                                                                                           & RFD51                  \\
	\hline
	\autoref{usecase:Modifica menù}                                                                                                                              & RFO52                  \\
	\hline
	\autoref{usecase:Modifica lista ingredienti}                                                                                                                 & RFO53                  \\
	\hline
	\autoref{usecase:Assegnamento ingredienti ad un piatto}                                                                                                      & RFO54                  \\
	\hline
	\autoref{usecase:Visualizzazione notifica annullamento ordine}                                                                                               & RFO55                  \\
	\hline
	\autoref{usecase:Visualizzazione notifica annullamento prenotazione}                                                                                         & RFO56                  \\
	\hline
	\caption{Tabella "Fonti - Requisito""}
\end{longtable}




\subsection{Di qualità}

Di seguito viene riportata la specifica relativa ai requisiti di qualità, che delineano le caratteristiche di come un Sistema
deve essere o comportarsi al fine di soddisfare le necessità dell'utente.
La presenza di ogni requisito viene giustificata riportando la fonte, che può essere un UC oppure presente
nel testo del capitolato d'appalto. Mentre i codici univoci sottostanti indicano:
\begin{enumerate}
	\item RQO: Requisito di Qualità Obbligatorio;
	\item RQF: Requisito di Qualità Facoltativo;
	\item RQD: Requisito di Qualità Desiderabile.
\end{enumerate}

\begin{table}[H]
	\renewcommand{\arraystretch}{1.5}
	\centering
	\begin{tabularx}{\textwidth}{l|X|c}
		\textbf{ID} & \textbf{Descrizione}                                                                                                                                                                                                 & \textbf{Fonte} \\
		\hline
		RQO1        & Il codice sorgente deve essere coperto da test almeno per il 80\%                                                                                                                                                    & Capitolato     \\
		\hline
		RQO2        & Deve essere prodotta della documentazione sulle scelte implementative e progettuali, che dovranno essere accompagnate da motivazioni.                                                                                & Capitolato     \\
		\hline
		RQF3        & Fornire un'analisi rispetto al carico massimo supportato in numero di dispositivi e di quale sarebbe il servizio \textit{cloud} più adatto per supportarlo analizzando prezzo, stabilità del servizio ed assistenza. & Capitolato     \\
		\hline
	\end{tabularx}
	\caption{Tabella dei requisiti di qualità}
\end{table}
\newpage
\subsection{Di vincolo}

Segue la specifica relativa ai requisiti di vincolo, i quali delineano i limiti e le restrizioni che il Sistema deve osservare per adempiere alle esigenze dell'utente.
La presenza di ogni requisito viene giustificata riportando la fonte, che può essere un UC oppure presente
nel testo del capitolato d'appalto. Mentre i codici univoci sottostanti indicano:
\begin{enumerate}
	\item RVO: Requisito di Vincolo Obbligatorio;
	\item RVF: Requisito di Vincolo Facoltativo;
	\item RVD: Requisito di Vincolo Desiderabile.
\end{enumerate}

\begin{table}[H]
	\renewcommand{\arraystretch}{1.5}
	\centering
	\begin{tabularx}{\textwidth}{l|X|c}
		\textbf{ID} & \textbf{Descrizione}                                                                                                     & \textbf{Fonte} \\
		\hline
		RVO1        & L'interfaccia degli utenti deve essere un'applicazione \textit{web responsive} del tipo \textit{single page application} & Capitolato     \\
		\hline
	\end{tabularx}
	\caption{Tabella dei requisiti di vincolo}
\end{table}
