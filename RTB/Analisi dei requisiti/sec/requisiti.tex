\section{\textit{Requirements baseline}}

In questa sezione sono presentati i requisiti individuati durante la fase di analisi, derivanti dai casi d'uso, dall'esame del capitolato d'appalto e dagli incontri 
sia interni che esterni con il proponente. Ciascun requisito è associato a un codice univoco, seguendo un formalismo stabilito tramite simboli.
Il significato del codice univoco verrà spiegato più avanti nel documento per ogni tipologia di requisito, ed è inoltre riportato all'interno del documento di "Norme di progetto".

Sono state individuate tre principali tipologie di requisiti:
\begin{enumerate}
	\item Funzionali, relativi all'usabilità del prodotto finale;
	\item Di qualità, riguardanti gli strumenti e la documentazione da fornire;
	\item Di vincolo, concernenti le tecnologie da presentare.
\end{enumerate}

\subsection{Funzionali}

Di seguito viene riportata la specifica relativa ai requisiti funzionali, che delineano le funzionalità del sistema, le azioni eseguibili 
da parte del sistema e le informazioni che il sistema può fornire. La presenza di ogni requisito viene giustificata riportando la fonte, che può essere un UC oppure presente 
nel testo del capitolato d'appalto. Mentre i codici univoci sottostanti indicano:
\begin{enumerate}
	\item RFO: Requisito Funzionale Obbligatorio;
	\item RFF: Requisito Funzionale Facoltativo;
	\item RFD: Requisito Funzionale Desiderabile.
\end{enumerate}


\begin{table}[H]
	\renewcommand{\arraystretch}{1.5}
	\centering
	\begin{tabularx}{\textwidth}{l|X|c}
		\textbf{ID} & \textbf{Descrizione}                                                                                                      & \textbf{Fonte} \\
		\hline
		RFO1        & L'Utente generico deve poter accedere al sistema                                                                           & \autoref{usecase:Effettua accesso}            \\
		\hline
		RFO2        & L'Utente base deve poter prenotare un tavolo                                                                              &             \\
		\hline
		RFO3        & L'Utente base deve poter modificare una prenotazione                                                                      & UC3            \\
		\hline
		RFO4        & L'Utente base deve poter porre delle domande al rispettivo ristorante                                                     & UC4            \\
		\hline
		RFO5        & L'Utente ristoratore deve poter rispondere alle domande poste dagli utenti                                                & UC4            \\
		\hline
		RFO6        & L'Utente base deve poter selezionare la modalità di divsione del conto                                                    & UC5            \\
		\hline
		RFO7        & L'Utente ristoratore deve poter visualizzare le prenotazioni effettuate raggruppate per giorno in un calendario           & UC6            \\
		\hline
		RFO8        & Il Sistema deve inviare una notifica all'Utente base quando una prenotazione è conclusa                                   & UC7            \\
		\hline
		RFO9        & L'Utente ristoratore deve poter visualizzare la lista degli ingredienti per arco temporale                                & UC8            \\
		\hline
		RFO10       & L'Utente generico deve poter uscire dal sistema tramite un'interfaccia web                                                 & UC9            \\
		\hline
		RFO11       & L'Utente generico deve poter visualizzare il dettaglio di un ristorante                                                    & UC10           \\
		\hline
		RFO12       & L'Utente base deve poter visualizzare il riepilogo delle proprie prenotazioni                                             & UC11           \\
		\hline
		RFO13       & L'Utente ristoratore deve poter confermare la prenotazione di un tavolo                                                   & UC12           \\
\end{tabularx}
\caption{Tabella dei requisiti funzionali}
\end{table}

\begin{table}[H]
	\renewcommand{\arraystretch}{1.5}
	\centering
	\begin{tabularx}{\textwidth}{l|X|c}
		\textbf{ID} & \textbf{Descrizione}                                                                                                      & \textbf{Fonte} \\
		\hline
		RFO14       & L'Utente ristoratore deve poter modificare lo stato di un tavolo                                                          & UC12           \\
		\hline
		RFO15       & L'Utente base deve poter cancellare una prenotazione                                                                      & UC13           \\
		\hline
		RFO16       & L'Utente base deve poter condividere la prenotazione tramite un \textit{link}                                                     & UC14           \\
		\hline
		RFO17       & L'Utente generico deve poter accedere ad una prenotazione tramite un \textit{link}                                                 & UC15           \\
		\hline
		RFO18       & Il Sistema deve inviare un messaggio di errore all'Utente ristoratore quando tenta di accedere ad una prenotazione        & UC16           \\
		\hline
		RFO19       & Il Sistema deve inviare un messaggio di errore all'Utente base quando tenta di confermare una prenotazione già confermata & UC17           \\
		\hline
		RFO20       & L'Utente base deve poter inserire un \textit{feedback} riguardante la prenotazione                                                 & UC18           \\
		\hline
		RFO19       & L'Utente ristoratore deve poter visualizzare le statistiche relative alle prenotazioni effettuate                         & UC19           \\
		\hline
\end{tabularx}
\caption{Tabella dei requisiti funzionali}
\end{table}

\subsection{Di qualità}

Di seguito viene riportata la specifica relativa ai requisiti di qualità, che delineano le caratteristiche di come un sistema 
deve essere o comportarsi al fine di soddisfare le necessità dell'utente.
La presenza di ogni requisito viene giustificata riportando la fonte, che può essere un UC oppure presente 
nel testo del capitolato d'appalto. Mentre i codici univoci sottostanti indicano:
\begin{enumerate}
	\item RQO: Requisito di Qualità Obbligatorio;
	\item RQF: Requisito di Qualità Facoltativo;
	\item RQD: Requisito di Qualità Desiderabile.
\end{enumerate}

\begin{table}[H]
	\renewcommand{\arraystretch}{1.5}
	\centering
	\begin{tabularx}{\textwidth}{l|X|c}
		\textbf{ID} & \textbf{Descrizione}                                                                                                                  & \textbf{Fonte} \\
		\hline
		RQO1        & Il codice sorgente deve essere coperto da test almeno per il 80\%                                                                     & Capitolato     \\
		\hline
		RQO2        & Deve essere prodotta della documentazione sulle scelte implementative e progettuali, che dovranno essere accompagnate da motivazioni. & Capitolato     \\
		\hline
	\end{tabularx}
	\caption{Tabella dei requisiti di qualità}
\end{table}

\subsection{Di vincolo}

Segue la specifica relativa ai requisiti di vincolo, i quali delineano i limiti e le restrizioni che il sistema deve osservare per adempiere alle esigenze dell'utente.
La presenza di ogni requisito viene giustificata riportando la fonte, che può essere un UC oppure presente 
nel testo del capitolato d'appalto. Mentre i codici univoci sottostanti indicano:
\begin{enumerate}
	\item RVO: Requisito di Vincolo Obbligatorio;
	\item RVF: Requisito di Vincolo Facoltativo;
	\item RVD: Requisito di Vincolo Desiderabile.
\end{enumerate}

\begin{table}[H]
	\renewcommand{\arraystretch}{1.5}
	\centering
	\begin{tabularx}{\textwidth}{l|X|c}
		\textbf{ID} & \textbf{Descrizione}                                                                                   & \textbf{Fonte} \\
		\hline
		RVO1        & L'interfaccia degli utenti deve essere un'applicazione \textit{web responsive} del tipo \textit{single page application} & Capitolato     \\
		\hline
	\end{tabularx}
	\caption{Tabella dei requisiti di vincolo}
\end{table}
