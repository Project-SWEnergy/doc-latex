\section{\textit{Requirements baseline}}

In questa sezione sono presentati i requisiti individuati durante la fase di analisi, derivanti dai casi d'uso, dall'esame del capitolato d'appalto e dagli incontri
sia interni che esterni con il proponente. Ciascun requisito è associato a un codice univoco, seguendo un formalismo stabilito tramite simboli.
Il significato del codice univoco verrà spiegato più avanti nel documento per ogni tipologia di requisito, ed è inoltre riportato all'interno del documento di "Norme di progetto".

Sono state individuate tre principali tipologie di requisiti:
\begin{enumerate}
	\item Funzionali, relativi all'usabilità del prodotto finale;
	\item Di qualità, riguardanti gli strumenti e la documentazione da fornire;
	\item Di vincolo, concernenti le tecnologie da presentare.
\end{enumerate}

\subsection{Funzionali}

Di seguito viene riportata la specifica relativa ai requisiti funzionali, che delineano le funzionalità del sistema, le azioni eseguibili
da parte del sistema e le informazioni che il sistema può fornire. La presenza di ogni requisito viene giustificata riportando la fonte, che può essere un UC oppure presente
nel testo del capitolato d'appalto. Mentre i codici univoci sottostanti indicano:
\begin{enumerate}
	\item RFO: Requisito Funzionale Obbligatorio;
	\item RFF: Requisito Funzionale Facoltativo;
	\item RFD: Requisito Funzionale Desiderabile.
\end{enumerate}


\begin{table}[H]
	\renewcommand{\arraystretch}{1.5}
	\centering
	\begin{tabularx}{\textwidth}{l|X|c}
		\textbf{ID} & \textbf{Descrizione}                                                                                      & \textbf{Fonte}                                                                         \\
		\hline
		RFO1        & L'Utente base e generico deve poter consultare l'elenco dei ristoranti disponibili.                       & \autoref{usecase:Consultazione elenco ristoranti}                                      \\
		\hline
		RFO2        & L'Utente base e generico devono poter visualizzare in dettaglio un ristorante.                            & \autoref{usecase:Visualizzazione di un ristorante}                                     \\
		\hline
		RFD3        & L'Utente base e generico devono poter condividere un link di un ristorante.                               & \autoref{usecase:Condivisione link del ristorante}                                     \\
		\hline
		RFD4        & L'Utente base e generico devono poter visualizzare la pagina delle FAQ.                                   & \autoref{usecase:Visualizzazione FAQ}                                                  \\
		\hline
		RFO5        & Il Sistema invia una notifica all'Utente base e Utente ristoratore ce sono collegati nella \textit{chat}. & \autoref{usecase:Notifica chat}                                                        \\
		\hline
		RFO6        & L'Utente generico effettua l'accesso al Sistema.                                                          & \autoref{usecase:Effettua accesso}                                                     \\
		\hline
		RFO7        & L'Utente generico effettua la registrazione al Sistema.                                                   & \autoref{usecase:Effettua registrazione}                                               \\
		\hline
		RFO8        & Il Sistema mostra un messaggio d'errore se l'accesso o la registrazione falliscono.                       & \autoref{usecase:Accesso fallito} e \autoref{usecase:Registrazione fallita}            \\
		\hline
		RFD9        & L'Utente base deve visualizzare e modificare i propri dati.                                               & \autoref{usecase:Visualizzazione dati utente} e \autoref{usecase:Modifica dati utente} \\
		\hline
		RFD10       & L'Utente base deve visualizzare lo storico dei suoi ordini.                                               & \autoref{usecase:Storico ordini}                                                       \\
		\hline
		RFO11       & L'Utente base deve visualizzare le proprie prenotazioni.                                                  & \autoref{usecase:Visualizzazione del riepilogo prenotazione}                           \\
		\hline
		RFD12       & L'Utente base e ristoratore devono effettuare il \textit{logout}.                                         & \autoref{usecase:Logout}                                                               \\
	\end{tabularx}
	\caption{Tabella dei requisiti funzionali}
\end{table}

\begin{table}[H]
	\renewcommand{\arraystretch}{1.5}
	\centering
	\begin{tabularx}{\textwidth}{l|X|c}
		\textbf{ID} & \textbf{Descrizione}                                                                                      & \textbf{Fonte}                                               \\
		\hline
		RFD13       & L'Utente base deve poter eliminare il suo \textit{account}.                                               & \autoref{usecase:Eliminazione account}                       \\
		\hline
		RFD14       & Il Sistema mostra un messaggio d'errore se l'eliminazione dell'\textit{account} non è andata a buon fine. & \autoref{usecase:Errore eliminazione account}                \\
		\hline
		RFO15       & L'Utente base deve poter prenotare un tavolo.                                                             & \autoref{usecase:Prenotazione di un tavolo}                  \\
		\hline
		RFO16       & Il Sistema notifica l'Utente ristoratore dell'avvenuta prenotazione.                                      & \autoref{usecase:Notifica prenotazione}                      \\
		\hline
		RFO17       & L'Utente base deve poter condividere la prenotazione con i commensali.                                    & \autoref{usecase:Condividi la prenotazione}                  \\
		\hline
		RFO18       & L'Utente base deve poter annullare la prenotazione.                                                       & \autoref{usecase:Annullamento della prenotazione}            \\
		\hline
		RFO19       & L'Utente base deve poter accedere ad una prenotazione.                                                    & \autoref{usecase:Accesso alla prenotazione}                  \\
		\hline
		RFD20       & Il Sistema mostra un messaggio d'errore in caso l'accesso alla prenotazione fallisca.                     & \autoref{usecase:Accesso prenotazione fallito}               \\
		\hline
		RFO21       & L'Utente base deve poter fare un ordinazione collaborativa dei pasti, e creare il suo ordine.             & \autoref{usecase:Ordinazione collaborativa dei pasti}        \\
		\hline
		RFO22       & L'Utente base deve poter annullare il proprio ordine.                                                     & \autoref{usecase:Annullamento dell'ordinazione}              \\
		\hline
		RFO23       & L'Utente base deve poter visualizzare il riepilogo della prenotazione.                                    & \autoref{usecase:Visualizzazione del riepilogo prenotazione} \\
		\hline
		RFO24       & Il Sistema notifica il ristoratore che è avvenuto un ordine.                                              & \autoref{usecase:Notifica ordine}                            \\
		\hline
		RFO25       & L'Utente base deve poter dividere il conto come preferisce.                                               & \autoref{usecase:Divisione del conto}                        \\
		\hline
		RFO26       & L'Utente base deve poter pagare il conto come preferisce.                                                 & \autoref{usecase:Pagamento del conto}                        \\
		\hline
		RFF27       & Il Sistema mostra un messaggio d'errore relativo al pagamento fallito.                                    & \autoref{usecase:Errore pagamento}                           \\
		\hline
		RFO28       & Il Sistema notifica il ristoratore dell'avvenuto pagamento.                                               & \autoref{usecase:Notifica avvenuto pagamento}                \\
		\hline
		RFO29       & L'Utente base deve poter inserire \textit{feedback} e recensioni.                                         & \autoref{usecase:Inserimento di feedback e recensioni}       \\
		\hline
	\end{tabularx}
	\caption{Tabella dei requisiti funzionali}
\end{table}


\begin{table}[H]
	\renewcommand{\arraystretch}{1.5}
	\centering
	\begin{tabularx}{\textwidth}{l|X|c}
		\textbf{ID} & \textbf{Descrizione}                                                                                                           & \textbf{Fonte}                                                                                                                                        \\
		\hline
		RFD30       & Il Sistema notifica il ristoratore che è stato inserito un \textit{feedback}.                                                  & \autoref{usecase:Notifica di inserimento feedback}                                                                                                    \\
		\hline
		RFF31       & Il Sistema notifica l'Utente base se vuole lasciare \textit{feedback}.                                                         & \autoref{usecase:Notifica di richiesta di inserimento feedback}                                                                                       \\
		\hline
		RFO32       & L'Utente base e ristoratore devono poter comunicare tra loro attraverso \textit{chat}.                                         & \autoref{usecase:Chat Utente base}, \autoref{usecase:Chat Utente ristoratore}, \autoref{usecase:Invio messaggio chat}, \autoref{usecase:Lettura chat} \\
		\hline
		RFO33       & L'Utente ristoratore deve poter consultare la lista delle prenotazioni (con la lista ingredienti), ed andare in dettaglio      & \autoref{usecase:Consultazione lista prenotazioni}, \autoref{usecase:Dettaglio lista prenotazioni}                                                    \\
		\hline
		RFO34       & Il Sistema deve aggiornare la lista degli ingredienti.                                                                         & \autoref{usecase:Aggiornamento lista ingredienti}                                                                                                     \\
		\hline
		RFO35       & L'Utente ristoratore deve poter accettare,rifiutare o terminare una prenotazione.                                              & \autoref{usecase:Accetta prenotazione}, \autoref{usecase:Rifiuta prenotazione}, \autoref{usecase:Termina prenotazione}                                \\
		\hline
		RFO36       & Il Sistema deve notificare il cliente sullo stato della sua prenotazione.                                                      & \autoref{usecase:Notifica stato della prenotazione}                                                                                                   \\
		\hline
		RFO37       & L'Utente ristoratore deve poter consultare la lista delle ordinazioni. Le può modificare e controllare lo stato del pagamento. & \autoref{usecase:Consultazione lista ordinazioni}                                                                                                     \\
		\hline
		RFD38       & Il Sistema notifica il cliente che il suo ordine è stato modificato.                                                           & \autoref{usecase:Notifica modifica ordinazione al cliente}                                                                                            \\
		\hline
		RFO39       & L'Utente ristoratore deve poter consultare una lista di \textit{feedback}.                                                     & \autoref{usecase:Consultazione lista feedback}                                                                                                        \\
		\hline
		RFD40       & L'Utente ristoratore deve poter rispondere o segnalare un \textit{feedback}.                                                   & \autoref{usecase:Risposta ad un feedback}, \autoref{usecase:Segnalazione di un feedback}                                                              \\
		\hline
	\end{tabularx}
	\caption{Tabella dei requisiti funzionali}
\end{table}


\begin{table}[H]
	\renewcommand{\arraystretch}{1.5}
	\centering
	\begin{tabularx}{\textwidth}{l|X|c}
		\textbf{ID} & \textbf{Descrizione}                                                                  & \textbf{Fonte}                                                                \\
		\hline
		RFD41       & Il Sistema notifica il cliente che il suo \textit{feedback} ha ricevuto una risposta. & \autoref{usecase:Notifica risposta feedback}                                  \\
		\hline
		RFO42       & L'Utente ristoratore deve poter gestire le informazioni del suo ristorante.           & \autoref{usecase:Gestione informazioni ristorante}                            \\
		\hline
		RFO43       & L'Utente ristoratore deve poter gestire il menù e la lista ingredienti.               & \autoref{usecase:Gestione menù}, \autoref{usecase:Gestione lista ingredienti} \\
		\hline
		RFD44       & L'Utente ristoratore deve poter gestire le impostazioni del suo \textit{account}.     & \autoref{usecase:Impostazioni account ristoratore}                            \\
		\hline
	\end{tabularx}
	\caption{Tabella dei requisiti funzionali}
\end{table}


\subsection{Di qualità}

Di seguito viene riportata la specifica relativa ai requisiti di qualità, che delineano le caratteristiche di come un sistema
deve essere o comportarsi al fine di soddisfare le necessità dell'utente.
La presenza di ogni requisito viene giustificata riportando la fonte, che può essere un UC oppure presente
nel testo del capitolato d'appalto. Mentre i codici univoci sottostanti indicano:
\begin{enumerate}
	\item RQO: Requisito di Qualità Obbligatorio;
	\item RQF: Requisito di Qualità Facoltativo;
	\item RQD: Requisito di Qualità Desiderabile.
\end{enumerate}

\begin{table}[H]
	\renewcommand{\arraystretch}{1.5}
	\centering
	\begin{tabularx}{\textwidth}{l|X|c}
		\textbf{ID} & \textbf{Descrizione}                                                                                                                                                                                                 & \textbf{Fonte} \\
		\hline
		RQO1        & Il codice sorgente deve essere coperto da test almeno per il 80\%                                                                                                                                                    & Capitolato     \\
		\hline
		RQO2        & Deve essere prodotta della documentazione sulle scelte implementative e progettuali, che dovranno essere accompagnate da motivazioni.                                                                                & Capitolato     \\
		\hline
		RQF3        & Fornire un'analisi rispetto al carico massimo supportato in numero di dispositivi e di quale sarebbe il servizio \textit{cloud} più adatto per supportarlo analizzando prezzo, stabilità del servizio ed assistenza. & Capitolato     \\
		\hline
	\end{tabularx}
	\caption{Tabella dei requisiti di qualità}
\end{table}

\subsection{Di vincolo}

Segue la specifica relativa ai requisiti di vincolo, i quali delineano i limiti e le restrizioni che il sistema deve osservare per adempiere alle esigenze dell'utente.
La presenza di ogni requisito viene giustificata riportando la fonte, che può essere un UC oppure presente
nel testo del capitolato d'appalto. Mentre i codici univoci sottostanti indicano:
\begin{enumerate}
	\item RVO: Requisito di Vincolo Obbligatorio;
	\item RVF: Requisito di Vincolo Facoltativo;
	\item RVD: Requisito di Vincolo Desiderabile.
\end{enumerate}

\begin{table}[H]
	\renewcommand{\arraystretch}{1.5}
	\centering
	\begin{tabularx}{\textwidth}{l|X|c}
		\textbf{ID} & \textbf{Descrizione}                                                                                                     & \textbf{Fonte} \\
		\hline
		RVO1        & L'interfaccia degli utenti deve essere un'applicazione \textit{web responsive} del tipo \textit{single page application} & Capitolato     \\
		\hline
	\end{tabularx}
	\caption{Tabella dei requisiti di vincolo}
\end{table}
