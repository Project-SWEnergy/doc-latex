\section{Casi d'uso}

Di seguito sono riportati i casi d'uso individuati
dall'analisi del gruppo sul capitolato d'appalto proposto. I casi d'uso sono stati raggruppati in tipologie, vengono riportate le nomenclature:

\begin{itemize}
	\item UC : indica un caso d'uso generale in cui sono coinvolti più attori principali.
	\item UCG : indica un caso d'uso strettamente legato all'Utente generico.
	\item UCB : indica un caso d'uso strettamente legato all'Utente base.
	\item UCR : indica un caso d'uso strettamente legato all'Utente ristoratore.
	\item UCE : indica un caso d'uso strettamente legato ad un errore.
\end{itemize}

\subsection{Attori}

Di seguito sono elencati gli attori individuati per il sistema e le relative
dipeendenze:
\begin{itemize}
	\item \textbf{Utente generico}: è un utente che non ha effettuato l'accesso al
	      sistema. Può essere un utente non registrato o un utente registrato che non ha
	      ancora effettuato l'accesso;

	\item \textbf{Utente base}: l'utente base può compiere tutte le azioni
	      dell'Utente generico. Inoltre, può effettuare l'accesso al sistema e può
	      effettuare delle prenotazioni e ciò che ne comporta. L'utente base rappresenta
	      il cliente del ristorante;

	\item \textbf{Utente ristoratore}: l'utente ristoratore può compiere tutte le
	      azioni dell'Utente generico. Inoltre, può gestire il proprio ristorante e le
	      prenotazioni ad esso associate. L'utente ristoratore rappresenta il gestore del
	      ristorante;

	\item \textbf{Utente autenticato}: l'utente autenticato rappresenta un utente
	      base oppure un utente ristoratore che ha effettuato l'accesso al sistema.
\end{itemize}
