\section{Casi d'uso}

Di seguito sono riportati i casi d'uso individuati per il sistema. Si noti che i
casi d'uso sono stati organizzati seguendo l'ordine del capitolato. SWEnergy
ritiene tuttavia, che l'ordine di presentazione dei casi d'uso debba essere
approfondito. Si è posta maggiore attenzione ai casi d'uso più impellenti.
Permangono dubbi e perplessità, principalmente di carattere generale. Siamo
interessati a ricevere un qualunque tipo di feedback, sia sulle cose
fatte bene, che su quelle fatte male. Anche i commenti più banali sono
benvenuti, perché è la prima volta che ci cimentiamo in questa attività.
Infine, non abbiamo approfondito tutti i casi d'uso che ci sono venuti in mente,
e nemmeno alcuni di quelli che abbiamo citato anche con il proponente. Non lo
riteniamo un problema, perché in ogni caso, ci saranno diversi casi d'uso che
andranno rivisti e modificati.

\subsection{Attori}

Di seguito sono elencati gli attori individuati per il sistema e le relative
dipeendenze:
\begin{itemize}
	\item \textbf{Utente esterno}: è un utente che non ha effettuato l'accesso al
	      sistema. Può essere un utente non registrato o un utente registrato che non ha
	      ancora effettuato l'accesso;

	\item \textbf{Utente base}: l'utente base può compiere tutte le azioni
	      dell'utente esterno. Inoltre, può effettuare l'accesso al sistema e può
	      effettuare delle prenotazioni e ciò che ne comporta. L'utente base rappresenta
	      il cliente del ristorante;

	\item \textbf{Utente ristoratore}: l'utente ristoratore può compiere tutte le
	      azioni dell'utente esterno. Inoltre, può gestire il proprio ristorante e le
	      prenotazioni ad esso associate. L'utente ristoratore rappresenta il gestore del
	      ristorante;

	\item \textbf{Utente autenticato}: l'utente autenticato rappresenta un utente
	      base oppure un utente ristoratore che ha effettuato l'accesso al sistema.
\end{itemize}
