\usecase{Chat Utente generico}
\label{usecase:Chat Utente generico}
\begin{itemize}
	\item \textbf{Attori principali:} 
	\begin{itemize}
        \item Utente generico.
        \item Utente ristoratore.
    \end{itemize}

	\item \textbf{Precondizioni:} L'Utente generico è connesso al sistema.


	\item \textbf{Postcondizioni:} La \textit{chat} è stata avviata.

	\item \textbf{Scenario principale:}
            \begin{enumerate}
                \item L'Utente generico inizia la \textit{chat} con l'invio di un messaggio (vedi \autoref{usecase:Invio messaggio chat});
                \item Il Sistema invia una notifica l'arrivo di un nuovo messaggio al destinatario (vedi \autoref{usecase:Notifica chat}).
                \item L'Utente ristoratore riceve la notifica e può leggere la \textit{chat} (vedi \autoref{usecase:Lettura chat}).
                \item Ora l'Utente generico e l'Utente ristoratore possono comunicare tra di loro attraverso l'uso della \textit{chat}, inviando e leggendo i messaggi vicendevolmente;
	      \end{enumerate}

    \item \textbf{Scenario secondario:}
		  \begin{itemize}
			  \item \autoref{usecase:Errore instaurazione chat} Errore instaurazione chat:
				\begin{enumerate}
					\item L'Utente generico invia un messaggio in chat all'Utente ristoratore.
					\item  Il Sistema mostra un messaggio di errore.
				\end{enumerate}
		  \end{itemize}
\end{itemize}