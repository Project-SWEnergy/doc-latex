\usecaseautenticato{Comunicazione attraverso chat}
\label{usecase:Comunicazione attraverso chat}
\begin{itemize}
	\item \textbf{Attore principale:} Utente autenticato.

	\item \textbf{Precondizione:} L'Utente autenticato ha effettuato l'accesso al Sistema (vedi \autoref{usecase:Effettua accesso}).

	\item \textbf{Postcondizione:} La \textit{chat} è stata avviata.

	\item \textbf{Scenario principale:}
            \begin{enumerate}
				\item L'Utente base oppure ristoratore inizia la \textit{chat} con l'invio di un messaggio (vedi \autoref{usecase:Invio messaggio chat});
                \begin{itemize}
					\item L'Utente ristoratore può iniziare la \textit{chat} con l'invio di un messaggio ad un Utente base soltanto se ha effettuato una prenotazione (ved \autoref{usecase:Prenotazione di un tavolo}).
				\end{itemize}
				\item La \textit{chat} viene creata e inizializzata;
				\item L'Utente base oppure ristoratore possono leggere i messaggi presenti in \textit{chat} (vedi \autoref{usecase:Lettura chat});
                \item Ora l'Utente base e l'Utente ristoratore possono comunicare tra di loro attraverso l'uso della \textit{chat}, inviando e leggendo i messaggi vicendevolmente.
	      \end{enumerate}

    \item \textbf{Scenario secondario:}
		  \begin{itemize}
			  \item \autoref{usecase:Errore instaurazione chat} Errore instaurazione chat:
				\begin{enumerate}
					\item L'Utente base oppure ristoratore invia un messaggio in chat al destinatario;
					\item Il Sistema mostra un messaggio di errore.
				\end{enumerate}
		  \end{itemize}
\end{itemize}