\usecase{Consulta lista degli ingredienti}
\label{usecase:Consulta lista degli ingredienti}
\begin{itemize}
	\item \textbf{Attore principale:} Utente ristoratore.

	\item \textbf{Attore secondario:} Sistema.

	\item \textbf{Precondizioni:}
	      L'Utente ristoratore ha effettuato l'accesso al sistema.

	\item \textbf{Postcondizioni:}
	      L'Utente ristoratore ha visualizzato la lista degli ingredienti.

	\item \textbf{Scenario principale:}
	      \begin{enumerate}
		      \item L'Utente ristoratore seleziona la funzionalità di
		            visualizzazione della lista degli ingredienti.

		      \item Il Sistema mostra la lista degli ingredienti;

		      \item L'Utente ristoratore applica un filtro temporale alla lista
		            degli ingredienti;

		      \item Il Sistema mostra la lista degli ingredienti
		            relativi all'arco temporale di interesse;

		      \item L'Utente ristoratore può esportare la lista degli
		            ingredienti.
	      \end{enumerate}

	\item \textbf{Descrizione:}
	      Il filtro temporale è espresso in termini di data di scadenza da oggi.
	      Probabilmente l'esportazione della lista degli ingredienti sarà
	      effettuata in formato \texttt{.csv}.
\end{itemize}
