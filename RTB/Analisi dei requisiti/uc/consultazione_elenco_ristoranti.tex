\usecasegenerico{Consultazione elenco ristoranti} % togliere utente base e quindi diventa un usecase del generico
\label{usecase:Consultazione elenco ristoranti}
\begin{itemize}
	\item \textbf{Attore principale:} Utente generico.

	\item \textbf{Precondizione:}
	      L'utente è connesso al Sistema.

	\item \textbf{Postcondizione:} L'utente visualizza una lista di ristoranti in base alla modalità da lui selezionata.
    Di \textit{default}, il Sistema mostra quelli con valutazione più alta.

	\item \textbf{Scenario principale:}
	      \begin{enumerate}
              \item Il Sistema mostra di \textit{default} all'utente un elenco di ristoranti in ordine decrescente di valutazione;
              
		      \item L'utente ha la possibilità di consultare l'elenco dei ristoranti in due modi:
		      \begin{itemize}
                \item Possibilità di ricercare il ristorante per nome o luogo(vedi \autoref{usecase:Ricerca ristoranti}).
                \item Basato sull'inserimento di filtri da parte dell'utente(vedi \autoref{usecase:Filtra ristoranti}).
              \end{itemize}

		      \item Il Sistema mostra l'elenco dei ristoranti in base alla modalità selezionata dall'utente.
		    
	      \end{enumerate}
\end{itemize}

\subusecasegenerico{Ricerca ristoranti}
\label{usecase:Ricerca ristoranti}
\begin{itemize}
	\item \textbf{Attore principale:} Utente generico.
	
	\item \textbf{Precondizione:} L'utente sta consultando un elenco di ristoranti (vedi \autoref{usecase:Consultazione elenco ristoranti}).

	\item \textbf{Postcondizione:} Il Sistema mostra l'elenco dei ristoranti in base ai risultati ottenuti dalla ricerca fatta dall'utente.
 
	      
	\item \textbf{Scenario principale:}
	      \begin{enumerate}
		      \item L'utente ricerca un ristorante in base al nome, oppure in base ad una località;

		      \item Il Sistema mostra l'elenco dei ristoranti aggiornato:
		      \begin{itemize}
                \item Mostra tutti i ristoranti con il nome ricercato dall'utente.
                \item Mostra tutti i ristoranti presenti nella località ricercata dall'utente.
              \end{itemize}
	      \end{enumerate}

\end{itemize}


\subusecasegenerico{Filtra ristoranti}
\label{usecase:Filtra ristoranti}
\begin{itemize}
	\item \textbf{Attore principale:} Utente generico.
	
	\item \textbf{Precondizione:} L'utente sta consultando un elenco di ristoranti (vedi \autoref{usecase:Consultazione elenco ristoranti}).

	\item \textbf{Postcondizione:} Il Sistema mostra l'elenco dei ristoranti in base ai filtri impostati dall'utente.
 
	      
	\item \textbf{Scenario principale:}
	      \begin{enumerate}
		      \item L'utente, per la ricerca di un ristorante, può impostare uno o più tra i seguenti filtri:
		      \begin{itemize}
                \item Orario.
                \item Voto.
                \item Cucina.
                \item Prezzo.
                \item Adatto a persone con ridotta mobilità.
                \item Adatto a bambini.
              \end{itemize}

		      \item Il Sistema mostra l'elenco dei ristoranti aggiornato.
	      \end{enumerate}

\end{itemize}