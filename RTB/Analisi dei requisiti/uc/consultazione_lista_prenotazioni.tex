\usecaseristoratore{Consultazione lista prenotazioni}
\label{usecase:Consultazione lista prenotazioni}
\begin{itemize}
	\item \textbf{Attore principale:} Utente ristoratore.

	\item \textbf{Precondizione:} L'Utente ristoratore ha effettuato l'accesso al Sistema (vedi \autoref{usecase:Effettua accesso}).

	\item \textbf{Postcondizione:} L'Utente ristoratore visualizza la lista di prenotazioni (prenotazioni che si trovano in qualsiasi stato) attraverso una vista a calendario settimanale.

	\item \textbf{Scenario principale:}
	      \begin{enumerate}
		      \item Il Sistema mostra una vista a calendario settimanale, in ogni giorno della settimana l'utente può consultare:
		      \begin{itemize}
				\item La lista di prenotazioni.
				\item La lista di ingredienti con le quantità necessarie per soddisfare la richiesta giornaliera (vedi \autoref{usecase:Consultazione lista ingredienti}).
			  \end{itemize} 

		      \item Il Sistema mostra la lista prenotazioni inerente ad un giorno nei due modi seguenti:
		      \begin{itemize}
				\item Di \textit{default}, nella lista, le prenotazioni che vengono mostrate per prima sono quelle che si trovano nello stato "Accettata".
				\item L'Utente ristoratore può applicare un filtro basato sullo stato della prenotazione, per facilitare la consultazione della lista a seconda dei suoi bisogni.
			  \end{itemize}

		      \item L'Utente ristoratore visualizza la lista di prenotazioni a vista settimanale;

		      \item L'Utente ristoratore visualizza la lista degli ingredienti necessari per soddisfare a richiesta giornaliera (vedi \autoref{usecase:Consultazione lista ingredienti}).
	      \end{enumerate}
\end{itemize}


\subusecaseristoratore{Consultazione lista ingredienti}
\label{usecase:Consultazione lista ingredienti}
\begin{itemize}

	\item \textbf{Attore principale:} Utente ristoratore.

	\item \textbf{Precondizione:} L'Utente ristoratore si trova nella sezione "Consultazione lista prenotazioni" (vedi \autoref{usecase:Consultazione lista prenotazioni}).

	\item \textbf{Postcondizione:} L'Utente ristoratore visualizza la lista ingredienti aggiornata.

	\item \textbf{Scenario principale:}
	\begin{enumerate}
		\item Il Sistema mostra la lista ingredienti necessari per la giornata (relativamente alle prenotazioni nello stato "Accettata");
		\item L'Utente ristoratore visualizza la lista ingredienti;
		\item Il Sistema aggiorna costantemente la lista ingredienti (vedi \autoref{usecase:Aggiornamento lista ingredienti}) della giornata non appena l'Utente ristoratore accetta una nuova prenotazione per quella giornata.
	\end{enumerate}

\end{itemize}
