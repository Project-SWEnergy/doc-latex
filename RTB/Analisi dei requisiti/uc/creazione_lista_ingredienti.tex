\usecase{Creazione di una lista di ingredienti da parte del business}
\label{usecase:Creazione di una lista ingredienti}
\begin{itemize}
	\item \textbf{Attore principale:} Utente ristoratore

	\item \textbf{Attore secondario:} Sistema

	\item \textbf{Precondizioni:} L'Utente ristoratore ha effettuato l'accesso
	      alla funzionalità di visualizzazione lista ingredienti 
		  ((vedi \autoref{usecase:Visualizzazione delle lista ingredienti}).)

	\item \textbf{Postcondizioni:}
	      L'Utente ristoratore ha creato una lista di ingredienti disponibili.

	\item \textbf{Scenario principale:}
	      \begin{enumerate}
		      \item L'Utente ristoratore seleziona la funzionalità di
		            creazione di una lista di ingredienti e associarla ad un ristorante;

		      \item Il Sistema attraverso un \textit{form} permette di inserire i vari ingredienti utilizzati dal ristorante;
					Si possono inserire varie informazioni, come il nome dell'ingrediente, la disponibilità, il prezzo (\SI{}{\euro\per\kilo\gram}) e a che pietanze assegnarle.
	      \end{enumerate}
\end{itemize}
