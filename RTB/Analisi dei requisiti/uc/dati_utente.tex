\usecasebase{Visualizzazione e modifica dati utente}
\label{usecase:Visualizzazione e modifica dati utente}
\begin{itemize}
	\item \textbf{Attore principale:} Utente base.

	\item \textbf{Precondizioni:}
	\begin{itemize}
        \item L'utente deve aver effettuato l'accesso al Sistema (vedi \autoref{usecase:Effettua accesso}).
        \item L'utente deve trovarsi nella sua area personale.
    \end{itemize}

	\item \textbf{Postcondizione:} L'Utente base visualizza i suoi dati, ed in caso modificarli.

	\item \textbf{Scenario principale:}
	      \begin{enumerate}
		      \item L'Utente base si trova nella sezione "area personale" e seleziona di visualizzare i propri dati;
		      \item Il Sistema mostra tutti i dati dell'Utente base:
              \begin{itemize}
                \item Nome.
                \item Cognome.
                \item \textit{E-mail}.
                \item \textit{Password}.
                \item Allergie e intolleranze.
              \end{itemize}
              \item L'Utente base può scegliere di modificare uno o più di questi dati;
              \item In caso di modifica, il Sistema memorizza ogni cambiamento.
	      \end{enumerate}
\end{itemize}
