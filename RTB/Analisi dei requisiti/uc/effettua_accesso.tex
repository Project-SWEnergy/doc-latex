\usecasegenerico{Effettua accesso}
\label{usecase:Effettua accesso}

\begin{itemize}
	\item \textbf{Descrizione:} Un Utente generico decide di effettuare l'accesso all'interno della \textit{web app}. Per effettuare tale
	operazione gli vengono presentate due opzioni:
	\begin{itemize}
		\item Accesso tramite inserimento di \textit{email} e \textit{password}.
		\item Accesso tramite un sistema di terze parti.
	\end{itemize}
	Nel caso in cui l'autenticazione fallisca, l'Utente generico deve inserire nuovamente le credenziali.

	\item \textbf{Attore principale:} Utente generico.
	\item \textbf{Attore secondario:} Sistema di autenticazione esterno.
	\item \textbf{Precondizione:}
	      Un Utente generico è connesso al Sistema ed è in possesso di un \textit{account}.
	\item \textbf{Postcondizione:}
	      L'Utente generico è stato identificato dal Sistema come uno solo dei seguenti:
	      \begin{itemize}
		      \item Utente ristoratore.
		      \item Utente base.
	      \end{itemize}
		  L'utente autenticato viene reindirizzato alla pagina \textit{Home} di pertinenza.

	\item \textbf{Scenario principale:}
	      \begin{enumerate}
		      \item L'Utente generico seleziona la tipologia di accesso: 

			  \begin{itemize}
				\item Accesso per terze parti (vedi \autoref{usecase:Accesso per terze parti}).
				\item Accesso tradizionale (vedi \autoref{usecase:Accesso tradizionale}).
			  \end{itemize}

		      \item Il Sistema verifica se l'Utente generico è un Utente ristoratore oppure Utente base;
		      \item L'Utente autenticato viene reindirizzato alla \textit{Home} impostata a seconda della tipologia di \textit{account}.		
	      \end{enumerate}
		
	\item \textbf{Scenario secondario:}
	\begin{itemize}
		\item \autoref{usecase:Accesso fallito} Accesso fallito:
		\begin{enumerate}
			\item L'autenticazione è fallita (vedi \autoref{usecase:Accesso fallito});
			\item L'Utente generico viene indirizzato nuovamente nella pagina di accesso.
		\end{enumerate}	
	\end{itemize}

		

\end{itemize}

\subusecasegenerico{Accesso per terze parti}
\label{usecase:Accesso per terze parti}
\begin{itemize}

	\item \textbf{Attore principale:} Utente generico.
	\item \textbf{Attore secondario:} Sistema di autenticazione esterno.

	\item \textbf{Precondizione:} Un Utente generico è connesso al Sistema ed è in possesso di un \textit{account}.

	\item \textbf{Postcondizione:} Sono state verificate \textit{email} e \textit{password} dal Sistema di autenticazione esterno.

	\item \textbf{Scenario principale:}
	\begin{enumerate}
		\item Il Sistema reindirizza l'Utente generico alla pagina di accesso per terze parti;
		\item Il Sistema di autenticazione esterno verifica le credenziali;
		\item Il Sistema di autenticazione esterno invia al Sistema le informazioni dell'Utente generico;
		\item Il Sistema di autenticazione esterno reindirizza l'Utente generico alla pagina di accesso.
	\end{enumerate}
	
\end{itemize}

\subusecasegenerico{Accesso tradizionale}
\label{usecase:Accesso tradizionale}
\begin{itemize}

	\item \textbf{Attore principale:} Utente generico.

	\item \textbf{Precondizione:} Un Utente generico è connesso al Sistema ed è in possesso di un \textit{account}.

	\item \textbf{Postcondizione:} Sono state verificate \textit{email} e \textit{password}.

	\item \textbf{Scenario principale:}
	\begin{enumerate}
		\item L'Utente generico inserisce le credenziali del suo \textit{account};
		\item Il Sistema verifica le credenziali.
	\end{enumerate}

\end{itemize}
