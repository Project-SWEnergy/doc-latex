\usecasegenerico{Effettua registrazione}
\label{usecase:Effettua registrazione}

\begin{itemize}
	\item \textbf{Descrizione:} Un Utente generico decide di effettuare la registrazione all'interno della \textit{web app}. Per effettuare tale
	operazione gli viene chiesto la tipologia di utente tra i due disponibili: Utente base oppure Utente ristoratore. 
    A seguito di tale decisione l'utente deve inserire una serie di dati relativi alla registrazione e legati alla tipologia di utenza.

	\item \textbf{Attore principale:} Utente generico.
	\item \textbf{Attore secondario:} Sistema di autenticazione esterno.
	\item \textbf{Precondizioni:}
        \begin{itemize}
            \item Un Utente generico è connesso al Sistema.
            \item L'Utente generico non dispone di un \textit{account}.
        \end{itemize}
	\item \textbf{Postcondizioni:}
	    \begin{enumerate}
            \item L'Utente generico ha creato un \textit{account} tra i seguenti:
            \begin{itemize}
                \item Utente ristoratore.
                \item Utente base.
            \end{itemize}
            \item Le informazioni del nuovo \textit{account} creato vengono tutte memorizzate all'interno del Sistema.
            \item L'utente autenticato viene reindirizzato alla pagina \textit{Home} di pertinenza.
        \end{enumerate}


	\item \textbf{Scenario principale:}
	      \begin{enumerate}
		      \item L'Utente generico seleziona la tipologia di \textit{account} che vuole creare: 
		      \begin{itemize}
				\item Utente base (vedi \autoref{usecase:Registrazione Utente base}).
				\item Utente ristoratore (vedi \autoref{usecase:Registrazione Utente ristoratore}).
			  \end{itemize} 
              \item Il Sistema memorizza l'\textit{account} appena creato;
		      \item L'utente è stato autenticato e viene reindirizzato alla \textit{Home} dedicata.
	      \end{enumerate}
		
    \item \textbf{Scenario secondario:}
          \begin{itemize}
              \item \autoref{usecase:Registrazione fallita} Registrazione fallita:
              \begin{enumerate}
                  \item La registrazione è fallita perchè esiste già un account con le credenziali inserite (vedi \autoref{usecase:Errore registrazione account esistente}).
                  \item La registrazione è fallita perchè l'Utente ristoratore ha inserito come recapito del ristorante uno già occupato (vedi \autoref{usecase:Errore registrazione recapito occupato}).
              \end{enumerate}	
          \end{itemize}
	
\end{itemize}


\subusecasegenerico{Registrazione Utente base}
\label{usecase:Registrazione Utente base}
\begin{itemize}

	\item \textbf{Attore principale:} Utente generico.
	\item \textbf{Attore secondario:} Sistema di autenticazione esterno.

	\item \textbf{Precondizioni:} 
	\begin{itemize}
        \item  Un Utente generico è connesso al Sistema.
        \item  Un Utente generico vuole registrarsi come Utente base.
    \end{itemize}
    

	\item \textbf{Postcondizioni:} 
    \begin{itemize}
        \item  Sono state verificate tutte le informazioni inserite durante la fase di registrazione.
        \item  L'Utente generico ha creato un \textit{account} come Utente base.
    \end{itemize}

	\item \textbf{Scenario principale:}
	\begin{enumerate}

            \item Se ha selezionato la creazione per terze parti allora il Sistema reindirizza l'utente alla pagina del Sistema esterno di terze parti (vedi \autoref{usecase:Accesso per terze parti});
            \item Se ha selezionato l'accesso tradizionale allora dovrà inserire:
            \begin{itemize}
                \item Nome.
                \item Cognome.
                \item \textit{Email}.
                \item \textit{Password}.
            \end{itemize}
            \item Indipendentemente dalla modalità di creazione dell' \textit{account}, l'utente infine dovrà inserire le proprie allergie e intolleranze.
            
	\end{enumerate}
	
\end{itemize}

\subusecasegenerico{Registrazione Utente ristoratore}
\label{usecase:Registrazione Utente ristoratore}
\begin{itemize}

	\item \textbf{Attore principale:} Utente generico.
	\item \textbf{Attore secondario:} Sistema di autenticazione esterno. 

	\item \textbf{Precondizioni:} 
	\begin{itemize}
        \item  Un Utente generico è connesso al Sistema.
        \item  Un Utente generico vuole registrarsi come Utente ristoratore.
    \end{itemize}
    

	\item \textbf{Postcondizioni:} 
    \begin{itemize}
        \item  Sono state verificate tutte le informazioni inserite durante la fase di registrazione.
        \item  L'Utente generico ha creato un \textit{account} come Utente ristoratore.
    \end{itemize}

	\item \textbf{Scenario principale:}
	\begin{enumerate}

            \item Se ha selezionato la creazione per terze parti allora il Sistema reindirizza l'utente alla pagina del Sistema esterno di terze parti (vedi \autoref{usecase:Accesso per terze parti});
            \item Se ha selezionato l'accesso tradizionale allora dovrà inserire:
            \begin{itemize}
                \item Nome.
                \item Cognome.
                \item \textit{Email}.
                \item \textit{Password}.
            \end{itemize}

            \item Indipendentemente dalla modalità di creazione dell' \textit{account}, l'utente infine dovrà inserire le seguenti informazioni:
                \begin{itemize}
                    \item Nome del ristorante.
                    \item Recapiti del risorante.
                    \item Orari di apertura del ristorante.
                    \item Numero dei coperti disponibili.
                    \item Tipologia di cucina.
                \end{itemize}
            
	\end{enumerate}
	
\end{itemize}