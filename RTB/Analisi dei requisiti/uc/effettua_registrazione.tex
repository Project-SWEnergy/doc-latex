\usecasegenerico{Effettua registrazione}
\label{usecase:Effettua registrazione}

\begin{itemize}
	\item \textbf{Descrizione:} Un Utente Generico decide di registrarsi all'interno della \textit{web app}. 
    Durante questa procedura, l'utente è tenuto a selezionare la tipologia di account desiderata tra le due opzioni disponibili: Utente Base o Utente Ristoratore. 
    Successivamente, all'utente viene richiesto di inserire una serie di dati correlati alla registrazione e specifici alla tipologia di utenza scelta.

	\item \textbf{Attore principale:} Utente generico.
	\item \textbf{Attore secondario:} Sistema di autenticazione esterno$^G$.
	\item \textbf{Precondizioni:}
        \begin{itemize}
            \item L'Utente generico è connesso al Sistema.
            \item L'Utente generico non dispone di un \textit{account}.
        \end{itemize}
	\item \textbf{Postcondizioni:}
        \begin{itemize} 
            \item L'Utente generico ha creato con successo un \textit{account} scegliendo tra le opzioni disponibili:
            \begin{itemize}
                \item Utente ristoratore.
                \item Utente base.
            \end{itemize}
            \item Tutte le informazioni relative al nuovo \textit{account} sono memorizzate all'interno del Sistema.
            \item L'utente autenticato viene reindirizzato alla pagina \textit{Home} di pertinenza.
        \end{itemize}


	\item \textbf{Scenario principale:}
	      \begin{enumerate}
		      \item L'Utente generico seleziona la tipologia di \textit{account} da creare: 
		      \begin{itemize}
				\item Utente base (vedi \autoref{usecase:Registrazione Utente base}).
				\item Utente ristoratore (vedi \autoref{usecase:Registrazione Utente ristoratore}).
			  \end{itemize} 
              \item Il Sistema memorizza con successo il nuovo \textit{account} creato.
		      \item L'utente è autenticato e viene reindirizzato alla pagina \textit{Home} corrispondente.
	      \end{enumerate}
		
    \item \textbf{Scenario secondario:}
                \begin{enumerate}
                    \item La registrazione fallisce (vedi \autoref{usecase:Registrazione fallita}) per due ragioni:
                    \begin{itemize}
                        \item "La registrazione fallisce a causa dell'esistenza di un \textit{account} preesistente con le stesse credenziali inserite (vedi \autoref{usecase:Errore registrazione account esistente}).
                        \item La registrazione non va a buon fine in quanto l'Utente Ristoratore ha inserito un recapito del ristorante già occupato (vedi \autoref{usecase:Errore registrazione recapito occupato}).
                    \end{itemize}
                    \item L’Utente generico viene nuovamente indirizzato alla pagina di accesso.
                \end{enumerate}	
          
	
\end{itemize}


\subusecasegenerico{Registrazione Utente base}
\label{usecase:Registrazione Utente base}
\begin{itemize}

	\item \textbf{Attore principale:} Utente generico.
	\item \textbf{Attore secondario:} Sistema di autenticazione esterno$^G$.

	\item \textbf{Precondizioni:} 
	\begin{itemize}
        \item  L'Utente generico è connesso al Sistema.
        \item  L'Utente generico desidera registrarsi come Utente base.
    \end{itemize}
    
	\item \textbf{Postcondizioni:} 
    \begin{itemize}
        \item  Tutte le informazioni inserite durante la fase di registrazione sono state verificate dal Sistema.
        \item  L'Utente generico ha creato un \textit{account} come Utente base.
    \end{itemize}

	\item \textbf{Scenario principale:}
	\begin{enumerate}

            \item Se l'Utente Generico ha selezionato la creazione attraverso terze parti, il Sistema reindirizza l'utente alla pagina del Sistema esterno di terze parti (vedi \autoref{usecase:Accesso per terze parti});
            \item Se l'Utente Generico ha scelto l'accesso tradizionale, dovrà inserire:
            \begin{itemize}
                \item Nome.
                \item Cognome.
                \item \textit{Email}.
                \item \textit{Password}.
            \end{itemize}
            \item Indipendentemente dalla modalità di creazione dell' \textit{account}, l'utente dovrà infine inserire le proprie informazioni relative alle:
            \begin{itemize}
                \item Allergie.
                \item Intolleranze.
            \end{itemize} 
	\end{enumerate}
\end{itemize}

\subusecasegenerico{Registrazione Utente ristoratore}
\label{usecase:Registrazione Utente ristoratore}
\begin{itemize}

	\item \textbf{Attore principale:} Utente generico.
	\item \textbf{Attore secondario:} Sistema di autenticazione esterno$^G$. 

	\item \textbf{Precondizioni:} 
	\begin{itemize}
        \item  L'Utente generico è connesso al Sistema.
        \item  L'Utente generico desidera registrarsi come Utente ristoratore.
    \end{itemize}
    

	\item \textbf{Postcondizioni:} 
    \begin{itemize}
        \item  Tutte le informazioni inserite durante la fase di registrazione sono state verificate dal Sistema.
        \item  L'Utente generico ha creato un \textit{account} come Utente ristoratore.
    \end{itemize}

	\item \textbf{Scenario principale:}
	\begin{enumerate}

            \item Se l'Utente generico ha selezionato la creazione attraverso terze parti, il Sistema reindirizza l'utente alla pagina del Sistema esterno di terze parti (vedi \autoref{usecase:Accesso per terze parti});
            \item Se l'Utente generico ha scelto l'accesso tradizionale, dovrà inserire:
            \begin{itemize}
                \item Nome.
                \item Cognome.
                \item \textit{Email}.
                \item \textit{Password}.
            \end{itemize}

            \item Indipendentemente dalla modalità di creazione dell' \textit{account}, l'utente dovrà infine inserire le seguenti informazioni relative al ristorante:
                \begin{itemize}
                    \item Denominazione.
                    \item Recapito.
                    \item Orari di apertura.
                    \item Numero dei coperti disponibili.
                    \item Tipologia di cucina.
                \end{itemize}
	\end{enumerate}
	
\end{itemize}