\usecasebase{Elimina account}
\label{usecase:Elimina account}
\begin{itemize}
	\item \textbf{Attore principale:} Utente base.

	\item \textbf{Precondizioni:}
	\begin{itemize}
        \item L'utente deve aver effettuato l'accesso (vedi \autoref{usecase:Effettua accesso}).
        \item L'utente deve trovarsi nella sua area personale.
    \end{itemize}

	\item \textbf{Postcondizioni:} L'Utente base elimina il suo \textit{account}.

	\item \textbf{Scenario principale:}
	      \begin{enumerate}
		      \item L'Utente base si trova nella sezione "area personale";
		      \item L'Utente base cancella il suo \textit{account};
              \item Il Sistema cancella l'\textit{account} e tutti i dati collegato ad esso;
              \item L'utente viene riportato alla \textit{home}.
	      \end{enumerate}
	\item \textbf{Scenario secondario:}
		  \begin{itemize}
			  \item \autoref{usecase:Errore eliminazione account} Errore eliminazione account:
				\begin{enumerate}
					\item L'Utente base prova ad eliminare il suo \textit{account}.
					\item Il Sistema mostra un messaggio di errore.
				\end{enumerate}
		  \end{itemize}
\end{itemize}
