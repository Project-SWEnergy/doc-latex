\usecaseerrore{Errore eliminazione account} %potenzialmente giusto l'attore princiale, siccome visualizza l'errore. non credo che serva come attore secondario il DB in questo caso, a differenza delle notifiche.
\label{usecase:Errore eliminazione account}
\begin{itemize}
	\item \textbf{Attore principale:} Utente base.

	\item \textbf{Precondizione:}
	      L'Utente base si trova all'interno della procedura di eliminazione dell'\textit{account} (vedi \autoref{usecase:Eliminazione account}).

	\item \textbf{Postcondizione:}
	      L'Utente base visualizza il messaggio di errore.

	\item \textbf{Scenario principale:}
	      \begin{enumerate}
		      \item L'Utente base tenta di eliminare il suo \textit{account}.
		      \item Il Sistema visualizza il messaggio di errore, comunicando che l'eliminazione non è stata completata con successo.
		      \item Il Sistema reindirizza l'Utente base nella sezione di eliminazione dell'\textit{account} (vedi \autoref{usecase:Eliminazione account}).
	      \end{enumerate}
\end{itemize}
