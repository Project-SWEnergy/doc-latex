\usecaseristoratore{Gestione informazioni ristorante}
\label{usecase:Gestione informazioni ristorante}
\begin{itemize}
	\item \textbf{Attore principale:} Utente ristoratore.

	\item \textbf{Precondizioni:} L'Utente ristoratore ha effettuato l'accesso al Sistema (vedi \autoref{usecase:Effettua accesso}).

	\item \textbf{Postcondizioni:} L'Utente ristoratore visualizza e modifica le informazioni del proprio ristorante.


	\item \textbf{Scenario principale:}
	      \begin{enumerate}

		      \item Il Sistema mostra le informazioni del ristorante:
		      \begin{itemize}
                \item Nome ristorante.
                \item Orario del ristorante.
                \item Descrizione del ristorante.
                \item Indirizzo del ristorante.
                \item Recapiti del ristorante.
                \item Password. 
                \item Link al sito del ristorante.
                \item Nome ristoratore.
                \item Costo ristorante.
                \item Disponibilità di sedie per bambini (numero).
                \item Adatto a persone con ridotta mobilità.
              \end{itemize}

		      \item L'Utente ristoratore visualizza tutte queste informazioni;
		      \item L'Utente ristoratore può modificare le seguenti informazioni;
		      \item Il Sistema registra le modifiche apporate alle informazioni dal ristoratore;

	      \end{enumerate}
\end{itemize}
