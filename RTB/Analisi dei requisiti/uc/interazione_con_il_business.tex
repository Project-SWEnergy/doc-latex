\usecase{Interazione con il business}
\label{usecase:Interazione con il business}
\begin{itemize}
	\item \textbf{Attore principale:} Utente base.
	\item \textbf{Attori secondari:}
	      \begin{itemize}
		      \item Utente ristoratore;
		      \item Sistema.
	      \end{itemize}
	\item \textbf{Precondizioni:} L'Utente base è connesso al Sistema.
	\item \textbf{Postcondizioni:} L'Utente base ha interagito con il business.
	\item \textbf{Scenario principale:}
	      \begin{enumerate}
		      \item L'Utente base accetta i termini di utilizzo delle
		            conversazioni, se non l'ha già fatto;

		      \item L'Utente base visualizza un ristorante (vedi
		            \autoref{usecase:Visualizzazione di un ristorante});

		      \item L'Utente base accede alla sezione per \textit{chattare} con
		            l'Utente ristoratore;

		      \item L'Utente base invia un messaggio all'Utente ristoratore;

		      \item Il Sistema instaura una canale di comunicazione
		            crittografato tra l'Utente base e l'Utente ristoratore, se
		            non l'ha già fatto;

		      \item L'Utente ristoratore e l'Utente base comunicano tra di loro
		            fino a quando l'Utente base non termina la conversazione (vedi
		            \autoref{usecase:Visualizza la conversazione}, \autoref{usecase:Invia un
			            messaggio} %e \autoref{usecase:Elimina la conversazione}
		            ).
	      \end{enumerate}
\end{itemize}

\subusecase{Visualizza la conversazione}
\label{usecase:Visualizza la conversazione}
\begin{itemize}
	\item \textbf{Attore principale:} Utente autenticato$_1$.
	\item \textbf{Attori secondari:}
	      \begin{itemize}
		      \item Utente autenticato$_2$;
		      \item Sistema.
	      \end{itemize}
	\item \textbf{Precondizioni:} L'Utente autenticato$_1$ è connesso al
	      Sistema.
	\item \textbf{Postcondizioni:} L'Utente autenticato$_1$ visualizza la conversazione
	      con l'Utente autenticato$_2$.

	\item \textbf{Scenario principale:}
	      \begin{enumerate}
		      \item L'Utente autenticato$_1$ accede alla sezione delle conversazioni;
		      \item L'Utente autenticato$_1$ seleziona la conversazione con l'Utente
		            autenticato$_2$;
		      \item Il Sistema mostra i messaggi scambiati tra l'Utente
		            autenticato$_1$ e l'Utente autenticato$_2$.
	      \end{enumerate}
\end{itemize}

\subusecase{Invia un messaggio}
\label{usecase:Invia un messaggio}
\begin{itemize}
	\item \textbf{Attore principale:} Utente autenticato$_1$.
	\item \textbf{Attori secondari:}
	      \begin{itemize}
		      \item Utente autenticato$_2$;
		      \item Sistema.
	      \end{itemize}
	\item \textbf{Precondizioni:} L'Utente autenticato$_1$ è connesso al
	      Sistema e sta visualizzando la conversazione con l'Utente autenticato$_2$ (vedi
	      \autoref{usecase:Visualizza la conversazione}).

	\item \textbf{Postcondizioni:} L'Utente autenticato$_1$ ha inviato un
	      messaggio all'Utente autenticato$_2$.
	\item \textbf{Scenario principale:}
	      \begin{enumerate}
		      \item L'Utente autenticato$_1$ invia un messaggio all'Utente
		            autenticato$_2$;
		      \item Il Sistema memorizza il messaggio nel database;
		      \item Il Sistema invia una notifica all'Utente autenticato$_2$.
	      \end{enumerate}
\end{itemize}

%\subusecase{Elimina la conversazione}
%\label{usecase:Elimina la conversazione}
%\begin{itemize}
%\item \textbf{Attore principale:} Utente autenticato$_1$.
%\item \textbf{Attore secondario:} Nessuno.
%\item \textbf{Precondizioni:} L'Utente autenticato$_1$ è connesso al
%Sistema e sta visualizzando la conversazione con l'Utente autenticato$_2$ (vedi
%\autoref{usecase:Visualizza la conversazione}).
%
%\item \textbf{Postcondizioni:} L'Utente autenticato$_1$ ha eliminato la
%conversazione con l'Utente autenticato$_2$.
%\item \textbf{Scenario principale:}
%\begin{enumerate}
%	\item L'Utente autenticato$_1$ elimina la conversazione con l'Utente
%	      autenticato$_2$;
%	\item Il Sistema cancella il \textit{link} tra l'Utente autenticato$_1$ e
%	      l'Utente autenticato$_2$ nel database.
%\end{enumerate}
