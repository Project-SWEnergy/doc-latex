\usecase{Modifica di una lista di ingredienti}
\label{usecase:Modifica di una lista ingredienti}
\begin{itemize}
	\item \textbf{Attore principale:} Utente ristoratore
	
	\item \textbf{Attore secondario:} Sistema

	\item \textbf{Precondizioni:} L'Utente ristoratore ha effettuato l'accesso
	alla funzionalità di visualizzazione lista ingredienti e ha selezionato una lista ingredienti.
	((vedi \autoref{usecase:Visualizzazione delle lista ingredienti}).)

	\item \textbf{Postcondizioni:}
	      L'Utente ristoratore ha modificato la lista di ingredienti selezionata.

	\item \textbf{Scenario principale:}
	      \begin{enumerate}
		      \item L'Utente ristoratore seleziona la funzionalità di
		            modifica della lista di ingredienti selezionata;

		      \item Il Sistema attraverso un \textit{form} permette di visualizzare e modificare i vari valori inseriti in precedenza.
					Inoltre da l'opportunità di aggiungere un ingrediente, modificarlo o eliminarlo;
				
			  \item Il Sistema salva le modifiche e aggiorna la lista di ingredienti;
			  
			  \item Il Sistema riporta l'Utente ristorante alla \textit{Visualizzazione delle liste ingredienti}.
	      \end{enumerate}
\end{itemize}
