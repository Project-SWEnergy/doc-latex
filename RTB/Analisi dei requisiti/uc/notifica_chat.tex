\usecase{Notifica chat}  %l'ipotesi è che l'attore principale interagisce attivamente relativamente all'UC quindi in questo caso vanno bene loro 3 come attori principali, perchè è un UC generale di ricezione notifica che tutti e 3 gli attori ricevono e cui interagiscono
%se il bisogno primario dell'attore è essere notificato, sarebbe corretto anche mettere come attore secondario il DB il quale viene interrogato dal sistema per soddifare il bisogno dell'attore principale.
\label{usecase:Notifica chat}
\begin{itemize}
    \item \textbf{Attore principale:} Utente autenticato.
	
	\item \textbf{Precondizione:} L'utente base oppure l'Utente ristoratore ha inviato un messaggio in \textit{chat} (vedi \autoref{usecase:Invio messaggio chat}).

	\item \textbf{Postcondizione:} Il Sistema invia la notifica al destinatario.
     
	\item \textbf{Scenario principale:}
	      \begin{enumerate}
                \item L'utente invia un messaggio;
                \item Il Sistema vede che al suo interno è stato memorizzato un nuovo messaggio;
                \item Il Sistema invia al destinatario la notifica di ricezione di un nuovo messaggio.
	      \end{enumerate}
\end{itemize}