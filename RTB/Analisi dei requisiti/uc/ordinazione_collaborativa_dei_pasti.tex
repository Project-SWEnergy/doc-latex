\usecasebase{Ordinazione collaborativa dei pasti}
\label{usecase:Ordinazione collaborativa dei pasti}
\begin{itemize}
	\item \textbf{Attore principale:} Utente base.

	\item \textbf{Precondizioni:}
		\begin{itemize}
			\item L'Utente base ha effettuato l'accesso al Sistema (vedi \autoref{usecase:Effettua accesso}).
			\item L'Utente base ha effettuato una prenotazione (bedi \autoref{usecase:Prenotazione di un tavolo}).
			\item L'Utente base sta visualizzando il riepilogo di una prenotazione (vedi \autoref{usecase:Visualizzazione del riepilogo prenotazione}) che si deve trovare nello stato: \textit{Confermata}  (vedi \autoref{usecase:Accetta prenotazione}).
		\end{itemize}
	      
	\item \textbf{Postcondizioni:} Un Utente base ha ordinato le pietanze per la prenotazione effettuata.

	\item \textbf{Scenario principale:}
	      \begin{enumerate}
		      \item \label{item:riepilogo}
		            Il Sistema mostra il riepilogo dell'ordinazione;

		      \item L'Utente base crea o modifica il proprio ordine
		            (vedi \autoref{usecase:Creazione e modifica del proprio ordine});

		      \item Se l'Utente base ha modificato il riepilogo dell'ordinazione: torna al punto \ref{item:riepilogo};

		      \item L'Utente base conferma il riepilogo dell'ordinazione;

		      \item Se tutti gli Utenti base che hanno accesso alla prenotazione
		            hanno confermato il riepilogo dell'ordinazione: il Sistema
		            notifica l'ordinazione all'Utente ristoratore (vedi\autoref{usecase:Notifica ordine}).
	      \end{enumerate}

	\item \textbf{Scenario secondario:}
	      \begin{itemize}
		      \item L'Utente base annulla l'ordinazione (vedi
		            \autoref{usecase:Annullamento dell'ordinazione}).
		            \begin{enumerate}
			            \item L'Utente base annulla l'ordinazione;
			            \item Il Sistema aggiorna l'ordinazione;
			            \item Il Sistema notifica l'annullamento dell'ordinazione
			                  all'Utente ristoratore.
		            \end{enumerate}
	      \end{itemize}
\end{itemize}


\subusecasebase{Creazione e modifica del proprio ordine}
\label{usecase:Creazione e modifica del proprio ordine}
\begin{itemize}
	\item \textbf{Attore principale:} Utente base.

	\item \textbf{Precondizioni:} L'Utente base sta effettuando un ordinazione collaborativa dei pasti (vedi \autoref{usecase:Ordinazione collaborativa dei pasti});

	\item \textbf{Postcondizioni:}
		\begin{itemize}
			\item L'Utente base ha creato o modificato il proprio ordine.
			\item Il Sistema aggiorna le informazioni inerenti al suo ordine.
		\end{itemize}

	\item \textbf{Scenario principale:}
	      \begin{enumerate}
		      \item L'Utente base seleziona una pietanza (vedi \autoref{usecase:Selezione pietanza});
		      \item L'Utente base modifica una propria pietanza (vedi \autoref{usecase:Modifica pietanza});
		      \item L'Utente base conferma il proprio ordine;
		      \item Il Sistema aggiorna il riepilogo dell'ordinazione.
	      \end{enumerate}
\end{itemize}


\subsubusecasebase{Selezione pietanza}
\label{usecase:Selezione pietanza}
\begin{itemize}
	\item \textbf{Attore principale:} Utente base.

	\item \textbf{Precondizioni:} L'Utente base si trova nella sezione Creazione e modifica del proprio ordine (vedi \autoref{usecase:Creazione e modifica del proprio ordine}).

	\item \textbf{Postcondizioni:} L'Utente base ha selezionato una pietanza.

	\item \textbf{Scenario principale:}
	      \begin{enumerate}
		      \item L'Utente base seleziona tra le lista delle pietanze una che vuole aggiungere al proprio ordine;
		      \item L'Utente base conferma la sua selezione;
		      \item Il Sistema registra la sua selezione.
	      \end{enumerate}
\end{itemize}

\subsubusecasebase{Modifica pietanza}
\label{usecase:Modifica pietanza}
\begin{itemize}
	\item \textbf{Attore principale:} Utente base.

	\item \textbf{Precondizioni:} L'Utente base si trova nella sezione Creazione e modifica del proprio ordine (vedi \autoref{usecase:Creazione e modifica del proprio ordine}).

	\item \textbf{Postcondizioni:} L'Utente base ha modificato una pietanza.

	\item \textbf{Scenario principale:}
	      \begin{enumerate}
		      \item L'Utente base seleziona tra le pietanze del suo ordine una che vuole modificare, ovvero:
		      \begin{itemize}
				\item Modifica della quantità della pietanza selezionata.
				\item Rimozione di ingredienti della pietanza selezionata.
				\item Aggiunta di ingredienti della pietanza selezionata.
			  \end{itemize}
		      \item L'Utente base conferma la sua modifica;
		      \item Il Sistema registra la sua modifica.
	      \end{enumerate}
\end{itemize}