\usecasegenerico{Ricerca di ristoranti}
\label{usecase:Ricerca di ristoranti}
\begin{itemize}
	\item \textbf{Attore principale:} Utente generico.

	\item \textbf{Precondizione:}
	      L'utente è connesso al Sistema.

	\item \textbf{Postcondizione:} L'utente ricerca dei ristoranti secondo tre modalità: per nome, per luogo, applicando dei filtri.

	\item \textbf{Scenario principale:}
	      \begin{enumerate}
		      \item Il Sistema presenta all'Utente generico tre modalità di ricerca dei ristoranti:
		            \begin{itemize}
			            \item Possibilità di ricercare il ristorante per nome (vedi \autoref{usecase:Ricerca ristoranti per nome}).
			            \item Possibilità di ricercare il ristorante per luogo (vedi \autoref{usecase:Ricerca ristoranti per luogo}).
			            \item Possibilità di ricercare il ristorante attraverso l'applicazione di filtri (vedi \autoref{usecase:Ricerca ristoranti per filtri}).
		            \end{itemize}

		      \item Il Sistema mostra l'elenco dei ristoranti in base alla modalità selezionata dall'utente.

	      \end{enumerate}
\end{itemize}


\subusecasegenerico{Ricerca ristoranti per nome}
\label{usecase:Ricerca ristoranti per nome}
\begin{itemize}
	\item \textbf{Attore principale:} Utente generico.

	\item \textbf{Precondizione:} L'utente è connesso al Sistema.

	\item \textbf{Postcondizione:} Il Sistema mostra l'elenco dei ristoranti secondo la ricerca per nome fatta dall'Utente generico.

	\item \textbf{Scenario principale:}
	      \begin{enumerate}
		      \item L'utente esegue una ricerca del ristorante in base al nome;
		      \item Il Sistema mostra il risultato della ricerca all'Utente generico.
	      \end{enumerate}
\end{itemize}

\subusecasegenerico{Ricerca ristoranti per luogo}
\label{usecase:Ricerca ristoranti per luogo}
\begin{itemize}
	\item \textbf{Attore principale:} Utente generico.

	\item \textbf{Precondizione:} L'utente è connesso al Sistema.

	\item \textbf{Postcondizione:} Il Sistema mostra l'elenco dei ristoranti secondo la ricerca per luogo fatta dall'Utente generico.

	\item \textbf{Scenario principale:}
	      \begin{enumerate}
		      \item L'utente esegue una ricerca del ristorante in base al luogo;
		      \item Il Sistema mostra il risultato della ricerca all'Utente generico.
	      \end{enumerate}
\end{itemize}


\subusecasegenerico{Ricerca ristoranti per filtri}
\label{usecase:Ricerca ristoranti per filtri}
\begin{itemize}
	\item \textbf{Attore principale:} Utente generico.

	\item \textbf{Precondizione:} L'utente è connesso al Sistema.

	\item \textbf{Postcondizione:} Il Sistema mostra l'elenco dei ristoranti secondo la ricerca basata sui filtri applicati dall'Utente generico.

	\item \textbf{Scenario principale:}
	      \begin{enumerate}
		      \item L'utente esegue una ricerca del ristorante in base all'applicazione di uno o più dei seguenti filtri:
		            \begin{itemize}
			            \item Orario.
			            \item Voto.
			            \item Cucina.
			            \item Prezzo.
			            \item Accessibilità per persone con ridotta mobilità.
			            \item Adatto a bambini.
		            \end{itemize}
		      \item Il Sistema mostra il risultato della ricerca all'Utente generico.
	      \end{enumerate}
\end{itemize}
