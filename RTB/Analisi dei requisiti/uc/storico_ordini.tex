\usecasebase{Storico ordini}
\label{usecase:Storico ordini}
\begin{itemize}
	\item \textbf{Attore principale:} Utente base.

	\item \textbf{Precondizioni:}
	\begin{itemize}
        \item L'utente deve aver effettuato l'accesso al Sistema (vedi \autoref{usecase:Effettua accesso}).
        \item L'utente deve trovarsi nella sua area personale.
    \end{itemize}

	\item \textbf{Postcondizione:} L'Utente base visualizza lo storico dei suoi ordini passati.

	\item \textbf{Scenario principale:}
	      \begin{enumerate}
		      \item L'Utente base si trova nella sezione "area personale" e seleziona di visualizzare lo storico dei suoi ordini;
		      \item Il Sistema mostra la lista dei suoi ordini passati in ordine dal più recente al meno recente;
              \item L'Utente base può scegliere un ordine che gli è particolarmente piaciuto e riaprire una nuova prenotazione con il ristorante (vedi \autoref{usecase:Prenotazione di un tavolo}).
	      \end{enumerate}
\end{itemize}
