\usecase{Visualizzazione delle liste di ingredienti da parte del business}
\label{usecase:Visualizzazione delle lista ingredienti}
\begin{itemize}
	\item \textbf{Attore principale:} Utente ristoratore
	
	\item \textbf{Attore secondario:} Sistema

	\item \textbf{Precondizioni:} L'Utente ristoratore ha effettuato l'accesso
	      al sistema.

	\item \textbf{Postcondizioni:}
	      L'Utente ristoratore visualizza le lista di ingredienti che ha creato.

	\item \textbf{Scenario principale:}
	      \begin{enumerate}
		      \item L'Utente ristoratore seleziona la funzionalità di
		            visualizzazione delle liste di ingredienti;

		      \item Il Sistema permette di visualizzare il nome della lista di ingredienti 
			  indicando anche il relativo ristorante a cui si riferisce.

			  \item L'Utente ristoratore può selezionare una di queste liste per scegliere se 
			  modificarla o eliminarla.
			  Inoltre può accedere alla funzionalità di creazione di una nuova lista di ingredienti.
	      \end{enumerate}
\end{itemize}
