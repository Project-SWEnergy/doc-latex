\usecase{Visualizzazione del riepilogo}
\label{usecase:Visualizzazione del riepilogo}
\begin{itemize}
	\item \textbf{Attore principale:} Utente base.

	\item \textbf{Attore secondario:} Sistema.

	\item \textbf{Precondizioni:}
	      Un Utente base ha effettuato l'accesso al Sistema ed è associato ad una
	      prenotazione (vedi \autoref{usecase:Prenotazione di un tavolo},
	      \autoref{usecase:Accedi alla prenotazione}).

	\item \textbf{Postcondizioni:}
	      L'Utente base visualizza il riepilogo della prenotazione.

	\item \textbf{Scenario principale:}
	      \begin{enumerate}
		      \item L'Utente base accede alla sezione delle prenotazioni;
		      \item L'Utente base seleziona una prenotazione;
		      \item Il Sistema mostra il riepilogo della prenotazione.
	      \end{enumerate}

	\item \textbf{Descrizione:}
	      I campi del riepilogo sono almeno:
	      \begin{itemize}
		      \item \textbf{Nome ristorante:} nome del ristorante;
		      \item \textbf{Data e ora:} data e ora della prenotazione;
		      \item \textbf{Numero di persone:} numero di persone per cui è
		            stata effettuata la prenotazione;
		      \item \textbf{Username:} username dell'Utente base
		            associato alla prenotazione o lista degli username degli
		            Utenti base associati alla prenotazione;

		      \item \textbf{Stato:} stato della prenotazione. Una
		            prenotazione si può trovare in uno dei seguenti stati:
		            \begin{itemize}
			            \item \textbf{In attesa:} la prenotazione è
			                  in attesa di conferma da parte del ristorante;

			            \item \textbf{Confermata dal ristoratore:} la prenotazione è
			                  stata confermata dal ristorante;

			            \item \textbf{Annullata:} la prenotazione è
			                  stata annullata dal ristorante o dall'Utente base
			                  oppure è scaduta;

			            \item \textbf{Confermata dal cliente:} da tutti gli
			                  Utenti base (vedi
			                  \autoref{usecase:Ordinazione collaborativa dei pasti});

			            \item \textbf{In corso:} la prenotazione è
			                  in corso;

			            \item \textbf{Terminata:} la prenotazione è
			                  terminata e il conto è stato pagato.
		            \end{itemize}

		            Ogni volta che lo stato della prenotazione cambia, il Sistema
		            invia una notifica a tutti gli Utenti autenticati associati alla
		            prenotazione che non hanno effettuato l'azione che ha portato al
		            cambio di stato (vedi UC-12).
	      \end{itemize}

\end{itemize}
