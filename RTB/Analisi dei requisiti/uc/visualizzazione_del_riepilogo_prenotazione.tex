\usecasebase{Visualizzazione del riepilogo prenotazione}
\label{usecase:Visualizzazione del riepilogo prenotazione}
\begin{itemize}
	\item \textbf{Attore principale:} Utente base.

	\item \textbf{Precondizioni:}
	\begin{itemize}
		\item L'Utente base ha effettuato l'accesso al Sistema (vedi \autoref{usecase:Effettua accesso}).
		\item L'Utente base ha effettuato una prenotazione (bedi \autoref{usecase:Prenotazione di un tavolo}).
	\end{itemize}


	\item \textbf{Postcondizione:}
	      L'Utente base visualizza il riepilogo della prenotazione.

	\item \textbf{Scenario principale:}
	      \begin{enumerate}
		      \item L'Utente base accede alla sezione delle prenotazioni;
		      \item Il Sistema mostra di default le prenotazioni nel seguente stato:
		      \begin{itemize}
                \item In attesa.
                \item Accettata.
                \item In corso.
              \end{itemize}
		      \item L'Utente base seleziona una prenotazione da visualizzare in dettaglio;
		      \item Il Sistema mostra il riepilogo della prenotazione.
	      \end{enumerate}

	\item \textbf{Descrizione:}
	      I campi del riepilogo sono almeno:
	      \begin{itemize}
		      \item \textbf{Nome ristorante:} nome del ristorante.
		      \item \textbf{Data e ora:} data e ora della prenotazione.
		      \item \textbf{Numero di persone:} numero di persone per cui è
		            stata effettuata la prenotazione.
		      \item \textbf{Username:} \textit{username} dell'Utente base
		            associato alla prenotazione o lista degli username degli
		            Utenti base associati alla prenotazione.

		      \item \textbf{Stato:} stato della prenotazione. Una
		            prenotazione si può trovare in uno dei seguenti stati:
		            \begin{itemize}
			            \item \textbf{In attesa:} la prenotazione è
			                  in attesa di accettazione da parte del ristorante.

			            \item \textbf{Accettata:} la prenotazione è
			                  stata accettata dal ristorante.

			            \item \textbf{Rifiutata:} la prenotazione è
			                  stata rifiutata dal ristorante.

			            \item \textbf{In corso:} la prenotazione è
			                  in corso.

			            \item \textbf{Terminata:} la prenotazione è
			                  terminata e il conto è stato pagato.
						\item \textbf{Annullata:} la prenotazione è stata
			                  annullata.
		            \end{itemize}

		            Ogni volta che lo stato della prenotazione cambia, il Sistema
		            invia una notifica a tutti gli Utenti autenticati associati alla
		            prenotazione che non hanno effettuato l'azione che ha portato al
		            cambio di stato (\autoref{usecase:Notifica stato della prenotazione}).
	      \end{itemize}

\end{itemize}
