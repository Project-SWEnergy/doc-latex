\usecase{Visualizzazione di un ristorante}
\label{usecase:Visualizzazione di un ristorante}
\begin{itemize}
	\item \textbf{Attore principale:}    
	\begin{itemize}
        \item Utente generico.
        \item Utente base
    \end{itemize}

	\item \textbf{Precondizioni:}
	\begin{itemize}
        \item L'utente è connesso al Sistema.
        \item L'utente ha selezionato un ristorante dalla lista di ristoranti proposta (vedi \autoref{usecase:Consultazione elenco ristoranti}).
    \end{itemize}

	\item \textbf{Postcondizioni:} L'utente visualizza le informazioni del ristorante.

	\item \textbf{Scenario principale:}
		\begin{enumerate}
		    \item L'Utente seleziona il ristorante di cui vuole vedere i dettagli;
		    \item Il Sistema mostra le informazioni del ristorante:
		    \begin{itemize}
				\item \textbf{Nome:} nome del ristorante;
				\item \textbf{Descrizione:} descrizione del ristorante;
				\item \textbf{Orario:} giorni settimanali e fascia oraria in cui il ristorante è aperto;
				\item \textbf{Indirizzo:} indirizzo del ristorante;
				\item \textbf{Recapiti:} numeri di telefono, \textit{e-mail} o altri modi per contattare il ristorante che non comprendano l'utilizzo della chat;
				\item \textbf{Prezzo:} fascia di prezzo del ristorante;
				\item \textbf{Voto:} voto minimo del ristorante;
				\item \textbf{Cucina:} tipologia di alimenti serviti dal ristorante;
				\item \textbf{Menù:} menù del ristorante;
				\item \textbf{Link:} link ad un eventuale sito \textit{web} del ristorante. 
			\end{itemize}
	    \end{enumerate}

\end{itemize}
