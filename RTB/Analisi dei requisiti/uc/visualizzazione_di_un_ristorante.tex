\usecasegenerico{Visualizzazione ristorante}  
\label{usecase:Visualizzazione di un ristorante}
\begin{itemize}
	\item \textbf{Attore principale:} Utente generico.


	\item \textbf{Precondizioni:}
	\begin{itemize}
        \item L'utente è connesso al Sistema.
        \item L'utente ha selezionato un ristorante dalla lista di ristoranti proposta (vedi \autoref{usecase:Visualizzazione elenco ristoranti}).
    \end{itemize}

	\item \textbf{Postcondizione:} L'utente visualizza le informazioni dettagliate del ristorante.

	\item \textbf{Scenario principale:}
		\begin{enumerate}
		    \item L'Utente generico visualizza le informazioni relative al ristorante.
		    \item Il Sistema mostra le seguenti informazioni del ristorante:
		    \begin{itemize}
				\item \textbf{Nome:} denominazione del ristorante.
				\item \textbf{Descrizione:} una breve descrizione del ristorante.
				\item \textbf{Orario:} giorni della settimanana e fascia oraria in cui il ristorante è aperto.
				\item \textbf{Indirizzo:} ubicazione fisica del ristorante.
				\item \textbf{Recapiti:} numeri di telefono, indirizzo \textit{e-mail} o altri metodi di contatto escludendo l'utilizzo della \textit{chat}.
				\item \textbf{Prezzo:} fascia di prezzo del ristorante.
				\item \textbf{Voto:} media delle valutazioni del ristorante.
				\item \textbf{Cucina:} tipologia delle pietanze servite dal ristorante.
				\item \textbf{Menù:} elenco dei piatti offerti dal ristorante.
				\item \textbf{\textit{Link}:} collegamento ad un eventuale sito \textit{web} del ristorante. 
			\end{itemize}
	    \end{enumerate}

\end{itemize}
