\subsec{A}

\subsubsection*{API}
Acronimo di \textit{Application Programming Interface}, si riferisce a un 
insieme di definizioni e protocolli che permettono la creazione e l'integrazione di software applicativi.
Le \textit{API} consentono ai componenti del software di comunicare tra loro e con altri componenti esterni, 
facilitando l'interazione tra diversi sistemi e applicazioni.

\subsubsection*{Area personale}
Si riferisce a una sezione dedicata all'utente registrato all'interno di un sito. 
Questa area fornisce all'utente un accesso riservato e personalizzato, consentendogli di visualizzare e gestire i propri dati personali in modo sicuro e privato. 

\subsubsection*{Attore}
Si tratta di un'entità che interagisce con il sistema svolgendo delle azioni.
Può essere una persona o un sistema esterno. Ciascuna entità è caratterizzata
dall'insieme delle azioni che può compiere.

\subsubsection*{Architettura di Deployment}
Struttura che descrive come i componenti di un sistema software vengono distribuiti su vari nodi dell'infrastruttura IT. 
Include la disposizione di server, reti, database e altre risorse, e definisce come questi elementi interagiscono tra loro. 
L'architettura di deployment considera aspetti come la scalabilità, la sicurezza, le prestazioni e la disponibilità del sistema, 
garantendo che ogni componente sia posizionato in modo ottimale per soddisfare i requisiti operativi e di business.

\newpage
