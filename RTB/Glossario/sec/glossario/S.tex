\subsection{S}

\subsubsection{SAL}
Lo "Stato di Avanzamento del Lavoro" (SAL) è un incontro con il proponente che avviene ogni due settimane al fine di condividere i \textit{feedback} in entrambe le direzioni. 
Durante il SAL, il gruppo presenta il lavoro svolto durante lo \textit{sprint}, vengono discusse le modalità di lavoro, si raccolgono consigli dal proponente e si concordano gli obiettivi da raggiungere al termine del successivo \textit{sprint}.

\subsubsection{Sistema di autenticazione esterno}
È un meccanismo utilizzato per verificare l'identità di un utente in un sistema informatico o applicazione mediante l'uso di risorse esterne. 
Invece di gestire autonomamente il processo di autenticazione, il sistema si affida a un servizio esterno specializzato per verificare le credenziali dell'utente.

\subsubsection{Stato (di un ordine)}
Condizione corrente in cui si trova un ordine (vedi \S\ref{ordine}) nell'ambito
di un processo di acquisto.

\newpage
