\section{Introduzione}

\subsection{Scopo del Documento}
Questo documento, redatto dal team SWEnergy, definisce le norme e le
metodologie adottate per lo sviluppo del progetto "Easy Meal". L'obiettivo è
fornire una guida chiara e strutturata che faciliti la collaborazione all'
interno del team e garantisca la coerenza e la qualità del lavoro svolto. Le
norme qui presentate si ispirano agli standard ISO 12207-1995, adattati alle
specificità del progetto universitario in questione.

\subsection{Struttura del Documento}
La struttura del documento è organizzata in modo da riflettere i diversi
aspetti e fasi del ciclo di vita del software, suddivisi in processi primari,
di supporto e organizzativi, come delineato dagli standard ISO 12207-1995:

\begin{itemize}
	\item \textbf{Processi Primari:} Questa sezione descrive i processi
	      fondamentali nello sviluppo del software, includendo le fasi di
	      acquisizione, fornitura, sviluppo, utilizzo e manutenzione del
	      prodotto software.
	\item \textbf{Processi di Supporto:} In questa parte vengono trattati i
	      processi che supportano lo sviluppo del software, come la gestione
	      della configurazione, la verifica, la validazione, la qualità e la
	      risoluzione dei problemi.
	\item \textbf{Processi Organizzativi:} Questa sezione copre i processi
	      trasversali che aiutano a migliorare e mantenere l'efficienza dell'
	      ambiente di sviluppo, inclusi la gestione dei processi, delle
	      infrastrutture, il miglioramento dei processi e la formazione del 
		  personale.
\end{itemize}

Ogni sezione del documento è strutturata per fornire una descrizione 
dettagliata dei processi, il loro scopo, le attività coinvolte e gli strumenti 
utilizzati, offrendo così una base solida per la gestione del progetto "Easy 
Meal".
