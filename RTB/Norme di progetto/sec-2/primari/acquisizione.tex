\subsection{Acquisizione}

Il processo di acquisizione coinvolge la definizione dei requisiti di sistema e
\textit{software}, la valutazione e selezione dei potenziali fornitori, e la gestione del
contratto con il fornitore selezionato.

\subsubsection{Scopo}
Garantire che il \textit{software} acquisito soddisfi i requisiti stabiliti, rispetti i
vincoli di budget e di tempo, e sia conforme agli standard di qualità previsti.

\subsubsection{Attività}
\begin{enumerate}
	\item \textbf{Definizione dei requisiti}: Identificazione delle necessità e
	      delle aspettative degli \textit{stakeholder}.
	\item \textbf{Selezione del fornitore}: Valutazione delle offerte e scelta
	      del fornitore più adatto.
	\item \textbf{Gestione del contratto}: Definizione degli accordi
	      contrattuali, monitoraggio della conformità e gestione delle
	      modifiche.
	\item \textbf{Accettazione del \textit{software}}: Verifica e validazione del
	      \textit{software} consegnato rispetto ai requisiti concordati.
\end{enumerate}

\subsubsection{Strumenti}
Gli strumenti adottati includono zoom e teams per le comunicazioni a distanza
rispettivamente con il cliente e con il proponente. Inoltre, si utilizzano
\textit{Presentazioni} di Google e \textit{Advanced Slides} di Obsidian per la
creazione delle presentazioni e rendere più efficace ed efficiente la
comunicazione con il cliente e con il proponente.

\textit{Nota: SWEnergy è il fornitore, dunque SWEnergy svolge solo l'attività
	"Gestione del contratto" con il committente e con il proponente.}
