\subsection{Accertamento della Qualità}

L'accertamento della qualità è un processo di supporto fondamentale che
garantisce che il software soddisfi i requisiti di qualità stabiliti e le
aspettative degli stakeholder.
Questo processo include la definizione, implementazione, valutazione e
manutenzione delle procedure e delle politiche di qualità per assicurare che
il software prodotto sia di alta qualità.

\subsubsection{Scopo}
Assicurare che il software e le pratiche di sviluppo rispettino gli standard
di qualità prefissati, migliorando così la soddisfazione del cliente e la
affidabilità del prodotto.

\subsubsection{Attività}
\begin{enumerate}
	\item \textbf{Definizione delle Politiche di Qualità:} Stabilire gli
	      standard e le metriche di qualità in base ai requisiti del progetto e
	      alle aspettative degli stakeholder.
	\item \textbf{Implementazione delle Procedure di Qualità:} Applicare le
	      politiche attraverso metodi concreti e pratiche di sviluppo, come
	      revisioni del codice e test.
	\item \textbf{Valutazione della Conformità:} Verificare periodicamente che
	      il software e i processi di sviluppo rispettino le politiche di
	      qualità stabilite.
	\item \textbf{Manutenzione e Miglioramento Continuo:} Aggiornare le
	      politiche e le procedure di qualità in base ai feedback e ai risultati
	      delle valutazioni per promuovere il miglioramento continuo.
\end{enumerate}

\subsubsection{Strumenti}
Gli strumenti utilizzati nel processo di accertamento della qualità possono
includere software autoprodotti di automazione dei test e di raccolta dei dati.
