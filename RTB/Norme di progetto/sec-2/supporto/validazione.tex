\subsection{Validazione}

Il processo di validazione si concentra sulla conferma che i requisiti e il
sistema software o prodotto finito soddisfino il loro uso inteso specifico. La
validazione può essere condotta in fasi precedenti dello sviluppo e può
variare in termini di indipendenza.

\subsubsection{Implementazione del Processo}
L'implementazione del processo di validazione include le seguenti attività
principali:

\begin{enumerate}
	\item Valutare se il progetto richieda uno sforzo di validazione e il
	      grado di indipendenza organizzativa di tale sforzo.
	\item Stabilire un processo di validazione per validare il sistema o il
	      prodotto software se il progetto lo richiede. Selezionare i compiti di
	      validazione, inclusi i metodi, le tecniche e gli strumenti associati.
	\item Sviluppare e documentare un piano di validazione che includa gli
	      elementi soggetti a validazione, i compiti di validazione da svolgere,
	      le risorse, le responsabilità e il programma per la validazione, e le
	      procedure per inoltrare i rapporti di validazione all'acquirente e ad
	      altre parti interessate.
\end{enumerate}

Questo processo assicura che il sistema software o il prodotto finito siano
adeguatamente validati rispetto al loro uso previsto, contribuendo
significativamente all'affidabilità e alla soddisfazione dell'utente finale.
