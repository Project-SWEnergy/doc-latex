\subsection{Validazione}

Il processo di validazione si concentra sulla conferma che i requisiti e il
sistema \textit{software} o prodotto finito soddisfino il loro uso inteso
specifico. La validazione può essere condotta in fasi precedenti dello sviluppo.

\subsubsection{Scopo}

Questo processo assicura che il sistema \textit{software} o il prodotto finito
siano adeguatamente validati rispetto al loro uso previsto, contribuendo
significativamente all'affidabilità e alla soddisfazione dell'utente finale.

\subsubsection{Attività}
L'implementazione del processo di validazione include le seguenti attività
principali:

\begin{enumerate}
	\item \textbf{Identificazione:} valutare se il progetto richieda uno sforzo
	      di validazione e il grado di indipendenza organizzativa di tale
	      sforzo.
	\item \textbf{Pianificazione:} stabilire un processo di validazione per
	      validare il sistema o il prodotto \textit{software} se il progetto lo
	      richiede. Selezionare i compiti di validazione, inclusi i metodi, le
	      tecniche e gli strumenti associati.
	\item \textbf{Esecuzione:} eseguire i compiti di validazione pianificati
	      (vedi \autoref{approvazione-documento}).
	\item \textbf{Rapporti:} inoltrare i rapporti di validazione al committente
	      e al proponente.
\end{enumerate}
