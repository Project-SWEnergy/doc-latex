\subsection{Verifica}

Il processo di verifica è essenziale per assicurare che il codice prodotto sia
conforme alle aspettative e agli standard definiti. Questo processo coinvolge
una serie di attività dettagliate per valutare la qualità e la correttezza del
codice.

\subsubsection{Attività di Verifica}
Le seguenti attività sono fondamentali per il processo di verifica:

\begin{enumerate}
	\item \textbf{Commenti:} Il verificatore esamina i commenti nel codice per
	      comprendere lo scopo delle funzioni o dei metodi.
	\item \textbf{Funzionamento:} Viene effettuata una verifica statica del
	      corpo delle funzioni o dei metodi per assicurare il loro corretto
	      funzionamento.
	\item \textbf{Test:} Il verificatore controlla la presenza di almeno un test
	      per ciascuna funzione o metodo, per validare il loro corretto
	      funzionamento.
	\item \textbf{Edge Cases:} Vengono verificati i test per assicurare che
	      coprano tutti i casi particolari, garantendo così una copertura
	      completa.
	\item \textbf{Nomi:} Il verificatore controlla che i nomi utilizzati nel
	      codice rispettino le convenzioni e le regole di forma predefinite.
	\item \textbf{Correzioni:} Eventuali errori riscontrati durante la verifica
	      vengono riportati ai programmatori per le necessarie correzioni.
	\item \textbf{Aggiornamento della Versione:} Dopo che le correzioni sono
	      state apportate, il verificatore aggiorna la versione del codice.
	\item \textbf{Gestione delle Versioni:} La versione del codice (X.Y.Z) viene
	      aggiornata incrementando Z di 1 per modifiche minori che non
	      introducono nuove funzionalità. Se le modifiche introducono nuove
	      funzionalità, Y viene incrementato di 1 e Z reimpostato a 0.
\end{enumerate}

Questo processo assicura che il codice sia non solo funzionale ma anche
conforme agli standard qualitativi stabiliti, contribuendo significativamente
alla qualità generale del prodotto software.
