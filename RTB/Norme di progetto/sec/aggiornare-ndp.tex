\subsection{Aggiornamento delle "Norme di progetto"}
\label{aggiornare-ndp}

\subsubsection{Descrizione}

L'amministratore deve aggiornare il documento "Norme di progetto".

\subsubsection{Svolgimento}
Di seguito sono riportate le attività da completare per aggiornare il documento
"Norme di progetto":
\begin{itemize}
	\item \textbf{Identificazione delle attività}: in quale modo
	      l'amministratore ed il gruppo possono individuare le attività da
	      includere nel documento. Di seguito sono riportati i passi da
	      seguire:
	      \begin{enumerate}
		      \item \textbf{Nuovo compito}: durante gli incontri,
		            SWEnergy si rende conto che alcune attività si
		            presentano di frequente;

		      \item \textbf{Ipotesi}: SWEnergy ipotizza il flusso di lavoro da
		            svolgere per completare il compito. Sono stesi degli appunti
		            che verranno poi inseriti nel documento "Norme di progetto";

		      \item \textbf{Sperimentazione}: i componenti del gruppo che
		            svolgono l'attività, sperimentano diverse tecniche per
		            completare il compito, partendo dall'ipotesi iniziale;

		      \item \textbf{Perfezionamento}: i componenti che hanno
		            svolto l'attività, spiegano al gruppo il processo
		            seguito. SWEnergy lo discute e lo valuta;

		      \item \textbf{Formalizzazione}: l'amministratore inserisce il
		            compito nel documento "Norme di progetto". Nota bene:
		            viene modificato un documento, quindi si rimanda alla
		            sottosezione che illustra come redigere un documento
		            (vedi \autoref{redazione-documento}).
	      \end{enumerate}

	\item \textbf{Aggiornamento delle attività}: in seguito ad una discussione
	      organica a SWEnergy, l'amministratore modifica il
	      compito nel documento "Norme di progetto". Nota bene:
	      viene modificato un documento, quindi si rimanda alla
	      sottosezione che illustra come redigere un documento
	      (vedi \autoref{redazione-documento}).
\end{itemize}
