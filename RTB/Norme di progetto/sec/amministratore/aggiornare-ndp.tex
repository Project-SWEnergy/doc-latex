\subsection{Aggiornamento delle "Norme di progetto"}
\label{aggiornare-ndp}

\subsubsection{\textit{Trigger}}
\begin{itemize}
	\item Si discute di qualche processo da aggiungere o modificare durante un
	      \textit{meeting}.
\end{itemize}

\subsubsection{Scopo}
\begin{itemize}
	\item Mantenere il documento coerente rispetto al modello di sviluppo e ai
	      processi adottati da SWEnergy;

	\item Formalizzare i processi adottati da SWEnergy, per chiarire eventuali
	      dubbi e per facilitare l'individuazione di attività e processi da
	      svolgere;

	\item Mostrare l'evoluzione dell'organizzazione del lavoro di SWEnergy;

	\item Evidenziare i dubbi e le lacune intestini ai processi di sviluppo.
\end{itemize}

\subsubsection{Svolgimento}
\begin{itemize}
	\item \textbf{Identificazione delle attività}: in quale modo
	      l'amministratore ed il gruppo possono individuare le attività da
	      includere nel documento. Di seguito sono riportati i passi da
	      seguire:
	      \begin{enumerate}
		      \item \textbf{Nuova attività}: durante gli incontri,
		            SWEnergy si rende conto che alcune attività si
		            presentano di frequente;

		      \item \textbf{Ipotesi}: SWEnergy ipotizza il flusso di lavoro da
		            svolgere per completare l'attività. Sono stesi degli appunti
		            che verranno poi inseriti nel documento "Norme di progetto";

		      \item \textbf{Sperimentazione}: i componenti del gruppo che
		            svolgono l'attività, sperimentano diverse tecniche per
		            il suo completare, partendo dall'ipotesi iniziale;

		      \item \textbf{Perfezionamento}: i componenti che hanno
		            svolto l'attività, spiegano al gruppo il processo
		            seguito. SWEnergy lo discute e lo valuta;

		      \item \textbf{Formalizzazione}: l'amministratore inserisce
		            l'attività nel documento "Norme di progetto". \textit{Nota:
		            viene modificato un documento, quindi si rimanda alla
		            sottosezione che illustra come redigere un documento
		            (vedi \cref{redazione-documento})}.
	      \end{enumerate}

	\item \textbf{Aggiornamento delle attività}: in seguito ad una discussione
	      organica a SWEnergy, l'amministratore modifica l'attivtà
	      nel documento "Norme di progetto". \textit{Nota:
	      viene modificato un documento, quindi si rimanda alla
	      sottosezione che illustra come redigere un documento
	      (vedi \cref{redazione-documento})}.
\end{itemize}
