\subsection{Aggiornamento del "Piano di progetto"}
\label{aggiornare-pdp}

\subsubsection{Descrizione}

Il responsabile deve aggiornare il documento "Piano di progetto".

\subsubsection{\textit{Trigger}}
\begin{itemize}
	\item Inizio di uno sprint\g;
	\item Fine di uno sprint\g;
\end{itemize}

\subsubsection{Scopo}
\begin{itemize}
	\item Formalizzare la pianificazione delle attività da svolgere durante
	      lo sprint\g;

	\item Disambiguare la pianificazione;

	\item Aggiornare le informazioni relative ai rischi e al modello di
	      sviluppo;

	\item Aggiornare le informazioni utili alla verifica dello stato di
	      avanzamento del progetto;
\end{itemize}

\subsubsection{Svolgimento}
Di seguito sono riportate le attività da completare per aggiornare il documento
"Piano di progetto":
\begin{itemize}
	\item \textbf{Rischi e modello di sviluppo}: per quanto riguarda le sezioni
	      relative ai rischi e al modello di sviluppo, il responsabile aggiorna
	      le informazioni in esse contenute in base all'esperienza maturata
	      durante il periodo da responsabile;

	\item \textbf{Pianificazione}: il responsabile aggiorna la sezione di
	      pianificazione rispettando la struttura già definita nel documento.
	      Eventualmente può proporre modifiche alla struttura di pianificazione
	      di perido. Queste sono discusse nelle riunioni interne.
	      Di seguito sono riportati i passi da seguire per aggiornare la sezione
	      di pianificazione:
	      \begin{enumerate}
		      \item \textbf{Creazione}: nella cartella \texttt{preventivi} viene
		            aggiunto un nuovo file \texttt{MM\_GG-P.tex} dove
		            \texttt{MM} e \texttt{GG} indicano rispettivamente il mese e
		            il giorno di inizio del periodo di riferimento;

		      \item \textbf{Stesura}: seguendo la struttura definita nei
		            preventivi precedenti, il responsabile stila la
		            sotto-sezione, riportando le informazioni di pianificazione
		            relative al periodo di riferimento presenti sul progetto di
		            \textit{GitHub\g}.
	      \end{enumerate}

	\item \textbf{Consuntivo}: medesimo procedimento della sezione di
	      pianificazione. Di seguito sono riportati i passi da seguire per
	      aggiornare la sezione di consuntivo:
	      \begin{enumerate}
		      \item \textbf{Creazione} nella cartella \texttt{consuntivi} viene
		            aggiunto un nuovo file \texttt{MM\_GG-C.tex} dove
		            \texttt{MM} e \texttt{GG} indicano rispettivamente il mese e
		            il giorno di inizio del periodo di riferimento;

		      \item \textbf{Stesura}: seguendo la struttura definita nei
		            consuntivi precedenti, il responsabile stila la
		            sotto-sezione, riportando le informazioni di consuntivo
		            relative al periodo di riferimento. Nota bene: le
		            \textit{issue\g} di \textit{GitHub\g} usate per tenere traccia
		            del consuntivo sono quelle generate e chiuse dai membri del
		            gruppo e non quelle create dal responsabile (che
		            sono invece usate per stilare il preventivo).
	      \end{enumerate}

	\item \textbf{Modifica di un documento}: dal momento che
	      l'aggiornamento del documento "Piano di progetto" rientra nella
	      casistica di modifica di un documento, si rimanda alla sezione
	      che illustra come redigere un documento (vedi
	      \ref{redazione-documento}).
\end{itemize}
