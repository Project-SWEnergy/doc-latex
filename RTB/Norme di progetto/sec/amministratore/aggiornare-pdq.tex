\subsection{Aggiornamento del "Piano di qualifica"}
\label{aggiornare-pdq}

\subsubsection{\textit{Trigger}}
\begin{itemize}
	\item Termina uno sprint;
\end{itemize}

\subsubsection{Scopo}
\begin{itemize}
	\item Mantenere sotto controllo la qualità del prodotto;

	\item Mantenere sotto controllo l'andamento del progetto;

	\item Documentare quanto qui sopra, per poterlo mostrare al committente
	      durante le revisioni e per evidenziarne l'evoluzione nel tempo.
\end{itemize}

\subsubsection{Svolgimento}
\begin{itemize}
	\item \textbf{Identificazione di una metrica}: in quale modo
	      l'amministratore ed il gruppo possono individuare le metriche utili a
	      controllare e valutare la qualità del prodotto. Di seguito sono
	      descritti i passi da seguire:
	      \begin{enumerate}
		      \item \textbf{Nuova metrica}: durante gli incontri, uno dei
		            componenti di SWEnergy propone una nuova metrica da adottare
		            per valutare la qualità del prodotto;

		      \item \textbf{Discussione}: i componenti del gruppo discutono in
		            merito alla metrica proposta: se è utile, se è applicabile
		            ed in quale modo verificare i risultati ottenuti e
		            formalizzarli;

		      \item \textbf{Formalizzazione}: l'amministratore inserisce la
		            metrica di qualità nel documento "Piano di qualifica". Nota
		            bene: viene modificato un documento, quindi si rimanda alla
		            sottosezione che illustra come redigere un documento
		            (vedi \cref{redazione-documento}).
	      \end{enumerate}

	\item \textbf{Aggiornamento di una metrica}: In seguito ad una discussione
	      organica a SWEnergy, l'amministratore modifica qualche caratteristica
	      di una metrica nel documento "Piano di qualifica". Nota bene:
	      viene modificato un documento, quindi si rimanda alla
	      sottosezione che illustra come redigere un documento
	      (vedi \cref{redazione-documento}).

	\item \textbf{Misurazione}: l'amministratore misura i risultati ottenuti
	      applicando le metriche di qualità. Di seguito sono descritti i passi
	      di aggiornamento del documento "Piano di qualifica":
	      \begin{enumerate}
		      \item \textbf{Nuovi risultati}: alla fine di ogni sprint,
		            l'amministratore e il gruppo valutano i risultati di qualità
		            ottenuti applicando le metriche concordate;

		      \item \textbf{Discussione}: I risultati sono discussi durante la
		            retrospettiva e, se ritenuto opportuno, sono modificati
		            gli obiettivi di qualità adottati da SWEnergy;

		      \item \textbf{Inserimento dei risultati}: l'amministratore
		            inserisce i risultati ottenuti nel documento "Piano di
		            qualifica".
		            Nota bene: viene modificato un documento, quindi si rimanda
		            alla sottosezione che illustra come redigere un documento
		            (vedi \cref{redazione-documento}).
	      \end{enumerate}
\end{itemize}
