\subsection{Approvare un documento}
\label{approvazione-documento}

Per approvare un documento, è necessario che due verificatori diversi
verifichino il documento. Questo non vale per i verbali, che hanno bisogno di
una sola verifica.

\subsubsection{Svolgimento}

Per verificare la correttezza di un documento, i verificatori devono
completare le seguenti attività:
\begin{itemize}
	\item \textbf{Approvazione:}
	      \begin{enumerate}
		      \item \textbf{Prima verifica:} un verificatore verifica il
		            documento (vedi \autoref{verifica-documento}) e segnala ad
		            un secondo verificatore che il documento è pronto per la
		            seconda verifica;

		      \item \textbf{Seconda verifica}: un verificatore diverso dal primo
		            verifica il documento (vedi \autoref{verifica-documento});

		      \item \textbf{Aggiornamento della versione}: dopo che il documento
		            viene corretto dall'autore, il secondo verificatore deve
		            aggiornare la versione del documento;

		      \item \textbf{Versione}: sia $X.Y.Z$ la versione del documento,
		            dopo l'approvazione, il valore di $X$ viene incrementato di
		            $1$, mentre $Y$ e $Z$ vengono azzerati.
	      \end{enumerate}
\end{itemize}
