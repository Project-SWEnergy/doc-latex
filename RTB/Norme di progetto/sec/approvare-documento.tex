\subsection{Approvare un documento}
\label{approvazione-documento}
Di seguito viene descritto il processo di approvazione di un documento.

\subsubsection{\textit{Trigger}}
\begin{itemize}
	\item Un documento viene completato.
\end{itemize}

\subsubsection{Scopo}
\begin{itemize}
	\item Assicurarsi che il documento soddisfi i requisiti ad esso associati;

	\item Convalidare il contenuto ed il completameto del documento.
\end{itemize}

\subsubsection{Svolgimento}

Per verificare la correttezza di un documento, il responsabile deve completare
le seguenti attività:
\begin{itemize}
	\item \textbf{Approvazione:}
	      \begin{enumerate}
		      \item \textbf{Seconda verifica}: il responsabile verifica il
		            documento (vedi \autoref{verifica-documento});

		      \item \textbf{Aggiornamento della versione}: dopo che il documento
		            viene corretto dall'autore, il responsabile aggiorna la
		            sua versione ed il suo stato;

		      \item \textbf{Versione}: sia $X.Y.Z$ la versione del documento,
		            dopo l'approvazione, il valore di $X$ viene incrementato di
		            $1$, mentre $Y$ e $Z$ vengono azzerati.
	      \end{enumerate}
\end{itemize}
