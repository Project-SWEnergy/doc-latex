\section{Documentazione}
\subsection{Strumenti}
Gli strumenti utilizzati per la creazione dei documenti sono:
\begin{itemize}
	\item \textbf{LaTeX}: linguaggio di \textit{markup} per la creazione di documenti \\
	      \href{https://www.latex-project.org/}{(www.latex-project.org)};
	\item \textbf{VisualStudio Code}: GUI con integrazioni per la creazione di documenti scritti in LaTeX e per la gestione delle repository git \\
	      \href{https://code.visualstudio.com/}{(code.visualstudio.com)}
	      \begin{itemize}
		      \item \textbf{LaTeX Workshop}: estensione utilizzata in VisualStudio Code per la compilazione e la scrittura dei documenti.
	      \end{itemize}
\end{itemize}


\subsection{Creazione e modifica di un documento}

A ciascun documento corrisponde un'omonima cartella che viene creata nella
\textit{repository}
\href{https://github.com/Project-SWEnergy/doc-latex}{\texttt{doc-latex}}
presente su GitHub. Il nome della cartella corrisponde al nome del documento che
deve avere la prima lettera maiuscola, sono previsti gli spazi tra le parole e
le parole successive alla prima sono in minuscolo.
La cartella di ciascun documento ha la seguente struttura:

\dirtree{%
	.1 / (Nome del documento).
	.2 main.tex.
	.2 sec.
	.3 registro\_modifiche.tex.
	.3 introduzione.tex.
	.3 le\_altre\_sezioni.tex.
}

\subsection{\textit{Template} dei documenti}

Di seguito sono elencati i \textit{template} che devono essere importati da
ciascun documento:
\begin{itemize}
	\item \textbf{Copertina}: il primo \textit{template} importato da ogni
	      documento è la copertina, che formatta la prima pagina del documento;
	      ciasun documento deve definire i comandi necessari per la costruzione
	      della copertina.

	\item \textbf{Header e footer}: ciascun documento deve importare il file
	      \texttt{header\_footer.tex} che definisce l'header e il footer di ogni
	      pagina del documento.

	\item \textbf{Variabile}: ciascun documento deve importare il file
	      \texttt{variable.tex} che definisce i comandi per la definizione delle
	      variabili globali tra i documenti; per esempio il nome del gruppo o la
	      mail del gruppo.

	\item \textbf{\textit{Template ad hoc}}: qualche documento potrebbe avere
	      dei comandi specifici per la propria creazione e stesura; per esempio,
	      l'"Analisi dei requisiti" ha bisogno di un comando per la creazione di
	      un caso d'uso.
\end{itemize}
Ogni documento creato dovrà quindi importare al suo interno i \textit{template}
appena descritti.
Nota bene, ogni documento include i \textit{template} sopra descritti con la
seguente sintassi:
\begin{lstlisting}[language=TeX]
	\documentclass[a4paper, 12pt]{article}
	\newcommand{\template}{../../templates}
	\usepackage{\template/package}
	\graphicspath{{../../assets}}

	\newcommand{\Titolo}{Norme di progetto}
	% ... altre variabili

	\newcommand{\Gruppo}{SWEnergy}
\newcommand{\Mail}{\href{mailto:project.swenergy@gmail.com}{project.swenergy@gmail.com}}
\newcommand{\g}{$^G$ }

\renewcommand\familydefault{\sfdefault} % Set default font family to sans-serif
\linespread{1.5}

\hypersetup{
	pdfmenubar=true,            % show Acrobat’s menu?
	pdfstartview={FitH},        % fits the width of the page to the window
	colorlinks=true,            % false: boxed links; true: colored links
	linkcolor=black,            % color of internal links (change box color with linkbordercolor)
	% citecolor=green,          % color of links to bibliography
	% filecolor=magenta,        % color of file links
	urlcolor=[RGB]{156,1,198}   % color of external links
}

% Define the subsubsubsection command
\newcounter{subsubsubsection}[subsubsection]
\renewcommand{\thesubsubsubsection}{\thesubsubsection.\arabic{subsubsubsection}}
\newcommand{\subsubsubsection}[1]{\refstepcounter{subsubsubsection}\subsubsection*{\thesubsubsubsection\quad#1}}

% Define the subsubsubsubsection command
\newcounter{subsubsubsubsection}[subsubsubsection]
\renewcommand{\thesubsubsubsubsection}{\thesubsubsubsection.\arabic{subsubsubsubsection}}
\newcommand{\subsubsubsubsection}[1]{\refstepcounter{subsubsubsubsection}\subsubsection*{\thesubsubsubsubsection\quad#1}}


% Define cref references
\crefname{section}{Sezione \S}{Sezioni}
\crefname{subsection}{Sottosezione \S}{Sottosezioni}
\crefname{subsubsection}{Sottosezione \S}{Sottosezioni}
\crefname{subsubsubsection}{Sottosezione \S}{Sottosezioni}
\crefname{subsubsubsubsection}{Sottosezione \S}{Sottosezioni}

\setcounter{secnumdepth}{5} % Set the section numbering depth

	\newcommand{\copertina}{
	\begin{titlepage}
		\vspace*{-3.5cm}
		\makebox[\textwidth]{\includegraphics[width=\paperwidth]{header.png}}
		\begin{center}
			\includegraphics[width=1\textwidth]{logo.png}	\\
			\vspace{1cm}
			\Mail	\\
			\vspace{0.5cm}
			\textbf{\begin{LARGE} \Titolo \end{LARGE}}		\\
			\vspace{1cm}
			\textbf{Descrizione:} \Descrizione{}			\\
			\vspace{1cm}
			\begin{tabular}{ll}
				\textbf{Stato}               & \Stato              \\
				\textbf{Data}                & \Data               \\
				\midrule
				\textbf{Redattori}           & \Redattori          \\
				\textbf{Verificatori}        & \Verificatori       \\

				\ifdefined\Approvatori
				\textbf{Approvatori}         & \Approvatori        \\
				\fi

				\ifdefined\ApprovatoriInterni
				\textbf{Approvatori interni} & \ApprovatoriInterni \\
				\fi

				\ifdefined\ApprovatoriEsterni
				\textbf{Approvatori esterni} & \ApprovatoriEsterni \\
				\fi

				\ifdefined\Destinatari
				\textbf{Destinatari}         & \Destinatari        \\
				\fi

				\midrule

				\ifdefined\Versione
				\textbf{Versione}            & \Versione           \\
				\fi
			\end{tabular}
		\end{center}
		\vspace{4cm}
	\end{titlepage}
	\newpage
}

	\fancypagestyle{plain}{
	\fancyhf{}
	\rhead{ \includegraphics[scale=0.05]{horizontal_logo.png}}
	\lhead{\Titolo \ifdefined\Versione \ \Versione \fi}
	%\lfoot{\Titolo}
	\rfoot{\thepage{} di \pageref{LastPage}}
	\renewcommand{\headrulewidth}{0.2pt}
	\renewcommand{\footrulewidth}{0.2pt}
}
\pagestyle{plain}

	% ... altri template
\end{lstlisting}

In questo modo è possibile mantenere uniforme la struttura di tutti i
documenti e semplificarne la creazione.

\subsection{Variabili dei documenti}

Ciascun documento deve definire alcune variabili che sono utilizzate per
costruirne la struttura e la copertina. Di seguito sono elencate le variabili da
definire per ciascun documento:
\begin{itemize}
	\item \textbf{Verbale interno}:
	      \begin{itemize}
		      \item \textbf{Titolo}: "Verbale interno".
		      \item \textbf{Data}: la data in cui ha avuto luogo la riunione.
		      \item \textbf{Versione}: la versione del verbale.
		      \item \textbf{Descrizione}: una breve descrizione del verbale.
		      \item \textbf{Stato}: lo stato del verbale, ovvero se è stato
		            approvato o meno.
		      \item \textbf{Redattori}: i redattori del verbale (in genere è il
		            responsabile).
		      \item \textbf{Verificatori}: i verificatori del verbale (in genere
		            è il verificatore).
	      \end{itemize}

	\item \textbf{Verbale esterno}:
	      \begin{itemize}
		      \item \textbf{Titolo}: "Verbale esterno - Imola Informatica".
		      \item \textbf{Data}: la data in cui ha avuto luogo la riunione.
		      \item \textbf{Versione}: la versione del verbale;
		      \item \textbf{Descrizione}: una breve descrizione del verbale.
		      \item \textbf{Stato}: lo stato del verbale, ovvero se è stato
		            approvato o meno.
		      \item \textbf{Redattori}: i redattori del verbale (in genere è il
		            responsabile).
		      \item \textbf{Verificatori}: i verificatori del verbale (in genere
		            è il verificatore).
		      \item \textbf{Approvatori esterni}: i nomi degli approvatori
		            esterni, ovvero dei proponenti.
	      \end{itemize}

	      Nota bene, ciascun verbale esterno deve essere firmato dai proponenti,
	      per questo motivo è necessario inserire uno spazio apposito alla fine
	      del documento attraverso il comando \texttt{\\firma\{\}} definito nel
	      \textit{template} \texttt{firma.tex}.

	\item \textbf{Documenti generici}:
	      \begin{itemize}
		      \item \textbf{Titolo}: il nome del documento.
		      \item \textbf{Data}: la data di creazione del documento.
		      \item \textbf{Versione}: la versione del documento.
		      \item \textbf{Descrizione}: una breve descrizione del documento.
		      \item \textbf{Stato}: lo stato del documento, ovvero se è stato
		            approvato o meno.
	      \end{itemize}

\end{itemize}

\subsection{Registro delle modifiche}
\label{documentazione_registromodifiche}
Ogni documento, esclusi verbali e presentazione, include subito dopo la
copertina un registro delle modifiche in forma tabellare, come indicato nei
riferimenti.
Vi sono le seguenti voci:
\begin{itemize}
	\item \textbf{Versione}: indica da versione del documento alla riga di
	      modifica.
	\item \textbf{Data}: indica la data di redazione, verifica o approvazione.
	\item \textbf{Redattore}: nome del componente del gruppo che ha effettuato
	      la redazione.
	\item \textbf{Verificatore}: nome del componente del gruppo che ha
	      effettuato la verifica.
	\item \textbf{Approvatore}: nome del componente del gruppo che ha effettuato
	      l'approvazione.
	\item \textbf{Descrizione}: breve descrizione della sezione oggetto di
	      redazione e verifica, in caso di approvazione indica l'azione svolta e
	      la fase di avanzamento attuale.
\end{itemize}
\noindent
La tabella viene riportata in ordine decrescente di modifica, così da mantenere
sempre in cima le azioni più recenti eseguite.

\subsection{Versionamento}
\label{documentazione_versionamento}
Ogni documento rilasciato dovrà presentare un versionamento interno indicato con
tre interi positivi nel formato \texttt{X.Y.Z}, dove:
\begin{itemize}
	\item \textbf{X}: da incrementare in di rilascio di un
	      documento, indica l'ultima versione ufficialmente rilasciata.
	      L'incremento di questa cifra comporta l'azzeramento delle cifre Y e Z
	      e avviene con modalità differenti a seconda del documento.

	\item \textbf{Y}: da incrementare in caso avvenga la verifica e
	      l'approvazione di un documento; indica il numero di verifiche
	      effettuate sul documento. L'incremento di questa cifra comporta
	      l'azzeramento della cifra Z.

	\item \textbf{Z}: da incrementare in caso avvenga modifica, creazione o
	      eliminazione di una qualunque porzione di testo del documento; indica
	      il numero di modifiche effettuate sul documento.
\end{itemize}

Ogni documento dovrà essere creato partendo dalla versione \texttt{0.0.1}, ogni
successivo incremento alla versione dovrà essere accompagnato da una nuova riga
nella tabella "Registro delle modifiche" che espliciti i cambiamenti effettuati.
Modifiche quali correzioni grammaticali o leggere variazioni nel testo non vanno
riportate nel registro delle modifiche e non producono un avanzamento di
versione.
La verifica di un documento comporta l'incremento della cifra Y, mentre
L'approvazione di un documento avviene solo quando il documento deve essere
rilasciato.

\subsection{Verifica di un documento}
Ogni documento deve essere necessariamente verificato da un Verificatore, il
quale si occuperà di revisionare i seguenti aspetti:
\begin{itemize}
	\item Correttezza grammaticale.
	\item Correttezza logica.
	\item Correttezza e completezza del contenuto in modo che risulti coerente
	      con il documento.
	\item Adesione al \textit{Way of Working}.
\end{itemize}
\noindent
In caso il verificatore dovesse riscontrare dei problemi questi vanno segnalati
su GitHub tramite i commenti presenti all'interno dell'apposita sezione
"Pull Request".
Questa azione genera una notifica immediata al redattore, il quale provvederà
ad apportare le modifiche necessarie. \\
In caso invece il documento non richieda modifiche si dovrà procedere ad un
avanzamento di versione come descritto in sezione
\S\ref{documentazione_versionamento}.


\subsection{Approvazione di un documento}
L'approvazione di un documento avviene solo quando il documento deve essere
rilasciato. A seconda del documento, l'approvazione interna è svolta in modo
diverso:
\begin{itemize}
	\item \textbf{Verbali}: un verbale viene approvato quando un verificatore
	      ritiene che il documento sia pronto per il rilascio.

	\item \textbf{Altri documenti}: l'incremento di questa cifra avviene
	      quando due verificatori ritengono che il documento sia pronto per il
	      rilascio.
\end{itemize}

Ogni documento approvato deve subire un avanzamento di versione come descritto
alla sezione \S\ref{documentazione_versionamento}.
\noindent
In caso il documento in questione sia un verbale esterno, il verificatore dovrà
occuparsi di effettuare una approvazione interna e, successivamente, inviare il
documento in allegato ad una \textit{email} al fine di ottenere l'approvazione
esterna del verbale. Una volta firmato il documento, il verificatore dovrà
inserire il documento nella \textit{repository}
\href{https://github.com/Project-SWEnergy/Project-SWEnergy.github.io}{\texttt{Project-SWEnergy.github.io}}
nella cartella \texttt{verbali\_esterni}.

\subsection{Riassunto della creazione e del rilascio di un documento}

\begin{enumerate}
	\item Creazione della cartella del documento nella \textit{repository}
	      \href{https://github.com/Project-SWEnergy/doc-latex}{\texttt{doc-latex}}.
	\item Creazione del documento con i \textit{template} descritti in
	      sezione \S\ref{documentazione_versionamento}.
	\item Creazione del registro delle modifiche.
	\item Inserimento delle variabili del documento nel file
	      \texttt{main.tex} che si trova nella cartella del documento.
	\item Creazione delle sezioni del documento.
	\item Verifica del documento con conseguenti correzioni e avanzamento di
	      versione.
	\item Approvazione del documento con conseguenti correzioni e avanzamento
	      di versione.
\end{enumerate}

La pubblicazione di un documento avviene in due modi:
\begin{itemize}
	\item \textbf{Verbali esterni}: il verificatore dovrà inserire il documento
	      nella \textit{repository}
	      \begin{sloppypar}
		      \href{https://github.com/Project-SWEnergy/Project-SWEnergy.github.io}{\texttt{Project-SWEnergy.github.io}}
		      nel percorso \texttt{verbali\_esterni}.
	      \end{sloppypar}
	\item \textbf{Altri documenti}: il documento verrà pubblicato in automatico
	      nella \textit{repository}
	      \begin{sloppypar}
		      \href{https://github.com/Project-SWEnergy/Project-SWEnergy.github.io}{\texttt{Project-SWEnergy.github.io}}
		      attraverso l'azione di \textit{GitHub Actions}.
	      \end{sloppypar}
\end{itemize}
