\section{Introduzione}

\subsection{Scopo del Documento}
Questo documento, redatto dal \textit{team} SWEnergy, definisce le norme e le
metodologie adottate per lo sviluppo del progetto "Easy Meal". L'obiettivo è
fornire una guida chiara e strutturata che faciliti la collaborazione all'interno
del \textit{team} e garantisca la coerenza e la qualità del lavoro svolto. Le
norme qui presentate si ispirano agli \textit{standard} ISO 12207-1995, adattati alle
specificità del progetto universitario in questione.

\subsection{Struttura del Documento}
Le sezioni \cref{sec:processi_primari}, \cref{sec:processi_supporto} e \cref{sec:processi_organizzativi} del documento riflettono i diversi aspetti e fasi del ciclo di vita del software, suddivisi in processi primari, di supporto e organizzativi, rispettivamente, come delineato dagli \textit{standard} ISO 12207-1995:

\begin{itemize}
	\item \textbf{Processi Primari:} Questa sezione descrive i processi
	      fondamentali nello sviluppo del software, includendo le fasi di
	      acquisizione, fornitura, sviluppo del prodotto software;
	\item \textbf{Processi di Supporto:} In questa parte vengono trattati i
	      processi che supportano lo sviluppo del software, come la gestione
	      della configurazione, la verifica, l'approvazione, la qualità e la
	      risoluzione dei problemi;
	\item \textbf{Processi Organizzativi:} Questa sezione copre i processi
	      trasversali che aiutano a migliorare e mantenere l'efficienza dell'
	      ambiente di sviluppo, inclusi la gestione dei processi, delle
	      infrastrutture, il miglioramento dei processi e la formazione del
	      personale.
\end{itemize}

Ciascun processo è descritto dalle seguenti sezioni:
\begin{enumerate}
	\item \textbf{Descrizione:} Fornisce una panoramica generale del processo,
	      fornendo informazioni aggiuntive rispetto al titolo del processo;
	\item \textbf{Scopo:} Specifica gli obiettivi e le finalità del processo;
	\item \textbf{Attività:} Elenca le attività principali che compongono il
	      processo;
	\item \textbf{Strumenti:} Specifica gli strumenti utilizzati per
	      l'attuazione del processo.
\end{enumerate}

Dopo le sezioni dedicate ai processi, il documento include una sezione per
ciascun ruolo. Oggi sotto-sezione rappresenta un'attività che il ruolo deve
svolgere.  La struttura delle attività è la seguente:
\begin{itemize}
	\item \textbf{Descrizione}: riguarda l'introduzione all'attività.
	      In aggiunta, sono contenute le informazioni necessarie per lo
	      svolgimento di qualche attività, per rendere la medesima più chiara e
	      comprensibile;

	\item \textbf{\textit{Trigger}}: spiega quando l'attività deve essere
	      svolta. Quindi sono elencate le condizioni che devono essere
	      verificate per attivare l'attività;

	\item \textbf{Scopo}: descrizione dello stato che si vuole raggiungere, in
	      seguito al completamento dell'attività;

	\item \textbf{Svolgimento}: contiene l'elenco dei \textit{task} che il
	      ruolo è tenuto a svolgere per completare l'attività. La dipendenza tra
	      i \textit{task} è specificata, se presente;

	\item \textbf{\textit{Task}}: per ogni \textit{task} viene fornita una
	      breve descrizione. Se necessario, viene fornita una serie di passi da
	      seguire per completare il \textit{task}. I passi da seguire sono
	      elencati in ordine e sono dipendenti tra loro.
\end{itemize}

\subsection{Glossario}
Al fine di evitare ambiguità linguistiche e garantire un’utilizzazione coerente delle terminologie nei documenti, il gruppo ha redatto un documento interno chiamato "Glossario". Questo
documento definisce in modo chiaro e preciso i termini che potrebbero generare ambiguità
o incomprensione nel testo. I termini presenti nel Glossario sono identificati da una ’G’ ad
apice (per esempio parola\g ).


\subsection{Riferimenti}

\subsubsection{Normativi}
\begin{itemize}
	\item \href{https://www.math.unipd.it/~tullio/IS-1/2009/Approfondimenti/ISO_12207-1995.pdf}{ISO/IEC 12207:1995} (ultimo accesso 10/01/2024).
\end{itemize}

\subsubsection{Informativi}
\begin{itemize}
	\item Glossario.
	\item \href{https://www.math.unipd.it/~tullio/IS-1/2023/Dispense/T4.pdf}{Gestione di progetto} (ultimo accesso 22/01/2024).
\end{itemize}
