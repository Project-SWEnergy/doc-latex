\section{Introduzione}
\subsection{Scopo del documento}

Questo documento ha solo scopo di contenere le procedure e le \textit{best
	practice (so far)} che SWEnergy ha deciso di adottare per sviluppare il
progetto "\textit{Easy Meal}".
Dunque, serve per aiutare gli elementi di SWEnergy ad affrontare il cambio
dei ruoli, in modo da rendere il passaggio più semplice e meno dispendioso in
termini di tempo.\\
Non si considera che questo documento sia esaustivo, ma che sia sufficiente per
aiutare i membri di SWEnergy a svolgere i compiti assegnati. Infatti, per
agevolare lo svolgimento dei compiti assegnati, oltre alla consultazione delle
"Norme di progetto", è necessario chiedere chiarimenti all'ultima persona che ha
svolto il ruolo ora assegnato al lettore.


\subsection{Struttura del documento}

Il documento è diviso in sezioni, una per ciascun ruolo. Oggi sotto-sezione
rappresenta un compito che il ruolo deve svolgere. La struttura dei compiti è la
seguente:
\begin{itemize}
	\item \textbf{Titolo};

	\item \textbf{Descrizione}: riguarda la descrizione del compito, viene usata
	      come introduzione al compito. In aggiunta, sono contenute le
	      informazioni necessarie per lo svolgimento di qualche attività, per
	      rendere le medesime più chiare e comprensibili;

	\item \textbf{\textit{Trigger}}: spiega quando il compito deve essere
	      svolto. Quindi sono elencate le condizioni che devono essere
	      verificate per attivare il compito;

	\item \textbf{Scopo}: descrizione dello stato che si vuole raggiungere, in
	      seguito al completamento del compito;

	\item \textbf{Svolgimento}: contiene l'elenco delle attività che il ruolo è
	      tenuto a svolgere per completare il compito. Le attività possono
	      essere tra loro dipendenti, oppure indipendenti. Si ritiene che il
	      linguaggio usato sia sufficiente per rendere le attività chiare e
	      comprensibili;

	\item \textbf{Attività}: per ogni attività viene fornita una breve
	      descrizione. Se necessario, viene fornita una serie di passi da
	      seguire per completare l'attività. I passi da seguire sono elencati in
	      ordine e sono dipendenti tra loro.
\end{itemize}

Nel caso in cui un componente di SWEnergy abbia qualche dubbio in merito a
qualche compito, è tenuto a chiedere chiarimenti.
