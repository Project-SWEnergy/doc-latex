\subsection{Organizzare un \textit{meeting} interno}
\label{organizzare-meeting-interno}

\subsubsection{Descrizione}
Il responsabile è tenuto ad organizzare i \textit{meeting} interni, ovvero le
\textit{stand-up}. Le \textit{stand-up} sono riunioni brevi, della durata di
circa 30 minuti, che si svolgono su \textit{Discord}. In esse sono trattati i
seguenti argomenti:
\begin{itemize}
	\item \textbf{\textit{Brainstorming}:} i membri del gruppo riassumono
	      brevemente il lavoro svolto nella settimana;

	\item \textbf{Problemi riscontrati:} i membri del gruppo espongono i
	      problemi riscontrati durante la settimana;

	\item \textbf{\textit{Todo list}:} sono discussi i compiti da svolgere nella
	      settimana successiva;

	\item \textbf{\textit{Dubbi}:} i membri del gruppo espongono i dubbi
	      riguardo alle attività da svolgere;

	\item \textbf{Restrospettiva:} i membri del gruppo espongono i problemi,
	      non inerenti alle attività, riscontrati durante la settimana e le
	      possibili soluzioni. I problemi possono, per esempio, riguardare
	      l'organizzazione del lavoro o la comunicazione tra i membri del
	      gruppo o con il proponente.
\end{itemize}

\subsubsection{Svolgimento}
Di seguito sono riportate le attività da completare per organizzare una
\textit{stand-up}:

\begin{itemize}

	\item \textbf{Pianificazione:} il responsabile deve decidere quando
	      svolgere la \textit{stand-up}. Di seguito i passi:
	      \begin{enumerate}
		      \item \textbf{Anticipare la data:} nella \textit{stand-up}
		            precedente il responsabile si informa sulle disponibilità
		            dei membri del gruppo per la prossima \textit{stand-up};

		      \item \textbf{Pianificare la data:} il responsabile propone delle
		            date e degli orari per la prossima \textit{stand-up} e le
		            propone sul gruppo \textit{Telegram} del gruppo. I membri
		            del gruppo esprimono la loro preferenza attraverlo un
		            sondaggio.
	      \end{enumerate}

	\item \textbf{Ordine del giorno:} il responsabile deve stilare un ordine del
	      giorno, ovvero una lista degli argomenti da trattare durante la
	      riunione. Di seguito i passi:
	      \begin{enumerate}
		      \item \textbf{Template:} il responsabile utilizza il template
		            delle \textit{stand-up} situato nella \textit{repository}
		            \texttt{appunti-swe};

		      \item \textbf{Brainstorming:} il responsabile si informa con i
		            membri del gruppo attraverso \textit{Telegram} in merito ai
		            punti che ciascun componente di SWEnergy intende trattare
		            durante la riunione;

		      \item \textbf{\textit{Todo list}:} il responsabile stila la lista
		            delle attività da svolgere nella settimana successiva. La
		            lista viene poi discussa e approvata durante la riunione.
	      \end{enumerate}

	\item \textbf{Verbale della riunione:} il responsabile deve redigere un
	      verbale della riunione, in cui vengono riportati gli argomenti
	      trattati e le decisioni prese. Di seguito i passi per redigere il
	      verbale interno:
	      \begin{enumerate}
		      \item \textbf{Appunti:} l'ordine del giorno (il punto precedente)
		            viene utilizzato come base per stilare il verbale interno;

		      \item \textbf{Template:} viene copiata la cartella della
		            riunione precedente e viene rinominata seguendo il formato:
		            \texttt{YYYY-MM-DD\_I};

		      \item \textbf{Stesura:} poichè si tratta di un documento, si
		            rimanda alla sottosezione che illustra come redigere un
		            documento (vedi \autoref{redazione-documento}).
	      \end{enumerate}
\end{itemize}
