\subsection{Miglioramento del Processo}

Il miglioramento del processo si basa sul Ciclo di Miglioramento Continuo PDCA.

\subsubsection{Scopo}
Lo scopo del processo consiste nell'ottimizzare i processi organizzativi e
incrementare l'efficacia e l'efficienza nel ciclo di vita del software.

\subsubsection{Attività}
\begin{enumerate}
	\item \textbf{\textit{Plan}}:
	      Definire gli obiettivi specifici di miglioramento, identificare le
	      attività necessarie per raggiungerli, stabilire le scadenze e
	      assegnare le responsabilità. Questo include la selezione di metriche
	      di processo per misurare l'efficacia delle azioni di miglioramento
	      (vedi \autoref{aggiornare-ndp}).

	\item \textbf{\textit{Do}}:
	      Implementare le attività pianificate, seguendo i piani stabiliti.
	      Questo può includere la formazione del personale, l'aggiornamento
	      delle procedure o l'introduzione di nuovi strumenti e tecnologie.

	\item \textbf{\textit{Check}}:
	      Monitorare e valutare l'esito delle azioni di miglioramento rispetto
	      agli obiettivi prefissati, utilizzando le metriche di processo
	      definite nella fase di pianificazione. Analizzare i dati raccolti per
	      identificare le tendenze, le deviazioni e le aree che necessitano di
	      ulteriori miglioramenti.
	\item \textbf{\textit{Act}}:
	      Sulla base dei risultati ottenuti nella fase di valutazione,
	      intraprendere azioni correttive per consolidare i miglioramenti
	      ottenuti e indirizzare le aree che non hanno raggiunto gli obiettivi
	      prefissati. Questa fase può anche includere la standardizzazione di
	      nuove pratiche di successo e la modifica dei piani di miglioramento
	      per i cicli futuri.

	\item \textbf{Ciclicità del Processo}:
	      Ripetere il ciclo PDCA per garantire un miglioramento continuo dei
	      processi, adattando gli obiettivi e le strategie in base ai risultati
	      ottenuti e alle nuove priorità identificate.
\end{enumerate}
