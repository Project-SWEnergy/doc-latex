\subsection{Sviluppo}

Questo processo comprende tutte le attività necessarie per trasformare i
requisiti in un \textit{software} funzionante e conforme alle aspettative degli
\textit{stakeholder}.

\subsubsection{Scopo}
Assicurare la creazione di un \textit{software} che risponda pienamente ai
bisogni degli utenti, sia tecnicamente valido, mantenibile e scalabile.

\subsubsection{Attività}
\begin{enumerate}
	\item \textbf{Analisi dei requisiti:} Comprensione e documentazione delle
	      necessità degli utenti e degli \textit{stakeholder}.
	\item \textbf{Progettazione del sistema:} Definizione dell'architettura del
	      sistema e dei principali componenti \textit{software} (vedi
	      \cref{progettazione}).
	\item \textbf{Implementazione:} Codifica effettiva del \textit{software} in
	      base alla progettazione (vedi \cref{codifica}).
	\item \textbf{Testing:} Verifica della correttezza del \textit{software}
	      attraverso test funzionali, di integrazione e di sistema.
\end{enumerate}

\subsubsection{Strumenti}
Per il processo di sviluppo sono utilizzati gli IDE \textit{VSCode} oppure
\textit{NeoVim}, il sistema di \textit{versioning} Git e l'organizzazione
\href{https://GitHub\g.com/Project-SWEnergy}{GitHub\g} per la gestione del codice
sorgente e altro materiale di progetto. Sono adottate le \textit{GitHub Actions}
per l'automazione di test e \textit{deployment}.
