\subsection{Organizzare un \textit{meeting} esterno}
\label{organizzare-meeting-esterno}

Il responsabile organizza i \textit{meeting} esterni, ovvero i SAL
tenuti tra il proponente e SWEnergy.
I \textit{SAL} sono riunioni brevi, della durata di circa 30 minuti, che hanno
luogo su \textit{Teams}. In esse
sono trattati i seguenti argomenti:
\begin{itemize}
	\item \textbf{Riassunto:} il responsabile riassume le attività svolte dal
	      gruppo durante lo sprint;

	\item \textbf{Problemi riscontrati:} il responsabile espone i problemi
	      riscontrati durante lo sprint;

	\item \textbf{\textit{To-do list}:} sono discussi i compiti da svolgere nella
	      settimana successiva tra il gruppo e il proponente;

	\item \textbf{\textit{Dubbi}:} il responsabile esponge i dubbi riguardo alle
	      attività da svolgere;

	\item \textbf{Restrospettiva:} il responsabile guida la discussione sulla
	      qualità del prodotto e soprattutto del processo.
\end{itemize}

\subsubsection{\textit{Trigger}}
\begin{itemize}
	\item La domanica precedente al SAL.
\end{itemize}

\subsubsection{Scopo}
\begin{itemize}
	\item Rendere la comunicazione tra i membri del gruppo più efficace ed
	      efficiente;

	\item Creare della documentazione usufruibile in caso di dubbi o
	      problematiche;

	\item Formalizzare le decisioni prese durante la riunione.
\end{itemize}

\subsubsection{Svolgimento}
\begin{itemize}

	\item \textbf{Pianificazione:} il responsabile deve decidere quando svolgere
	      un SAL. Di seguito i passi:
	      \begin{enumerate}
		      \item \textbf{Anticipare la data:} nel \textit{SAL}
		            precedente il responsabile e il proponente concordano
		            la data del prossimo \textit{SAL};

		      \item \textbf{Pianificare l'ora:} il responsabile contatta su
		            \textit{Telegram}\g il proponente, gli condivide l'ordine del
		            giorno e concorda l'ora del \textit{SAL}. Le due attività
		            sono svolte in concomitanza, perché si può prevedere la
		            durata del \textit{SAL} solo dopo aver stilato l'ordine del
		            giorno.
	      \end{enumerate}

	\item \textbf{Ordine del giorno:} il responsabile stila l'ordine del
	      giorno, ovvero una lista degli argomenti da trattare durante la
	      riunione. Di seguito i passi:
	      \begin{enumerate}
		      \item \textbf{Template:} il responsabile utilizza il template
		            dei SAL precedenti, situato nella \textit{repository\g}
		            \texttt{appunti-swe};

		      \item \textbf{Brainstorming:} il responsabile si informa con i
		            membri del gruppo attraverso le \textit{stand-up} in merito
		            allo \textit{status quo} del progetto;

		      \item \textbf{\textit{To-do list}:} il responsabile stila la lista
		            delle attività da svolgere nello sprint successivo. La
		            lista viene poi discussa e approvata durante la riunione.
	      \end{enumerate}

	\item \textbf{Verbale della riunione:} il responsabile deve redigere il
	      verbale della riunione, in cui vengono riportati gli argomenti
	      trattati e le decisioni prese. Di seguito i passi per redigere il
	      verbale interno:
	      \begin{enumerate}
		      \item \textbf{Appunti:} l'ordine del giorno (il punto precedente)
		            viene utilizzato come base per stilare il verbale esterno;

		      \item \textbf{Template:} viene copiata la cartella di template dei
		            verbali esterni e viene rinominata seguendo il formato:
		            \texttt{YYYY-MM-DD\_E};

		      \item \textbf{Stesura:} poichè si tratta di un documento, si
		            rimanda alla sottosezione che illustra come redigere un
		            documento (vedi \autoref{redazione-documento}).
	      \end{enumerate}
\end{itemize}

\subsubsection{Strumenti}
\begin{itemize}
	\item \textbf{Telegram\g:} per comunicare con il proponente e con i membri del
	      gruppo;

	\item \textbf{Teams:} per svolgere il \textit{SAL};

	\item \textbf{GitHub\g:} per condividere l'ordine del giorno e il verbale
	      esterno;

	\item \textbf{Mail:} per la condivisione di documenti e per la
	      comunicazione con il proponente.
\end{itemize}
