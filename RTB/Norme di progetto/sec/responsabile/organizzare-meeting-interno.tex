\subsection{Organizzare un \textit{meeting} interno}
\label{organizzare-meeting-interno}

Il responsabile coordina gli incontri interni, noti anche come \textit{stand-up}.
Queste riunioni, della durata approssimativa di 30 minuti, si tengono sulla piattaforma Discord e affrontano i seguenti punti chiave:
\begin{itemize}
	\item \textbf{\textit{Brainstorming}:} breve riassunto delle attività svolte durante la settimana;
	\item \textbf{Problemi riscontrati:} vengono presentati e discussi i problemi emersi nel corso della settimana;
	\item \textbf{\textit{To-do list}:} si discutono i compiti previsti per la settimana successiva;
	\item \textbf{Dubbi:} si chiariscono eventuali incertezze relative alle attività imminenti;
	\item \textbf{Restrospettiva:} i membri del gruppo condividono riflessioni sui successi e sulle difficoltà incontrate durante la settimana, esplorando insieme possibili soluzioni.
	      Questi problemi possono riguardare, ad esempio, l'organizzazione del lavoro o la comunicazione all'interno del \textit{team} o con il proponente.
\end{itemize}

\subsubsection{\textit{Trigger}}
\begin{itemize}
	\item Ogni venerdì, per dare tempo al responsabile di preparare il materiale
	      per la \textit{stand-up}.
\end{itemize}

\subsubsection{Scopo}
\begin{itemize}
	\item Rendere la comunicazione tra i membri del gruppo più efficace ed
	      efficiente;

	\item Creare della documentazione usufruibile in caso di dubbi o
	      problematiche;

	\item Formalizzare le decisioni prese durante la riunione.
\end{itemize}

\subsubsection{Svolgimento}
\begin{itemize}

	\item \textbf{Pianificazione:} il responsabile deve decidere quando
	      svolgere la \textit{stand-up}. Di seguito i passi:
	      \begin{enumerate}
		      \item \textbf{Anticipare la data:} nella \textit{stand-up}
		            precedente il responsabile si informa sulle disponibilità
		            dei membri del gruppo rispetto alla prossima
		            \textit{stand-up};

		      \item \textbf{Pianificare la data:} il responsabile propone delle
		            date e degli orari per la prossima \textit{stand-up} sul
		            gruppo Telegram\g del gruppo. I membri di SWEnergy esprimono
		            la loro preferenza attraverlo un sondaggio.
	      \end{enumerate}

	\item \textbf{Ordine del giorno:} il responsabile stila l'ordine del
	      giorno, ovvero una lista degli argomenti da trattare durante la
	      riunione. Di seguito i passi:
	      \begin{enumerate}
		      \item \textbf{\textit{Template}:} il responsabile utilizza il \textit{template}
		            delle \textit{stand-up} situato nella \textit{repository\g}
		            \texttt{appunti-swe/stand-up/template-stand-up.md};

		      \item \textbf{Brainstorming:} il responsabile si informa con i
		            membri del gruppo attraverso Telegram\g in merito ai
		            punti che bisogna trattare durante la riunione;

		      \item \textbf{\textit{To-do list}:} il responsabile stila la lista
		            delle attività da svolgere nella settimana successiva. La
		            lista viene poi discussa e approvata durante la riunione.
	      \end{enumerate}

	\item \textbf{Verbale della riunione:} il responsabile redige il
	      verbale della riunione, in cui vengono riportati gli argomenti
	      trattati e le decisioni prese. Di seguito i passi per redigere il
	      verbale interno:
	      \begin{enumerate}
		      \item \textbf{Appunti:} l'ordine del giorno (il punto precedente)
		            viene utilizzato come base per stilare il verbale interno;

		      \item \textbf{\textit{Template}:} viene copiata la cartella di \textit{template} dei
		            verbali interni e viene rinominata
		            seguendo il formato: \texttt{YYYY-MM-DD\_I};

		      \item \textbf{Stesura:} poichè si tratta di un documento, si
		            rimanda alla sottosezione che illustra come redigere un
		            documento (vedi \cref{redazione-documento}).
	      \end{enumerate}
\end{itemize}

\subsubsection{Strumenti}
\begin{itemize}
	\item \textbf{Discord\g:} per svolgere la riunione;
	\item \textbf{Telegram\g:} per comunicare con i membri del gruppo;
	\item \textbf{GitHub\g:} per la gestione del codice sorgente e altro
	      materiale di progetto.
\end{itemize}
