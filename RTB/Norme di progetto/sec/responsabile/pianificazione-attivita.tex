\subsection{Pianificazione delle attività}
\label{pianificazione-attivia}

Il responsabile pianifica le attività da svolgere durante lo sprint
e le suddivider tra i membri del gruppo. Inoltre, aggiorna il piano di progetto
in base alle attività svolte e a quelle da svolgere. La pianificazione avviene
tramite l'uso dei diagrammi di Gantt disponibili su \textit{GitHub\g}.

\subsubsection{\textit{Trigger}}
\begin{itemize}
	\item Comincia una nuova iterazione, che sia uno sprint\g od un
	      mini-sptrint\g.
\end{itemize}

\subsubsection{Scopo}
\begin{itemize}
	\item Aggiornare il piano di progetto in base alle attività svolte e a
	      quelle da svolgere.

	\item Pianificare le attività da svolgere durante l'iterazione corrente;

	\item Guidare lo svolgimento delle attività;

	\item Produrre la documentazione che permette di tenere traccia delle
	      attività svolte e da svolgere.
\end{itemize}

\subsubsection{Svolgimento}
\begin{itemize}
	\item \textbf{Creazione delle \textit{issue\g}}: il responsabile crea
	      delle \textit{issue}\g che descrivono le attività da svolgere e guidano i
	      lavoratori nella loro esecuzione. Di seguito sono riportati i passi
	      per definire le \textit{issue\g}:
	      \begin{enumerate}
		      \item \textbf{Identificazione}: il responsabile identifica le
		            attività da svolgere e le aggiunge su \textit{GitHub\g};

		      \item \textbf{Priorità}: il responsabile assegna una priorità
		            alle \textit{issue}\g in base all'urgenza e all'importanza;

		      \item \textbf{Scadenza}: il responsabile assegna una data di
		            scadenza alle \textit{issue}\g in base alla priorità e alla durata
		            dell'attività. L'attività viene quindi inserita nel
		            \textit{project} di \textit{GitHub\g} corrispondente alla
		            milestone di riferimento. In questo modo viene aggiornato il
		            diagramma di Gantt;

		      \item \textbf{Perfezionamento}: il responsabile guida la
		            discussione in merito alle \textit{issue\g} durante le
		            riunioni. In questo modo sono aggiornate priorità,
		            scadenza e descrizione;

		      \item \textbf{Assegnazione}: il responsabile assegna le \textit{issue}\g
		            ai membri del gruppo in base alle loro competenze e
		            disponibilità.
	      \end{enumerate}
\end{itemize}
