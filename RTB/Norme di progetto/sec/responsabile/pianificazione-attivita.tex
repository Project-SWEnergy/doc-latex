\subsection{Pianificazione delle attività}
\label{pianificazione-attivia}

Il responsabile pianifica le attività da svolgere durante lo \textit{sprint}
e le suddivide tra i membri del gruppo. Inoltre, aggiorna il "Piano di progetto"
in base alle attività svolte e a quelle da svolgere. La pianificazione avviene
tramite l'uso dei diagrammi di Gantt disponibili su GitHub\g.

\subsubsection{\textit{Trigger}}
\begin{itemize}
	\item Comincia una nuova iterazione, che sia uno \textit{sprint}\g o un
	      \textit{mini-sprint}\g.
\end{itemize}

\subsubsection{Scopo}
\begin{itemize}
	\item Pianificare le attività da svolgere durante l'iterazione corrente;

	\item Guidare lo svolgimento delle attività;

\end{itemize}

\subsubsection{Svolgimento}
\begin{itemize}
	\item \textbf{Creazione delle \textit{issue\g}}: il responsabile crea
	      delle \textit{issue}\g che descrivono le attività da svolgere fornendo informazioni utili alla loro esecuzione.
	      Di seguito sono riportati i passi per definire le \textit{issue\g}:
	      \begin{enumerate}
		      \item \textbf{Identificazione}: il responsabile identifica le
		            attività da svolgere e le aggiunge su GitHub\g;

		      \item \textbf{Scadenza}: il responsabile assegna una data di
		            scadenza alle \textit{issue}\g in base alla priorità e alla durata
		            dell'attività. L'attività viene quindi inserita nel
		            \textit{project} di GitHub\g corrispondente alla
		            \textit{milestone} di riferimento. In questo modo viene
		            aggiornato il diagramma di Gantt;

		      \item \textbf{Perfezionamento}: il responsabile guida la
		            discussione in merito alle \textit{issue\g} durante le
		            riunioni. In questo modo sono aggiornate scadenza e
		            descrizione;

		      \item \textbf{Assegnazione}: il responsabile assegna le \textit{issue}\g
		            ai membri del gruppo in base alle loro competenze e
		            disponibilità.
	      \end{enumerate}
\end{itemize}
