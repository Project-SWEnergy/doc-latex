\subsection{Revisione del codice}
\label{revisione-codice}

\subsubsection{Descrizione}

Il verificatore deve effettuare dei controlli di conformità sul codice
prodotto. Questo controllo deve essere effettuato in modo sistematico e
ripetitivo.

\subsubsection{\textit{Trigger}}
\begin{itemize}
	\item Viene prodotto un incremento sulla \textit{code base};

	\item Un componente di SWEnergy segnala la necessità di una verifica.
\end{itemize}

\subsubsection{Scopo}
\begin{itemize}
	\item Evidenziare gli errori nel codice e segnalarli al programmatore;

	\item Assicurarsi che il codice soddisfi le norme qui sotto descritte;

	\item Convalidare l'incremento di codice per garantirne l'integrità agli
	      altri componenti di SWEnergy
\end{itemize}

\subsubsection{Norme}
\begin{itemize}
	\item \textbf{Commenti:} per ciascuna funzione o metodo, deve essere
	      spiegato lo scopo. In particolare, deve essere sempre presenta la
	      spiegazione dei parametri in ingresso e del valore di ritorno;

	\item \textbf{Test:} per ciascuna funzione o metodo, deve essere presente
	      almeno un test che ne verifica il corretto funzionamento e fornisce un
	      esempio di utilizzo;

	\item \textbf{Nomi:} i nomi delle variabili devono essere significativi e
	      devono essere scritti in lingua italiana. Di seguito sono riportate
	      le regole di forma per ciascun tipo di variabile:
	      \begin{itemize}
		      \item \textbf{Variabili:} devono essere scritte in minuscolo e
		            devono essere separate da underscore (es.
		            \texttt{nome\_variabile});

		      \item \textbf{Costanti:} devono essere scritte in maiuscolo e
		            devono essere separate da underscore (es.
		            \texttt{NOME\_COSTANTE});

		      \item \textbf{Interfacce:} la prima lettera di ogni parola è
		            maiuscola e le parole sono unite senza spazi (es.
		            \texttt{NomeInterfaccia});

		      \item \textbf{Classi:} la prima lettera di ogni parola è
		            maiuscola e le parole sono unite senza spazi (es.
		            \texttt{NomeClasse});

		      \item \textbf{Metodi:} devono essere scritte in minuscolo e
		            devono essere separate da underscore (es.
		            \texttt{nome\_metodo});

		      \item \textbf{Funzioni:} devono essere scritte in minuscolo e
		            devono essere separate da underscore (es.
		            \texttt{nome\_funzione});

		      \item \textbf{\textit{File}:} devono essere scritte in minuscolo e
		            devono essere separate da underscore (es.
		            \texttt{nome\_file}). In ogni \textit{file} ci può essere al
		            più una classe o un'interfaccia che ha lo stesso nome del
		            \textit{file}.
	      \end{itemize}
\end{itemize}

\subsubsection{Svolgimento}
Di seguito sono riportate le attività da completare per effettuare i controlli
di conformità sul codice prodotto:
\begin{itemize}
	\item \textbf{Correzione del codice:} il verificatore deve controllare che per
	      ciascuna funzione o metodo sia presente una descrizione dello scopo,
	      dei parametri in ingresso e del valore di ritorno. Di seguito sono
	      riportati i passi da seguire:
	      \begin{enumerate}
		      \item \textbf{Commenti:} il verificatore legge i commenti della
		            funzione e ne intuisce lo scopo;

		      \item \textbf{Funzionamento:} il verificatore legge il corpo
		            della funzione o del metodo e ne verifica il funzionamento
		            staticamente;

		      \item \textbf{Test:} il verificatore verifica che sia
		            presente almeno un test per la funzione o il metodo;

		      \item \textbf{\textit{Edge cases}:} il verificatore verifica
		            che i test siano completi e che coprano tutti i casi
		            particolari;

		      \item \textbf{Nomi:} il verificatore controlla che i nomi definiti
		            dal programmatore rispettino le regole di forma definite
		            precedentemente;

		      \item \textbf{Correzioni:} il verificatore riporta
		            gli errori riscontrati al programmatore;

		      \item \textbf{Aggiornamento della versione:} dopo che il codice
		            viene corretto dal programmatore, il verificare deve aggiornare
		            la versione del codice;

		      \item \textbf{Versione:} sia $X.Y.Z$ la versione del codice,
		            dopo la verifica, il valore di $Z$ viene incrementato di
		            $1$, se le modifiche apportate al codice non aggiungono
		            nuove funzionalità. Se invece le modifiche apportate al
		            codice aggiungono nuove funzionalità, il valore di $Y$ viene
		            incrementato di $1$ e il valore di $Z$ viene reimpostato a
		            $0$. Una funzionalità coincide con un requisito.
	      \end{enumerate}
\end{itemize}
