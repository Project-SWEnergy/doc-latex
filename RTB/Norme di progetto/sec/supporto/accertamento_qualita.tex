\subsection{Accertamento della Qualità}

L'accertamento della qualità è un processo di supporto fondamentale che
garantisce che il \textit{software} soddisfi i requisiti di qualità stabiliti e
le aspettative degli \textit{stakeholder}.
Questo processo include la definizione, implementazione, valutazione e
manutenzione delle procedure e delle politiche di qualità per assicurare che
il \textit{software} prodotto sia di alta qualità.

\subsubsection{Scopo}
Assicurare che il \textit{software} e le pratiche di sviluppo rispettino gli
\textit{standard} di qualità prefissati, migliorando così la soddisfazione del cliente e
l'affidabilità del prodotto.

\subsubsection{Attività}
\begin{enumerate}
	\item \textbf{Definizione delle Politiche di Qualità:} Stabilire gli
	      \textit{standard} e le metriche di qualità in base ai requisiti del progetto e
	      alle aspettative degli \textit{stakeholder}.
	\item \textbf{Implementazione delle Procedure di Qualità:} Applicare le
	      politiche attraverso metodi concreti e pratiche di sviluppo, come
	      revisioni del codice e test.
	\item \textbf{Valutazione della Conformità:} Verificare periodicamente che
	      il \textit{software} e i processi di sviluppo rispettino le politiche
	      di qualità stabilite.
	\item \textbf{Manutenzione e Miglioramento Continuo:} Aggiornare le
	      politiche e le procedure di qualità in base ai \textit{feedback\g} e ai
	      risultati delle valutazioni per promuovere il miglioramento continuo
	      (vedi \cref{aggiornare-pdq}).
\end{enumerate}

\subsubsection{Strumenti}
Gli strumenti utilizzati nel processo di accertamento della qualità possono
includere \textit{software} autoprodotti di automazione dei test e di raccolta
dei dati.
