\subsection{Documentazione}

La documentazione è un processo di supporto essenziale che fornisce un insieme
di informazioni e dati strutturati necessari per comprendere, utilizzare, e
manutenere il \textit{software}.\\
La documentazione comprende tutti i materiali scritti o elettronici che
descrivono le caratteristiche, le operazioni o l'uso del \textit{software},
come i manuali utente, le specifiche tecniche, i rapporti di test, e i piani di
progetto.

\subsubsection{Scopo}
Fornire una chiara comprensione del \textit{software}, facilitare la
comunicazione tra i membri del \textit{team}, consentire un uso efficace del
\textit{software} da parte degli utenti e supportare le future attività di
manutenzione e sviluppo.

\subsubsection{Attività}
\begin{enumerate}
	\item \textbf{Pianificazione della documentazione:} Definire gli obiettivi,
	      il pubblico e la portata della documentazione.
	\item \textbf{Redazione dei documenti:} Creare i documenti necessari
	      seguendo le linee guida e gli \textit{standard} stabiliti
	      (vedi \cref{redazione-documento}).
	\item \textbf{Revisione e aggiornamento:} Valutare e aggiornare i documenti
	      per garantirne la precisione e la rilevanza nel tempo (vedi
	      \cref{verifica-documento}).
	\item \textbf{Gestione della documentazione:} Organizzare, archiviare e
	      rendere facilmente accessibili i documenti a tutti gli
	      \textit{stakeholder} interessati.
\end{enumerate}

\subsubsection{Strumenti}
Gli strumenti utilizzati per la creazione dei documenti sono:
\begin{itemize}
	\item \textbf{LaTeX}: linguaggio di \textit{markup} per la creazione di
	      documenti \\
	      \href{https://www.latex-project.org/}{(www.latex-project.org)};
	\item \textbf{VisualStudio Code}: GUI con integrazioni per la creazione di
	      documenti scritti in LaTeX e per la gestione delle \textit{repository\g} git\g \\
	      \href{https://code.visualstudio.com/}{(code.visualstudio.com)}
	      \begin{itemize}
		      \item \textbf{LaTeX Workshop}: estensione utilizzata in
		            VisualStudio Code per la compilazione e la scrittura dei
		            documenti.
	      \end{itemize}
\end{itemize}
