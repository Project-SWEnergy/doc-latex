\subsection{Lavoro sul progetto}
\label{lavoro-sul-progetto}

\subsubsection{Descrizione}
Questa attività descrive i passi da seguire per svolgere qualunque compito
assegnato dal responsabile di progetto. Un compito è l'istanza di un'attività
che deve essere svolta da un membro del gruppo.

% si potrebbe richiedere di inserire un'ora da qualche parte, tipo quanto ci si
% mette per completare l'attività

\subsubsection{\textit{Trigger}}
\begin{itemize}
	\item Il responsabile di progetto assegna un compito ad un membro del
	      gruppo.
\end{itemize}

\subsubsection{Scopo}
\begin{itemize}
	\item Svolgere il compito assegnato;
	\item Risulta conclusa un'\textit{issue} nella \textit{repository}
	      corrispondente;
	\item Il compito è stato verificato e convalidato da una persona diversa
	      da chi lo ha svolto.
\end{itemize}

\subsubsection{Svolgimento}
In questo caso viene descritto il flusso di lavoro da seguire per completare
un compito assegnato:

\begin{enumerate}
	\item \textbf{Analisi}: si crea una \textit{issue} nella \textit{repository}
	      nella quale verrà svolto il compito. La \textit{issue} sarà assegnata
	      a se stessi. All'interno della issue si descrive l'insieme degli
	      obiettivi da raggiungere il compito risulti completato.
	      Il compito deve essere collegata al \textit{project} corrispondente
	      alla fase di sviluppo in cui si trova il progetto. In aggiunta, deve
	      essere marcata con un \textit{tag} che ne identifichi la tipologia;

	\item \textbf{Creazione appunti}: si inseriscono i file degli
	      appunti nel proprio \textit{branch} personale all'interno
	      della \textit{repository} \texttt{appunti-swe}.
	      Nella \textit{repository}, deve essere presente un \texttt{README.md}
	      contenente l'organizzazione della cartella per permettere agli altri
	      membri di orientarsi;

	\item \textbf{Svolgimento}: si svolge il compito assegnato, al meglio
	      delle proprie capacità e cercando di rispettare le scadenze;

	\item \textbf{Integrazione appunti}: si modificano i file degli
	      appunti precedentemente generati.

	\item \textbf{Verifica}: si chiede ad un membro del gruppo,
	      tendenzialmente al verificatore, di controllare la conformità del
	      lavoro svolto.
\end{enumerate}
