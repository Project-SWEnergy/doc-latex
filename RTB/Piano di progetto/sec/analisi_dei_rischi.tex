\section{Analisi dei rischi}

La seguente sezione ha lo scopo di identificare e catalogare i rischi che
potrebbero verificarsi durante lo svolgimento del progetto, per poterli
prevenire o almeno provare a mitigarli.
Ciascun rischio è descritto seguendo la struttura:
\begin{itemize}
	\item \textbf{Codice identificativo} seguito da un numero progressivo:
	      \begin{itemize}
		      \item \textbf{RT}: rischi legati alle tecnologie;
		      \item \textbf{RC}: rischi legati alla comunicazione;
		      \item \textbf{RP}: rischi legati alla pianificazione.
	      \end{itemize}

	\item \textbf{Titolo}: il nome che identifica il rischio;

	\item \textbf{Descrizione}: una breve descrizione del rischio;

	\item \textbf{Identificazione}: in quale modo il \textit{team} può capire se
	      si sta verificando qualche danno;

	\item \textbf{Mitigazione}: come il \textit{team} ha modo di prevenire o
	      attenuare i danni causati dal rischio;
\end{itemize}

Dopo la descrizione di ciascun rischio, viene presentata una tabella che
riassume i rischi individuati, associando a ciascuno un indice di gravità e un
indice di frequenza.

\subsection{Rischi legati alle tecnologie}
\risktech{Conoscenza delle tecnologie carente}
\label{risk:conoscenza tecnologie carente}
\begin{itemize}
	\item \textbf{Descrizione}:
		Durante lo sviluppo del progetto, potrebbe verificarsi la situazione 
		in cui almeno un membro del \textit{team} non possiede una conoscenza 
		sufficiente di una tecnologia adottata dal gruppo e necessaria per 
		lo sviluppo del progetto.

	\item \textbf{Identificazione}: 
		Il \textit{team} ha identificato le tecnologie conosciute dal gruppo 
		attraverso discussioni e accordi con il proponente. 
		Questo processo ha permesso di individuare le tecnologie non conosciute dal gruppo.

	\item \textbf{Mitigazione}:
	      \begin{itemize}
		      \item \textbf{\textit{Workshop} interni}: si rimanda alla
		            sotto-sezione "Organizzare un \textit{workshop}" del
		            documento "Norme di progetto" sotto il ruolo di progettista;

		      \item \textbf{Seminari con il proponente}: il \textit{team}
		            partecipa a seminari organizzati con il proponente, per approfondire
		            le tecnologie non conosciute. 
					Il proponente spiegherà le tecnologie e fornirà esempi di codice
		            per illustrarne l'utilizzo;

		      \item \textbf{Dialogo con il proponente}: il \textit{team} può
		            contattare il proponente per chiarimenti sulle
		            tecnologie non conosciute;

		      \item \textbf{\textit{Code review}}: si rimanda alla sotto-sezione
		            "Verifica del codice" del documento "Norme di progetto"
		            sotto il ruolo di verificatore;

		      \item \textbf{Divisione del \textit{front-end} e del \textit{back-end}}: 
			  		il \textit{team} si suddivide in due sottogruppi, uno responsabile del 
					\textit{front-end} e l'altro del \textit{back-end}. 
					Questa divisione riduce l'\textit{overhead} di comunicazione e di cambio di
		            contesto. I due gruppi si scambiano i ruoli al termine della prima 
					fase del progetto: RTB.
	      \end{itemize}
	
	\item \textbf{Riscontro}: Nessuna conseguenza significativa è stata riscontrata in quanto le misure di mitigazione necessarie sono state tempestivamente implementate
	ed i componenti del gruppo si sono riuniti al più presto per poter risolvere la problematica. Particolarmente utili si sono rilevati i \textit{workshop} interni,
	un dialogo attivo con il proponente e la divisione del \textit{team} in due sottogruppi per il \textit{front-end} e il \textit{back-end}. Mentre per quanto riguarda la mitigazione 
	prevista "Seminari con il proponente" nononostante sia stata attuata non si è rivelata del tutto efficace, per questo motivo il gruppo non ha ritenuto necessario farne uso ulteriormente.
	Una criticità riscontrata per quanto riguarda la divisione del \textit{team} in due sottogruppi sono state le difficoltà di comunicazione tra i due gruppi e formazione dei componenti, 
	che ha portato ad un rallentamento nello sviluppo del progetto.
\end{itemize}

\risktech{Strumenti \textit{software} inadeguati}
\label{risk:strumenti software inadeguati}
\begin{itemize}
	\item \textbf{Descrizione}: l'utilizzo di strumenti \textit{software} datati o poco
	      efficienti potrebbe causare ritardi nello sviluppo del progetto;

	\item \textbf{Identificazione}:
	      \begin{itemize}
		      \item Durante le riunioni interne, è cruciale prestare attenzione 
			  		ai \textit{feedback}\g dei membri del gruppo che potrebbero esprimere 
					preoccupazioni sull'efficienza o l'adeguatezza degli strumenti \textit{software} utilizzati;

		      \item I membri del gruppo potrebbero segnalare procedure troppo lunghe o 
			  		che possono essere facilmente automatizzate;

		      \item I membri del gruppo potrebbero segnalare procedure troppo lunghe o 
			  		che possono essere facilmente automatizzate. 
					Questo tipo di \textit{feedback} può indicare che gli strumenti attuali 
					potrebbero non essere ottimali per il processo di sviluppo.
	      \end{itemize}

	\item \textbf{Mitigazione}:
	      \begin{itemize}

			\item \textbf{Controllo delle Versioni da Parte dell'Amministratore}: 
				L'amministratore del progetto deve monitorare attentamente le versioni 
				degli strumenti \textit{software} utilizzati per assicurare che siano aggiornate e efficienti.
				
			\item \textbf{Informazione da parte dei membri del gruppo}: 
				I membri del gruppo devono essere proattivi nell'informarsi su nuove tecnologie e 
				strumenti \textit{software} che potrebbero migliorare l'efficienza del processo di sviluppo.

			\item \textbf{Automazione}: 
				I membri del gruppo analizzano e controllano se le procedure utilizzate siano 
				automatizzabili per migliorare l'efficienza.
	      \end{itemize}
\end{itemize}

\risktech{Codice incomprensibile}
\label{risk:codice incomprensibile}
\begin{itemize}
	\item \textbf{Descrizione}: il codice prodotto da qualche membro del gruppo
	      è di difficile comprensione per gli altri membri del gruppo.
	\item \textbf{Identificazione}:
	      \begin{itemize}
		      \item \textbf{\textit{Code review}}: durante la verifica del
		            codice, i verificatori possono riscontrare difficoltà nella
		            comprensione del codice;

	      \end{itemize}
	\item \textbf{Mitigazione}:
	      \begin{itemize}
		      \item \textbf{"Norme di progetto"}: il gruppo stila delle linee
		            guida da seguire per la stesura del codice, in modo da
		            uniformare la stesura del codice e facilitarne la
		            comprensione. Le norme sono disponibili nel documento
		            "Norme di progetto" alla sotto-sezione "Codifica" sotto il
		            ruolo di programmatore;

		      \item \textbf{\textit{Testing}}: il codice deve essere testato
		            in modo approfondito, per facilitarne la comprensione e
		            illustrarne i casi d'uso. Si rimanda alla sotto-sezione
		            "Revisione del codice" del documento "Norme di progetto"
		            sotto il ruolo di verificatore;
	      \end{itemize}
\end{itemize}


\subsection{Rischi legati alla comunicazione}
\riskcom{Comunicazione interna carente}
\label{risk:comunicazione interna carente}
\begin{itemize}
	\item \textbf{Descrizione}:
	      La comunicazione interna non è efficace ed efficiente, causando riunioni
	      interne più lunghe del previsto e rallentando le attività.
	\item \textbf{Identificazione}:
	      \begin{itemize}
		      \item \textbf{Dubbi ripetuti}: durante le riunioni interne, i
		            membri del gruppo possono porre domande già presentate in
		            precedenza;

		      \item \textbf{Riunioni interne lunghe}: le riunioni interne
		            possono protrarsi oltre il tempo previsto;

		      \item \textbf{Fraintendimenti frequenti}: i membri del gruppo
		            possono fraintendersi frequentemente.

	      \end{itemize}
	\item \textbf{Mitigazione}:
	      \begin{itemize}
		      \item \textbf{Documentazione}: il gruppo stila una documentazione
		            adeguata per facilitare la comunicazione interna. A seconda
		            dell'argomento la documentazione può avere diverse forme;

		      \item \textbf{\textit{Meeting} frequenti}: il gruppo si impegna a
		            tenere riunioni interne frequenti, in modo da ridurre la
		            durata delle riunioni interne e facilitare la comunicazione
		            interna;

		      \item \textbf{Ordine del giorno}: ciascuna riunione deve avere
		            l'ordine del
		            giorno ben definito, per discutere di tutti gli argomenti
		            utili allo sviluppo del progetto e per definire la durata di
		            ciascuno dei punti dell'ordine del giorno;

		      \item \textbf{Retrospettiva}: durante la retrospettiva, il gruppo
		            deve pensare a soluzioni \textit{ad hoc} per migliorare la
		            comunicazione interna.
	      \end{itemize}
\end{itemize}

\riskcom{Conflitti decisionali}
\label{risk:conflitti decisionali}
\begin{itemize}
	\item \textbf{Descrizione}:
	      Il gruppo potrebbe dilungarsi nella discussione di una sola idea, senza
	      raggiungere una decisione finale.
	\item \textbf{Identificazione}:
	      \begin{itemize}
		      \item un punto dell'ordine del giorno subisce un ritardo grave;
	      \end{itemize}
	\item \textbf{Mitigazione}:
	      \begin{itemize}
		      \item dibattito: i membri del gruppo discutono riguardo
		            all'importanza del punto dell'ordine del giorno, per capire se
		            è necessario approfondire la discussione o meno;

		      \item approfondimento: se il punto dell'ordine del giorno è
		            ritenuto importante, almeno due membri del gruppo si impegnano
		            a studiare i pro ed i contro delle varie soluzioni possibili.
		            Può essere chiesto un supporto al proponente oppure al
		            committente per chiarire i dubbi;

		      \item votazione: alla fine del dibattito i membri del gruppo
		            votano per la soluzione che ritengono più opportuna. La
		            votazione si ritiene conclusa quando la maggioranza dei
		            membri del gruppo ha espresso la propria preferenza e il
		            risultato non è un pareggio.

		      \item il responsabile ha il compito di vigilare sul corretto
		            svolgimento del dibattito e della votazione, in modo da
		            evitare che si dilunghi troppo.
	      \end{itemize}
\end{itemize}

\riskcom{Comunicazione esterna carente}
\label{risk:comunicazione esterna carente}
\begin{itemize}
	\item \textbf{Descrizione}:
	      Le comunicazioni con il proponente o con il committente non sono
	      efficaci ed efficienti, causando riunioni esterne più lunghe del
	      previsto e rallentando le attività; oppure rallentando le attività
	      del gruppo a causa di risposte tardive o mancanti.

	\item \textbf{Identificazione}:
	      \begin{itemize}
		      \item \textbf{Dubbi ripetuti}: durante le riunioni esterne, i
		            membri del gruppo possono porre domande già presentate in
		            precedenza;

		      \item \textbf{Riunioni esterne lunghe}: le riunioni esterne
		            possono protrarsi oltre il tempo previsto;

		      \item \textbf{Risposte tardive o mancanti}: il proponente o il
		            committente può rispondere in ritardo o non rispondere
		            affatto alle comunicazioni del gruppo.
	      \end{itemize}

	\item \textbf{Mitigazione}:
	      \begin{itemize}
		      \item \textbf{Ordine del giorno}: il responsabile si impegna a
		            stilare l'ordine del giorno delle riunioni esterne, per
		            tempo, ne discute la struttura con il gruppo e lo condivide
		            con il proponente e con il committente in anticipo;

		      \item \textbf{SAL}: il gruppo si impegna a mantenere il
		            proponente aggiornato sullo stato di avanzamento del
		            progetto, in modo da ridurre la durata delle riunioni
		            esterne e migliorare la qualità del supporto del proponente;

		      \item \textbf{Retrospettive}: sono previste delle retrospettive
		            all'interno dei SAL con il proponente, durante le quali, si
		            discute la qualità delle comunicazioni e si pensa a
		            soluzioni \textit{ad hoc} per migliorare la comunicazione
		            esterna;

		      \item \textbf{Comunicazioni frequenti}: il proponente viene tenuto
		            aggiornato frequentemente sullo stato di avanzamento del
		            progetto mediante gli appositi canali di comunicazione:
		            \textit{Telegram} e \textit{email};

		      \item \textbf{Diario di bordo}: il gruppo si impegna a tenere
		            dei diari di
		            bordo, quando richiesti dal committente, per aggiornarlo
		            sullo stato di avanzamento del progetto;

		      \item \textbf{\textit{Meeting} supplementari}: se il gruppo
		            manifesta dei dubbi o delle incertezze, può richiedere dei
		            \textit{meeting} supplementare con il proponente o con il
		            committente;

		      \item \textbf{Documentazione}: il responsabile si impegna ad
		            aggiornare la documentazione inerente agli argomenti
		            trattati durante le riunioni esterne, per dare modo ai
		            membri del gruppo di consultarla in caso di dubbi o
		            incertezze.
	      \end{itemize}
\end{itemize}


\subsection{Rischi legati alla pianificazione}
I membri del gruppo non hanno mai assunto un ruolo manageriale in
precedenza e non hanno mai lavorato in un gruppo di lavoro così
numeroso. Questo porta a problemi di gestione del tempo e delle
risorse. D'altro canto, SWEnergy si rende conto che lo scopo del
progetto è proprio quello di acquisire esperienza, anche in questi
termini. Per cui, il gruppo ha deciso di individuare alcuni
rischi legati alla pianificazione, per poterli prevenire o mitigare.

\riskplan{Organizzazione carente}
\label{risk:organizzazione carente}
\begin{itemize}
	\item \textbf{Descrizione}:
	      Il gruppo, oppure qualche membro, potrebbe non essere in grado di
	      svolgere le proprie attività, oppure potrebbe riscontrare delle
	      difficoltà a causa di una cattiva organizzazione.
	\item \textbf{Identificazione}:
	      \begin{itemize}
		      \item membri confusi: i membri del gruppo non sanno quali sono i
		            compiti a loro assegnati, oppure non sanno come svolgerli;

		      \item carenza di risorse: sono stati assegnati più incarichi di
		            quelli sostenibili con le risorse disponibili.

		      \item scadenze non aggiornate: il gruppo o qualche suo membro non
		            è in grado di rispettare le scadenze e non sono aggiornate.
		            Si tratta di un modo molto semplice, per ricadere nel
		            sintomo individuato precedentemente.
	      \end{itemize}
	\item \textbf{Mitigazione}:
	      \begin{itemize}
		      \item il responsabile mantiene aggiornate le \textit{issue}.
		            Ciascuna \textit{issue} deve essere ben documentata nella
		            propria descrizione; se è il caso, si possono aggiungere i
		            riferimenti a della documentazione supplementare. Maggiori
		            informazioni sono presenti nel documento "\textit{Way of
			            working}";

		      \item ciasun componente di SWEnergy deve aggiornare le
		            \textit{issue} a cui è assegnato, in modo da tenere il
		            responsabile e l'intera organizzazione aggiornati sullo
		            stato di avanzamento dei compiti; inoltre, deve aggiungere
		            delle \textit{issue} se ritiene che ci siano delle attività
		            da svolgere. Maggiori informazioni sono presenti nel
		            documento "\textit{Way of working}";

		      \item in caso di dubbi, i membri di SWEnergy possono rivolgersi
		            al responsabile, che si occuperà di chiarire la situazione,
		            o di indirizzare il membro verso chi può aiutarlo;

		      \item il responsabile mantiene aggiornato il \textit{project} su
		            \textit{GitHub}, in particolare per quanto riguarda
		            l'assegnamento delle \textit{issue} e delle scadenze.
		            Maggiori informazioni sono presenti nel documento
		            "\textit{Way of working}";

		      \item durante le retrospettive, il gruppo discute di eventuali
		            problemi organizzativi e cerca di trovare soluzioni per
		            migliorare la pianificazione;

		      \item dialogo con il proponente: sono chiesti consigli al
		            proponente in merito, in quanto ha più esperienza
		            nel settore e ha modo di collaborare con figure manageriali.
	      \end{itemize}
\end{itemize}

\riskplan{Comprensione dei requisiti carente}
\label{risk:comprensione dei requisiti carente}
\begin{itemize}
	\item \textbf{Descrizione}:
	      Il gruppo o qualche suo membro potrebbe non essere in grado di
	      comprendere i requisiti del progetto, oppure potrebbe riscontrare
	      delle difficoltà a causa di una cattiva comprensione dei requisiti.
	\item \textbf{Identificazione}:
	      \begin{itemize}
		      \item \textbf{Dubbi}: i membri del gruppo hanno dei dubbi in merito ai
		            requisiti;

		      \item \textbf{Dibattiti sui requisiti}: i membri del gruppo
		            discutono tra loro in merito ai requisiti;

		      \item \textbf{Discrepanza nella progettazione}: i membri del
		            gruppo progettano in modo diverso, a causa di una cattiva
		            comprensione dei requisiti.
	      \end{itemize}

	\item \textbf{Mitigazione}:
	      \begin{itemize}
		      \item \textbf{Dibattito interno}: SWEnergy si è diviso in coppie
		            per approfondire i casi d'uso e i requisiti del progetto.
		            Successivamente, si è tenuta una riunione interna in cui ciascuna coppia
		            ha esposto  le proprie considerazioni e i propri dubbi. In
		            questo modo, si è cercato di chiarire i dubbi e di
		            uniformare la comprensione dei requisiti;

		      \item \textbf{"Analisi dei requisiti"}: il metodo più formale per
		            ovviare a questa situazione risulta essere
		            l'"Analisi dei requisiti".
		            I requisiti devono essere chiari e completi. Inoltre, il documento 
					include i casi d’uso, che facilitano una migliore comprensione 
					dei requisiti concordati con il proponente;

		      \item \textbf{Dialogo con il proponente}: si instaura un dialogo attivo 
			  con il proponente per discutere dei requisiti, chiarire eventuali dubbi 
			  e definire in maggior dettaglio le funzionalità del prodotto;

		      \item \textbf{Messaggi tempestivi con il proponente}: in caso di dubbi
		            semplici e veloci da risolvere, si inviano dei messaggi al
		            proponente per ottenere una risposta tempestiva, riducendo così 
					eventuali incertezze e ritardi nella comprensione dei requisiti.
	      \end{itemize}
\end{itemize}

\riskplan{Interfacce incoerenti}
\label{risk:interfacce incoerenti}
\begin{itemize}
	\item \textbf{Descrizione}:
	      Durante la fase integrativa di più componenti, risultano delle
	      incongruenze che rendono impossibile l'integrazione.
	\item \textbf{Identificazione}:
	      \begin{itemize}
		      \item \textbf{Test di integrazione falliti}: i test di
		            integrazione falliscono a causa di incongruenze tra le
		            interfacce delle componenti;

		      \item \textbf{Discussioni interne in merito alle interfacce}: i
		            membri del gruppo discutono tra loro in merito alle
		            interfacce delle componenti, per capire come risolvere le
		            incongruenze;

		      \item \textbf{Fallimento del sistema}: l'applicativo non funziona
		            in seguito ad un'integrazione.

	      \end{itemize}
	\item \textbf{Mitigazione}:
	      \begin{itemize}
		      \item \textbf{Dialogo interno}: i membri del gruppo discutono tra loro
		            in merito alle interfacce delle componenti prima di iniziare lo 
					sviluppo delle stesse. Questo anticipato confronto aiuta a identificare 
					e risolvere potenziali incongruenze;

		      \item \textbf{Test di integrazione}: si effettuano test di integrazione 
			  		approfonditi per verificare la compatibilità tra le componenti, facilitando 
			  		l'identificazione tempestiva di eventuali problemi e garantendo 
			  		un'integrazione più efficiente;

		      \item \textbf{Documentazione}: le interfacce delle componenti sono
		            documentate in modo chiaro e completo, per evitare
		            incomprensioni ed esplicitarne la struttura e la
		            compatibilità.
	      \end{itemize}
\end{itemize}

\riskplan{Costi e tempi imprevisti}
\label{risk:costi e tempi imprevisti}
\begin{itemize}
	\item \textbf{Descrizione}:
	      Durante lo sviluppo del progetto, si può incorrere in costi o
	      rallentamenti imprevisti. Si tratta, a tutti gli effetti, di arginare
	      il danno prodotto da un rischio che si è verificato.
	\item \textbf{Identificazione}:
	      \begin{itemize}
		      \item \textbf{Monitoraggio costante}: si effettua un monitoraggio 
			  		continuo dei costi e delle tempistiche al raggiungimento delle 
					\textit{milestone} e durante gli \textit{stand-up}.
			  
	      \end{itemize}
	\item \textbf{Mitigazione}:
	      \begin{itemize}
		      \item \textbf{\textit{Buffer} di tempo}: Il team ha proattivamente 
			  		inserito margini temporali tra le diverse attività, creando 
					\textit{buffer} di tempo che consentono di gestire eventuali 
					ritardi senza compromettere la pianificazione principale;

		      \item \textbf{\textit{Buffer} di costi}: il \textit{team} ha 
			  		preventivamente allocato risorse finanziarie extra, 
					sotto forma di \textit{buffer} di costi, per far fronte 
					a spese impreviste e mantenere il controllo del budget;

		      \item \textbf{Pianificazione in itinere}: il \textit{team} si adatta
		            alle variazioni dei costi e delle tempistiche di
		            completamento, per poter gestire eventuali costi
		            imprevisti. In questo caso, sono aggiornate le scadenze
		            nel \textit{project} su \textit{GitHub} e i costi.
		            A seconda della situazione, le \textit{issue} sono
		            riassegnate e le \textit{milestone} sono adattate allo
		            \textit{status quo}.
	      \end{itemize}
\end{itemize}


\subsection{Pericolosità e occorrenze}

Per ogni rischio, il \textit{team} ha individuato un indice di gravità e un
indice di frequenza, per poter stimare il rischio residuo. L'indice di
gravità e quello di frequenza sono due numeri compresi tra 1 e 5. L'indice di
rischio residuo è il prodotto tra i due indici, può quindi assumere valori
compresi tra 1 e 25. Più il rischio residuo è alto, maggiori sono i danni che
può causare e più è probabile che si verifichi. Si noti che non è detto che il
verificarsi del rischio causi i danni massimi, le strategie di
mitigazione servono proprio per prevenire e attenuare i danni.

\begin{table}[H]
	\centering

	\begin{tabular}{l|r|r|r}
		\hline
		\textbf{Rischi tecnologici}                                                          & \textbf{Gravità} & \textbf{Frequenza} & \textbf{Rischio residuo} \\
		\hline
		\autoref{risk:conoscenza tecnologie carente} Conoscenza delle tecnologie carente     & 5                & 4                  & 20                       \\
		\autoref{risk:strumenti software inadeguati} Strumenti \textit{software} inadeguati           & 1                & 2                  & 2                        \\
		\autoref{risk:codice incomprensibile} Codice incomprensibile                         & 2                & 2                  & 4                        \\
		\autoref{risk:incompatibilità delle versioni del software} 
		Incompatibilità delle versioni del \textit{software}											 & 3                & 1                  & 3                        \\
		\autoref{risk:scarsa documentazione delle tecnologie utilizzate} 
		Scarsa documentazione delle tecnologie utilizzate	   								 & 2                & 2                  & 4                        \\
		\autoref{risk:problemi di sicurezza delle tecnologie utilizzate} 
		Problemi di sicurezza delle tecnologie utilizzate	   								 & 5                & 2                  & 10                        \\
		\hline
		\multicolumn{4}{l}{}                                                                                                                                    \\
		\hline
		\textbf{Rischi comunicativi}                                                         & \textbf{Gravità} & \textbf{Frequenza} & \textbf{Rischio residuo} \\
		\hline
		\autoref{risk:comunicazione interna carente} Comunicazione interna carente           & 3                & 3                  & 9                        \\
		\autoref{risk:conflitti decisionali} Conflitti decisionali                           & 1                & 2                  & 2                        \\
		\autoref{risk:comunicazione esterna carente} Comunicazine esterna carente            & 2                & 2                  & 4                        \\
		\autoref{risk:mancanza di chiarezza nei ruoli e responsabilità}
		Mancanza di chiarezza nei ruoli e responsabilità									 & 4                & 3                  & 12                        \\
		\autoref{risk:comunicazione asincrona inefficace}
		Comunicazione asincrona inefficace									 				 & 2                & 3                  & 6                        \\
		\hline
		\multicolumn{4}{l}{}                                                                                                                                    \\
		\hline
		\textbf{Rischi organizzativi}                                                        & \textbf{Gravità} & \textbf{Frequenza} & \textbf{Rischio residuo} \\
		\hline
		\autoref{risk:organizzazione carente} Organizzazione carente                         & 3                & 4                  & 12                       \\
		\autoref{risk:comprensione dei requisiti carente} Comprensione dei requisiti carente & 2                & 3                  & 6                        \\
		\autoref{risk:interfacce incoerenti} Interfacce incoerenti                           & 4                & 2                  & 8                        \\
		\autoref{risk:costi e tempi imprevisti} Costi e tempi imprevisti                     & 5                & 3                  & 15                       \\
		\autoref{risk:cambiamenti nei requisiti} Cambiamenti nei requisiti                   & 5                & 1                  & 5                        \\
		\hline
	\end{tabular}
	\caption{Tabella della pericolosità e dell'occorrenza dei rischi.}
\end{table}

