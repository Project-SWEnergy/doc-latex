\section{Analisi dei rischi}

La seguente sezione ha lo scopo di identificare e catalogare i rischi che
potrebbero verificarsi durante lo svolgimento del progetto, per poterli
prevenire o almeno provare a mitigarli.
Ciascun rischio è descritto seguendo la struttura:
\begin{itemize}
	\item codice identificativo seguito da un numero progressivo:
	      \begin{itemize}
		      \item \textbf{RT}: rischi legati alle tecnologie;
		      \item \textbf{RC}: rischi legati alla comunicazione;
		      \item \textbf{RP}: rischi legati alla pianificazione.
	      \end{itemize}

	\item titolo: il nome che identifica il rischio;

	\item descrizione: una breve descrizione del rischio;

	\item identificazione: in quale modo il \textit{team} può capire se si sta
	      verificando qualche danno;

	\item mitigazione: come il \textit{team} ha modo di prevenire o
	      attenuare i danni causati dal rischio;
\end{itemize}

Dopo la descrizione di ciascun rischio, viene presentata una tabella che
riassume i rischi individuati, associando a ciascuno un indice di gravità e un
indice di frequenza. Entrambi assumono un valore contenuto in
$\{Alto, Medio, Basso\}$.


\subsection{Rischi legati alle tecnologie}
\risktech{Conoscenza delle tecnologie carente}
\label{risk:conoscenza-tecnologie-carente}
\begin{itemize}
	\item \textbf{Descrizione}:
	      nello sviluppo del progetto, si può incorrere nella situazione in cui
	      almeno qualche membro non conosce almeno una tecnologia adottata dal
	      gruppo e necessaria per lo sviluppo del progetto.

	\item \textbf{Identificazione}: il \textit{team} ha individuato le
	      tecnologie conosciute dal gruppo. Con il proponente sono state
	      discusse e concordate le tecnologie da utilizzare per lo sviluppo del
	      progetto. In questo modo, sono state individuate le tecnologie
	      non conosciute dal gruppo.

	\item \textbf{Mitigazione}:
	      \begin{itemize}
		      \item \textit{workshop} interni: il \textit{team} sceglie
		            una o due persone per ogni tecnologia non conosciuta dal
		            gruppo. Le persone scelte si occupano di approfondire la
		            tecnologia e di organizzare un \textit{workshop} interno.
		            Le persone scelte svilupperanno inoltre degli esempi di
		            codice per illustrare l'utilizzo della tecnologia e degli
		            appunti da condividere;

		      \item seminari con il proponente: il \textit{team} partecipa a
		            dei seminari organizzati con il proponente, per approfondire
		            le tecnologie non conosciute dal gruppo. Il proponente
		            spiegherà le tecnologie e fornirà degli esempi di codice
		            per illustrarne l'utilizzo;

		      \item dialogo con il proponente: il \textit{team} può
		            contattare il proponente per chiedere chiarimenti sulle
		            tecnologie non conosciute dal gruppo.

		      \item documentazione: il \textit{team} può consultare la
		            documentazione ufficiale delle tecnologie non conosciute
		            dal gruppo.

		      \item \textit{pair programming}: il codice viene sviluppato con
		            almeno un altro membro del gruppo. Le modalità di lavoro
		            sono meglio descritte nel documento "\textit{Way of
			            working}".

		      \item \textit{code review}: il codice viene revisionato da
		            almeno un altro membro del gruppo. Le modalità di lavoro
		            sono meglio descritte nel documento "\textit{Way of
			            working}".

		      \item divisione del \textit{front-end} e del
		            \textit{back-end}: il \textit{team} si divide in due
		            sottogruppi, uno che si occupa del \textit{front-end} e
		            l'altro del \textit{back-end}. In questo modo, i membri
		            del gruppo possono concentrarsi su un numero ridotto di
		            tecnologie. I due gruppi si scambiano i ruoli al termine
		            della prima fase del progetto: RTB.
	      \end{itemize}
\end{itemize}

\risktech{Strumenti software datati}
\label{strumenti-software-datati}
\begin{itemize}
	\item \textbf{Descrizione}: l'utilizzo di strumenti software datati o poco
	      efficienti può portare a ritardi nello sviluppo del progetto;

	\item \textbf{Identificazione}:
	      I membri del gruppo possono lamentare l'utilizzo di strumenti
	      software poco efficienti durante le riunioni interne;

	\item \textbf{Mitigazione}:
	      \begin{itemize}
		      \item l'amministratore deve tenere sotto controllo le versioni
		            degli strumenti software utilizzati;

		      \item i membri del gruppo si informano in merito a nuove
		            tecnologie da adottare.
	      \end{itemize}
\end{itemize}


\subsection{Rischi legati alla comunicazione}

\subsection{Rischi legati alla pianificazione}
