\section{Analisi dei rischi}

La seguente sezione ha lo scopo di identificare e catalogare i rischi che
potrebbero verificarsi durante lo svolgimento del progetto, per poterli
prevenire o almeno provare a mitigarli.
Ciascun rischio è descritto seguendo la struttura:
\begin{itemize}
	\item codice identificativo seguito da un numero progressivo:
	      \begin{itemize}
		      \item \textbf{RT}: rischi legati alle tecnologie;
		      \item \textbf{RC}: rischi legati alla comunicazione;
		      \item \textbf{RP}: rischi legati alla pianificazione.
	      \end{itemize}

	\item titolo: il nome che identifica il rischio;

	\item descrizione: una breve descrizione del rischio;

	\item identificazione: in quale modo il \textit{team} può capire se si sta
	      verificando qualche danno;

	\item mitigazione: come il \textit{team} ha modo di prevenire o
	      attenuare i danni causati dal rischio;
\end{itemize}

Dopo la descrizione di ciascun rischio, viene presentata una tabella che
riassume i rischi individuati, associando a ciascuno un indice di gravità e un
indice di frequenza.

\subsection{Rischi legati alle tecnologie}
\risktech{Conoscenza delle tecnologie carente}
\label{risk:conoscenza-tecnologie-carente}
\begin{itemize}
	\item \textbf{Descrizione}:
	      nello sviluppo del progetto, si può incorrere nella situazione in cui
	      almeno qualche membro non conosce almeno una tecnologia adottata dal
	      gruppo e necessaria per lo sviluppo del progetto.

	\item \textbf{Identificazione}: il \textit{team} ha individuato le
	      tecnologie conosciute dal gruppo. Con il proponente sono state
	      discusse e concordate le tecnologie da utilizzare per lo sviluppo del
	      progetto. In questo modo, sono state individuate le tecnologie
	      non conosciute dal gruppo.

	\item \textbf{Mitigazione}:
	      \begin{itemize}
		      \item \textit{workshop} interni: il \textit{team} sceglie
		            una o due persone per ogni tecnologia non conosciuta dal
		            gruppo. Le persone scelte si occupano di approfondire la
		            tecnologia e di organizzare un \textit{workshop} interno.
		            Le persone scelte svilupperanno inoltre degli esempi di
		            codice per illustrare l'utilizzo della tecnologia e degli
		            appunti da condividere;

		      \item seminari con il proponente: il \textit{team} partecipa a
		            dei seminari organizzati con il proponente, per approfondire
		            le tecnologie non conosciute dal gruppo. Il proponente
		            spiegherà le tecnologie e fornirà degli esempi di codice
		            per illustrarne l'utilizzo;

		      \item dialogo con il proponente: il \textit{team} può
		            contattare il proponente per chiedere chiarimenti sulle
		            tecnologie non conosciute dal gruppo.

		      \item documentazione: il \textit{team} può consultare la
		            documentazione ufficiale delle tecnologie non conosciute
		            dal gruppo.

		      \item \textit{pair programming}: il codice viene sviluppato con
		            almeno un altro membro del gruppo. Le modalità di lavoro
		            sono meglio descritte nel documento "\textit{Way of
			            working}".

		      \item \textit{code review}: il codice viene revisionato da
		            almeno un altro membro del gruppo. Le modalità di lavoro
		            sono meglio descritte nel documento "\textit{Way of
			            working}".

		      \item divisione del \textit{front-end} e del
		            \textit{back-end}: il \textit{team} si divide in due
		            sottogruppi, uno che si occupa del \textit{front-end} e
		            l'altro del \textit{back-end}. In questo modo, i membri
		            del gruppo possono concentrarsi su un numero ridotto di
		            tecnologie. I due gruppi si scambiano i ruoli al termine
		            della prima fase del progetto: RTB.
	      \end{itemize}
\end{itemize}

\risktech{Strumenti software inadeguati}
\label{risk:strumenti software inadeguati}
\begin{itemize}
	\item \textbf{Descrizione}: l'utilizzo di strumenti software datati o poco
	      efficienti può portare a ritardi nello sviluppo del progetto;

	\item \textbf{Identificazione}:
	      \begin{itemize}
		      \item I membri del gruppo possono lamentare l'utilizzo di
		            strumenti software poco efficienti durante le riunioni
		            interne;

		      \item I membri del gruppo possono lamentare procedure troppo
		            lunghe e automatizzabili;

		      \item Nella rendicontazione delle ore, si nota che la medesima
		            attività subisce continui ritardi.
	      \end{itemize}

	\item \textbf{Mitigazione}:
	      \begin{itemize}
		      \item l'amministratore deve tenere sotto controllo le versioni
		            degli strumenti software utilizzati;

		      \item i membri del gruppo si informano in merito a nuove
		            tecnologie da adottare.
	      \end{itemize}
\end{itemize}

\risktech{Codice incomprensibile}
\label{risk:codice incomprensibile}
\begin{itemize}
	\item \textbf{Descrizione}: questo rischio riguarda la produzione di codice
	      da parte di alcuni membri del gruppo che risulta
	      difficile da comprendere per gli altri membri del \textit{team}.
	\item \textbf{Identificazione}:
	      \begin{itemize}
		      \item \textbf{\textit{Code review}}: durante la fase di verifica del codice,
		            i verificatori potrebbero riscontrare difficoltà
		            nella comprensione del codice, evidenziando
		            potenziali problemi di chiarezza e leggibilità.
	      \end{itemize}

	\item \textbf{Mitigazione}:
	      \begin{itemize}
		      \item \textbf{"Norme di progetto"}: il gruppo ha definito delle linee guida dettagliate
		            per la stesura del codice, al fine di uniformare lo stile di scrittura e facilitare
		            la comprensione. Le norme sono disponibili nel documento "Norme di progetto"
		            nella sotto-sezione "Codifica" sotto il ruolo di programmatore;

		      \item \textbf{\textit{Testing}}: il codice deve essere sottoposto a un processo di
		            \textit{testing} approfondito. Questo non solo aiuta a individuare eventuali errori o \textit{bug},
		            ma contribuisce anche a facilitare la comprensione del codice, illustrando
		            chiaramente i casi d'uso. Si rimanda alla sotto-sezione "Verifica del codice"
		            del documento "Norme di progetto" sotto il ruolo di verificatore.
	      \end{itemize}
\end{itemize}


\subsection{Rischi legati alla comunicazione}
\riskcom{Comunicazione interna carente}
\label{risk:comunicazione interna carente}
\begin{itemize}
	\item \textbf{Descrizione}:
	      La comunicazione interna non è efficace ed efficiente, causando riunioni
	      interne più lunghe del previsto e rallentando le attività.
	\item \textbf{Identificazione}:
	      \begin{itemize}
		      \item \textbf{Dubbi ripetuti}: durante le riunioni interne, i
		            membri del gruppo possono porre domande già presentate in
		            precedenza;

		      \item \textbf{Riunioni interne lunghe}: le riunioni interne
		            possono protrarsi oltre il tempo previsto;

		      \item \textbf{Fraintendimenti frequenti}: i membri del gruppo
		            possono fraintendersi frequentemente.

	      \end{itemize}
	\item \textbf{Mitigazione}:
	      \begin{itemize}
		      \item \textbf{Documentazione}: il gruppo si impegna a redigere 
			  		documentazione adeguata per facilitare la comunicazione interna. 
					La documentazione può assumere forme diverse a seconda dell'argomento;

		      \item \textbf{\textit{Meeting} frequenti}: il gruppo stabilisce incontri interni 
			  		frequenti per ridurre la durata delle riunioni e migliorare la comunicazione 
					interna. Questo permette un flusso costante di informazioni e la risoluzione 
					tempestiva di eventuali dubbi;

		      \item \textbf{Ordine del giorno}: ogni riunione viene pianificata con un ordine 
			  		del giorno ben definito, garantendo la discussione di tutti gli argomenti 
					rilevanti per lo sviluppo del progetto e definendo il tempo dedicato a 
					ciascun punto;

		      \item \textbf{Retrospettiva}: il gruppo riflette sulle sfide riscontrate nella 
			  		comunicazione interna e sviluppa soluzioni \textit{ad hoc} per migliorare 
					il flusso delle informazioni e prevenire futuri fraintendimenti.
	      \end{itemize}
\end{itemize}

\riskcom{Conflitti decisionali}
\label{risk:conflitti decisionali}
\begin{itemize}
	\item \textbf{Descrizione}:
	      Il gruppo potrebbe dilungarsi nella discussione di una sola idea, senza
	      raggiungere una decisione finale.
	\item \textbf{Identificazione}:
	      \begin{itemize}
		      \item un punto dell'ordine del giorno subisce un ritardo grave;
	      \end{itemize}
	\item \textbf{Mitigazione}:
	      \begin{itemize}

		      \item \textbf{Dibattito}: i membri del gruppo si impegnano in una 
			  		discussione riguardo all'importanza del punto dell'ordine del 
					giorno per determinare se è necessario approfondire ulteriormente 
					la discussione o meno.

		      \item \textbf{Approfondimento}: se il punto dell'ordine del giorno è 
			  ritenuto importante, almeno due membri del gruppo si dedicano a uno studio 
			  approfondito dei pro e contro delle varie soluzioni possibili. 
			  Possono richiedere supporto al proponente o al committente per chiarire i dubbi.

		      \item \textbf{Votazione}: alla fine del dibattito, i membri del gruppo 
			  votano per la soluzione che ritengono più opportuna. 
			  La votazione è considerata conclusa quando la maggioranza dei membri 
			  del gruppo ha espresso la propria preferenza e il risultato non è un pareggio.

		      \item \textbf{Arbitro imparziale}: il responsabile del progetto ha il compito 
			  di vigilare sul corretto svolgimento del dibattito e della votazione, 
			  intervenendo se la discussione si dilunga eccessivamente. 
			  Il suo ruolo è quello di garantire l'efficienza e l'imparzialità del processo decisionale.
	      \end{itemize}
\end{itemize}

\riskcom{Comunicazione esterna carente}
\label{risk:comunicazione esterna carente}
\begin{itemize}
	\item \textbf{Descrizione}:
	      Le comunicazioni con il proponente o con il committente non sono
	      efficaci ed efficienti, causando riunioni esterne più lunghe del
	      previsto e rallentando le attività; oppure rallentando le attività
	      del gruppo a causa di risposte tardive o mancanti.

	\item \textbf{Identificazione}:
	      \begin{itemize}
		      \item dubbi ripetuti: durante le riunioni esterne, i membri del
		            gruppo possono porre domande già presentate in precedenza;

		      \item riunioni esterne lunghe: le riunioni esterne possono
		            protrarsi oltre il tempo previsto;

		      \item risposte tardive o mancanti: il proponente o il committente
		            può rispondere in ritardo o non rispondere affatto alle
		            comunicazioni del gruppo;

		      \item durante le retrospettive, i membri del gruppo possono
		            lamentarsi di una comunicazione esterna carente.
	      \end{itemize}

	\item \textbf{\gls{Mitigazione}$^G$}:
	      \begin{itemize}
		      \item \gls{Ordine}$^G$ del giorno: il responsabile si impegna a stilare
		            l'\gls{Ordine}$^G$ del giorno delle riunioni esterne, per tempo, ne
		            discute la struttura con il gruppo e lo condivide con il
		            proponente e con il committente in anticipo;

		      \item SAL: il gruppo si impegna a mantenere il
		            proponente aggiornato sullo \gls{Stato}$^G$ di avanzamento del progetto,
		            in modo da ridurre la durata delle riunioni esterne e
		            migliorare la qualità del supporto del proponente;

		      \item retrospettive: sono previste delle retrospettive
		            all'interno dei SAL con il proponente, si discute della
		            qualità delle comunicazioni e si pensa a soluzioni
		            \textit{ad hoc} per migliorare la comunicazione esterna;

		      \item comunicazioni frequenti: il proponente viene tenuto
		            aggiornato frequentemente sullo \gls{Stato}$^G$ di avanzamento del
		            progetto mediante gli appositi canali di comunicazione, per
		            esempio \textit{\gls{Telegram}$^G$};

		      \item diario di bordo: il gruppo si impegna a tenere dei diari di
		            bordo, quando richiesti dal committente, per aggiornarlo
		            sullo \gls{Stato}$^G$ di avanzamento del progetto;

		      \item \textit{meeting} supplementari: se il gruppo manifesta dei
		            dubbi o delle incertezze, può richiedere un \textit{meeting}
		            supplementare con il proponente o con il committente;

		      \item documentazione: il responsabile si impegna ad aggiornare
		            la documentazione inerente agli argomenti trattati durante
		            le riunioni esterne, per dare modo ai membri del gruppo di
		            consultarla in caso di dubbi o incertezze.
	      \end{itemize}
\end{itemize}


\subsection{Rischi legati alla pianificazione}
I membri del gruppo non hanno mai assunto un ruolo manageriale in
precedenza e non hanno mai lavorato in un gruppo di lavoro così
numeroso. Questo porta a problemi di gestione del tempo e delle
risorse. D'altro canto, SWEnergy si rende conto che lo scopo del
progetto è proprio quello di acquisire esperienza, anche in questi
termini. Per cui, il gruppo ha deciso di individuare alcuni
rischi legati alla pianificazione, per poterli prevenire o mitigare.

\riskplan{Organizzazione carente}
\label{risk:organizzazione carente}
\begin{itemize}
	\item \textbf{Descrizione}:
	      Il gruppo, oppure qualche membro, potrebbe non essere in grado di
	      svolgere le proprie attività, oppure potrebbe riscontrare delle
	      difficoltà a causa di una cattiva organizzazione.
	\item \textbf{Identificazione}:
	      \begin{itemize}
		      \item \textbf{Membri confusi}: i membri del gruppo non sanno quali
		            sono i compiti a loro assegnati, oppure non sanno come
		            svolgerli;

		      \item \textbf{Carenza di risorse}: sono stati assegnati più
		            incarichi di quelli sostenibili con le risorse disponibili;

		      \item \textbf{Scadenze non aggiornate}: il gruppo o qualche suo
		            membro non è in grado di rispettare le scadenze e queste non
		            sono aggiornate. Si tratta di un modo molto semplice, per
		            ricadere nel sintomo individuato precedentemente.
	      \end{itemize}

	\item \textbf{Mitigazione}:
	      \begin{itemize}
		      \item \textbf{Pianificazione delle \textit{issue}}: si rimanda
		            alla sotto-sezione "Pianificazione delle attività" del
		            documento "Norme di progetto" sotto il ruolo di
		            responsabile;

		      \item \textbf{Aggiornamento delle \textit{issue}}:
		            ciascun componente di SWEnergy deve aggiornare le
		            \textit{issue} a cui è assegnato, in modo da tenere il
		            responsabile e l'intera organizzazione aggiornati sullo
		            stato di avanzamento dei compiti; inoltre, deve aggiungere
		            delle \textit{issue} se ritiene che ci siano delle attività
		            da svolgere;

		      \item \textbf{Persona di riferimento}: in caso di dubbi, i
		            membri di SWEnergy possono rivolgersi
		            al responsabile, che si occuperà di chiarire la situazione,
		            o di indirizzare il membro verso chi può aiutarlo;

		      \item \textbf{Retrospettiva}: durante le retrospettive, il gruppo
		            discute di eventuali problemi organizzativi e cerca di
		            trovare soluzioni per migliorare la pianificazione;

		      \item \textbf{Dialogo con il proponente}: sono chiesti consigli al
		            proponente in merito, in quanto ha più esperienza
		            nel settore e ha modo di collaborare con figure manageriali.
	      \end{itemize}
\end{itemize}

\riskplan{Comprensione dei requisiti carente}
\label{risk:comprensione dei requisiti carente}
\begin{itemize}
	\item \textbf{Descrizione}:
	      Il gruppo o qualche suo membro potrebbe non essere in grado di
	      comprendere i requisiti del progetto, oppure potrebbe riscontrare
	      delle difficoltà a causa di una cattiva comprensione dei requisiti.
	\item \textbf{Identificazione}:
	      \begin{itemize}
		      \item dubbi: i membri del gruppo hanno dei dubbi in merito ai
		            requisiti;

		      \item dibattiti sui requisiti: i membri del gruppo
		            discutono tra loro in merito ai requisiti;

		      \item discrepanza nella progettazione: i membri del gruppo
		            progettano in modo diverso, a causa di una cattiva
		            comprensione dei requisiti.
	      \end{itemize}
	\item \textbf{Mitigazione}:
	      \begin{itemize}
		      \item Dibattito interno: SWEnergy si è diviso in coppie per
		            approfondire i casi d'uso e i requisiti del progetto. Poi si
		            è tenuta una riunione interna in cui ciasuna coppia ha
		            esposto i propri dubbi e le proprie considerazioni. In
		            questo modo, si è cercato di chiarire i dubbi e di
		            uniformare la comprensione dei requisiti.

		      \item "Analisi dei requisiti": il metodo più formale per ovviare a
		            questa situazione risulta essere l'"Analisi dei requisiti".
		            I requisiti dovrebbero essere chiari e completi. Inoltre,
		            il documento contiene i casi d'uso, che aiutano a
		            comprendere meglio i requisiti concordati con il proponente.

		      \item dialogo con il proponente: si discute con il proponente in
		            merito ai requisiti, per chiarire eventuali dubbi e per
		            decidere in maggiore dettaglio le funzionalità del prodotto.

		      \item messaggi tempestivi con il proponente: in caso di dubbi
		            semplici e veloci da risolvere, si inviano dei messaggi al
		            proponente, per ottenere una risposta tempestiva.
	      \end{itemize}
\end{itemize}

\riskplan{Interfacce incoerenti}
\label{risk:interfacce incoerenti}
\begin{itemize}
	\item \textbf{Descrizione}:
	      Durante la fase integrativa di più componenti, risultano delle
	      incongruenze che rendono impossibile l'integrazione.
	\item \textbf{Identificazione}:
	      \begin{itemize}
		      \item \textbf{Test di integrazione falliti}: i test di
		            integrazione falliscono a causa di incongruenze tra le
		            interfacce delle componenti;

		      \item \textbf{Discussioni interne in merito alle interfacce}: i
		            membri del gruppo discutono tra loro in merito alle
		            interfacce delle componenti, per capire come risolvere le
		            incongruenze;

		      \item \textbf{Fallimento del sistema}: l'applicativo non funziona
		            in seguito ad un'integrazione.

	      \end{itemize}
	\item \textbf{Mitigazione}:
	      \begin{itemize}
		      \item \textbf{Dialogo interno}: i membri del gruppo discutono tra loro
		            in merito alle interfacce delle componenti, prima di
		            cominciare a sviluppare le componenti stesse;

		      \item \textbf{Test di integrazione}: vengono effettuati dei test di
		            integrazione, per verificare che le componenti siano
		            compatibili tra loro, in modo da agevolare l'identificazione
		            del problema;

		      \item \textbf{Documentazione}: le interfacce delle componenti sono
		            documentate in modo chiaro e completo, per evitare
		            incomprensioni ed esplicitarne la struttura e la
		            compatibilità.
	      \end{itemize}
\end{itemize}

\riskplan{Costi e tempi imprevisti}
\label{risk:costi e tempi imprevisti}
\begin{itemize}
	\item \textbf{Descrizione}:
	      Durante lo sviluppo del progetto, si può incorrere in costi o
	      rallentamenti imprevisti. Si tratta, a tutti gli effetti, di arginare
	      il danno prodotto da un rischio che si è verificato.
	\item \textbf{Identificazione}:
	      \begin{itemize}
		      \item cambiamenti significativi nelle tempistiche di completamento
		            del progetto;

		      \item variazioni notevoli dei costi di realizzazione.

		      \item monitoraggio costante: monitoraggio dei costi e delle
		            tempistiche al completamento delle \textit{milestone} e
		            ad ogni \textit{stand-up}.
	      \end{itemize}
	\item \textbf{Mitigazione}:
	      \begin{itemize}
		      \item \textit{buffer} di tempo: il \textit{team} ha
		            preventivamente inserito dei \textit{buffer} di tempo
		            tra le varie attività, per poter gestire eventuali
		            ritardi;

		      \item \textit{buffer} di costi: il \textit{team} ha
		            preventivamente inserito dei \textit{buffer} di costi
		            tra le varie attività, per poter gestire eventuali
		            costi imprevisti;

		      \item pianificazione in itinere: il \textit{team} si adatta
		            alle variazioni dei costi e delle tempistiche di
		            completamento, per poter gestire eventuali costi
		            imprevisti. In questo caso, sono aggiornate le scadenze
		            nel \textit{project} su \textit{GitHub} e i costi.
		            A seconda della situazione, le \textit{issue} sono
		            riassegnate e le \textit{milestone} sono adattate allo
		            \textit{status quo}.
	      \end{itemize}
\end{itemize}


\subsection{Pericolosità e occorrenze}

Per ogni rischio, il \textit{team} ha individuato un indice di gravità e un
indice di frequenza, per poter stimare il rischio residuo. L'indice di
gravità e quello di frequenza sono due numeri compresi tra 1 e 5. L'indice di
rischio residuo è il prodotto tra i due indici, può quindi assumere valori
compresi tra 1 e 25. Più il rischio residuo è alto, maggiori sono i danni che
può causare e più è probabile che si verifichi. Si noti che non è detto che il
verificarsi del rischio causi i danni massimi, le strategie di
mitigazione servono proprio per prevenire e attenuare i danni.

\begin{table}[H]
	\centering

	\begin{tabular}{l|r|r|r}
		\hline
		\textbf{Rischi tecnologici}                                                          & \textbf{Gravità} & \textbf{Frequenza} & \textbf{Rischio residuo} \\
		\hline
		\autoref{risk:conoscenza tecnologie carente} Conoscenza delle tecnologie carente     & 5                & 4                  & 20                       \\
		\autoref{risk:strumenti software inadeguati} Strumenti \textit{software} inadeguati           & 1                & 2                  & 2                        \\
		\autoref{risk:codice incomprensibile} Codice incomprensibile                         & 2                & 2                  & 4                        \\
		\hline
		\multicolumn{4}{l}{}                                                                                                                                    \\
		\hline
		\textbf{Rischi comunicativi}                                                         & \textbf{Gravità} & \textbf{Frequenza} & \textbf{Rischio residuo} \\
		\hline
		\autoref{risk:comunicazione interna carente} Comunicazione interna carente           & 3                & 3                  & 9                        \\
		\autoref{risk:conflitti decisionali} Conflitti decisionali                           & 1                & 2                  & 2                        \\
		\autoref{risk:comunicazione esterna carente} Comunicazine esterna carente            & 2                & 2                  & 4                        \\
		\hline
		\multicolumn{4}{l}{}                                                                                                                                    \\
		\hline
		\textbf{Rischi organizzativi}                                                        & \textbf{Gravità} & \textbf{Frequenza} & \textbf{Rischio residuo} \\
		\hline
		\autoref{risk:organizzazione carente} Organizzazione carente                         & 3                & 4                  & 12                       \\
		\autoref{risk:comprensione dei requisiti carente} Comprensione dei requisiti carente & 2                & 3                  & 6                        \\
		\autoref{risk:interfacce incoerenti} Interfacce incoerenti                           & 4                & 2                  & 8                        \\
		\autoref{risk:costi e tempi imprevisti} Costi e tempi imprevisti                     & 5                & 3                  & 15                       \\
		\hline
	\end{tabular}
	\caption{Tabella della pericolosità e dell'occorrenza dei rischi.}
\end{table}

