\section{Analisi dei rischi}

La seguente sezione ha lo scopo di identificare e catalogare i rischi che
potrebbero verificarsi durante lo svolgimento del progetto, per poterli
prevenire o almeno provare a mitigarli.
Ciascun rischio è descritto seguendo la struttura:
\begin{itemize}
	\item codice identificativo seguito da un numero progressivo:
	      \begin{itemize}
		      \item \textbf{RT}: rischi legati alle tecnologie;
		      \item \textbf{RC}: rischi legati alla comunicazione;
		      \item \textbf{RP}: rischi legati alla pianificazione.
	      \end{itemize}

	\item titolo: il nome che identifica il rischio;

	\item descrizione: una breve descrizione del rischio;

	\item identificazione: in quale modo il \textit{team} può capire se si sta
	      verificando qualche danno;

	\item mitigazione: come il \textit{team} ha modo di prevenire o
	      attenuare i danni causati dal rischio;
\end{itemize}

Dopo la descrizione di ciascun rischio, viene presentata una tabella che
riassume i rischi individuati, associando a ciascuno un indice di gravità e un
indice di frequenza. Entrambi assumono un valore contenuto in
$\{Alto, Medio, Basso\}$.


\subsection{Rischi legati alle tecnologie}
\risktech{Conoscenza delle tecnologie carente}
\label{risk:conoscenza tecnologie carente}
\begin{itemize}
	\item \textbf{Descrizione}:
		Durante lo sviluppo del progetto, potrebbe verificarsi la situazione 
		in cui almeno un membro del \textit{team} non possiede una conoscenza 
		sufficiente di una tecnologia adottata dal gruppo e necessaria per 
		lo sviluppo del progetto.

	\item \textbf{Identificazione}: 
		Il \textit{team} ha identificato le tecnologie conosciute dal gruppo 
		attraverso discussioni e accordi con il proponente. 
		Questo processo ha permesso di individuare le tecnologie non conosciute dal gruppo.

	\item \textbf{Mitigazione}:
	      \begin{itemize}
		      \item \textbf{\textit{Workshop} interni}: si rimanda alla
		            sotto-sezione "Organizzare un \textit{workshop}" del
		            documento "Norme di progetto" sotto il ruolo di progettista;

		      \item \textbf{Seminari con il proponente}: il \textit{team}
		            partecipa a seminari organizzati con il proponente, per approfondire
		            le tecnologie non conosciute. 
					Il proponente spiegherà le tecnologie e fornirà esempi di codice
		            per illustrarne l'utilizzo;

		      \item \textbf{Dialogo con il proponente}: il \textit{team} può
		            contattare il proponente per chiarimenti sulle
		            tecnologie non conosciute;

		      \item \textbf{\textit{Code review}}: si rimanda alla sotto-sezione
		            "Verifica del codice" del documento "Norme di progetto"
		            sotto il ruolo di verificatore;

		      \item \textbf{Divisione del \textit{front-end} e del \textit{back-end}}: 
			  		il \textit{team} si suddivide in due sottogruppi, uno responsabile del 
					\textit{front-end} e l'altro del \textit{back-end}. 
					Questa divisione riduce l'\textit{overhead} di comunicazione e di cambio di
		            contesto. I due gruppi si scambiano i ruoli al termine della prima 
					fase del progetto: RTB.
	      \end{itemize}
	
	\item \textbf{Riscontro}: Nessuna conseguenza significativa è stata riscontrata in quanto le misure di mitigazione necessarie sono state tempestivamente implementate
	ed i componenti del gruppo si sono riuniti al più presto per poter risolvere la problematica. Particolarmente utili si sono rilevati i \textit{workshop} interni,
	un dialogo attivo con il proponente e la divisione del \textit{team} in due sottogruppi per il \textit{front-end} e il \textit{back-end}. Mentre per quanto riguarda la mitigazione 
	prevista "Seminari con il proponente" nononostante sia stata attuata non si è rivelata del tutto efficace, per questo motivo il gruppo non ha ritenuto necessario farne uso ulteriormente.
	Una criticità riscontrata per quanto riguarda la divisione del \textit{team} in due sottogruppi sono state le difficoltà di comunicazione tra i due gruppi e formazione dei componenti, 
	che ha portato ad un rallentamento nello sviluppo del progetto.
\end{itemize}

\risktech{Strumenti software datati}
\label{strumenti-software-datati}
\begin{itemize}
	\item \textbf{Descrizione}: l'utilizzo di strumenti software datati o poco
	      efficienti può portare a ritardi nello sviluppo del progetto;

	\item \textbf{Identificazione}:
	      I membri del gruppo possono lamentare l'utilizzo di strumenti
	      software poco efficienti durante le riunioni interne;

	\item \textbf{Mitigazione}:
	      \begin{itemize}
		      \item l'amministratore deve tenere sotto controllo le versioni
		            degli strumenti software utilizzati;

		      \item i membri del gruppo si informano in merito a nuove
		            tecnologie da adottare.
	      \end{itemize}
\end{itemize}


\subsection{Rischi legati alla comunicazione}

\subsection{Rischi legati alla pianificazione}
