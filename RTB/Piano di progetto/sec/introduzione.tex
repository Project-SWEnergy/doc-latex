\section{Introduzione}

Il presente documento, intitolato "Piano di Progetto", descrive e spiegare le
decisioni organizzative adottate dal gruppo SWEnergy per lo sviluppo del
progetto "\textit{Easy Meal}", proposto dall'azienda
\href{https://imolainformatica.it/}{Imola Informatica}. Il "Piano di Progetto" è
suddiviso nelle seguenti sezioni:

\begin{itemize}
	\item \textbf{Analisi dei rischi}: identifica i rischi individuati dal
	      gruppo e le strategie per mitigarli;

	\item \textbf{Modello di sviluppo}: descrive l'organizzazione temporale del
	      team di SWEnergy;

	\item \textbf{Pianificazione}: dettaglia la pianificazione del lavoro del
	      gruppo, incluse le attività, le risorse e i tempi necessari per lo
	      sviluppo del progetto;

	\item \textbf{Preventivo}: presenta il preventivo delle ore di lavoro e il
	      costo totale del progetto;

	\item \textbf{Consuntivo}: riporta le ore di lavoro e il costo effettivo del
	      progetto fino al momento della stesura del piano di progetto della
	      fase corrente: RTB.
\end{itemize}

\subsection{Scopo del documento}

Questo documento ha lo scopo di raccogliere in modo organico, coerente e
uniforme tutte le informazioni riguardanti la pianificazione del progetto, al
fine di fornire un riferimento per la gestione dello stesso. Al termine della
prima fase del progetto (RTB), verrà utilizzato per valutare l'andamento del
lavoro e per spiegare le decisioni adottate durante la pianificazione.

\subsection{Scopo del prodotto}

"\textit{Easy Meal}" è una web app progettata per gestire le prenotazioni
presso i ristoranti, sia dal lato dei clienti che dei ristoratori. Il prodotto
finale sarà composto da due parti:

\begin{itemize}
	\item \textbf{Cliente}: consente ai clienti di prenotare un tavolo presso un
	      ristorante, visualizzare il menù e effettuare un ordine;

	\item \textbf{Ristoratore}: consente ai ristoratori di gestire le
	      prenotazioni e gli ordini dei clienti, oltre a visualizzare la lista
	      degli ingredienti necessari per preparare i piatti ordinati.
\end{itemize}

\subsection{Glossario}

Al fine di evitare ambiguità linguistiche e garantire un'utilizzazione coerente
delle terminologie nei documenti, il gruppo ha redatto un documento interno
chiamato "Glossario". Questo documento definisce in modo chiaro e preciso i
termini che potrebbero generare ambiguità o incomprensione nel testo. I termini
presenti nel Glossario sono identificati da una 'G' (per esempio parola$_G$) a
pedice.

\subsection{Riferimenti}

\subsubsection{Normativi}
\begin{itemize}
	\item "\textit{Way of Working}";
	\item 	\href{https://www.math.unipd.it/~tullio/IS-1/2023/Progetto/C3.pdf}
	      {Documento del capitolato d'appalto C3 - \textit{Easy Meal}};
	\item \href{https://www.math.unipd.it/~tullio/IS-1/2023/Dispense/PD2.pdf}
	      {Regolamento del progetto};
\end{itemize}

\subsubsection{Informativi}

Slide dell'insegnamento di Ingegneria del Software:
\begin{itemize}
	\item \href{https://www.math.unipd.it/~tullio/IS-1/2023/Dispense/T3.pdf}
	      {Modelli di sviluppo del software};
	\item \href{https://www.math.unipd.it/~tullio/IS-1/2023/Dispense/T4.pdf}
	      {Gestione di progetto};
	\item \href{https://www.math.unipd.it/~tullio/IS-1/2023/Dispense/T5.pdf}
	      {Analisi dei requisiti};
\end{itemize}

\subsection{Scadenze}
Il \textit{team} di SWEnergy si impegna a rispettare le seguenti scadenze per il
completamento del progetto:
\begin{itemize}
	\item \textbf{Prima revisione (avanzamento RTB)}: 21 dicembre 2023;
	\item \textbf{Seconda revisione (avanzamento PB)}: da definire;
	\item \textbf{Terza revisione (avanzamento CA)}: da definire;
\end{itemize}
