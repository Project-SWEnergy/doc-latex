\section{Modello di sviluppo}

\subsection{Modello agile}
SWEnergy ha adottato un modello di sviluppo agile, con alcune personalizzazioni atte 
a massimizzare l'efficacia del processo. 
Uno dei potenziali svantaggi del modello agile è rappresentato dal rischio di cadere 
in un'iterazione continua, ovvero un ritorno all'indietro nella direzione opposta all'avanzamento del progetto, per correzioni o
rifacimenti, il che potrebbe rivelarsi un processo distruttivo. 
Per mitigare questo rischio, abbiamo scelto di stabilire un rapporto collaborativo 
con il proponente, garantendo un \textit{feedback} costante al fine di evitare la 
necessità di ritornare su decisioni già prese e ridurre i costi delle iterazioni. 
Ogni progresso nel progetto viene presentato al proponente durante le sessioni di revisione. \\

Poiché il \textit{team} ha una limitata esperienza professionale, il modello di sviluppo 
trae ispirazione dal \textit{framework} Scrum, utilizzando \textit{sprint} di due settimane 
con alcune personalizzazioni per adattarlo alle specifiche del progetto. 
In particolare, SWEnergy ha introdotto una retrospettiva \textit{in media res} per valutare 
il lavoro svolto e apportare modifiche al processo di sviluppo. 
Allo stesso tempo, abbiamo scelto di non includere i \textit{daily stand-up meeting}, poiché 
il gruppo ritiene che la loro frequenza sia eccessiva, considerando il contesto degli 
studenti universitari coinvolti a tempo parziale nel progetto.

\subsection{Iterazioni}

\subsubsection{\textit{Sprint}}
Uno \textit{sprint}, della durata di due settimane, impegna il gruppo nello 
sviluppo del prodotto concordato con il proponente. 
Durante questo periodo, il \textit{team} segue una strategia interna per raggiungere 
gli obiettivi di avanzamento stabiliti. 
La durata dello \textit{sprint} facilita il ricevimento di \textit{feedback} 
regolari dal proponente e consente al \textit{team} di apportare modifiche tempestive 
al prodotto. 
Inoltre, offre la possibilità di risolvere dinamicamente eventuali problemi 
o dubbi con il proponente.

\subsubsection{Mini-\textit{sprint}}
Questa iterazione, basata sul \textit{framework} Scrum, si svolge settimanalmente 
come uno \textit{sprint} interno al gruppo, ma senza coinvolgere il proponente. 
Al termine di un mini-\textit{sprint}, potrebbe verificarsi un cambio dei ruoli 
in base alle esigenze del progetto e del gruppo. 
I mini-\textit{sprint} aumentano la frequenza delle retrospettive, consentendo 
di valutare il progresso delle attività e apportare modifiche dinamiche al processo 
di sviluppo, adattando il lavoro alle esigenze del progetto. 
È importante notare che un mini-\textit{sprint} si verifica all'interno di uno 
\textit{sprint} principale.

\subsection{Eventi}

\subsubsection{SAL}
Lo \SAL{} è un incontro con il proponente che
avviene ogni due settimane di venerdì, ed , è fondamentale per condividere i 
\textit{feedback} in entrambe le direzioni. 
Durante il SAL, avvengono attività cruciali:

\begin{itemize}
	\item \textbf{\textit{Sprint review}}: Il gruppo presenta il lavoro svolto 
			durante lo \textit{sprint}, ricevendo \textit{feedback} dal proponente. 
			Vengono affrontati eventuali dubbi sui requisiti o sulle funzionalità implementate;
	
	\item \textbf{\textit{Sprint retrospective}}: Si discute sulle modalità di lavoro, 
			valutando l'efficacia del processo di sviluppo e identificando possibili miglioramenti. 
			Si richiedono consigli al proponente sull'organizzazione del lavoro, si segnalano 
			problemi riscontrati durante lo \textit{sprint} e si propongono soluzioni;

	\item \textbf{\textit{Sprint planning}}: Il gruppo e il proponente concordano lo stato 
			di avanzamento del prodotto da raggiungere durante lo \textit{sprint} successivo, 
			determinando cosa includere nello \textit{sprint backlog}.
\end{itemize}

\subsubsection{\textit{Stand-up}}
Il nome \textit{stand-up} è ispirato ai \textit{daily stand-up meeting} del
\textit{framework} Scrum. 
L'incontro con il proponente si svolge ogni due settimane di venerdì, 
mentre le \textit{stand-up}, sono incontri posti all'inizio e alla fine 
di un mini-\textit{sprint}, hanno luogo ogni domenica, per considerare 
i \textit{feedback} del proponente e di pianificare l'iterazione successiva.
Durante la \textit{stand-up} avvengono le seguenti attività:

\begin{itemize}
	\item \textbf{\textit{Brainstorming}}: il responsabile riassume il lavoro 
			svolto durante la settimana e ogni membro del gruppo arricchisce 
			la discussione con le proprie esperienze, con particolare attenzione 
			alle difficoltà incontrate e alle soluzioni adottate;

	\item \textbf{Retrospettiva}: il gruppo discute sulle modalità di lavoro, 
			valutando l'efficacia del processo di sviluppo e proponendo miglioramenti. 
			Si affrontano eventuali problemi riscontrati durante il mini-\textit{sprint} 
			e si propongono soluzioni. 
			I problemi possono essere successivamente discussi con il proponente 
			durante il SAL o con il committente per ottenere consigli.

	\item \textbf{Pianificazione}: il responsabile presenta la pianificazione del 
			mini-\textit{sprint} successivo. I membri del gruppo intervengono per proporre 
			miglioramenti o una migliore ripartizione del lavoro. 
			Infine, il responsabile assegna i compiti ai membri del gruppo, 
			considerando le loro disponibilità, capacità e preferenze.
\end{itemize}

\subsection{Motivazioni}
La scelta di SWEnergy di adottare questo approccio è guidata principalmente 
dalla necessità. 
I membri del gruppo hanno acquisito competenze fondamentali del 
\textit{framework} Scrum durante i corsi di Ingegneria del Software e di Metodi 
e Tecnologie per lo Sviluppo Software. 
Dato che il \textit{team} non ha esperienza professionale, questa scelta 
fornisce un quadro organizzativo solido. 
Inoltre, la richiesta del proponente di una pianificazione di almeno due settimane 
si adatta bene a questa organizzazione. 
Questo approccio consente a SWEnergy di soddisfare le esigenze del proponente, 
mitigando i rischi individuati durante l'analisi dei rischi e consentendo a 
ogni membro del gruppo di comprendere appieno il ruolo assegnato durante 
ogni ciclo di due settimane. 
Il \textit{framework} Scrum offre diversi vantaggi, tra cui:

\begin{enumerate}
	\item \textbf{Flessibilità}: il gruppo può adattare il processo di sviluppo 
			alle esigenze del progetto, modificando la pianificazione in base alle 
			richieste del proponente e alle sfide riscontrate.

	\item \textbf{Comunicazione trasparente}: il gruppo rilascia regolarmente 
			\textit{feedback} sul prodotto, mantenendo il proponente aggiornato sullo 
			\textit{status quo} del progetto. 

	\item \textbf{Miglioramento continuo}: le retrospettive consentono al gruppo 
			di valutare costantemente il processo di sviluppo e di apportare modifiche 
			per migliorare l'efficienza. 
			Il confronto con il proponente e il committente fornisce ulteriori consigli 
			e suggerimenti per ottimizzare l'organizzazione e il metodo di lavoro.

	\item \textbf{Monitoraggio costante}: la pianificazione basata sugli \textit{sprint} 
			permette al gruppo di identificare e affrontare tempestivamente i rischi, 
			riducendo la possibilità di gravi ritardi e di aumenti di costo nel progetto.
\end{enumerate}
