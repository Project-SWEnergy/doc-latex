\section{Preventivo}

Nella seguente sezioni sono riportate le attività di cui si prevede lo
svolgimento in ciascuno sprint e la suddivisione delle ore preventivate per
ciascun componente del gruppo.

\sprint{Sprint}

In seguito al colloquio con il proponente, il gruppo ha deciso di completare le
seguenti attività:

\begin{itemize}
	\item \textbf{"Analisi dei requisiti"}: risulta necessario approfondire la
	      comprensione dei casi d'uso e dei requisiti ad essi associati, in modo da
	      poterli documentare in maniera precisa e completa. Non solo, è anche necessario
	      completare l'analisi dei requisiti, in modo da poter procedere con la
	      progettazione delle PoC. Per questo motivo, è stato deciso di dedicare 30 ore a
	      questa attività.

	\item \textbf{"Analisi delle tecnologie"}: è necessario approfondire l'analisi
	      delle tecnologie, per avere modo di comprendere le loro caratteristiche e quali
	      requisiti possono soddisfare. Non solo, l'analisi delle tecnologie aiuta a
	      sviluppare i casi d'uso e i requisiti, in quanto permette di comprendere quali
	      siano le funzionalità che è possibile implementare. Per questo motivo, è stato
	      deciso di dedicare 10 ore a questa attività.

	\item \textbf{"Piano di progetto"}: è necessario stendere il piano di progetto,
	      per formalizzare la pianificazione delle attività e la suddivisione dei ruoli.
	      Per questo motivo, è stato deciso di dedicare 10 ore a questa attività.

	\item \textbf{"Piano di qualifica"}: è necessario stendere il piano di
	      qualifica, per formalizzare le strategie di verifica e validazione del prodotto
	      e dei processi. Per questo motivo, è stato deciso di dedicare 5 ore a questa
	      attività. Più che altro, SWEnergy ritiene che sia necessario comprendere quali
	      argomenti trattare nel piano di qualifica, in modo da poterlo sviluppare
	      completamente in seguito.

	\item \textbf{Database}: in seguito all'"Analisi dei requisiti", risulta
	      opportuno progettare il database che verrà utilizzato per la realizzazione delle
	      PoC. Per questo motivo, è stato deciso di dedicare 5 ore a questa attività.
\end{itemize}

\subsubsection{Suddivisione delle ore}

\begin{table}[H]
	\centering
	\begin{tabular}{l|r|r|r|r|r|r|r}
		\textbf{Membro}        & \textbf{Re} & \textbf{Am} & \textbf{An} & \textbf{Pt} & \textbf{Pr} & \textbf{Ve} & \textbf{Tot} \\
		\hline
		Alessandro Tigani Sava & 5           & -           & -           & 5           & -           & -           & 10           \\
		Carlo Rosso            & 10          & 5           & -           & 5           & -           & -           & 20           \\
		Davide Maffei          & -           & -           & 10          & -           & -           & -           & 10           \\
		Giacomo Gualato        & -           & -           & 10          & -           & -           & -           & 10           \\
		Matteo Bando           & -           & -           & 5           & -           & -           & -           & 5            \\
		Niccolò Carlesso       & -           & -           & 5           & -           & -           & -           & 5            \\
		\hline
		\textbf{Totale}        & 15          & 5           & 30          & 10          & 0           & 0           & 60           \\
	\end{tabular}

	\caption{Re: Responsabile, Am: Amministratore, An: Analista, Pt:
		Progettista, Pr: Programmatore, Ve: Verificatore, Tot: Totale ore per
		membro; valori espressi in ore.}
\end{table}

\subsubsection{Diagramma di Gantt}

\begin{ganttchart}[
		x unit=0.6cm, % Adjust the width of each day
		y unit chart=0.6cm,
		bar/.style={fill=blue!50},
		bar height=0.5,
		time slot format=isodate,
		time slot unit=day,
		vgrid,
		today=2023-12-04,
		today rule/.style={draw=red, ultra thick}
	]{2023-12-04}{2023-12-15}
	\gantttitlecalendar{day} \\
	\ganttbar{Analisi dei requisiti}{2023-12-04}{2023-12-15} \\
	\ganttbar{Piano di qualifica}{2023-12-04}{2023-12-9} \\
	\ganttbar{Piano di progetto}{2023-12-04}{2023-12-06} \\
	\ganttbar{Database}{2023-12-08}{2023-12-9} \\
	\ganttbar{Analisi delle tecnologie}{2023-12-11}{2023-12-15}
\end{ganttchart}

Dove:
\begin{itemize}
	\item \textbf{"Analisi dei requisiti"}: questa \textit{issue} è svolta da
	      Davide Maffei, Giacomo Gualato, Matteo Bando e Niccolò Carlesso;

	\item \textbf{"Piano di qualifica"}: questa \textit{issue} è svolta da
	      Alessandro Tigani Sava;

	\item \textbf{"Piano di progetto"}: questa \textit{issue} è svolta da Carlo
	      Rosso;

	\item \textbf{"Database"}: questa \textit{issue} è svolta da Carlo Rosso;

	\item \textbf{"Analisi delle tecnologie"}: questa \textit{issue} è svolta da
	      Alessandro Tigani Sava e Carlo Rosso.
\end{itemize}
