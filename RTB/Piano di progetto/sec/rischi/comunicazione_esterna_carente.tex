\riskcom{Comunicazione esterna carente}
\label{risk:comunicazione esterna carente}
\begin{itemize}
	\item \textbf{Descrizione}:
	      Le comunicazioni con il proponente o con il committente non sono
	      efficaci ed efficienti, causando riunioni esterne più lunghe del
	      previsto e rallentando le attività; oppure rallentando le attività
	      del gruppo a causa di risposte tardive o mancanti.

	\item \textbf{Identificazione}:
	      \begin{itemize}
		      \item \textbf{Dubbi ripetuti}: durante le riunioni esterne, i
		            membri del gruppo possono porre domande già presentate in
		            precedenza;

		      \item \textbf{Riunioni esterne lunghe}: le riunioni esterne
		            possono protrarsi oltre il tempo previsto;

		      \item \textbf{Risposte tardive o mancanti}: il proponente o il
		            committente può rispondere in ritardo o non rispondere
		            affatto alle comunicazioni del gruppo.
	      \end{itemize}

	\item \textbf{Mitigazione}:
	      \begin{itemize}
		      \item \textbf{Ordine del giorno}: il responsabile si impegna a
		            stilare l'ordine del giorno delle riunioni esterne, per
		            tempo, ne discute la struttura con il gruppo e lo condivide
		            con il proponente e con il committente in anticipo;

		      \item \textbf{SAL}: il gruppo si impegna a mantenere il
		            proponente aggiornato sullo stato di avanzamento del
		            progetto, in modo da ridurre la durata delle riunioni
		            esterne e migliorare la qualità del supporto del proponente;

		      \item \textbf{Retrospettive}: sono previste delle retrospettive
		            all'interno dei SAL con il proponente, durante le quali, si
		            discute la qualità delle comunicazioni e si pensa a
		            soluzioni \textit{ad hoc} per migliorare la comunicazione
		            esterna;

		      \item \textbf{Comunicazioni frequenti}: il proponente viene tenuto
		            aggiornato frequentemente sullo stato di avanzamento del
		            progetto mediante gli appositi canali di comunicazione:
		            \textit{Telegram} e \textit{email};

		      \item \textbf{Diario di bordo}: il gruppo si impegna a tenere
		            dei diari di
		            bordo, quando richiesti dal committente, per aggiornarlo
		            sullo stato di avanzamento del progetto;

		      \item \textbf{\textit{Meeting} supplementari}: se il gruppo
		            manifesta dei dubbi o delle incertezze, può richiedere dei
		            \textit{meeting} supplementare con il proponente o con il
		            committente;

		      \item \textbf{Documentazione}: il responsabile si impegna ad
		            aggiornare la documentazione inerente agli argomenti
		            trattati durante le riunioni esterne, per dare modo ai
		            membri del gruppo di consultarla in caso di dubbi o
		            incertezze.
	      \end{itemize}
\end{itemize}
