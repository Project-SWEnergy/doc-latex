\riskcom{Comunicazione interna carente}
\label{risk:comunicazione interna carente}
\begin{itemize}
	\item \textbf{Descrizione}:
	      La comunicazione interna non è efficace ed efficiente, causando riunioni
	      interne più lunghe del previsto e rallentando le attività.
	\item \textbf{Identificazione}:
	      \begin{itemize}
		      \item \textbf{Dubbi ripetuti}: durante le riunioni interne, i
		            membri del gruppo possono porre domande già presentate in
		            precedenza;

		      \item \textbf{Riunioni interne lunghe}: le riunioni interne
		            possono protrarsi oltre il tempo previsto;

		      \item \textbf{Fraintendimenti frequenti}: i membri del gruppo
		            possono fraintendersi frequentemente.

	      \end{itemize}
	\item \textbf{Mitigazione}:
	      \begin{itemize}
		      \item \textbf{Documentazione}: il gruppo stila una documentazione
		            adeguata per facilitare la comunicazione interna. A seconda
		            dell'argomento la documentazione può avere diverse forme;

		      \item \textbf{\textit{Meeting} frequenti}: il gruppo si impegna a
		            tenere riunioni interne frequenti, in modo da ridurre la
		            durata delle riunioni interne e facilitare la comunicazione
		            interna;

		      \item \textbf{Ordine del giorno}: ciascuna riunione deve avere
		            l'ordine del
		            giorno ben definito, per discutere di tutti gli argomenti
		            utili allo sviluppo del progetto e per definire la durata di
		            ciascuno dei punti dell'ordine del giorno;

		      \item \textbf{Retrospettiva}: durante la retrospettiva, il gruppo
		            deve pensare a soluzioni \textit{ad hoc} per migliorare la
		            comunicazione interna.
	      \end{itemize}
\end{itemize}
