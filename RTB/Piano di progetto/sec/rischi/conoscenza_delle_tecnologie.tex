\risktech{Conoscenza delle tecnologie carente}
\label{risk:conoscenza tecnologie carente}
\begin{itemize}
	\item \textbf{Descrizione}:
	      nello sviluppo del progetto, si può incorrere nella situazione in cui
	      almeno qualche membro non conosce almeno una tecnologia adottata dal
	      gruppo e necessaria per lo sviluppo del progetto.

	\item \textbf{Identificazione}: il \textit{team} ha individuato le
	      tecnologie conosciute dal gruppo. Con il proponente sono state
	      discusse e concordate le tecnologie da utilizzare per lo sviluppo del
	      progetto. In questo modo, sono state individuate le tecnologie
	      non conosciute dal gruppo.

	\item \textbf{Mitigazione}:
	      \begin{itemize}
		      \item \textbf{\textit{Workshop} interni}: si rimanda alla
		            sotto-sezione "Organizzare un \textit{workshop}" del
		            documento "Norme di progetto" sotto il ruolo di progettista;

		      \item \textbf{Seminari con il proponente}: il \textit{team}
		            partecipa a
		            dei seminari organizzati con il proponente, per approfondire
		            le tecnologie non conosciute dal gruppo. Il proponente
		            spiegherà le tecnologie e fornirà degli esempi di codice
		            per illustrarne l'utilizzo;

		      \item \textbf{Dialogo con il proponente}: il \textit{team} può
		            contattare il proponente per chiedere chiarimenti sulle
		            tecnologie non conosciute dal gruppo;

		      \item \textbf{\textit{Code review}}: si rimanda alla sotto-sezione
		            "Revisione del codice" del documento "Norme di progetto"
		            sotto il ruolo di verificatore;

		      \item \textbf{Divisione del \textit{front-end} e del
			            \textit{back-end}}: il \textit{team} si divide in due
		            sottogruppi, uno che si occupa del \textit{front-end} e
		            l'altro del \textit{back-end}. In questo modo, si diminuisce
		            l'\textit{overhead} di comunicazione e di cambio di
		            contesto. I due gruppi si scambiano i ruoli al
		            termine della prima fase del progetto: RTB.
	      \end{itemize}
\end{itemize}
