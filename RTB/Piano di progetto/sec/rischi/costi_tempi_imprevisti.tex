\riskplan{Costi e tempi imprevisti}
\label{risk:costi e tempi imprevisti}
\begin{itemize}
	\item \textbf{Descrizione}:
	      Durante lo sviluppo del progetto, si può incorrere in costi o
	      rallentamenti imprevisti. Si tratta, a tutti gli effetti, di arginare
	      il danno prodotto da un rischio che si è verificato.
	\item \textbf{Identificazione}:
	      \begin{itemize}
		      \item cambiamenti significativi nelle tempistiche di completamento
		            del progetto;

		      \item variazioni notevoli dei costi di realizzazione.

		      \item monitoraggio costante: monitoraggio dei costi e delle
		            tempistiche al completamento delle \textit{milestone} e
		            ad ogni \textit{stand-up}.
	      \end{itemize}
	\item \textbf{Mitigazione}:
	      \begin{itemize}
		      \item \textit{buffer} di tempo: il \textit{team} ha
		            preventivamente inserito dei \textit{buffer} di tempo
		            tra le varie attività, per poter gestire eventuali
		            ritardi;

		      \item \textit{buffer} di costi: il \textit{team} ha
		            preventivamente inserito dei \textit{buffer} di costi
		            tra le varie attività, per poter gestire eventuali
		            costi imprevisti;

		      \item pianificazione in itinere: il \textit{team} si adatta
		            alle variazioni dei costi e delle tempistiche di
		            completamento, per poter gestire eventuali costi
		            imprevisti. In questo caso, sono aggiornate le scadenze
		            nel \textit{project} su \textit{GitHub} e i costi.
		            A seconda della situazione, le \textit{issue} sono
		            riassegnate e le \textit{milestone} sono adattate allo
		            \textit{status quo}.
	      \end{itemize}
\end{itemize}
