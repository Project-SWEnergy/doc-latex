\riskplan{Costi e tempi imprevisti}
\label{risk:costi e tempi imprevisti}
\begin{itemize}
	\item \textbf{Descrizione}:
	      Durante lo sviluppo del progetto, si può incorrere in costi o
	      rallentamenti imprevisti. Si tratta, a tutti gli effetti, di arginare
	      il danno prodotto da un rischio che si è verificato.
	\item \textbf{Identificazione}:
	      \begin{itemize}
		      \item \textbf{Monitoraggio costante}: si effettua un monitoraggio 
			  		continuo dei costi e delle tempistiche al raggiungimento delle 
					\textit{milestone} e durante gli \textit{stand-up}.
			  
	      \end{itemize}
	\item \textbf{Mitigazione}:
	      \begin{itemize}
		      \item \textbf{\textit{Buffer} di tempo}: Il team ha proattivamente 
			  		inserito margini temporali tra le diverse attività, creando 
					\textit{buffer} di tempo che consentono di gestire eventuali 
					ritardi senza compromettere la pianificazione principale;

		      \item \textbf{\textit{Buffer} di costi}: il \textit{team} ha 
			  		preventivamente allocato risorse finanziarie extra, 
					sotto forma di \textit{buffer} di costi, per far fronte 
					a spese impreviste e mantenere il controllo del \textit{budget};

		      \item \textbf{Pianificazione in itinere}: il \textit{team} si adatta
		            alle variazioni dei costi e delle tempistiche di
		            completamento, per poter gestire eventuali costi
		            imprevisti. In questo caso, sono aggiornate le scadenze
		            nel \textit{project} su \textit{GitHub}\g e i costi.
		            A seconda della situazione, le \textit{issue} sono
		            riassegnate e le \textit{milestone} sono adattate allo
		            \textit{status quo}.
	      \end{itemize}
\end{itemize}
