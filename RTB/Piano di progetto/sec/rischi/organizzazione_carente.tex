\riskplan{Organizzazione carente}
\label{risk:organizzazione carente}
\begin{itemize}
	\item \textbf{Descrizione}:
	      Il gruppo, oppure qualche membro, potrebbe non essere in grado di
	      svolgere le proprie attività, oppure potrebbe riscontrare delle
	      difficoltà a causa di una cattiva organizzazione.
	\item \textbf{Identificazione}:
	      \begin{itemize}
		      \item \textbf{Membri confusi}: i membri del gruppo non sanno quali
		            sono i compiti a loro assegnati, oppure non sanno come
		            svolgerli;

		      \item \textbf{Carenza di risorse}: sono stati assegnati più
		            incarichi di quelli sostenibili con le risorse disponibili;

		      \item \textbf{Scadenze non aggiornate}: il gruppo o qualche suo
		            membro non è in grado di rispettare le scadenze e queste non
		            sono aggiornate. Si tratta di un modo molto semplice, per
		            ricadere nel sintomo individuato precedentemente.
	      \end{itemize}

	\item \textbf{Mitigazione}:
	      \begin{itemize}
		      \item \textbf{Pianificazione delle \textit{issue}\g}: si rimanda
		            alla sotto-sezione "Pianificazione delle attività" del
		            documento "Norme di progetto" sotto il ruolo di
		            responsabile;

		      \item \textbf{Aggiornamento delle \textit{issue}}:
		            ciascun componente di SWEnergy deve aggiornare le
		            \textit{issue} a cui è assegnato, in modo da tenere il
		            responsabile e l'intera organizzazione aggiornati sullo
		            stato di avanzamento dei compiti; inoltre, deve aggiungere
		            delle \textit{issue} se ritiene che ci siano delle attività
		            da svolgere;

		      \item \textbf{Persona di riferimento}: in caso di dubbi, i
		            membri di SWEnergy possono rivolgersi
		            al responsabile, che si occuperà di chiarire la situazione,
		            o di indirizzare il membro verso chi può aiutarlo;

		      \item \textbf{Retrospettiva}: durante le retrospettive, il gruppo
		            discute di eventuali problemi organizzativi e cerca di
		            trovare soluzioni per migliorare la pianificazione;

		      \item \textbf{Dialogo con il proponente}: sono chiesti consigli al
		            proponente in merito, poiché ha più esperienza nel settore e 
					può collaborare con figure manageriali.
	      \end{itemize}
\end{itemize}
