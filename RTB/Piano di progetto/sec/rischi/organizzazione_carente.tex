\riskplan{Organizzazione carente}
\label{risk:organizzazione carente}
\begin{itemize}
	\item \textbf{Descrizione}:
	      Il gruppo, oppure qualche membro, potrebbe non essere in grado di
	      svolgere le proprie attività, oppure potrebbe riscontrare delle
	      difficoltà a causa di una cattiva organizzazione.
	\item \textbf{Identificazione}:
	      \begin{itemize}
		      \item membri confusi: i membri del gruppo non sanno quali sono i
		            compiti a loro assegnati, oppure non sanno come svolgerli;

		      \item carenza di risorse: sono stati assegnati più incarichi di
		            quelli sostenibili con le risorse disponibili.

		      \item scadenze non aggiornate: il gruppo o qualche suo membro non
		            è in grado di rispettare le scadenze e non sono aggiornate.
		            Si tratta di un modo molto semplice, per ricadere nel
		            sintomo individuato precedentemente.
	      \end{itemize}
	\item \textbf{Mitigazione}:
	      \begin{itemize}
		      \item il responsabile mantiene aggiornate le \textit{issue}.
		            Ciascuna \textit{issue} deve essere ben documentata nella
		            propria descrizione; se è il caso, si possono aggiungere i
		            riferimenti a della documentazione supplementare. Maggiori
		            informazioni sono presenti nel documento "\textit{Way of
			            working}";

		      \item ciasun componente di SWEnergy deve aggiornare le
		            \textit{issue} a cui è assegnato, in modo da tenere il
		            responsabile e l'intera organizzazione aggiornati sullo
		            stato di avanzamento dei compiti; inoltre, deve aggiungere
		            delle \textit{issue} se ritiene che ci siano delle attività
		            da svolgere. Maggiori informazioni sono presenti nel
		            documento "\textit{Way of working}";

		      \item in caso di dubbi, i membri di SWEnergy possono rivolgersi
		            al responsabile, che si occuperà di chiarire la situazione,
		            o di indirizzare il membro verso chi può aiutarlo;

		      \item il responsabile mantiene aggiornato il \textit{project} su
		            \textit{GitHub}, in particolare per quanto riguarda
		            l'assegnamento delle \textit{issue} e delle scadenze.
		            Maggiori informazioni sono presenti nel documento
		            "\textit{Way of working}";

		      \item durante le retrospettive, il gruppo discute di eventuali
		            problemi organizzativi e cerca di trovare soluzioni per
		            migliorare la pianificazione;

		      \item dialogo con il proponente: sono chiesti consigli al
		            proponente in merito, in quanto ha più esperienza
		            nel settore e ha modo di collaborare con figure manageriali.
	      \end{itemize}
\end{itemize}
