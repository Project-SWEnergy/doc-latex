\risktech{Strumenti \textit{software} inadeguati}
\label{risk:strumenti software inadeguati}
\begin{itemize}
	\item \textbf{Descrizione}: l'utilizzo di strumenti \textit{software} datati o poco
	      efficienti potrebbe causare ritardi nello sviluppo del progetto;

	\item \textbf{Identificazione}:
	      \begin{itemize}
		      \item Durante le riunioni interne, è cruciale prestare attenzione 
			  		ai \textit{feedback}\g dei membri del gruppo che potrebbero esprimere 
					preoccupazioni sull'efficienza o l'adeguatezza degli strumenti \textit{software} utilizzati;

		      \item I membri del gruppo potrebbero segnalare procedure troppo lunghe o 
			  		che possono essere facilmente automatizzate;

		      \item I membri del gruppo potrebbero segnalare procedure troppo lunghe o 
			  		che possono essere facilmente automatizzate. 
					Questo tipo di \textit{feedback} può indicare che gli strumenti attuali 
					potrebbero non essere ottimali per il processo di sviluppo.
	      \end{itemize}

	\item \textbf{Mitigazione}:
	      \begin{itemize}

			\item \textbf{Controllo delle Versioni da Parte dell'Amministratore}: 
				L'amministratore del progetto deve monitorare attentamente le versioni 
				degli strumenti \textit{software} utilizzati per assicurare che siano aggiornate e efficienti.
				
			\item \textbf{Informazione da parte dei membri del gruppo}: 
				I membri del gruppo devono essere proattivi nell'informarsi su nuove tecnologie e 
				strumenti \textit{software} che potrebbero migliorare l'efficienza del processo di sviluppo.

			\item \textbf{Automazione}: 
				I membri del gruppo analizzano e controllano se le procedure utilizzate siano 
				automatizzabili per migliorare l'efficienza.
	      \end{itemize}
\end{itemize}
