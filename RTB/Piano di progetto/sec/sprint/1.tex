\subsection{Sprint - 1}
\textbf{Inizio}: 04-12-2023 \\
\textbf{Fine}: 15-12-2023

\subsubsection{Gestione dei rischi}
\textbf{Rischi attesi verificati}:

\begin{itemize}
	\item RC-4 Mancanza di chiarezza nei ruoli e responsabilità
	\begin{itemize}
		\item \textbf{Esito mitigazione}: L'inesperienza dei componenti del gruppo nel lavorare ad un progetto di tale portata con altre persone 
		e la mancanza di chiarezza riguardo ai ruoli e alle responsabilità, ha generato confusione, conflitti e ritardi nelle attività. Di conseguenza 
		si è deciso di effettuare una stesura e aggiornamento costante di una chiara matrice dei ruoli e responsabilità. L'azione di mitigazione adottata si è dimostrata efficace.

		\item \textbf{Impatto}: Nessuna conseguenza significativa è stata riscontrata in quanto le misure di mitigazione necessarie sono state tempestivamente implementate
		ed i componenti del gruppo si sono riuniti al più presto per poter risolvere la problematica.

	\end{itemize}
	\item RP-2 Comprensione dei requisiti carente
	\begin{itemize}
		\item \textbf{Esito mitigazione}: A causa della loro inesperienza, i membri del gruppo hanno avuto dei dubbi in merito ai requisiti del progetto siccome 
	 	non sapevano con esattezza quanto in profondità dovessero essere analizzati. L'azione di mitigazione adottata si è dimostrata efficace.
		\item \textbf{Impatto}: Non sono emerse conseguenze significative. Conformemente al processo di mitigazione si è deciso di svolgere un dibattito interno tra i vari membri di SWEnergy
		e di effettuare un dialogo con il proponente per poter chiarire i dubbi e le incertezze emerse.
	\end{itemize}		
\end{itemize}  

\subsubsection{Diagramma di Gantt}

\begin{ganttchart}[
		x unit=0.6cm, % Adjust the width of each day
		y unit chart=0.6cm,
		bar/.style={fill=blue!50},
		bar height=0.5,
		time slot format=isodate,
		time slot unit=day,
		vgrid,
		today=2023-12-04,
		today rule/.style={draw=red, ultra thick}
	]{2023-12-04}{2023-12-15}
	\gantttitlecalendar{day} \\
	\ganttbar{Analisi dei requisiti}{2023-12-04}{2023-12-15} \\
	\ganttbar{Piano di progetto}{2023-12-04}{2023-12-15} \\
	\ganttbar{Piano di qualifica}{2023-12-04}{2023-12-9} \\
	\ganttbar{Database}{2023-12-04}{2023-12-9} \\
	\ganttbar{Analisi delle tecnologie}{2023-12-11}{2023-12-15}
\end{ganttchart}

Dove:
\begin{itemize}
	\item \textbf{"Analisi dei requisiti"}: questa \textit{issue} è eseguita da
	      Davide Maffei, Niccolò Carlesso e Matteo Bando. Per
	      svolgere questa attività, il gruppo ha deciso di dedicare 30 ore;

	\item \textbf{"Piano di progetto"}: questa \textit{issue} è eseguita da
	      Carlo Rosso. Per svolgere questa attività, il gruppo ha deciso
	      di dedicare 10 ore;

	\item \textbf{"Piano di qualifica"}: questa \textit{issue} è eseguita da
	      Alessandro Tigani Sava. Per svolgere questa attività, il gruppo ha
	      deciso di dedicare 5 ore;

	\item \textbf{"Database"}: questa \textit{issue} è eseguita da Giacomo
	      Gualato. Per svolgere questa attività, il gruppo ha deciso di
	      dedicare 5 ore;

	\item \textbf{"Analisi delle tecnologie"}: questa \textit{issue} è eseguita
	      da Alessandro Tigani Sava e Giacomo Gualato. Per svolgere questa
	      attività, il gruppo ha deciso di dedicare 10 ore.
\end{itemize}

\subsubsection{Preventivo}

\begin{table}[H]
	\centering
	\begin{tabular}{l|r|r|r|r|r|r|r}
		\textbf{Nome}         & \textbf{Re} & \textbf{Am} & \textbf{An} & \textbf{Pt} & \textbf{Pr} & \textbf{Ve} & \textbf{Totale} \\
		\hline
		Alessandro            & 5           & -           & -           & 5           & -           & -           & 10              \\
		Carlo                 & 5           & 5           & -           & -           & -           & -           & 10              \\
		Davide                & -           & -           & 10          & -           & -           & -           & 10              \\
		Giacomo               & -           & -           & 10          & -           & -           & -           & 10              \\
		Matteo                & -           & -           & 10          & -           & -           & -           & 10              \\
		Niccolò               & -           & -           & 10          & -           & -           & -           & 10              \\
		\hline
		\textbf{Ore totali}   & 10          & 5           & 40          & 5           & -           & -           & 60              \\
		\textbf{Costo totale} & 300         & 100         & 1000        & 125         & -           & -           & 1525
	\end{tabular}
	\caption{Re: Responsabile, Am: Amministratore, An: Analista, Pt: Progettista,
		Pr: Programmatore, Ve: Verificatore, Totale: Totale per persona; valori espressi in ore; Costo totale espresso in euro.}
\end{table}

\subsubsection{Riassunto delle attività svolte}

\begin{enumerate}
	\item \textbf{Verbale esterno}: stesura e verifica del verbale esterno del
	      1/12/2023;

	\item \textbf{Verbale interno}: stesura e verifica del verbale interno del
	      3/12/2023;

	\item \textbf{Piano di progetto}: stesura e verifica della prima bozza del
	      piano di progetto;

	\item \textbf{Analisi dei requisiti}: stesura e verifica della prima bozza
	      dell'analisi dei requisiti;

	\item \textbf{\textit{Template}}: aggiornamento e riorganizzazione dei
	      \textit{template} LaTeX per la documentazione;

	\item \textbf{\textit{Build} automatizzata della documentazione}:
	      corretto lo \textit{script} di \textit{build} della documentazione per
	      automatizzare la compilazione dei documenti;

	\item \textbf{Automatizzazione del glossario}: creato uno \textit{script} per
	      automatizzare l'individuazione delle parole del Glossario nei
	      documenti;

	\item \textbf{Piano di qualifica}: stesura dell'introduzione del Piano
	      di Qualifica;

	\item \textbf{Analisi delle tecnologie}: PoC\g containerizzato in
	      \textit{Docker} del \textit{database} \textit{PostgreSQL} e dei
	      \textit{framework} \textit{Nest.js} e \textit{Drizzle}.
\end{enumerate}

\subsubsection{Consuntivo}

\begin{table}[H]
	\centering
	\begin{tabular}{l|r|r|r|r|r|r|r}
		\textbf{Nome}         & \textbf{Re} & \textbf{Am} & \textbf{An} & \textbf{Pt} & \textbf{Pr} & \textbf{Ve} & \textbf{Totale} \\
		\hline
		Alessandro            & 5           & -           & -           & 5           & -           & -           & 10              \\
		Carlo                 & 5           & 5           & -           & -           & -           & -           & 10              \\
		Davide                & -           & -           & 10          & -           & -           & -           & 10              \\
		Giacomo               & -           & -           & -           & 10          & -           & -           & 10              \\
		Matteo                & -           & -           & 10          & -           & -           & -           & 10              \\
		Niccolò               & -           & -           & 10          & -           & -           & -           & 10              \\
		\hline
		\textbf{Ore totali}   & 10          & 5           & 30          & 15          & -           & -           & 60              \\
		\textbf{Costo totale} & 300         & 100         & 750         & 375         & -           & -           & 1525
	\end{tabular}
	\caption{Re: Responsabile, Am: Amministratore, An: Analista, Pt: Progettista,
		Pr: Programmatore, Ve: Verificatore, Totale: Totale per persona; valori espressi in ore; Costo totale espresso in euro.}
\end{table}

\newpage
\subsubsection{Gestione dei ruoli}

\begin{figure}[h]
	\centering
	\begin{tikzpicture}
		\pie[text=legend]{
			17/Responsabile,
			8/Amministratore,
			50/Analista,
			25/Progettista,
			0/Programmatore,
			0/Verificatore
		}
	\end{tikzpicture}
	\caption{Grafico delle proporzioni dei ruoli ricoperti dai membri del gruppo}
\end{figure}


La distribuzione delle risorse nello \textit{sprint} è stata principalmente focalizzata sul ruolo di Analista, poiché il gruppo ha attribuito 
un'enorme importanza alla fase iniziale di analisi dei requisiti.
Il 25\% delle risorse è stato destinato al ruolo di Progettista, evidenziando l'orientamento verso una prima progettazione delle soluzioni 
successivamente all'analisi.
Il 17\% delle risorse è stato dedicato al ruolo di Responsabile, indicando un'importante attività di coordinamento e gestione durante lo 
\textit{sprint}.
L'8\% delle risorse è stato assegnato al ruolo di Amministratore, segnalando che le attività amministrative e di supporto sono state gestite
in misura inferiore rispetto ad altri ruoli.
Non sono state assegnate risorse alle figure di Programmatore e Verificatore in questo \textit{sprint}, poiché l'attenzione è stata prevalentemente 
concentrata sull'analisi e la progettazione, piuttosto che sulla codifica (ancora prematura durante il primo \textit{sprint}) e sulla verifica.

\subsubsection{Analisi retrospettiva}

Analizzando le ore impiegate nel primo \textit{sprint} in relazione allo stato di avanzamento del progetto, lo svolgimento del lavoro risulta positivo nel suo complesso. 
Durante lo \textit{sprint}, il gruppo ha dedicato particolare attenzione all'analisi dei requisiti, riflettendo la consapevolezza dell'importanza di una corretta definizione degli 
obiettivi del progetto fin dalle fasi iniziali.
La scelta di non assegnare risorse alle figure di Programmatore e Verificatore in questo \textit{sprint} indica la consapevolezza del gruppo riguardo alla fase prematura per 
attività di codifica e verifica. Complessivamente, il gruppo ha allocato le risorse in modo coerente con le esigenze dello \textit{sprint}, orientandosi principalmente verso la definizione 
dei requisiti e la progettazione iniziale del progetto.

\textbf{Obiettivi raggiunti}:
\begin{itemize}
	\item Definizione chiara dei ruoli e la loro conseguente rotazione.
	\item Creazione dei \textit{template} per la documentazione.
	\item Stesura delle bozze per ogni documento.
	\item Automatizzazione della compilazione della documentazione e del glossario.
\end{itemize}

\textbf{Obiettivi mancati}:
\begin{itemize}
	\item Implementazione di automazioni per il versionamento dei documenti.
\end{itemize}

\textbf{Problematiche riscontrate}:
\begin{itemize}
	\item Corretta gestione e tracciamento delle \textit{issues}.
\end{itemize}

\textbf{Soluzioni attuate}:
\begin{itemize}
	\item Utilizzo dei \textit{project} presenti su \textit{GitHub} per la gestione delle \textit{issues}.
\end{itemize}





