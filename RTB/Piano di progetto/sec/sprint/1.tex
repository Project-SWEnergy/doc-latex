\sprint{Sprint}

\subsubsection{Diagramma di Gantt}

\begin{ganttchart}[
		x unit=0.6cm, % Adjust the width of each day
		y unit chart=0.6cm,
		bar/.style={fill=blue!50},
		bar height=0.5,
		time slot format=isodate,
		time slot unit=day,
		vgrid,
		today=2023-12-04,
		today rule/.style={draw=red, ultra thick}
	]{2023-12-04}{2023-12-15}
	\gantttitlecalendar{day} \\
	\ganttbar{Analisi dei requisiti}{2023-12-04}{2023-12-15} \\
	\ganttbar{Piano di progetto}{2023-12-04}{2023-12-15} \\
	\ganttbar{Piano di qualifica}{2023-12-04}{2023-12-9} \\
	\ganttbar{Database}{2023-12-04}{2023-12-9} \\
	\ganttbar{Analisi delle tecnologie}{2023-12-11}{2023-12-15}
\end{ganttchart}

Dove:
\begin{itemize}
	\item \textbf{"Analisi dei requisiti"}: questa \textit{issue} è eseguita da
	      Davide Maffei, Niccolò Carlesso e Matteo Bando. Per
	      svolgere questa attività, il gruppo ha deciso di dedicare 30 ore;

	\item \textbf{"Piano di progetto"}: questa \textit{issue} è eseguita da
	      Carlo Rosso. Per svolgere questa attività, il gruppo ha deciso
	      di dedicare 10 ore;

	\item \textbf{"Piano di qualifica"}: questa \textit{issue} è eseguita da
	      Alessandro Tigani Sava. Per svolgere questa attività, il gruppo ha
	      deciso di dedicare 5 ore;

	\item \textbf{"Database"}: questa \textit{issue} è eseguita da Giacomo
	      Gualato. Per svolgere questa attività, il gruppo ha deciso di
	      dedicare 5 ore;

	\item \textbf{"Analisi delle tecnologie"}: questa \textit{issue} è eseguita
	      da Alessandro Tigani Sava e Giacomo Gualato. Per svolgere questa
	      attività, il gruppo ha deciso di dedicare 10 ore.
\end{itemize}

\subsubsection{Preventivo}

\begin{table}[H]
	\centering
	\begin{tabular}{l|r|r|r|r|r|r|r}
		\textbf{Nome}         & \textbf{Re} & \textbf{Am} & \textbf{An} & \textbf{Pt} & \textbf{Pr} & \textbf{Ve} & \textbf{Totale} \\
		\hline
		Alessandro            & 5           & -           & -           & 5           & -           & -           & 10              \\
		Carlo                 & 5           & 5           & -           & -           & -           & -           & 10              \\
		Davide                & -           & -           & 10          & -           & -           & -           & 10              \\
		Giacomo               & -           & -           & 10          & -           & -           & -           & 10              \\
		Matteo                & -           & -           & 10          & -           & -           & -           & 10              \\
		Niccolò               & -           & -           & 10          & -           & -           & -           & 10              \\
		\hline
		\textbf{Ore totali}   & 10          & 5           & 40          & 5           & -           & -           & 60              \\
		\textbf{Costo totale} & 300         & 100         & 1000        & 125         & -           & -           & 1525
	\end{tabular}
	\caption{Re: Responsabile, Am: Amministratore, An: Analista, Pt: Progettista,
		Pr: Programmatore, Ve: Verificatore, Totale: Totale per persona; valori espressi in ore; Costo totale espresso in euro.}
\end{table}

\subsubsection{Riassunto delle attività svolte}

\begin{enumerate}
	\item \textbf{Verbale esterno}: stesura e verifica del verbale esterno del
	      1/12/2023;

	\item \textbf{Verbale interno}: stesura e verifica del verbale interno del
	      3/12/2023;

	\item \textbf{Piano di progetto}: stesura e verifica della prima bozza del
	      piano di progetto;

	\item \textbf{Analisi dei requisiti}: stesura e verifica della prima bozza
	      dell'analisi dei requisiti;

	\item \textbf{\textit{Template}}: aggiornamento e riorganizzazione dei
	      \textit{template} \LaTeX{} per la documentazione;

	\item \textbf{\textit{Build} automatizzata della documentazione}:
	      corretto lo script di \textit{build} della documentazione per
	      automatizzare la compilazione dei documenti;

	\item \textbf{Automatizzazione del glossario}: creato uno script per
	      automatizzare l'individuazione delle parole del glossario nei
	      documenti;

	\item \textbf{Piano di qualifica}: stesura dell'introduzione del piano
	      di qualifica;

	\item \textbf{Analisi delle tecnologie}: PoC containerizzato in
	      \textit{Docker} del \textit{database} \textit{PostgreSQL} e dei
	      \textit{framework} \textit{Nest.js} e \textit{Drizzle}.
\end{enumerate}

\subsubsection{Consuntivo}

\begin{table}[H]
	\centering
	\begin{tabular}{l|r|r|r|r|r|r|r}
		\textbf{Nome}         & \textbf{Re} & \textbf{Am} & \textbf{An} & \textbf{Pt} & \textbf{Pr} & \textbf{Ve} & \textbf{Totale} \\
		\hline
		Alessandro            & 5           & -           & -           & 5           & -           & -           & 10              \\
		Carlo                 & 5           & 5           & -           & -           & -           & -           & 10              \\
		Davide                & -           & -           & 10          & -           & -           & -           & 10              \\
		Giacomo               & -           & -           & -           & 10          & -           & -           & 10              \\
		Matteo                & -           & -           & 10          & -           & -           & -           & 10              \\
		Niccolò               & -           & -           & 10          & -           & -           & -           & 10              \\
		\hline
		\textbf{Ore totali}   & 10          & 5           & 30          & 15          & -           & -           & 60              \\
		\textbf{Costo totale} & 300         & 100         & 750         & 375         & -           & -           & 1525
	\end{tabular}
	\caption{Re: Responsabile, Am: Amministratore, An: Analista, Pt: Progettista,
		Pr: Programmatore, Ve: Verificatore, Totale: Totale per persona; valori espressi in ore; Costo totale espresso in euro.}
\end{table}

\subsubsection{Gestione dei ruoli}

\begin{figure}[h]
	\centering
	\begin{tikzpicture}
		\pie[text=legend]{
			17/Responsabile,
			8/Amministratore,
			50/Analista,
			25/Progettista,
			0/Programmatore,
			0/Verificatore
		}
	\end{tikzpicture}
	\caption{Grafico delle proporzioni dei ruoli ricoperti dai membri del gruppo}
\end{figure}
