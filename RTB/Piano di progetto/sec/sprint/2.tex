\sprint{Sprint}

\subsubsection{Diagramma di Gantt}

\begin{ganttchart}[
		x unit=0.6cm, % Adjust the width of each day
		y unit chart=0.6cm,
		bar/.style={fill=blue!50},
		bar height=0.5,
		time slot format=isodate,
		time slot unit=day,
		vgrid,
		today=2023-12-16,
		today rule/.style={draw=red, ultra thick}
	]{2023-12-16}{2023-12-29}
	\gantttitlecalendar{day} \\
	\ganttbar{Analisi dei requisiti}{2023-12-16}{2023-12-29} \\
	\ganttbar{Piano di qualifica}{2023-12-16}{2023-12-19} \\
	\ganttbar{Piano di progetto}{2023-12-16}{2023-12-19} \\
	\ganttbar{PoC \textit{front-end}}{2023-12-21}{2023-12-29} \\
	\ganttbar{PoC \textit{back-end}}{2023-12-24}{2023-12-29}
\end{ganttchart}

Dove:
\begin{itemize}
	\item \textbf{"Analisi dei requisiti"}: questa \textit{issue} è eseguita da
	      Davide Maffei. Per svolgere questa attività, il gruppo ha deciso di
	      dedicare 5 ore;

	\item \textbf{"Piano di progetto"}: questa \textit{issue} è eseguita da
	      Giacomo Gualato. Per svolgere questa attività, il gruppo ha deciso
	      di dedicare 5 ore;

	\item \textbf{"PoC \textit{front-end}"}: questa \textit{issue} è eseguita
	      da Alessandro Tigani Sava e Giacomo Gualato. Per svolgere questa
	      attività, il gruppo ha deciso di dedicare 20 ore.

	\item \textbf{"PoC \textit{back-end}"}: questa \textit{issue} è eseguita
	      da Carlo Rosso. Per svolgere questa attività, il gruppo ha deciso di
	      dedicare 10 ore.

	\item \textbf{Verifica dei documenti}: questo compito è eseguito da
	      Niccolò Carlesso. Per svolgere questa attività, il gruppo ha deciso
	      di dedicare 10 ore.
\end{itemize}

\subsubsection{Preventivo}

\begin{table}[H]
	\centering
	\begin{tabular}{l|r|r|r|r|r|r|r}
		\textbf{Nome}         & \textbf{Re} & \textbf{Am} & \textbf{An} & \textbf{Pt} & \textbf{Pr} & \textbf{Ve} & \textbf{Totale} \\
		\hline
		Alessandro            & -           & -           & -           & 10          & -           & -           & 10              \\
		Carlo                 & -           & -           & -           & 10          & -           & -           & 10              \\
		Davide                & -           & -           & 10          & -           & -           & -           & 10              \\
		Giacomo               & 5           & -           & -           & -           & -           & -           & 5               \\
		Matteo                & -           & -           & -           & -           & 10          & -           & 10              \\
		Niccolò               & -           & -           & -           & -           & -           & 10          & 10              \\
		\hline
		\textbf{Ore totali}   & 5           & -           & 10          & 20          & 10          & 10          & 55              \\
		\textbf{Costo totale} & 150         & -           & 250         & 500         & 150         & 150         & 1200
	\end{tabular}
	\caption{Re: Responsabile, Am: Amministratore, An: Analista, Pt: Progettista,
		Pr: Programmatore, Ve: Verificatore, Totale: Totale per persona; valori espressi in ore; Costo totale espresso in euro.}
\end{table}

\subsubsection{Riassunto delle attività svolte}

\begin{itemize}
	\item \textbf{Verbale interno}: stesura e verifica del verbale interno del
	      17/12/2023;

	\item \textbf{Analisi dei requisiti}: ristrutturazione del documento con
	      divisione dei casi d'uso per attore;

	\item \textbf{Piano di progetto}: aggiornamento del documento;

	\item \textbf{Piano di qualifica}: stesura della bozza, con individuazione
	      dei controlli di qualità e dei test da effettuare;

	\item \textbf{Fix dei \textit{template}};

	\item \textbf{PoC \textit{front-end}}: realizzazione di un prototipo
	      dell'interfaccia grafica del prodotto in \textit{Figma}.
\end{itemize}

\subsubsection{Consuntivo}

\begin{table}[H]
	\centering
	\begin{tabular}{l|r|r|r|r|r|r|r}
		\textbf{Nome}         & \textbf{Re} & \textbf{Am} & \textbf{An} & \textbf{Pt} & \textbf{Pr} & \textbf{Ve} & \textbf{Totale} \\
		\hline
		Alessandro            & -           & 5           & -           & 5           & -           & -           & 10              \\
		Carlo                 & 5           & 5           & -           & -           & -           & -           & 10              \\
		Davide                & -           & -           & 10          & -           & -           & -           & 10              \\
		Giacomo               & 5           & -           & -           & -           & -           & 5           & 10              \\
		Matteo                & -           & -           & -           & 5           & 5           & -           & 10              \\
		Niccolò               & -           & -           & -           & -           & -           & 10          & 10              \\
		\hline
		\textbf{Ore totali}   & 10          & 10          & 10          & 10          & 5           & 15          & 60              \\
		\textbf{Costo totale} & 300         & 200         & 250         & 250         & 75          & 225         & 1300
	\end{tabular}
	\caption{Re: Responsabile, Am: Amministratore, An: Analista, Pt: Progettista,
		Pr: Programmatore, Ve: Verificatore, Totale: Totale per persona; valori espressi in ore; Costo totale espresso in euro.}
\end{table}

\subsection{Gestione dei ruoli}

\begin{figure}[h]
	\centering
	\begin{tikzpicture}
		\pie[text=legend]{
			19/Responsabile,
			16/Amministratore,
			16/Analista,
			16/Progettista,
			8/Programmatore,
			25/Verificatore
		}
	\end{tikzpicture}
	\caption{Grafico delle proporzioni dei ruoli ricoperti dai membri del gruppo}
\end{figure}
