\sprint{Sprint}

\subsubsection{Diagramma di Gantt}

\begin{ganttchart}[
		x unit=0.6cm, % Adjust the width of each day
		y unit chart=0.6cm,
		bar/.style={fill=blue!50},
		bar height=0.5,
		time slot format=isodate,
		time slot unit=day,
		vgrid,
		today=2024-01-8,
		today rule/.style={draw=red, ultra thick}
	]{2023-12-30}{2024-01-12}
	\gantttitlecalendar{day} \\
	\ganttbar{Analisi dei requisiti}{2023-12-30}{2024-01-05} \\
	\ganttbar{Piano di qualifica}{2023-12-30}{2024-01-05} \\
	\ganttbar{Piano di progetto}{2024-01-07}{2024-01-12} \\
	\ganttbar{PoC \textit{front-end}}{2023-12-30}{2024-01-12} \\
	\ganttbar{PoC \textit{back-end}}{2023-12-30}{2024-01-12}
\end{ganttchart}

Dove:
\begin{itemize}
	\item \textbf{"Analisi dei requisiti"}: questa \textit{issue} è eseguita da
	      Carlo Rosso. Per svolgere questa attività, il gruppo ha deciso di
	      dedicare 5 ore;

	\item \textbf{"Piano di progetto"}: questa \textit{issue} è eseguita da
	      Giacomo Gualato. Per svolgere questa attività, il gruppo ha deciso
	      di dedicare 5 ore;

	\item \textbf{"PoC \textit{front-end}"}: questa \textit{issue} è eseguita
	      da Alessandro Tigani Sava e Matteo Bando. Per svolgere questa
	      attività, il gruppo ha deciso di dedicare 15 ore.

	\item \textbf{"PoC \textit{back-end}"}: questa \textit{issue} è eseguita
	      da Carlo Rosso, Davide Maffei, Niccolò Carlesso. Per svolgere questa
	      attività, il gruppo ha deciso di dedicare 20 ore.

	\item \textbf{Verifica dei documenti}: questo compito è eseguito da
	      Matteo Bando. Per svolgere questa attività, il gruppo ha deciso
	      di dedicare 5 ore.
\end{itemize}

\subsubsection{Preventivo}

\begin{table}[H]
	\centering
	\begin{tabular}{l|r|r|r|r|r|r|r}
		\textbf{Nome}         & \textbf{Re} & \textbf{Am} & \textbf{An} & \textbf{Pt} & \textbf{Pr} & \textbf{Ve} & \textbf{Totale} \\
		\hline
		Alessandro            & -           & -           & -           & -           & 10          & -           & 10              \\
		Carlo                 & -           & -           & 5           & -           & 5           & -           & 10              \\
		Davide                & 5           & -           & -           & -           & 5           & -           & 10              \\
		Giacomo               & -           & -           & 5           & -           & -           & -           & 5               \\
		Matteo                & -           & -           & -           & -           & -           & 10          & 10              \\
		Niccolò               & -           & -           & -           & 10          & -           & -           & 10              \\
		\hline
		\textbf{Ore totali}   & 5           & -           & 10          & 10          & 20          & 10          & 55              \\
		\textbf{Costo totale} & 150         & -           & 250         & 250         & 300         & 150         & 1100
	\end{tabular}
	\caption{Re: Responsabile, Am: Amministratore, An: Analista, Pt: Progettista,
		Pr: Programmatore, Ve: Verificatore, Totale: Totale per persona; valori espressi in ore; Costo totale espresso in euro.}
\end{table}

\subsubsection{Riassunto delle attività svolte}

\begin{itemize}
	\item \textbf{Verifica documenti}: verifica delle "Norme di progetto" e
	      "Piano di progetto";

	\item \textbf{Glossario}: aggiornamento del documento;

	\item \textbf{Norme di progetto}: aggiornamento del documento;

	\item \textbf{Analisi dei requisiti}: inserimento degli UML;

	\item \textbf{PoC front-end}: realizzazione di un PoC in TypeScript, secondo
	      i requisiti concordati con il proponente;

	\item \textbf{PoC back-end}: realizzazione di un PoC in TypeScript, secondo
	      i requisiti concordati con il proponente;
\end{itemize}

\subsubsection{Consuntivo}

\begin{table}[H]
	\centering
	\begin{tabular}{l|r|r|r|r|r|r|r}
		\textbf{Nome}         & \textbf{Re} & \textbf{Am} & \textbf{An} & \textbf{Pt} & \textbf{Pr} & \textbf{Ve} & \textbf{Totale} \\
		\hline
		Alessandro            & -           & -           & -           & -           & 10          & -           & 10              \\
		Carlo                 & -           & -           & 5           & -           & 5           & -           & 10              \\
		Davide                & 5           & -           & -           & -           & 5           & -           & 10              \\
		Giacomo               & -           & 5           & 5           & -           & -           & -           & 10              \\
		Matteo                & -           & -           & -           & -           & 5           & 5           & 10              \\
		Niccolò               & -           & -           & -           & 10          & -           & -           & 10              \\
		\hline
		\textbf{Ore totali}   & 5           & 5           & 10          & 10          & 25          & 5           & 60              \\
		\textbf{Costo totale} & 150         & 100         & 250         & 250         & 375         & 75          & 1200
	\end{tabular}
	\caption{Re: Responsabile, Am: Amministratore, An: Analista, Pt: Progettista,
		Pr: Programmatore, Ve: Verificatore, Totale: Totale per persona; valori espressi in ore; Costo totale espresso in euro.}
\end{table}

\subsection{Gestione dei ruoli}

\begin{figure}[h]
	\centering
	\begin{tikzpicture}
		\pie[text=legend]{
            11/Responsabile,
            8/Amministratore,
            16/Analista,
            16/Progettista,
            41/Programmatore,
            8/Verificatore
        }
	\end{tikzpicture}
    \caption{Grafico delle proporzioni dei ruoli ricoperti dai membri del gruppo}
\end{figure}
