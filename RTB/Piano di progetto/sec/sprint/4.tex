\sprint{Sprint}

\subsubsection{Diagramma di Gantt}

\begin{ganttchart}[
		x unit=0.6cm, % Adjust the width of each day
		y unit chart=0.6cm,
		bar/.style={fill=blue!50},
		bar height=0.5,
		time slot format=isodate,
		time slot unit=day,
		vgrid,
		today=2024-02-19,
		today rule/.style={draw=red, ultra thick}
	]{2024-02-19}{2024-03-03}
	\gantttitlecalendar{day} \\
	\ganttbar{Analisi dei requisiti}{2024-02-19}{2024-02-22} \\
	\ganttbar{Piano di progetto}{2024-02-21}{2024-02-27} \\
	\ganttbar{Piano di qualifica}{2024-02-21}{2024-02-27} \\
	\ganttbar{Glossario}{2024-02-19}{2024-02-27} \\
	\ganttbar{Presentazione RTB}{2024-02-22}{2024-02-29}
\end{ganttchart}

Dove:
\begin{itemize}
	\item \textbf{"Analisi dei requisiti"}: questa \textit{issue} è eseguita da
	      X, Y e Z. Per
	      svolgere questa attività, il gruppo ha deciso di dedicare TOT ore;

	\item \textbf{"Piano di progetto"}: questa \textit{issue} è eseguita da
	      X. Per svolgere questa attività, il gruppo ha deciso
	      di dedicare TOT ore;

	\item \textbf{"Piano di qualifica"}: questa \textit{issue} è eseguita da
	      X. Per svolgere questa attività, il gruppo ha
	      deciso di dedicare TOT ore;

	\item \textbf{"Glossario"}: questa \textit{issue} è eseguita da X. 
			Per svolgere questa attività, il gruppo ha deciso di
	      dedicare TOT ore;

	\item \textbf{"Presentazione RTB"}: questa \textit{issue} è eseguita
	      da X e Y. Per svolgere questa
	      attività, il gruppo ha deciso di dedicare TOT ore.
\end{itemize}

\subsubsection{Preventivo}

\begin{table}[H]
	\centering
	\begin{tabular}{l|r|r|r|r|r|r|r}
	\textbf{Nome} & \textbf{Re} & \textbf{Am} & \textbf{An} & \textbf{Pt} & \textbf{Pr} & \textbf{Ve} & \textbf{Totale} \\
	\hline
		Alessandro & - & - & 5 & - & - & - & 5 \\
		Carlo & - & - & 5 & - & - & - & 5 \\
		Davide & - & - & - & - & - & 5 & 5 \\
		Giacomo & - & - & - & - & - & 5 & 5 \\
		Matteo & - & 5 & - & - & - & - & 5 \\
		Niccolò & 5 & - & - & - & - & - & 5 \\
	\hline
	\textbf{Ore totali} & 5 & 5 & 10 & - & - & 10 & 30 \\
	\textbf{Costo totale} & 150 & 100 & 250 & - & - & 150 & 650
	\end{tabular}
	\caption{Re: Responsabile, Am: Amministratore, An: Analista, Pt: Progettista,
		Pr: Programmatore, Ve: Verificatore, Totale: Totale per persona; valori espressi in ore; Costo totale espresso in euro.} 
\end{table}

\subsubsection{Riassunto delle attività svolte}
ANCORA DA REDIGERE, QUESTE SOTTO SONO SBAGLIATE
\begin{enumerate}
	\item \textbf{Verbale esterno}: stesura e verifica del verbale esterno del
	      1/12/2023;

	\item \textbf{Verbale interno}: stesura e verifica del verbale interno del
	      3/12/2023;

	\item \textbf{Piano di progetto}: stesura e verifica della prima bozza del
	      piano di progetto;

	\item \textbf{Analisi dei requisiti}: stesura e verifica della prima bozza
	      dell'analisi dei requisiti;

	\item \textbf{\textit{Template}}: aggiornamento e riorganizzazione dei
	      \textit{template} LaTeX per la documentazione;

	\item \textbf{\textit{Build} automatizzata della documentazione}:
	      corretto lo script di \textit{build} della documentazione per
	      automatizzare la compilazione dei documenti;

	\item \textbf{Automatizzazione del glossario}: creato uno script per
	      automatizzare l'individuazione delle parole del glossario nei
	      documenti;

	\item \textbf{Piano di qualifica}: stesura dell'introduzione del piano
	      di qualifica;

	\item \textbf{Analisi delle tecnologie}: PoC containerizzato in
	      \textit{Docker} del \textit{database} \textit{PostgreSQL} e dei
	      \textit{framework} \textit{Nest.js} e \textit{Drizzle}.
\end{enumerate}

\subsubsection{Consuntivo}

\begin{table}[H]
	\centering
	\begin{tabular}{l|r|r|r|r|r|r|r}
	\textbf{Nome} & \textbf{Re} & \textbf{Am} & \textbf{An} & \textbf{Pt} & \textbf{Pr} & \textbf{Ve} & \textbf{Totale} \\
	\hline
	Alessandro & - & - & 5 & - & - & - & 5 \\
	Carlo & - & - & 5 & - & - & - & 5 \\
	Davide & - & - & - & - & - & 5 & 5 \\
	Giacomo & - & - & - & - & - & 5 & 5 \\
	Matteo & - & 5 & - & - & - & - & 5 \\
	Niccolò & 5 & - & - & - & - & - & 5 \\
	\hline
	\textbf{Ore totali} & 5 & 5 & 10 & - & - & 10 & 30 \\
	\textbf{Costo totale} & 150 & 100 & 250 & - & - & 150 & 650
	\end{tabular}
	\caption{Re: Responsabile, Am: Amministratore, An: Analista, Pt: Progettista,
		Pr: Programmatore, Ve: Verificatore, Totale: Totale per persona; valori espressi in ore; Costo totale espresso in euro.} 
\end{table}

\subsubsection{Gestione dei ruoli}

\begin{figure}[h]
	\centering
	\begin{tikzpicture}
		\pie[text=legend]{
            18/Responsabile,
            16/Amministratore,
            33/Analista,
            0/Progettista,
            0/Programmatore,
            33/Verificatore
        }
	\end{tikzpicture}
    \caption{Grafico delle proporzioni dei ruoli ricoperti dai membri del gruppo}
\end{figure}
