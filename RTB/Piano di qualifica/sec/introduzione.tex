\section{Introduzione}
\subsection{Scopo del documento}
Il seguente documento mira ad illustrare dettagliatamente tutte le informazioni relative al controllo di qualità del \textit{software}. 
Questo obiettivo si concretizza attraverso l'implementazione di test standardizzati che non solo permettono di valutare la funzionalità del \textit{software}, ma anche di fornire una base per il suo continuo perfezionamento. \\
Il Piano di Qualifica è concepito come un documento dinamico e incrementale, la sua evoluzione è prevista in modo graduale, specialmente per la definizione di metriche di valutazione del prodotto. 
Queste metriche sono strettamente allineate ai requisiti e alle aspettative del proponente, costituendo così un parametro affidabile per la valutazione della qualità del prodotto nel corso del suo sviluppo. \\
Nel complesso, il documento è realizzato per guidare l'adozione di processi mirati a definire metriche di misurazione di efficacia ed efficienza, fornendo misure quantitative che saranno utilizzate per valutare il progresso nel corso del progetto \textit{software}.

\subsection{Scopo del capitolato}
Il capitolato\g C3 ha come obiettivo la realizzazione di una applicazione \textit{web} per la gestione di prenotazioni e ordinazioni nei ristoranti. \\
L'utente avrà modo di effettuare le operazioni di prenotazione ed ordinazione in modo semplice, personalizzando gli ordini in base alle proprie esigenze alimentari e collaborando con altri utenti nella divisione del conto. \\
I ristoratori avranno modo di visualizzare, oltre alla prenotazione, l'eventuale il preordine effettuato dai clienti. 
In questo modo sarà loro possibile ottimizzare l'uso delle materie prime necessarie alle preparazioni. \\
L'applicativo sviluppato richiede una copertura dei \textit{test} di almeno l'$80\%$ ed una analisi dei principali servizi \textit{cloud} per individuare quello più adatto allo scopo.

\subsection{Glossario}
Al fine di chiarire possibili ambiguità relative alla terminologia utilizzata all'interno dei documenti è stato redatto il Glossario contenente i termini ritenuti di particolare rilievo.
All'interno del presente documento questi sono contrassegnati con una ”G” ad apice.

\subsection{Riferimenti}
Riferimenti normativi:
\begin{itemize}
    \item \href{https://www.math.unipd.it/~tullio/IS-1/2023/Progetto/C3.pdf}{Capitolato C3 - Easy Meal}.
\end{itemize}

\noindent
Riferimenti informativi:
\begin{itemize}
    \item \href{https://www.math.unipd.it/~tullio/IS-1/2023/Dispense/T7.pdf}{Dispense T7 - Qualità del \textit{software}}.
    \item \href{https://www.math.unipd.it/~tullio/IS-1/2023/Dispense/T8.pdf}{Dispense T8 - Qualità di processo}.
    \item \href{https://www.math.unipd.it/~tullio/IS-1/2023/Dispense/T9.pdf}{Dispense T9 - Verifica e validazione}.
    \item \href{https://it.wikipedia.org/wiki/ISO/IEC_9126}{ISO\g/ IECG\g 9126}.
    \item \href{https://www.math.unipd.it/~tullio/IS-1/2009/Approfondimenti/ISO_12207-1995.pdf}{ISO\g/ IECG\g 12207:1995}.
\end{itemize}