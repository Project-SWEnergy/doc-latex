\section{Test}
La verifica del prodotto necessita l'esecuzione di test mirati, questi verranno eseguiti contemporaneamente allo sviluppo per garantire la correttezza del software in ogni momento.
Le tipologie di test previste sono:
\begin{itemize}
	\item \textbf{Test di unità}: utilizzati per valutare le singole componenti del sistema come classi e metodi, non considerano le dipendenze tra le classi (che dovranno esistere solo in caso gli elementi debbano essere testati insieme).
	      Sono definiti durante la progettazione.
	\item \textbf{Test di integrazione}: verificano l'integrazione tra le componenti del sistema.
	\item \textbf{Test di sistema}: vengono effettuati basandosi su quanto emerso durante l'analisi dei requisiti per verificarne la corretta implementazione.
	\item \textbf{Test di accettazione}: utilizzati per verificare che il software soddisfi i requisiti presentati nel capitolato in modo da essere pronto per la consegna, svolti in presenza del committente.
\end{itemize}

\subsection{Test di sistema}
Di seguito è riportata la tabella relativa ai \textit{test} di sistema. In quest'ultima vengono fornite descrizioni
dei singoli \textit{test} e viene specificato lo stato di ciascuno, che può essere:
\begin{itemize}
	\item Implementato.
	\item Non implementato.
\end{itemize}
Ciascun \textit{test} è identificato da un codice univoco che segue la seguente nomenclatura:
\begin{itemize}
	\item TS.RFO: dove "TS" indica che si tratta di un \textit{test} di sistema, mentre "RFO" indica che è un \textit{test} di sistema relativo a un requisito funzionale obbligatorio.
	\item TS.RFF: dove "TS" indica che si tratta di un \textit{test} di sistema, mentre "RFF" indica che è un \textit{test} di sistema relativo a un requisito funzionale facoltativo.
	\item TS.RFD: dove "TS" indica che si tratta di un \textit{test} di sistema, mentre "RFD" indica che è un \textit{test} di sistema relativo a un requisito funzionale desiderabile.
\end{itemize}
La descrizione e la nomenclatura utilizzata per identificare i requisiti funzionali è spiegata nel documento Analisi dei requisiti v2.0.0.

\setlist{nolistsep}
\setlist[itemize]{after=\vspace{-\baselineskip}, left=5pt}
\fontsize{10}{12}\selectfont
\begin{longtable}{|p{0.10\linewidth}|p{0.70\linewidth}|p{0.12\linewidth}|}
	\hline
	\textbf{Codice}                                                                                                                 & \textbf{Descrizione} & \textbf{Stato} \\
	\hline
	TS.RFO1                                                                                                                         &
	L'Utente base\g e generico\g deve poter consultare l'elenco dei ristoranti disponibili. \par
	Verificare che:
	\begin{itemize}
		\item la \textit{Home} della piattaforma sia visualizzabile senza richiedere l'accesso.
		\item la \textit{Home} della piattaforma mostri l'elenco dei ristoranti disponibili.
		\item il pulsante per il ritorno alla \textit{Home} sia ben riconoscibile.
		\item sia possibile selezionare la città di interesse dalla \textit{Home}.
	\end{itemize}                                         &
	Non \par implementato                                                                                                                                                   \\
	\hline
	TS.RFO2                                                                                                                         &
	L’Utente generico e l’Utente base devono poter ricercare un ristorante attraverso filtri:
	\begin{itemize}
		\item nome;
		\item luogo;
		\item tipologia di cucina.
	\end{itemize}                                                                                                      &
	Non \par implementato                                                                                                                                                   \\
	\hline
	TS.RFO3                                                                                                                         &
	L'Utente base e generico devono poter visualizzare in dettaglio un ristorante \par
	Verificare che:
	\begin{itemize}
		\item la \textit{Home} della piattaforma sia visualizzabile senza richiedere l'accesso.
		\item il dettaglio di un ristorante deve essere visualizzabile senza richiedere l'accesso.
		\item siano presenti ristoranti nell'elenco visualizzato dall'utente.
		\item la scheda del ristorante sia selezionabile.
	\end{itemize}                                      &
	Non \par implementato                                                                                                                                                   \\
	\hline
	TS.RFD4                                                                                                                         &
	L'Utente base e generico devono poter condividere un link di un ristorante. \par
	Verificare che:
	\begin{itemize}
		\item la \textit{Home} della piattaforma sia visualizzabile senza richiedere l'accesso.
		\item il dettaglio di un ristorante deve essere visualizzabile senza richiedere l'accesso.
		\item la pagina di un ristorante deve essersi caricata correttamnte.
		\item il pulsante di condivisione deve essere individuabile.
	\end{itemize}                                      &
	Non \par implementato                                                                                                                                                   \\
	\hline
	TS.RFD5                                                                                                                         &
	L'Utente base e generico devono poter visualizzare la pagina delle FAQ\g. \par
	Verificare che:
	\begin{itemize}
		\item il pulsante per la visualizzazione delle FAQ deve essere individuabile.
		\item la pagina di visualizzazione delle FAQ deve essere visitabile senza richiedere l'accesso.
	\end{itemize}                                 &
	Non \par implementato                                                                                                                                                   \\
	\hline
	TS.RFO6                                                                                                                         &
	L'Utente generico effettua l'accesso al Sistema \par
	Verificare che:
	\begin{itemize}
		\item l'utente sia registrato nel sistema.
		\item l'utente non abbia già effettuato l'acesso al sistema.
		\item il pulsante per effettuare l'accesso sia ben visibile e chiaro.
		\item il pulsante per effettuare l'accesso rimandi alla pagina di inserimento credenziali.
	\end{itemize}                                      &
	Non \par implementato                                                                                                                                                   \\
	\hline
	TS.RFO7                                                                                                                         &
	L'Utente generico effettua la registrazione al Sistema. \par
	Verificare che:
	\begin{itemize}
		\item l'utente non sia registrato nel sistema.
		\item l'utente non abbia già effettuato l'accesso al sistema.
		\item il pulsante per effettuare la registrazione sia ben visibile e chiaro.
		\item il pulsante per effettuare la registrazione rimandi alla pagina di inserimento dati.
	\end{itemize}                                      &
	Non \par implementato                                                                                                                                                   \\
	\hline
	TS.RFO8                                                                                                                         &
	L’Utente generico deve visualizzare un messaggio d’errore se l’accesso fallisce. \par
	Verificare che:
	\begin{itemize}
		\item l'utente non abbia già effettuato l'accesso al sistema.
		\item il pulsante per effettuare l'accesso sia ben visibile e chiaro.
		\item il pulsante per effettuare l'accesso rimandi alla pagina di inserimento dati.
		\item il sistema effettui correttamente la procedura di controllo.
		\item il messaggio venga visualizzato correttamente.
	\end{itemize}                                             &
	Non \par implementato                                                                                                                                                   \\
	\hline
	TS.RFO9                                                                                                                         &
	L’Utente generico deve visualizzare un messaggio d’errore se la registrazione fallisce. \par
	Verificare che:
	\begin{itemize}
		\item l'utente non sia registrato nel sistema.
		\item l'utente non abbia già effettuato l'accesso al sistema.
		\item il pulsante per effettuare la registrazione sia ben visibile e chiaro.
		\item il pulsante per effettuare la registrazione rimandi alla pagina di inserimento dati.
		\item il sistema avvii correttamente la procedura di registrazione.
		\item il messaggio venga visualizzato correttamente.
	\end{itemize}                                      &
	Non \par implementato                                                                                                                                                   \\
	\hline
	TS.RFD10                                                                                                                        &
	L'Utente base deve poter visualizzare i propri dati. \par
	Verificare che:
	\begin{itemize}
		\item l'utente deve essere correttamente autenticato.
		\item il pulsante per accedere alla visualizzazione dei dati sia ben visibile e chiaro.
		\item il pulsante rimandi alla pagina di visualizzazione dati.
	\end{itemize}                                         &
	Non \par implementato                                                                                                                                                   \\
	\hline
	TS.RFD11                                                                                                                        &
	L'Utente base deve poter modificare i propri dati. \par
	Verificare che:
	\begin{itemize}
		\item l'utente deve essere correttamente autenticato.
		\item il pulsante per accedere alla modifica dei dati sia ben visibile e chiaro.
		\item il pulsante rimandi alla pagina di modifica dati.
		\item i nuovi dati inseriti non contengano caratteri non permessi dal sistema.
	\end{itemize}                                                &
	Non \par implementato                                                                                                                                                   \\
	\hline
	TS.RFD12                                                                                                                        &
	L’Utente base deve poter visualizzare lo storico dei suoi ordini. \par
	Verificare che:
	\begin{itemize}
		\item l'utente deve essere correttamente autenticato.
		\item il pulsante per accedere allo storico degli ordini sia ben visibile e chiaro.
		\item il pulsante rimandi alla pagina di visualizzazione dello storico ordini.
	\end{itemize}                                             &
	Non \par implementato                                                                                                                                                   \\
	\hline
	TS.RFO13                                                                                                                        &
	L'Utente base deve visualizzare le proprie prenotazioni. \par
	Verificare che:
	\begin{itemize}
		\item l'utente deve essere correttamente autenticato.
		\item il pulsante per accedere alla visualizzazione delle prenotazioni sia ben visibile e chiaro.
		\item il pulsante rimandi correttamente alla pagina di visualizzazione delle prenotazioni.
	\end{itemize}                               &
	Non \par implementato                                                                                                                                                   \\
	\hline
	TS.RFO13                                                                                                                        &
	L’Utente base deve poter visualizzare il dettaglio della prenotazione.   \par
	Verificare che:
	\begin{itemize}
		\item l'utente base sia correttamente autenticato nel sistema.
		\item l'utente base abbia una prenotazione in corso.
		\item il pulsante per la visualizzazione del dettaglio sia ben visibile e chiaro.
		\item le pietanze ordinate siano correttamente visualizzate.
	\end{itemize}                                               &
	Non \par implementato                                                                                                                                                   \\
	\hline
	TS.RFO14                                                                                                                        &
	Il Sistema deve notificare il cliente sullo stato della sua prenotazione.  \par
	Verificare che:
	\begin{itemize}
		\item l'utente base sia correttamente autenticato.
		\item la prenotazione faccia riferimento al corretto utente.
		\item la prenotazione sia visualizzata correttamente.
	\end{itemize}                                                                    &
	Non \par implementato                                                                                                                                                   \\
	\hline
	TS.RFO14                                                                                                                        &
	Il Sistema notifica il cliente che il suo ordine è stato modificato. \par
	Verificare che:
	\begin{itemize}
		\item l'utente base sia correttamente autenticato.
		\item la prenotazione sia stata accettata dal ristoratore.
		\item la prenotazione faccia riferimento al giusto utente.
	\end{itemize}                                                                      &
	Non \par implementato                                                                                                                                                   \\
	\hline
	TS.RFD15                                                                                                                        &
	L’Utente base deve poter eliminare il suo account. \par
	Verificare che:
	\begin{itemize}
		\item l'utente deve essere correttamente autenticato.
		\item il pulsante per richiedere l'eliminazione dei propri dati sia ben visibile e chiaro.
		\item il pulsante avvii correttamente la procedura di eliminazione dati relativi allo specifico utente.
	\end{itemize}                         &
	Non \par implementato                                                                                                                                                   \\
	\hline
	TS.RFO16                                                                                                                        &
	L’Utente base deve poter prenotare un tavolo.  \par
	Verificare che:
	\begin{itemize}
		\item l'utente deve essere correttamente autenticato.
		\item un ristorante deve essere attualmente selezionato.
		\item i dati necessari alla prenotazione siano stati inseriti e siano corretti.
		\item il pulsante per richiedere la prenotazione sia ben visibile e chiaro.
		\item il pulsante invii correttamente la richiesta di prenotazione.
	\end{itemize}                                                 &
	Non \par implementato                                                                                                                                                   \\
	\hline
	TS.RFO17                                                                                                                        &
	L’Utente base deve poter condividere la prenotazione con i commensali\g.  \par
	Verificare che:
	\begin{itemize}
		\item l'utente base è correttamente autenticato nel sistema.
		\item l'utente base ha una prenotazione in corso che è stata accettata dal ristoratore.
		\item il pulsante per la condivisione sia ben visibile e chiaro.
		\item il pulsante per la condivisione salvi il riferimento alla prenotazione nelle note del dispositivo utilizzato dall'utente.
	\end{itemize} &
	Non \par implementato                                                                                                                                                   \\
	\hline
	TS.RFO18                                                                                                                        &
	L’Utente base deve poter annullare la prenotazione.  \par
	Verificare che:
	\begin{itemize}
		\item l'utente base sia correttamente autenticato nel sistema.
		\item l'utente base abbia una prenotazione in corso.
		\item il pulsante per l'annullamento sia ben visibile e chiaro.
		\item il pulsante per l'annullamento modifichi correttamente lo stato\g della prenotazione.
	\end{itemize}                                     &
	Non \par implementato                                                                                                                                                   \\
	\hline
	TS.RFO19                                                                                                                        &
	L’Utente base deve poter accedere ad una prenotazione.  \par
	Verificare che:
	\begin{itemize}
		\item l'utente base sia correttamente autenticato nel sistema.
		\item il \textit{link} relativo alla condivisione di una prenotazione è corretto.
		\item la pagina relativa alla prenotazione viene visualizzata correttamente.
	\end{itemize}                                               &
	Non \par implementato                                                                                                                                                   \\
	\hline
	TS.RFO20                                                                                                                        &
	L’Utente base deve poter annullare il proprio ordine\g.\par
	Verificare che:
	\begin{itemize}
		\item l'utente base sia correttamente autenticato nel sistema.
		\item l'utente base abbia una prenotazione in corso con un ordine associato.
		\item il pulsante per l'annullamento dell'ordine sia ben visibile e chiaro.
		\item il pulsante per l'annullamento dell'ordine elimini correttamente l'ordine associato alla prenotazione.
	\end{itemize}                    &
	Non \par implementato                                                                                                                                                   \\
	\hline
	TS.RFO21                                                                                                                        &
	L’Utente base deve poter fare un'ordinazione\g collaborativa dei pasti e creare il suo ordine.  \par
	Verificare che:
	\begin{itemize}
		\item l'utente base sia correttamente autenticato nel sistema.
		\item l'utente base abbia una prenotazione in corso.
		\item le ordinazioni degli altri utenti vengano correttamente visualizzate.
		\item le pietanze scelte dall'utente vengano correttamente aggiunte all'ordine.
		\item il pulsante di conferma sia ben visibile e chiaro.
	\end{itemize}                                                 &
	Non \par implementato                                                                                                                                                   \\
	\hline
	TS.RFO22                                                                                                                        &
	L’Utente base deve poter annullare il proprio ordine.   \par
	Verificare che:
	\begin{itemize}
		\item l'utente base sia correttamente autenticato nel sistema.
		\item l'utente base abbia una prenotazione in corso.
		\item le pietanze ordinate siano correttamente visualizzate.
		\item il pulsante di rimozione sia chiaro e ben visibile.
	\end{itemize}                                                                  &
	Non \par implementato                                                                                                                                                   \\
	\hline
	TS.RFO23                                                                                                                        &
	L’Utente base deve poter dividere il conto come preferisce.   \par
	Verificare che:
	\begin{itemize}
		\item l'utente base sia correttamente autenticato.
		\item l'utente base possieda il permesso di visualizzare la prenotazione.
		\item lo stato della prenotazione sia relativo alla sua conclusione.
		\item il pulsante per la scelta della divisione del conto sia chiaro e ben visibile.
	\end{itemize}                                            &
	Non \par implementato                                                                                                                                                   \\
	\hline
	TS.RFO24                                                                                                                        &
	L’Utente base deve poter visualizzare il messaggio d’errore che la divisione del conto è stata già effettuata.   \par
	Verificare che:
	\begin{itemize}
		\item l'utente base sia correttamente autenticato.
		\item l'utente base possieda il permesso di visualizzare la prenotazione.
		\item lo stato della prenotazione sia relativo alla sua conclusione.
		\item il pulsante per la scelta della divisione del conto sia chiaro e ben visibile.
		\item il sistema abbia correttamente memorizzato l'avvenuta divisione del conto.
		\item il sistema notifichi correttamente che la divisione del conto sia già stata effettuata.
	\end{itemize}                                   &
	Non \par implementato                                                                                                                                                   \\
	\hline
	TS.RFO25                                                                                                                        &
	L’Utente base deve poter pagare il conto come preferisce.   \par
	Verificare che:
	\begin{itemize}
		\item l'utente base sia correttamente autenticato.
		\item l'utente base possieda il permesso di visualizzare la prenotazione.
		\item lo stato della prenotazione sia relativo alla sua conclusione.
		\item il pulsante per la scelta del metodo di pagamento sia chiaro e ben visibile.
	\end{itemize}                                              &
	Non \par implementato                                                                                                                                                   \\
	\hline
	TS.RFO26                                                                                                                        &
	L’Utente base deve poter visualizzare l’errore relativo al pagamento fallito.   \par
	Verificare che:
	\begin{itemize}
		\item l'utente base sia correttamente autenticato.
		\item l'utente base possieda il permesso di visualizzare la prenotazione.
		\item lo stato della prenotazione sia relativo alla sua conclusione.
		\item l'utente abbia effettuato un tentativo di pagamento.
		\item la notifica relativa all'errore venga recapitata all'utente corretto.
	\end{itemize}                                                     &
	Non \par implementato                                                                                                                                                   \\
	\hline
	TS.RFO27                                                                                                                        &
	L’Utente base deve poter inserire recensioni.   \par
	Verificare che:
	\begin{itemize}
		\item l'utente sia correttamente autenticato.
		\item l'utente abbia i permessi per visualizzare la prenotazione.
		\item la prenotazione da recensire sia conclusa con successo.
		\item non vengano inseriti caratteri non permessi.
		\item non sia già stata lasciata una recensione dallo stesso cliente\g relativa alla stessa prenotazione
	\end{itemize}                        &
	Non \par implementato                                                                                                                                                   \\
	\hline
	TS.RFO28                                                                                                                        &
	Il Sistema notifica l’Utente base se vuole lasciare un \textit{feedback}.\par
	Verificare che:
	\begin{itemize}
		\item l'utente sia correttamente autenticato.
		\item l'utente abbia concluso con successo una prenotazione.
		\item la prenotazione non sia ancora stata recensita dall'utente.
	\end{itemize}                                                               &
	Non \par implementato                                                                                                                                                   \\
	\hline
	TS.RFO29                                                                                                                        &
	L’Utente base deve poter visualizzare la notifica relativa alla modifica della sua ordinazione.\par
	Verificare che:
	\begin{itemize}
		\item l'utente sia correttamente autenticato.
		\item l'utente abbia effettuato con successo una prenotazione.
		\item la notifica venga inviata solo in caso la prenotazione abbia subito una modifica.
		\item la notifica venga recapitata all'utente corretto.
	\end{itemize}                                         &
	Non \par implementato                                                                                                                                                   \\
	\hline
	TS.RFD30                                                                                                                        &
	Il Sistema notifica il cliente che il suo \textit{feedback} ha ricevuto una risposta. \par
	Verificare che:
	\begin{itemize}
		\item l'utente base sia correttamente autenticato.
		\item si faccia riferimento al corretto utente.
	\end{itemize}                                                                              &
	Non \par implementato                                                                                                                                                   \\
	\hline
	TS.RFD31                                                                                                                        &
	L’Utente base deve poter inserire le proprie allergie. \par
	Verificare che:
	\begin{itemize}
		\item l'utente sia correttamente autenticato.
		\item il pulsante per l'inserimento di una allergia sia ben visibile e chiaro.
		\item i dati vengano correttamente inseriti.
		\item i dati vengano inseriti in riferimento all'utente corretto.
	\end{itemize}                                                  &
	Non \par implementato                                                                                                                                                   \\
	\hline
	TS.RFD31                                                                                                                        &
	L’Utente base deve poter modificare le proprie allergie. \par
	Verificare che:
	\begin{itemize}
		\item l'utente sia correttamente autenticato.
		\item l'utente abbia precedentemente inserito almeno una allergia.
		\item il pulsante per la modifica di una allergia sia ben visibile e chiaro.
		\item i dati vengano correttamente modificati.
		\item i dati vengano modificati in riferimento all'utente corretto.
	\end{itemize}                                                    &
	Non \par implementato                                                                                                                                                   \\
	\hline
	TS.RFD32                                                                                                                        &
	L’Utente base deve poter visualizzare un messaggio se seleziona un piatto di cui è allergico. \par
	Verificare che:
	\begin{itemize}
		\item l'utente sia correttamente autenticato.
		\item l'utente abbia precedentemente inserito almeno una allergia.
		\item l'utente abbia selezionato un piatto contenente un allergene.
		\item l'allergene presente nel piatto corrisponda ad uno presente nell'elenco delle allergie dell'utente.
		\item il messaggio venga visualizzato correttamente.
	\end{itemize}                       &
	Non \par implementato                                                                                                                                                   \\
	\hline
	TS.RFO33                                                                                                                        &
	L’Utente base deve poter visualizzare il menù di un ristorante. \par
	Verificare che:
	\begin{itemize}
		\item l'utente sia correttamente autenticato.
		\item l'utente abbia selezionato un ristorante.
		\item il sistema mostri il menù relativo al ristorante corretto.
	\end{itemize}                                                                &
	Non \par implementato                                                                                                                                                   \\
	\hline
	TS.RFD34                                                                                                                        &
	L'Utente autenticato deve effettuare il \textit{logout}. \par
	Verificare che:
	\begin{itemize}
		\item l'utente deve essere correttamente autenticato.
		\item il pulsante per effettuare la disconnessione sia ben visibile e chiaro.
		\item il pulsante disconnetta correttamente l'utente e lo rimandi ala \textit{Home} della piattaforma.
	\end{itemize}                          &
	Non \par implementato                                                                                                                                                   \\
	\hline
	TS.RFO35                                                                                                                        &
	L'Utente generico e ristoratore\g devono poter comunicare tra loro attraverso chat \par
	Verificare che:
	\begin{itemize}
		\item l'utente generico sia autenticato nel sistema.
		\item il pulsante per accederea alla chat sia ben visibile e chiaro.
		\item il messaggio venga inviato al corretto destinatario.
	\end{itemize}                                                            &
	Non \par implementato                                                                                                                                                   \\
	\hline
	TS.RFD36                                                                                                                        &
	Il Sistema invia una notifica agli utenti collegati nella chat \par
	Verificare che:
	\begin{itemize}
		\item l'utente sia registrato nel sistema.
		\item l'utente a cui è destinata la notifica sia connesso al sistema.
	\end{itemize}                                                           &
	Non \par implementato                                                                                                                                                   \\
	\hline
	TS.RFO37                                                                                                                        &
	Il Sistema notifica l’Utente ristoratore dell’avvenuta prenotazione.  \par
	Verificare che:
	\begin{itemize}
		\item il corretto utente ristoratore sia autenticato nel sistema.
		\item il pulsante per visualizzare le prenotazioni mostri un indicatore di notifica.
	\end{itemize}                                            &
	Non \par implementato                                                                                                                                                   \\
	\hline
	TS.RFO38                                                                                                                        &
	Il Sistema notifica il ristoratore che è avvenuto un ordine.   \par
	Verificare che:
	\begin{itemize}
		\item la notifica sia recapitata al ristoratore corretto.
	\end{itemize}                                                                       &
	Non \par implementato                                                                                                                                                   \\
	\hline
	TS.RFO39                                                                                                                        &
	Il Sistema notifica il ristoratore dell’avvenuto pagamento.   \par
	Verificare che:
	\begin{itemize}
		\item la notifica sia recapitata al ristoratore corretto.
		\item il codice identificativo della prenotazione sia corretto.
		\item la prenotazione sia in uno stato che ne indichi la conclusione.
		\item il nuovo stato della prenotazione ne indichi il pagamento avvenuto con successo.
	\end{itemize}                                          &
	Non \par implementato                                                                                                                                                   \\
	\hline
	TS.RFD40                                                                                                                        &
	Il Sistema notifica il ristoratore che è stato inserito un \textit{feedback}\g   \par
	Verificare che:
	\begin{itemize}
		\item il ristoratore a cui è destinata la notifica sia quello corretto.
		\item la prenotazione sia stata conclusa con successo.
	\end{itemize}                                                         &
	Non \par implementato                                                                                                                                                   \\
	\hline
	TS.RFO41                                                                                                                        &
	L’Utente ristoratore deve poter consultare la lista delle prenotazioni (con la lista ingredienti), ed andare in dettaglio. \par
	Verificare che:
	\begin{itemize}
		\item l'utente ristoratore sia correttamente autenticato.
		\item l'elenco delle prenotazioni sia relativo al ristorante dell'utente.
		\item il pulsante per la visualizzazione della lista sia visibile e chiaro.
		\item il pulsante per la visualizzazione degli ingredienti sia ben visibile e chiaro.
	\end{itemize}                                           &
	Non \par implementato                                                                                                                                                   \\
	\hline
	TS.RFO42                                                                                                                        &
	L’Utente ristoratore deve poter accettare una prenotazione. \par
	Verificare che:
	\begin{itemize}
		\item l'utente ristoratore sia correttamente autenticato.
		\item la richiesta di prenotazione sia correttamente visualizzata.
		\item il polsante per la modifica dello stato sia ben visibile e chiaro.
		\item il pulsante modifichi correttamente lo stato della prenotazione.
	\end{itemize}                                                        &
	Non \par implementato                                                                                                                                                   \\
	\hline

	TS.RFO43                                                                                                                        &
	L’Utente ristoratore deve poter rifiutare una prenotazione. \par
	Verificare che:
	\begin{itemize}
		\item l'utente ristoratore sia correttamente autenticato.
		\item la richiesta di prenotazione sia correttamente visualizzata.
		\item il polsante per la modifica dello stato sia ben visibile e chiaro.
		\item il pulsante modifichi correttamente lo stato della prenotazione.
	\end{itemize}                                                        &
	Non \par implementato                                                                                                                                                   \\
	\hline
	TS.RFO44                                                                                                                        &
	L’Utente ristoratore deve poter terminare una prenotazione. \par
	Verificare che:
	\begin{itemize}
		\item l'utente ristoratore sia correttamente autenticato.
		\item la richiesta di prenotazione sia correttamente visualizzata.
		\item il polsante per la modifica dello stato sia ben visibile e chiaro.
		\item il pulsante modifichi correttamente lo stato della prenotazione.
	\end{itemize}                                                        &
	Non \par implementato                                                                                                                                                   \\
	\hline
	TS.RFO45                                                                                                                        &
	L’Utente ristoratore deve poter consultare la lista delle ordinazioni. \par
	Verificare che:
	\begin{itemize}
		\item l'utente ristoratore sia correttamente autenticato.
		\item la prenotazione sia stata accettata dal ristoratore.
		\item la prenotazione sia visualizzata correttamente.
		\item le ordinazioni siano visualizzate correttamente.
	\end{itemize}                                                                      &
	Non \par implementato                                                                                                                                                   \\
	\hline
	TS.RFO46                                                                                                                        &
	L’Utente ristoratore deve poter modificare una ordinazione. \par
	Verificare che:
	\begin{itemize}
		\item l'utente ristoratore sia correttamente autenticato.
		\item la prenotazione sia stata accettata dal ristoratore.
		\item la prenotazione sia visualizzata correttamente.
		\item le ordinazioni siano visualizzate correttamente.
		\item il pulsante per la modifica dei piatti sia ben visibile e chiaro.
	\end{itemize}                                                         &
	Non \par implementato                                                                                                                                                   \\
	\hline
	TS.RF047                                                                                                                        &
	L’Utente ristoratore deve poter controllare lo stato del pagamento  \par
	Verificare che:
	\begin{itemize}
		\item l'utente ristoratore sia correttamente autenticato.
		\item la prenotazione sia stata accettata dal ristoratore.
		\item la prenotazione sia visualizzata correttamente.
		\item lo stato del pagamento della prenotazione sia ben visibile e chiaro.
	\end{itemize}                                                      &
	Non \par implementato                                                                                                                                                   \\
	\hline
	TS.RFO48                                                                                                                        &
	L’Utente ristoratore deve poter consultare una lista di \textit{feedback} \par
	Verificare che:
	\begin{itemize}
		\item l'utente ristoratore sia correttamente autenticato.
		\item i \textit{feedback} siano ben visualizzati e facciano riferimento al corretto ristorante.
		\item il pulsante per la visualizzazione dei \textit{feedback} sia ben visibile e chiaro.
	\end{itemize}                                 &
	Non \par implementato                                                                                                                                                   \\
	\hline
	TS.RFO49                                                                                                                        &
	L’Utente ristoratore deve poter rispondere ad un \textit{feedback}. \par
	Verificare che:
	\begin{itemize}
		\item l'utente ristoratore sia correttamente autenticato.
		\item il pulsante di risposta sia ben visibile e chiaro.
		\item la risposta non includa caratteri non accettati.
		\item la risposta venga assegnata al \textit{feedback} corretto.
	\end{itemize}                                                                &
	Non \par implementato                                                                                                                                                   \\
	\hline
	TS.RF050                                                                                                                        &
	L’Utente ristoratore deve poter segnalare un \textit{feedback}. \par
	Verificare che:
	\begin{itemize}
		\item l'utente ristoratore sia correttamente autenticato.
		\item il pulsante di segnalazione sia ben visibile e chiaro.
		\item la segnalazione avvenga relativamente al \textit{feedback} corretto.
	\end{itemize}                                                      &
	Non \par implementato                                                                                                                                                   \\
	\hline
	TS.RFD51                                                                                                                        &
	L’Utente ristoratore deve poter gestire le informazioni del suo ristorante. \par
	Verificare che:
	\begin{itemize}
		\item l'utente ristoratore sia correttamente autenticato.
		\item il pulsante per la visualizzazione del profilo ristorante sia ben visibile e chiaro.
		\item il pulsante per la modifica dei dati sia ben visibile e chiaro.
	\end{itemize}                                      &
	Non \par implementato                                                                                                                                                   \\
	\hline
	TS.RFO52                                                                                                                        &
	L’Utente ristoratore deve poter gestire il menù. \par
	Verificare che:
	\begin{itemize}
		\item l'utente ristoratore sia correttamente autenticato.
		\item il pulsante per la visualizzazione del menù sia ben visibile e chiaro.
		\item il pulsante per la modifica dei dati sia ben visibile e chiaro.
		\item il sistema apporti le modifiche correttamente.
	\end{itemize}                                                    &
	Non \par implementato                                                                                                                                                   \\
	\hline
	TS.RFO53                                                                                                                        &
	L’Utente ristoratore deve poter gestire la lista ingredienti. \par
	Verificare che:
	\begin{itemize}
		\item l'utente ristoratore sia correttamente autenticato.
		\item l'utente ristoratore abbia selezionato un piatto da modificare.
		\item il pulsante per la modifica dei dati sia ben visibile e chiaro.
		\item il sistema apporti le modifiche correttamente.
	\end{itemize}                                                           &
	Non \par implementato                                                                                                                                                   \\
	\hline
	TS.RFO54                                                                                                                        &
	L’Utente ristoratore deve poter assegnare gli ingredienti ad un piatto. \par
	Verificare che:
	\begin{itemize}
		\item l'utente ristoratore sia correttamente autenticato.
		\item l'utente ristoratore abbia selezionato un piatto da modificare.
		\item il pulsante per la modifica dei dati sia ben visibile e chiaro.
		\item il sistema apporti le modifiche correttamente.
	\end{itemize}                                                           &
	Non \par implementato                                                                                                                                                   \\
	\hline
	TS.RFO55                                                                                                                        &
	L’Utente ristoratore deve poter visualizzare la notifica relativa all’annullamento di un ordinazione. \par
	Verificare che:
	\begin{itemize}
		\item l'utente ristoratore sia correttamente autenticato.
		\item l'utente ristoratore abbia ricevuto una ordinazione.
		\item la notifica venga recapitata al ristoratore corretto.
		\item la notifica venga inviata correttamente al seguito dell'annullamento di una ordinazione.
	\end{itemize}                                 &
	Non \par implementato                                                                                                                                                   \\
	\hline
	TS.RFO56                                                                                                                        &
	L’Utente ristoratore deve poter visualizzare la notifica relativa all’annullamento di una prenotazione. \par
	Verificare che:
	\begin{itemize}
		\item l'utente ristoratore sia correttamente autenticato.
		\item l'utente ristoratore abbia ricevuto una prenotazione.
		\item la notifica venga recapitata al ristoratore corretto.
		\item la notifica venga inviata correttamente al seguito dell'annullamento di una prenotazione.
	\end{itemize}                                &
	Non \par implementato                                                                                                                                                   \\
	\hline
\end{longtable}


