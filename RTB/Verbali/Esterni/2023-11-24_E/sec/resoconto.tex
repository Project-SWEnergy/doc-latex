\section{Tecnologie conosciute}

Prima dell'incontro con il proponente, il gruppo ha preparato un file Excel
per riassumere le conoscenze di ciascun membro di SWEnergy.

All'inizio dell'incontro, il proponente è stato consultato per ottenere
suggerimenti sulle tecnologie consigliate per lo sviluppo del progetto.

\subsection{Database}

Il proponente ha notato che la maggior parte del gruppo conosce \textit{php} e
\textit{Postgres}, consigliando quindi l'utilizzo di queste due tecnologie per
lo sviluppo del backend. Stefano ha sottolineato che l'80\%, in linea di
massima, del backend riguarda il salvataggio dei dati e le query. Pertanto,
suggerisce l'adozione di una tecnologia conosciuta dall'intero gruppo, almeno
per quanto riguarda il database.

\subsection{APIs}

Il proponente ha aggiunto che l'azienda lavora con \textit{Java} e
\textit{JavaScript}. Ha sconsigliato l'uso di \textit{Java}, a meno che non sia
necessaria correttezza e sicurezza, poiché il linguaggio può risultare
eccessivamente verboso. Al contrario, Imola Informatica offre supporto per
\textit{JavaScript}. In particolare, Stefano ha spiegato che per progetti di
grandi dimensioni, con più persone che scrivono codice, è preferibile
utilizzare \textit{TypeScript}. L'indicazione dei tipi fornisce una
descrizione più precisa delle funzioni, rendendo il codice più leggibile. Il
framework per lo sviluppo del backend in TypeScript è Node.js.

\subsection{Front-end}

Per il front-end, sono stati consigliati in particolare i due framework Angular
e React-Native. Anche in questo caso, il linguaggio consigliato è
\textit{JavaScript}, o meglio, \textit{TypeScript}.

\subsection{Conclusioni}

Il gruppo è orientato verso l'adozione di \textit{TypeScript} come linguaggio
unico sia per il \textit{backend} che per il \textit{frontend}. Tuttavia, è
ancora prematuro prendere una decisione definitiva, ma i membri del gruppo
stanno iniziando a familiarizzare con questo linguaggio costruendo alcuni
piccoli prototipi nel tempo libero.

\section{Domande}

Durante le ultime due settimane, SWEnergy ha identificato i casi d'uso e i
requisiti. Di conseguenza, sono emerse delle domande riguardo alle esigenze
del proponente. Di seguito sono riassunte brevemente solo le risposte alle
domande:

\begin{itemize}
	\item Se un ristoratore lo ritenesse necessario, è apprezzabile se ha la
	      possibilità di avviare una conversazione con il cliente, ad esempio,
	      nel caso in cui il ristoratore abbia dubbi sulle allergie del cliente.

	\item Le chat possono rimanere attive per tutto il tempo necessario, a meno
	      di restrizioni sulla privacy. Il proponente fornirà ulteriori
	      informazioni dopo aver consultato i suoi colleghi.

	\item È preferibile includere uno stato dell'ordine in modo che il sistema
	      verifichi autonomamente la disponibilità dei posti e informi l'utente
	      base. Successivamente, il ristoratore conferma o rifiuta la
	      prenotazione in modo definitivo.

	\item Il ristoratore dovrebbe poter indicare la cucina del proprio
	      ristorante da una lista di opzioni, migliorando la precisione delle
	      query e riducendo le ambiguità.

	\item Il proponente richiede lo sviluppo di una web app strutturata come
	      \textit{single page application}.

	\item Il proponente consente l'implementazione del login anche attraverso
	      un servizio di terze parti.

	\item Al momento, non è richiesto un moderatore dei contenuti.
\end{itemize}

\section{Organizzazione del tempo}

Il gruppo ha concordato con il proponente di tenere incontri ogni due settimane
, mantenendo comunque un contatto costante tramite la chat su Telegram.
Tuttavia, questa organizzazione non è definitiva: se notiamo che gli incontri
sono troppo ravvicinati o distanti, affronteremo la questione con il
proponente per una diversa organizzazione.

Poiché il gruppo non ha avuto modo di discutere tutti i punti preparati, il
prossimo incontro è stato programmato per la prossima settimana, precisamente
venerdì 1 dicembre. L'obiettivo principale da completare per il prossimo
incontro è la stesura della bozza del documento "Analisi dei requisiti". Gli
obiettivi successivi riguarderanno l'analisi delle tecnologie e
l'identificazione e definizione delle entità.

\section{Conclusioni}

Il responsabile del gruppo SWEnergy redigerà il verbale esterno dell'incontro 
e organizzerà il meeting con i membri di SWEnergy per domenica sera, intorno 
alle 20. Il gruppo si organizzerà in base a quanto discusso con il proponente 
in questo incontro.
