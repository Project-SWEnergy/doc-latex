\section{\textit{Brainstorming}}

Prima dell'incontro, è stato condiviso con Alessandro Staffolani l'ordine del giorno. 
All'inizio dell'incontro, il
responsabile del gruppo SWEnergy ha presentato il lavoro svolto nel primo
sprint ponendo attenzione agli argomenti di interesse del proponente. Sono stati
citati il completamento dell'"Analisi dei requisiti" e la progettazione del
database anche in SQL. In aggiunta, è stata velocemente mostrata la PoC del
\textit{back-end} fino ad ora sviluppata.

\section{PoC del \textit{back-end}}

Mostrando la PoC, il gruppo ha chiesto qualche consiglio al proponente riguardo
all'utilizzo di \textit{OpenAPI} per la documentazione del \textit{back-end}.
Staffolani ha suggerito di implementare anche la descrizione delle classi di
ritorno delle API. Ha aggiunto che la documentazione può essere completata con
qualche esempio di utilizzo delle API, con dati di esempio, chiarendo che non è
il caso di svilupparla in questa fase del progetto; piuttosto consiglia di
completarla almeno nella PB.

\section{Retrospettiva}

Sono state discusse eventuali criticità riscontrate dal gruppo e dal proponente
durante lo sprint. In realtà, non sono state riscontrate criticità.

\section{Organizzazione del lavoro}

Il prossimo incontro è stato fissato in linea di massima per il 12 gennaio del
2024. Poiché ci sono delle festività in mezzo, SWEnergy ha concordato una certa
flessibilità con il proponente. Anche il lavoro da svolgere è stato concordato
in considerazione delle festività.

\section{Conclusioni}

Gli obiettivi concordati con il proponente per il secondo \textit{sprint}
includono:

\begin{itemize}
	\item \textbf{"Analisi dei requisiti"}: conclusione del documento.

	\item \textbf{"Piano di progetto"}: conclusione del documento.

	\item \textbf{"Piano di qualifica"}: conclusione del documento.

	\item \textbf{"Glossario"}: conclusione del documento.

	\item \textbf{"Norme di progetto"}: conclusione del documento.

	\item \textbf{Progettazione del PoC}: progettazione delle interazioni tra
	      il \textit{back-end} e il \textit{front-end}.

	\item \textbf{Progettazione più dettagliata}: ciascuno dei due \textit{team}
	      di sviluppo si occuperà di progettare più in dettaglio la parte di
	      competenza.

	\item \textbf{Implementazione del PoC}: implementazione di due PoC, una per
	      il \textit{back-end} e una per il \textit{front-end}.
\end{itemize}

Si noti che le attività riportate coincidono con la consegna dell'RTB.
SWEnergy si impegna a tenere un colloquio con il proponente prima della
consegna per verificare che il lavoro svolto sia conforme alle aspettative.
Il responsabile del gruppo SWEnergy redigerà il verbale esterno dell'incontro
e organizzerà una riunione con i membri di SWEnergy per domenica sera, intorno
alle 20, per pianificare le azioni basate su quanto discusso con il proponente
in questo incontro.
