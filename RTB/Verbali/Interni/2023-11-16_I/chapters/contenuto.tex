\section{Ordine del giorno}
\begin{itemize}
	\item Brainstorming.
	\item Individuazione dei ruoli.
	\item Spartizione dei ruoli.
	\item Individuazione dei compiti.
	\item Prossima riunione.
	\item Assegnazione degli incarichi.
\end{itemize}

\section{Resoconto}

\subsection{Brainstorming}
In seguito all'acquisizione del capitolato, è stato tenuto un incontro tra i
membri del gruppo per comprendere la situazione e le problematiche emergenti.
Il gruppo si prepara alla riunione con il proponente in data 24/11/2023.
SWEnergy ha notato che nella fase iniziale, il responsabile non ha molti
compiti, per questo motivo il gruppo ha deciso di assegnare al responsabile
i compiti di stesura dei verbali.

\subsection{Individuazione dei ruoli}
Il gruppo ha individuato i seguenti ruoli:
\begin{itemize}
	\item \textbf{Responsabile:} organizza l'ordine del giorno prima della
	      riunione, coordina il gruppo durante la riunione e redige i verbali.

	\item \textbf{Verificatore:} controlla la correttezza dei documenti prodotti.

	\item \textbf{Amministratore:} aggiorna il "\textit{Way of working}" e il
	      "Glossario", stende il "Piano di qualifica".

	\item \textbf{Analista:} redige la "Analisi dei requisiti".
\end{itemize}

Il gruppo ha diviso i ruoli in un modo piuttosto rigido, in realtà ogni membro
del gruppo, in queste fasi iniziali, può svolgere compiti di qualsiasi ruolo,
per aumentare la flessibilità e per lavorare in coppia.

\subsection{Spartizione dei ruoli}
I ruoli sono stati così suddivisi:
\begin{itemize}
	\item \textbf{Responsabile:} Carlo Rosso.
	\item \textbf{Verificatore:} Davide Maffei.
	\item \textbf{Amministratore:} Alessandro Tigani Sava.
	\item \textbf{Analista:} Giacomo Gualato, Matteo Bando, Niccolò Carlesso.
\end{itemize}

\subsection{Individuazione dei compiti}
Il gruppo ha individuato la  \textit{milestone Requirements and Technology
	Baseline}, composta dal completamento dei seguenti documenti:
\begin{itemize}
	\item "Analisi dei requisiti";
	\item "\textit{Way of working}";
	\item "Glossario";
	\item "Piano di qualifica";
	\item "Piano di progetto".
\end{itemize}

\subsubsection{"Analisi dei requisiti"}
Il gruppo ha compreso quali informazioni devono essere presenti nella "Analisi
dei requisiti". Il gruppo ha deciso di realizzare
una bozza per l'incontro con il proponente, in modo da avere un
\textit{feedback} immediato sulle scelte effettuate. La bozza approfondirà i
casi d'uso di primo livello. Inoltre, il gruppo raccoglierà degli appunti per
approfondire eventuali casi d'uso di secondo livello.

\subsubsection{"\textit{Way of working}"}
L'aggiornamento del "\textit{Way of working}" è un compito dell'Amministratore,
che dovrà aggiornare il documento con le nuove norme e le nuove procedure
decise dal gruppo durante gli incontri. Per includere una nuova norma o
procedura, il gruppo dovrà approvare la modifica tramite un voto: sono
necessari almeno 4 voti favorevoli per approvare una modifica.

\subsubsection{"Glossario"}
L'aggiornamento del "Glossario" è un compito dell'Amministratore, ma ogni membro
del gruppo che produce qualche documento, deve aggiornare il "Glossario" con i
termini che ritiene opportuni.

\subsubsection{"Piano di qualifica"}
Il "Piano di qualifica" è un compito dell'Amministratore. Per il momento, il
gruppo ha deciso di non approfondire questo documento. Si aspetterà la prossima
riunione per decidere come procedere.

\subsubsection{"Piano di progetto"}
Il "Piano di progetto" è un compito del Responsabile. Per il momento, il gruppo
ha deciso di non approfondire questo documento. Tuttavia, sono state chiarite
alcune scadenze per i compiti da svolgere. In particolare, il gruppo ritiene
che al momento la "Analisi dei requisiti" sia il compito con la maggiore priorità.
Per sviluppare questo documento, il team è stato diviso in coppie di lavoro di
due persone. Ogni coppia si occuperà di individuare i casi d'uso di primo
livello e di produrre una bozza di essi entro il prossimo incontro. Le bozze da
produrre non hanno alcun vincolo di formato, ma si deve evincere facilmente la
struttura del caso d'uso, presentata a lezione. In particolare, ciascun caso
d'uso di primo livello individuato dovrà essere spiegato almeno con i punti:
\begin{itemize}
	\item diagramma UML;
	\item attori;
	\item precondizioni;
	\item postcondizioni;
	\item scenario principale;
	\item generalizzazioni;
	\item estensioni.
\end{itemize}

\subsection{Prossima riunione}
La prossima riunione è stata fissata per il 19/11/2023 alle ore 20:00. Nella
prossima riunione il gruppo si occuperà di:
\begin{itemize}
	\item Brainstorming del materiale prodotto;
	\item Redigere la bozza della "Analisi dei requisiti";
	\item Aggiornare il "\textit{Way of working}";
	\item Aggiornare il "Glossario";
	\item Discutere del "Piano di qualifica";
	\item Discutere del "Piano di progetto".
\end{itemize}

Ciascuna coppia di lavoro presenterà i casi d'uso di primo livello individuati.
Il gruppo discuterà i casi d'uso e individuerà due persone per la stesura della
bozza della "Analisi dei requisiti". In questa fase, la bozza comprenderà i casi
d'uso di primo livello e i requisiti ad essi collegati. Inoltre,
saranno individuati due analisti per l'analisi delle tecnologie da adottare.
L'analisi delle tecnologie così affrontata sarà di carattere generale, in
modo da avere un'idea di quali tecnologie potrebbero essere adottate
per sviluppare il "\textit{Proof of Concept}" e per discuterne con il
proponente. L'analisi delle tecnologie sarà poi presentata al gruppo in
una riunione precedente all'incontro con il proponente.

\section{Assegnazione degli incarichi}
In tabella vengono riportate le attività assegnate ai membri del gruppo.
\begin{center}
	{
		\renewcommand{\arraystretch}{1.5}
		\begin{tabular}{p{0.30\linewidth}|p{0.55\linewidth}|p{0.10\linewidth}}
			\textbf{Assegnatario}                   & \textbf{Descrizione}                       & \textbf{Rif.} \\
			\hline
			\multirow{2}{*}{Alessandro Tigani Sava} & Use case di primo livello                  & \#91          \\
			\cline{2-3}
			                                        & Documento "\textit{Way of working}"        & \#18          \\
			\hline
			\multirow{2}{*}{Carlo Rosso}            & Use case di primo livello                  & \#91          \\
			\cline{2-3}
			                                        & Stesura del verbale interno del 16/11/2023 & \#87          \\
			\hline
			\multirow{2}{*}{Davide Maffei}          & Use case di primo livello                  & \#91          \\
			\cline{2-3}
			                                        & Verifica verbale interno del 16/11/2023    & \#87          \\
			\hline
			Giacomo Gualato                         & Use case di primo livello                  & \#91          \\
			\hline
			Matteo Bando                            & Use case di primo livello                  & \#91          \\
			\hline
			Niccolò Carlesso                        & Use case di primo livello                  & \#91          \\
			\hline
		\end{tabular}
	}
\end{center}

