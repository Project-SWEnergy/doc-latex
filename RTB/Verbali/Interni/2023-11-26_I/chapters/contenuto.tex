\section{\textit{Brainstorming}}

Il relatore ha riassunto la situazione attuale del progetto e l'incontro con il
proponente avvenuto il 24/11/2023. Il gruppo ha quindi discusso in merito alle
tecnologie da analizzare in futuro. In base a quanto emerso con il proponente e
in base alle conoscenze dei membri del gruppo, si è deciso di analizzare le
seguenti tecnologie:

\begin{itemize}
	\item \textbf{\textit{TypeScript}}: linguaggio di programmazione che estende
	      \textit{JavaScript} aggiungendo il supporto ai tipi. Optiamo per
	      l'utilizzo di questo linguaggio perché l'azienda proponente è in grado
	      di supportarci; permette di sviluppare sia il \textit{frontend} che il
	      \textit{backend} dell'applicazione \textit{web} con un unico
	      linguaggio; è un linguaggio più comprensibile di \textit{JavaScript} e
	      individua più facilmente gli errori a tempo di compilazione.

	\item \textbf{\textit{Angular}}: \textit{framework} per lo sviluppo di
	      applicazioni \textit{web single page} in \textit{TypeScript}. Optiamo
	      per l'utilizzo di questo \textit{framework} perché l'azienda
	      proponente ha posto maggiore enfasi su questo \textit{framework}; un
	      membro del gruppo ha già avuto modo di utilizzarlo in passato; il
	      gruppo si rende conto  che esistano alternative molto simili, per
	      questo motivo non si esclude la possibilità di utilizzare un
	      \textit{framework} diverso in futuro. [NON E' VERO]


	\item \textbf{\textit{Node.js}}: \textit{framework} per lo sviluppo di
	      applicazioni \textit{web} in \textit{JavaScript}. Optiamo per
	      l'adozine di questo \textit{framework} perché il gruppo ha già avuto
	      modo di utilizzarlo in passato e il proponente consiglia di utilizzare
	      questo \textit{framework} per lo sviluppo del \textit{backend}.

	      % \item \textbf{\textit{Express}}: \textit{framework} per lo sviluppo di
	      % applicazioni \textit{web} in \textit{Node.js};

	\item \textbf{\textit{PostgreSQL}}: \textit{database} relazionale
	      \textit{open source}. Optiamo per l'utilizzo di questo
	      \textit{database} perché l'intero gruppo ha familiarità con questo
	      \textit{database} e SWEnergy ritiene che le tecnologie fino ad ora
	      elencate comportino un sufficiente grado di complessità per il
	      progetto.
\end{itemize}

\section{\textit{Stand-up}}

Il gruppo ha deciso di organizzare degli \textit{stand-up} settimanali per
tenersi aggiornati sullo stato di avanzamento del progetto. Gli \textit{stand-up}
avranno le seguenti caratteristiche:

\begin{itemize}
	\item \textbf{Frequenza}: settimanale, ogni domenica alle 20:30.

	\item \textbf{Durata}: 30 minuti.

	\item \textbf{Modalità}: videochiamata su \textit{Discord}.

	\item \textbf{Obiettivi:} l'obiettivo principale degli \textit{stand-up}
	      riguarda la risoluzione in itinere dei problemi riscontrati dai membri
	      del gruppo. Inoltre, gli \textit{stand-up} servono per tenere
	      aggiornati tutti i membri del gruppo sullo stato di avanzamento del
	      progetto e aumentare i \textit{feedback}; aumentano la flessibilità dei
	      membri del gruppo, in quanto permettono di adattare il piano di lavoro
	      in base alle difficoltà rilevate.

	\item \textbf{Argomenti:} ciascuno \textit{stand-up} avrà almeno la
	      seguente struttura:
	      \begin{enumerate}
		      \item \textbf{Brainstorming}: riassunto dello stato del progetto.

		      \item \textbf{Overview}: descrizione delle attività da svolgere la
		            settimana che viene.

		      \item \textbf{Assegnazione degli incarichi}: assegnazione degli
		            incarichi per la settimana che viene, se necessaria.

		      \item \textbf{Retrospettiva}: discussione sulle difficoltà
		            riscontrate dai membri del gruppo durante la settimana che si
		            è appena conclusa. Oppure opportunità di miglioramento del
		            progetto o del piano di lavoro.
	      \end{enumerate}
\end{itemize}

Le decisione prese durante gli altri incontri del gruppo saranno riportate nello
\textit{stand-up} successivo.

\section{\textit{Backlog}}

Il relatore ha inserito nel \textit{backlog} le attività da svolgere per
il completamento della prima fase del progetto, l'RTB. Nell'incontro di oggi ha
discusso l'organizzazione del lavoro per le settimane che verrano. Sono state
identificare le prime scadenze e sono stati assegnati gli incarichi. \\
Il gruppo ha deciso di indicare come data di scadenza della consegna dell'RTB il
21/12/2023, perché ritiene che sia una data ragionevole per la consegna della
PoC. In realtà, alcuni membri di SWEnergy sono titubanti in merito a questa
data, perché ritengono che sia troppo presto. Il \textit{team} ha deciso di
mantenere questa data di scadenza, perché, nel qual caso non fosse possibile
rispettarla, il gruppo avrebbe comunque tempo per recuperare durante le feste
natalizie. \\
