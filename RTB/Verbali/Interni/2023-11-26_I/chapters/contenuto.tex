\section{\textit{Brainstorming}}

Durante questa sessione, il relatore ha sintetizzato la situazione attuale del
progetto, evidenziando l'incontro avvenuto con il proponente il 24/11/2023. Il
gruppo ha poi discusso le tecnologie da esaminare in futuro, basandosi sia sui
dettagli emersi dal proponente sia sulle competenze dei membri del gruppo. Le
tecnologie scelte per l'analisi sono le
seguenti:

\begin{itemize}
	\item \textbf{\textit{TypeScript}}: un linguaggio di programmazione che
	      estende \textit{JavaScript} aggiungendo il supporto ai tipi. La scelta è
	      motivata dalla capacità dell'azienda proponente di fornire supporto, dalla
	      possibilità di sviluppare sia il \textit{frontend} che il \textit{backend}
	      dell'applicazione \textit{web} utilizzando un unico linguaggio e dalla
	      maggiore comprensibilità rispetto a \textit{JavaScript}, oltre alla facilità
	      di individuare errori durante la
	      compilazione.

	\item \textbf{\textit{Angular}}: un \textit{framework} per lo sviluppo di
	      applicazioni \textit{web single page} in \textit{TypeScript}. La scelta si
	      basa sull'enfasi posta dall'azienda proponente su questo \textit{framework},
	      sull'esperienza pregressa di uno dei membri del gruppo e sulla consapevolezza
	      che esistono alternative simili, anche se non si esclude l'utilizzo di un
	      \textit{framework} diverso in futuro.

	\item \textbf{\textit{Node.js}}: un \textit{framework} per lo sviluppo di
	      applicazioni \textit{web} in \textit{JavaScript}. Si opta per questo
	      \textit{framework} poiché il gruppo ne ha già esperienza e il
	      proponente lo consiglia esplicitamente per lo sviluppo del
	      \textit{backend}.

	\item \textbf{\textit{PostgreSQL}}: un \textit{database} relazionale
	      \textit{open source}. La decisione di utilizzare questo
	      \textit{database} è supportata dalla familiarità dell'intero gruppo con
	      questa tecnologia e dalla valutazione di SWEnergy, secondo cui le tecnologie
	      elencate fino a questo punto comportano un grado di complessità adeguato per
	      il progetto.
\end{itemize}

\section{\textit{Stand-up}}

Il gruppo ha deciso di organizzare degli \textit{stand-up} settimanali per
monitorare lo stato di avanzamento del progetto. Gli \textit{stand-up}
seguiranno queste
specifiche:

\begin{itemize}
	\item \textbf{Frequenza}: settimanale, ogni domenica alle 20:30.

	\item \textbf{Durata}: 30 minuti.

	\item \textbf{Modalità}: videochiamata su \textit{Discord}.

	\item \textbf{Obiettivi}: gli \textit{stand-up} mirano a risolvere i
	      problemi riscontrati dai membri del gruppo durante il processo, a tenere
	      aggiornati tutti sullo stato del progetto e a favorire i
	      \textit{feedback}. Offrono inoltre flessibilità nel piano di lavoro per
	      adattarsi alle difficoltà rilevate.

	\item \textbf{Agenda}: ogni \textit{stand-up} includerà almeno le seguenti
	      sezioni:
	      \begin{enumerate}
		      \item \textbf{Brainstorming}: riepilogo dello stato attuale del
		            progetto.

		      \item \textbf{Overview}: descrizione delle attività previste per
		            la settimana successiva.

		      \item \textbf{Assegnazione dei compiti}: se necessario,
		            assegnazione dei compiti per la settimana successiva.

		      \item \textbf{Retrospettiva}: discussione sulle difficoltà
		            incontrate durante la settimana conclusa e opportunità di
		            miglioramento del progetto o del piano di lavoro.
	      \end{enumerate}
\end{itemize}

Le decisioni prese durante gli incontri precedenti saranno riportate negli
\textit{stand-up} successivi.

\section{\textit{Backlog}}

Il relatore ha inserito nel \textit{backlog} le attività necessarie per
completare la prima fase del progetto, l'RTB. Durante l'incontro odierno, si è
discusso sull'organizzazione del lavoro nelle prossime settimane,
identificando le prime scadenze e assegnando i compiti. \\
Il gruppo ha deciso di fissare al 21/12/2023 la scadenza per la consegna dell'
RTB, ritenendola una data ragionevole per la PoC. Tuttavia, alcuni membri di
SWEnergy esprimono titubanza riguardo a questa data, reputandola troppo precoce.
Nonostante questo, il \textit{team} ha deciso di mantenerla, considerando la
possibilità di recuperare eventuali ritardi durante le festività natalizie.
