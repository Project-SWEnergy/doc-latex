\section{\textit{Brainstorming}}

La \textit{stand-up} è iniziata riassumento brevemente lo stato di avanzamento
del progetto. Infatti, il resoponsabile ha scorso velocemente l'ultimo verbale
esterno per ricordare al gruppo i punti salienti dell'incontro con il
proponente. Successivamente, il resoponsabile ha dato una panoramica dello stato del
"Piano di progetto", evidenziando eventuali domande o dubbi. Infine, i vari
membri del gruppo hanno rissunto il lavoro svolto in merito all'"Analisi dei
requisiti", anche in questo caso evidenziando eventuali domande o dubbi.

\section{"\textit{Way of working}"}

Per condividere gli appunti tra i membri del gruppo è stata creata una
\textit{repository} denominata \texttt{appunti-swe} all'interno
dell'organizzazione \textit{GitHub} del gruppo. Ciascun membro creerà un
\textit{branch} con il proprio nome. Viene richiesto a ciascun membro di
inserire tutto il materiale inerente al progetto di ingegneria del software
all'interno di tale \textit{repository}. Inoltre, il \texttt{README.md} della
\textit{repository} dovrà contenere una breve descrizione del contenuto della
\textit{repository} stessa, per facilitare la consultazione del materiale.
Infine, si richiede ai membri del gruppo di prediligere gli appunti nel formato
\texttt{Markdown}, per facilitare la consultazione del materiale.

\section{"Analisi dei requisiti"}

Il resoponsabile e l'amministratore hanno spiegato uno schema
generico e ad alto livello per illustrare i casi d'uso al resto del gruppo. Lo
schema è stato diviso in tre porzioni per facilitare l'assegnazione degli
incarichi ed è stato condiviso mediante la \textit{repository}
\texttt{appunti-swe}, qui sopra citata. Poi alcuni membri del gruppo hanno
spiegato il lavoro svolto in merito ai casi d'uso assegnati, per spiegare la
struttura del documento e per chiarire eventuali dubbi.

\section{Cambio dei ruoli}

Sono stati assegnati i ruoli per lo sprint che comincerà da domani. I ruoli
indicati nella tabella dei partecipanti (vedi \autoref{tab:partecipanti}) sono
validi fino al termine dello sprint. Tuttavia, gli incarichi assegnati prima del
cambio dei ruoli rimangono validi fino al termine dell'incarico stesso. Per
esempio, Carlo Rosso continuerà la stesura del "Piano di progetto".

\section{Retrospettiva}

Alla fine della riunione, il gruppo ha discusso brevemente per migliorare
l'organizzazione del lavoro. In particolare, Giacomo Gualato ha evidenziato la
necessità di individuare un metodo per tenere traccia delle discussioni avvenute
in merito ai requisiti. Infatti, nello schema generico e ad alto livello
illustrato dal resoponsabile e dall'amministratore, sono state evidenziate alcune
decisioni, ma non è presente un appunto per spiegare il motivo di tali
decisioni. Il gruppo, prova a utilizzare le \textit{Discussion} di
\textit{GitHub} per tenere traccia delle decisioni prese e valuta l'adozione di
questo strumento per il futuro.
