\section{\textit{Brainstorming}}
Il gruppo ha discusso l'avanzamento del lavoro all'interno dell'attuale \textit{sprint}.
Sono stati elencati i problemi presenti all'interno dei documenti realizzati e sono state discusse le modifiche necessarie per la presentazione inerente alla fase di avanzamento RTB.\\

Il \textit{team} ha organizzato la durata dell'attuale sprint, che avrà la consueta durata di due settimane ma presenterà un minore numero di ore lavorative a causa dei numerosi impegni derivanti dalla sessione d'esame e dagli impegni personali dei membri.

\section{Considerazioni}
Il gruppo ha individuato le seguenti modifiche da apportare al fine di presentare il lavoro svolto per l'avanzamento della fase RTB.\\
In particolare, le informazioni dettagliate in merito ad ogni documento verranno inserite all'interno delle relative \textit{issue}.

\subsection{Glossario}
Aggiornamento del documento "Glossario" inserendo la terminologia suggerita durante la riunione.\\
All'interno della \textit{issue} \#154 verranno elencati tali termini, insieme a quelli che potrebbero essere suggeriti in fase di revisione della documentazione.

\subsection{Analisi dei requisiti}
Correzione del documento "Analisi dei requisiti" secondo le indicazioni del professor Riccardo Cardin. \\
I dettagli inerenti tali modifiche sono stati discussi in precedenza alla riunione odierna.
Le informazioni utili alla modifica del documento sono reperibili nella \textit{issue} \#153 ad esso dedicata e nel \textit{repository} dedicato agli appunti, oltre che nella correzione fornita dal docente.

\subsection{Piano di progetto e Piano di qualifica}
Le modifiche da apportare ai documenti "Piano di progetto" e "Piano di qualifica" sono indicate nelle relative \textit{issue}, rispettivamente \#155, \#157.\\
Il piano di progetto dovrà presentare le informazioni relative all'ultimo \textit{sprint}, che avrà una durata pari a quella dei precedenti ma con un apporto inferiore di ore di lavoro, verranno apportate correzioni relative al suo contenuto e saranno aggiunte maggiori informazioni inerenti ai grafici presenti.\\
Il piano di qualifica sarà aggiornato con l'inserimento delle informazioni inerenti l'ultimo sprint e la conseguente modifica dei grafici riportanti le informazioni ricavate dall'applicazione delle metriche utili alla fase di progetto corrente.
Tali grafici dovranno riportare una descrizione più approfondita di quella attuale.

\subsection{Norme di progetto}
Aggiornamento del documento "Norme di progetto", le correzioni da apportare vengono esplicitate nella \textit{issue} \#156.\\
Oltre alla correzione di errori minori all'interno del testo, si sceglie di modificare la presentazione di alcune informazioni in esso presenti.