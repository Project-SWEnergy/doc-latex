\section{Introduzione}

\subsection{Scopo del documento}
Nel presente documento, viene presentata una descrizione dettagliata del prodotto, basata sull'analisi dei bisogni dell'utente 
emersi durante l'esaminazione del \gls{Capitolato}$^G$ e attraverso gli incontri con l'azienda proponente. 
La modellazione viene realizzata tramite \gls{UML}$^G$$^G$, identificando in modo approfondito requisiti e attori presenti nel progetto. 
Questo approccio consente di descrivere in dettaglio le varie componenti del prodotto e di indicare la struttura di ciascuna funzionalità.\\


\subsection{Riferimenti}
\subsubsection{Riferimenti normativi}
\begin{itemize}
    \item \href{https://www.math.unipd.it/~tullio/IS-1/2023/Progetto/C3.pdf}{\gls{Capitolato}$^G$ C3 - \textit{Easy Meal}}.
    \item \href{https://project-swenergy.\gls{\gls{Git}$^G$Hub}.io/}{Norme di progetto v2.0.0}.
    \item \href{https://www.math.unipd.it/~tullio/IS-1/2023/Dispense/PD2.pdf}{Regolamento progetto didattico}.
\end{itemize}

\subsubsection{Riferimenti informativi}
\begin{itemize}
    \item \href{https://www.math.unipd.it/~tullio/IS-1/2023/Dispense/T5.pdf}{Lezione T05 - Analisi dei requisiti}.
    \item \href{https://www.math.unipd.it/~rcardin/swea/2023/Diagrammi%20delle%20Classi.pdf}{Diagrammi delle classi}.
    \item \href{https://www.math.unipd.it/~rcardin/swea/2022/Diagrammi%20Use%20Case.pdf}{Diagrammi dei casi d'uso}.
\end{itemize}