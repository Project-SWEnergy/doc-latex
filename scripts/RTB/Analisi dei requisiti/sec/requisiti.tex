\section{\textit{Requirements baseline}}

In questa sezione sono presentati i requisiti individuati durante la fase di analisi, derivanti dai casi d'uso, dall'esame del \gls{Capitolato}$^G$ d'appalto e dagli incontri
sia interni che esterni con il proponente. Ciascun requisito è associato a un codice univoco, seguendo un formalismo stabilito tramite simboli.

Sono state individuate tre principali tipologie di requisiti:
\begin{enumerate}
	\item Funzionali, relativi all'usabilità del prodotto finale;
	\item Di qualità, riguardanti gli strumenti e la documentazione da fornire;
	\item Di vincolo, concernenti le tecnologie da presentare.
\end{enumerate}

\subsection{Funzionali}

Di seguito viene riportata la specifica relativa ai requisiti funzionali, che delineano le funzionalità del sistema, le azioni eseguibili
da parte del sistema e le informazioni che il sistema può fornire. La presenza di ogni requisito viene giustificata riportando la fonte, che può essere un \gls{UC}$^G$ oppure presente
nel testo del \gls{Capitolato}$^G$ d'appalto. Mentre i codici univoci sottostanti indicano:
\begin{enumerate}
	\item RFO: Requisito Funzionale Obbligatorio;
	\item RFF: Requisito Funzionale Facoltativo;
	\item RFD: Requisito Funzionale Desiderabile.
\end{enumerate}


\begin{table}[H]
	\renewcommand{\arraystretch}{1.5}
	\centering
	\begin{tabularx}{\textwidth}{l|X|p{2cm}}
		\textbf{ID} & \textbf{Descrizione}                                                                                       & \textbf{Fonte}                                                                                                                                               \\
		\hline
		RFO1        & L'\gls{Utente generico}$^G$ deve poter visualizzare l'elenco dei ristoranti disponibili.                             & \autoref{usecase:Visualizzazione elenco ristoranti}                                                                                                          \\
		\hline
		RFO2        & L'\gls{Utente generico}$^G$ deve poter ricercare un ristorante attraverso il nome,luogo e filtri.                    & \autoref{usecase:Ricerca di ristoranti}                                                                                                                      \\
		\hline
		RFO3        & L'\gls{Utente generico}$^G$ deve poter visualizzare un ristorante.                                                   & \autoref{usecase:Visualizzazione di un ristorante}                                                                                                           \\
		\hline
		RFD4        & L'\gls{Utente generico}$^G$ deve poter condividere un link di un ristorante.                                         & \autoref{usecase:Condivisione link del ristorante}                                                                                                           \\
		\hline
		RFD5        & L'\gls{Utente generico}$^G$ deve poter visualizzare la pagina delle FAQ.                                             & \autoref{usecase:Visualizzazione FAQ}                                                                                                                        \\
		\hline
		RFO6        & L'\gls{Utente generico}$^G$ deve poter effettuare l'accesso al Sistema.                                              & \autoref{usecase:Effettua accesso}, \autoref{usecase:Effettua accesso tradizionale}, \autoref{usecase:Effettua accesso per terze parti}                      \\
		\hline
		RFO7        & L'\gls{Utente generico}$^G$ deve poter effettuare la registrazione al Sistema come \gls{Utente base}$^G$ o \gls{Utente ristoratore}$^G$. & \autoref{usecase:Effettua registrazione}, \autoref{usecase:Effettua registrazione \gls{Utente base}$^G$} e \autoref{usecase:Effettua registrazione \gls{Utente ristoratore}$^G$} \\
		\hline
		RFO8        & L'\gls{Utente generico}$^G$ deve visualizzare un messaggio d'errore se l'accesso fallisce.                           & \autoref{usecase:Visualizzazione errore d'accesso}                                                                                                           \\
		\hline
		RFO9        & L'\gls{Utente generico}$^G$ deve visualizzare un messaggio d'errore se la registrazione fallisce.                    & \autoref{usecase:Errore registrazione account esistente} e \autoref{usecase:Errore registrazione recapito occupato}                                          \\
		\hline
		RFD10       & L'\gls{Utente base}$^G$ deve poter visualizzare i suoi dati utente.                                                  & \autoref{usecase:Visualizzazione dati utente}                                                                                                                \\
		\hline
		RFD11       & L'\gls{Utente base}$^G$ deve poter modificare i suoi dati utente.                                                    & \autoref{usecase:Modifica dati utente}                                                                                                                       \\
		\hline
		RFD12       & L'\gls{Utente base}$^G$ deve poter visualizzare lo storico dei suoi ordini.                                          & \autoref{usecase:Storico ordini}                                                                                                                             \\
	\end{tabularx}
	\caption{Tabella dei requisiti funzionali}
\end{table}

\begin{table}[H]
	\renewcommand{\arraystretch}{1.5}
	\centering
	\begin{tabularx}{\textwidth}{l|X|p{2cm}}
		\textbf{ID} & \textbf{Descrizione}                                                                                           & \textbf{Fonte}                                                              \\
		\hline
		RFO13       & L'\gls{Utente base}$^G$ deve poter visualizzare la lista delle sue prenotazioni, ed in caso andare in dettaglio.          & \autoref{usecase:Visualizzazione lista prenotazioni}, \autoref{usecase:Visualizzazione del riepilogo prenotazione}                \\
		\hline
		RFO14       & L'\gls{Utente base}$^G$ deve poter visualizzare la notifica dello \gls{Stato}$^G$ della sua prenotazione.                          & \autoref{usecase:Visualizzazione notifica \gls{Stato}$^G$ della prenotazione}         \\
		\hline
		RFD15       & L'\gls{Utente base}$^G$ deve poter elimare il proprio account.                                                           & \autoref{usecase:Eliminazione account}                                      \\
		\hline
		RFO16       & L'\gls{Utente base}$^G$ deve poter prenotare un tavolo.                                                                  & \autoref{usecase:Prenotazione di un tavolo}                                 \\
		\hline
		RFO17       & L'\gls{Utente base}$^G$ deve poter condividere la prenotazione.                                                          & \autoref{usecase:Condivisione della prenotazione}                           \\
		\hline
		RFO18       & L'\gls{Utente base}$^G$ deve poter annullare la prenotazione.                                                            & \autoref{usecase:Annullamento della prenotazione}                           \\
		\hline
		RFO19       & L'\gls{Utente base}$^G$ deve poter accedere ad una prenotazione mediante link di condivisione                            & \autoref{usecase:Accesso alla prenotazione}                                 \\
		\hline
		RFO20       & L'\gls{Utente base}$^G$ deve poter annullare il proprio \gls{Ordine}$^G$.                                                          & \autoref{usecase:Annullamento dell'\gls{Ordinazione}$^G$}                             \\
		\hline
		RFO21       & L'\gls{Utente base}$^G$ deve poter creare un \gls{Ordinazione}$^G$ collaborativa dei pasti.                                        & \autoref{usecase:Creazione dell'\gls{Ordinazione}$^G$ collaborativa dei pasti}        \\
		\hline
		RFO22       & L'\gls{Utente base}$^G$ deve poter annullare la propria \gls{Ordinazione}$^G$.                                                     & \autoref{usecase:Annullamento dell'\gls{Ordinazione}$^G$}                             \\
		\hline
		RFO23       & L'\gls{Utente base}$^G$ deve poter dividere il conto in maniera equa oppure proporzionale.                                                    & \autoref{usecase:Selezione della modalità di divisione del conto}           \\
		\hline
		RFO24       & L'\gls{Utente base}$^G$ deve poter visualizzare il messaggio d'errore che la divisione del conto è stata già effettuata. & \autoref{usecase:Visualizzazione errore divisione del conto già effettuata} \\
		\hline
		RFO25       & L'\gls{Utente base}$^G$ deve poter pagare il conto.                                                                      & \autoref{usecase:Pagamento del conto}                                       \\
	\end{tabularx}
	\caption{Tabella dei requisiti funzionali}
\end{table}


\begin{table}[H]
	\renewcommand{\arraystretch}{1.5}
	\centering
	\begin{tabularx}{\textwidth}{l|X|p{2cm}}
		\textbf{ID} & \textbf{Descrizione}                                                                                              & \textbf{Fonte}                                                                        \\
		\hline
		RFO26       & L'\gls{Utente base}$^G$ deve poter visualizzare l'errore relativo al pagamento fallito.                                     & \autoref{usecase:Visualizzazione errore pagamento}                                    \\
		\hline
		RFO27       & L'\gls{Utente base}$^G$ deve poter inserire \textit{\gls{Feedback}$^G$} e recensioni.                                                 & \autoref{usecase:Inserimento di \gls{Feedback}$^G$ e recensioni}                                \\
		\hline
		RFO28       & L'\gls{Utente base}$^G$ deve poter visualizzare la notifica di richiesta di inserimento \textit{\gls{Feedback}$^G$}.                  & \autoref{usecase:Visualizzazione della notifica di richiesta di inserimento \gls{Feedback}$^G$} \\
		\hline
		RFO29       & L'\gls{Utente base}$^G$ deve poter visualizzare la notifica relativa alla modifica della sua \gls{Ordinazione}$^G$.                   & \autoref{usecase:Visualizzazione notifica modifica \gls{Ordinazione}$^G$}                       \\
		\hline
		RFD30       & L'\gls{Utente base}$^G$ deve poter visualizzare la notifica relativa al suo \textit{\gls{Feedback}$^G$} che ha ricevuto una risposta. & \autoref{usecase:Visualizzazione notifica risposta \gls{Feedback}$^G$}                          \\
		\hline
		RFD31       & L'\gls{Utente base}$^G$ deve poter inserire e modificare le proprie allergie. 												& \autoref{usecase:Effettua registrazione \gls{Utente base}$^G$} e \autoref{usecase:Modifica dati utente}                      \\
		\hline
		RFD32       & L'\gls{Utente base}$^G$ deve poter visualizzare un messaggio se seleziona un piatto di cui è allergico. 					& \autoref{usecase:Visualizzazione messaggio di selezione di una pietanza con allergene}                      \\
		\hline
		RFD33       & L'\gls{Utente autenticato}$^G$ deve poter effettuare il \textit{logout}.                                                    & \autoref{usecase:Effettua Logout}                                                     \\
		\hline
		RFO34       & L'\gls{Utente autenticato}$^G$ deve poter comunicare attraverso la \textit{chat}.                                           & \autoref{usecase:Comunicazione attraverso chat}                                       \\
		\hline
		RFD35       & L'\gls{Utente autenticato}$^G$ deve poter visualizzare la notifica relativa all'arrivo di un nuovo messaggio in chat.       & \autoref{usecase:Visualizzazione notifica nuovo messaggio in chat}                    \\
		\hline
		RFO36       & L'\gls{Utente ristoratore}$^G$ deve poter visualizzare la notifica relativa ad una nuova prenotazione.                      & \autoref{usecase:Visualizzazione notifica nuova prenotazione}                               \\
		\hline
		RFD37       & L'\gls{Utente ristoratore}$^G$ deve poter visualizzare la notifica relativa ad un nuovo \gls{Ordine}$^G$.                             & \autoref{usecase:Visualizzazione notifica nuovo \gls{Ordine}$^G$}                               \\
		\hline
		RFO38       & L'\gls{Utente ristoratore}$^G$ deve poter visualizzare la notifica relativa all'avvenuto pagamento.                         & \autoref{usecase:Visualizzazione notifica di avvenuto pagamento}                      \\
		\hline
		RFD39       & L'\gls{Utente ristoratore}$^G$ deve poter visualizzare la notifica relativa all'inserimento di un \gls{Feedback}$^G$.                 & \autoref{usecase:Visualizzazione notifica di inserimento \gls{Feedback}$^G$}                    \\
	\end{tabularx}
	\caption{Tabella dei requisiti funzionali}
\end{table}


\begin{table}[H]
	\renewcommand{\arraystretch}{1.5}
	\centering
	\begin{tabularx}{\textwidth}{l|X|c}
		\textbf{ID} & \textbf{Descrizione}                                                                                                    & \textbf{Fonte}                                       \\
		\hline
		RFO40       & L'\gls{Utente ristoratore}$^G$ deve poter visualizzare la lista delle prenotazioni in dettaglio e con la lista degli ingredienti. & \autoref{usecase:Visualizzazione lista prenotazioni} \\
		\hline
		RFO41       & L'\gls{Utente ristoratore}$^G$ deve poter accettare una prenotazione.                                                             & \autoref{usecase:Accetta prenotazione}               \\
		\hline
		RFO42       & L'\gls{Utente ristoratore}$^G$ deve poter rifiutare una prenotazione.                                                             & \autoref{usecase:Rifiuta prenotazione}               \\
		\hline
		RFO43       & L'\gls{Utente ristoratore}$^G$ deve poter terminare una prenotazione.                                                             & \autoref{usecase:Termina prenotazione}               \\
		\hline
		RFO44       & L'\gls{Utente ristoratore}$^G$ deve poter visualizzare la lista delle ordinazioni.                                                & \autoref{usecase:Visualizzazione lista ordinazioni}  \\
		\hline
		RFD45       & L'\gls{Utente ristoratore}$^G$ deve poter modificare un \gls{Ordinazione}$^G$.                                                              & \autoref{usecase:Modifica \gls{Ordinazione}$^G$}               \\
		\hline
		RFO46       & L'\gls{Utente ristoratore}$^G$ deve poter visualizzare lo \gls{Stato}$^G$ di pagamento di una prenotazione.                                 & \autoref{usecase:Visualizzazione \gls{Stato}$^G$ di pagamento} \\
		\hline
		RFO47       & L'\gls{Utente ristoratore}$^G$ deve poter visualizzare la lista dei \gls{Feedback}$^G$.                                                     & \autoref{usecase:Visualizzazione lista \gls{Feedback}$^G$}     \\
		\hline
		RFD48       & L'\gls{Utente ristoratore}$^G$ deve poter segnalare un \gls{Feedback}$^G$.                                                                  & \autoref{usecase:Segnalazione di un \gls{Feedback}$^G$}        \\
		\hline
		RFD49       & L'\gls{Utente ristoratore}$^G$ deve poter rispondere ad un \gls{Feedback}$^G$.                                                              & \autoref{usecase:Risposta ad un \gls{Feedback}$^G$}            \\
		\hline
		RFD50       & L'\gls{Utente ristoratore}$^G$ deve poter modificare le informazioni del suo ristorante.                                          & \autoref{usecase:Modifica informazioni ristorante}   \\
		\hline
		RFO51       & L'\gls{Utente ristoratore}$^G$ deve poter gestire il menù, inserendo, eliminando e modificando dei piatti.                        & \autoref{usecase:Modifica menù}                      \\
		\hline
		RFO52       & L'\gls{Utente ristoratore}$^G$ deve poter gestire gli ingredienti, inserendo e eliminando degli ingredienti.                      & \autoref{usecase:Modifica lista ingredienti}         \\
		\hline
		RFO53       & L'\gls{Utente ristoratore}$^G$ deve poter assegnare gli ingredienti ad un piatto.                      							  & \autoref{usecase:Assegnamento ingredienti ad un piatto}         \\
		\hline
		RFO54       & L'\gls{Utente ristoratore}$^G$ deve poter visualizzare la notifica relativa all'annullamento di un \gls{Ordinazione}$^G$.                   & \autoref{usecase:Visualizzazione notifica annullamento \gls{Ordine}$^G$}         \\
		\hline
		RFO55       & L'\gls{Utente ristoratore}$^G$ deve poter visualizzare la notifica relativa all'annullamento di una prenotazione.                  & \autoref{usecase:Visualizzazione notifica annullamento prenotazione}         \\
	\end{tabularx}
	\caption{Tabella dei requisiti funzionali}
\end{table}


\subsection{Di qualità}

Di seguito viene riportata la specifica relativa ai requisiti di qualità, che delineano le caratteristiche di come un sistema
deve essere o comportarsi al fine di soddisfare le necessità dell'utente.
La presenza di ogni requisito viene giustificata riportando la fonte, che può essere un \gls{UC}$^G$ oppure presente
nel testo del \gls{Capitolato}$^G$ d'appalto. Mentre i codici univoci sottostanti indicano:
\begin{enumerate}
	\item RQO: Requisito di Qualità Obbligatorio;
	\item RQF: Requisito di Qualità Facoltativo;
	\item RQD: Requisito di Qualità Desiderabile.
\end{enumerate}

\begin{table}[H]
	\renewcommand{\arraystretch}{1.5}
	\centering
	\begin{tabularx}{\textwidth}{l|X|c}
		\textbf{ID} & \textbf{Descrizione}                                                                                                                                                                                                 & \textbf{Fonte} \\
		\hline
		RQO1        & Il codice sorgente deve essere coperto da test almeno per il 80\%                                                                                                                                                    & \gls{Capitolato}$^G$     \\
		\hline
		RQO2        & Deve essere prodotta della documentazione sulle scelte implementative e progettuali, che dovranno essere accompagnate da motivazioni.                                                                                & \gls{Capitolato}$^G$     \\
		\hline
		RQF3        & Fornire un'analisi rispetto al carico massimo supportato in numero di dispositivi e di quale sarebbe il servizio \textit{cloud} più adatto per supportarlo analizzando prezzo, stabilità del servizio ed assistenza. & \gls{Capitolato}$^G$     \\
		\hline
	\end{tabularx}
	\caption{Tabella dei requisiti di qualità}
\end{table}

\subsection{Di vincolo}

Segue la specifica relativa ai requisiti di vincolo, i quali delineano i limiti e le restrizioni che il sistema deve osservare per adempiere alle esigenze dell'utente.
La presenza di ogni requisito viene giustificata riportando la fonte, che può essere un \gls{UC}$^G$ oppure presente
nel testo del \gls{Capitolato}$^G$ d'appalto. Mentre i codici univoci sottostanti indicano:
\begin{enumerate}
	\item RVO: Requisito di Vincolo Obbligatorio;
	\item RVF: Requisito di Vincolo Facoltativo;
	\item RVD: Requisito di Vincolo Desiderabile.
\end{enumerate}

\begin{table}[H]
	\renewcommand{\arraystretch}{1.5}
	\centering
	\begin{tabularx}{\textwidth}{l|X|c}
		\textbf{ID} & \textbf{Descrizione}                                                                                                     & \textbf{Fonte} \\
		\hline
		RVO1        & L'interfaccia degli utenti deve essere un'applicazione \textit{web responsive} del tipo \textit{single page application} & \gls{Capitolato}$^G$     \\
		\hline
	\end{tabularx}
	\caption{Tabella dei requisiti di vincolo}
\end{table}
