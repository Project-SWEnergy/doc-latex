\usecaseristoratore{Modifica lista ingredienti}
\label{usecase:Modifica lista ingredienti}
\begin{itemize}
	\item \textbf{\gls{Attore}$^G$ principale:} \gls{Utente ristoratore}$^G$.

	\item \textbf{Precondizione:} L'\gls{Utente ristoratore}$^G$ ha effettuato l'accesso al Sistema (vedi \autoref{usecase:Effettua accesso}).

	\item \textbf{Postcondizione:} L'\gls{Utente ristoratore}$^G$ gestisce la lista degli ingredienti del proprio ristorante.


	\item \textbf{Scenario principale:}
	      \begin{enumerate}

		      \item L'\gls{Utente ristoratore}$^G$ può compiere le seguenti azioni per quanto riguarda la gestione della lista degli ingredienti:
		      \begin{itemize}
                \item Inserimento di un nuovo ingrediente (vedi \autoref{usecase:Inserimento ingrediente}).
                \item Eliminazione di un ingrediente (vedi \autoref{usecase:Eliminazione ingrediente}).
              \end{itemize}
		      \item Il Sistema registra le modifiche apporate alla lista ingredienti da parte del ristoratore.

	      \end{enumerate}
\end{itemize}

\subusecaseristoratore{Inserimento ingrediente}
\label{usecase:Inserimento ingrediente}
\begin{itemize}

	\item \textbf{\gls{Attore}$^G$ principale:} \gls{Utente ristoratore}$^G$.

	\item \textbf{Precondizione:} L'\gls{Utente ristoratore}$^G$ si trova nella sezione di gestione della lista ingredienti (vedi \autoref{usecase:Modifica lista ingredienti}).

	\item \textbf{Postcondizione:} L'\gls{Utente ristoratore}$^G$ ha inserito un ingrediente nella lista.

	\item \textbf{Scenario principale:}
	\begin{enumerate}
		\item L'\gls{Utente ristoratore}$^G$ inserisce un nuovo ingrediente nella lista;
		\item Il Sistema aggiorna la lista con il nuovo ingrediente inserito dal ristoratore.
	\end{enumerate}

\end{itemize}

\subusecaseristoratore{Eliminazione ingrediente}
\label{usecase:Eliminazione ingrediente}
\begin{itemize}

	\item \textbf{\gls{Attore}$^G$ principale:} \gls{Utente ristoratore}$^G$.

	\item \textbf{Precondizione:} L'\gls{Utente ristoratore}$^G$ si trova nella sezione di gestione della lista ingredienti (vedi \autoref{usecase:Modifica lista ingredienti}).

	\item \textbf{Postcondizione:} L'\gls{Utente ristoratore}$^G$ ha eliato un ingrediente dalla lista.

	\item \textbf{Scenario principale:}
	\begin{enumerate}
		\item L'\gls{Utente ristoratore}$^G$ elimina un ingrediente dalla lista;
		\item Il Sistema aggiorna la lista con l'ingrediente che è \gls{Stato}$^G$ eliminato dal ristoratore.
	\end{enumerate}

\end{itemize}
