\usecasebase{Creazione dell'\gls{Ordinazione}$^G$ collaborativa dei pasti}
\label{usecase:Creazione dell'\gls{Ordinazione}$^G$ collaborativa dei pasti}
\begin{itemize}
	\item \textbf{\gls{Attore}$^G$ principale:} \gls{Utente base}$^G$.

	\item \textbf{Precondizioni:}
	      \begin{itemize}
		      \item L'\gls{Utente base}$^G$ ha effettuato l'accesso al Sistema (vedi \autoref{usecase:Effettua accesso}).
		      \item L'\gls{Utente base}$^G$ ha effettuato una prenotazione (bedi \autoref{usecase:Prenotazione di un tavolo}).
		      \item L'\gls{Utente base}$^G$ sta visualizzando il riepilogo di una prenotazione (vedi \autoref{usecase:Visualizzazione del riepilogo prenotazione}) che si deve trovare nello \gls{Stato}$^G$: \textit{Accettata}  (vedi \autoref{usecase:Accetta prenotazione}).
	      \end{itemize}

	\item \textbf{Postcondizione:} Un \gls{Utente base}$^G$ ha ordinato le pietanze per la prenotazione effettuata.

	\item \textbf{Scenario principale:}
	      \begin{enumerate}
		      \item L'\gls{Utente base}$^G$ visualizza le ordinazioni di tutti gli altri commensali collegati alla stessa prenotazione;
		      \item L'\gls{Utente base}$^G$ crea il proprio \gls{Ordine}$^G$ (vedi \autoref{usecase:Creazione della propria \gls{Ordinazione}$^G$});
		      \item L'\gls{Utente base}$^G$ modifica il proprio \gls{Ordine}$^G$ (vedi \autoref{usecase:Modifica della propria \gls{Ordinazione}$^G$});

		      \item L'\gls{Utente base}$^G$ conferma il riepilogo dell'\gls{Ordinazione}$^G$;

		      \item Tutti gli Utenti base che partecipano all'\gls{Ordinazione}$^G$ collaborativa dei pasti devono
		            confermare il riepilogo della loro \gls{Ordinazione}$^G$;
		      \item Il Sistema memorizza l'\gls{Ordine}$^G$.
	      \end{enumerate}

	\item \textbf{Scenario secondario:}
	      \begin{itemize}
		      \item L'\gls{Utente base}$^G$ annulla l'\gls{Ordinazione}$^G$ (vedi
		            \autoref{usecase:Annullamento dell'\gls{Ordinazione}$^G$}).
		            \begin{enumerate}
			            \item L'\gls{Utente base}$^G$ annulla l'\gls{Ordinazione}$^G$;
			            \item Il Sistema aggiorna l'\gls{Ordinazione}$^G$.
		            \end{enumerate}
	      \end{itemize}
\end{itemize}


\subusecasebase{Creazione della propria \gls{Ordinazione}$^G$}
\label{usecase:Creazione della propria \gls{Ordinazione}$^G$}
\begin{itemize}
	\item \textbf{\gls{Attore}$^G$ principale:} \gls{Utente base}$^G$.

	\item \textbf{Precondizione:} L'\gls{Utente base}$^G$ sta effettuando un \gls{Ordinazione}$^G$ collaborativa dei pasti (vedi \autoref{usecase:Creazione dell'\gls{Ordinazione}$^G$ collaborativa dei pasti});

	\item \textbf{Postcondizioni:}
	      \begin{itemize}
		      \item L'\gls{Utente base}$^G$ ha creato il proprio \gls{Ordine}$^G$.
		      \item Il Sistema aggiorna le informazioni inerenti al suo \gls{Ordine}$^G$.
	      \end{itemize}

	\item \textbf{Scenario principale:}
	      \begin{enumerate}
		      \item L'\gls{Utente base}$^G$ seleziona delle pietanze (vedi \autoref{usecase:Seleziona pietanza});
		      \item L'\gls{Utente base}$^G$ conferma il proprio \gls{Ordine}$^G$;
		      \item Il Sistema aggiorna il riepilogo dell'\gls{Ordinazione}$^G$.
	      \end{enumerate}
\end{itemize}


\subsubusecasebase{Seleziona pietanza}
\label{usecase:Seleziona pietanza}
\begin{itemize}
	\item \textbf{\gls{Attore}$^G$ principale:} \gls{Utente base}$^G$.

	\item \textbf{Precondizione:} L'\gls{Utente base}$^G$ si trova nella sezione Creazione della propria \gls{Ordinazione}$^G$ (vedi \autoref{usecase:Creazione della propria \gls{Ordinazione}$^G$}).

	\item \textbf{Postcondizione:} L'\gls{Utente base}$^G$ ha selezionato una pietanza.

	\item \textbf{Scenario principale:}
	      \begin{enumerate}
		      \item L'\gls{Utente base}$^G$ seleziona tra le lista delle pietanze una che vuole aggiungere al proprio \gls{Ordine}$^G$;
		      \item L'\gls{Utente base}$^G$ conferma la sua selezione;
		      \item Il Sistema registra la sua selezione.
	      \end{enumerate}
\end{itemize}

\subusecasebase{Modifica della propria \gls{Ordinazione}$^G$}
\label{usecase:Modifica della propria \gls{Ordinazione}$^G$}
\begin{itemize}
	\item \textbf{\gls{Attore}$^G$ principale:} \gls{Utente base}$^G$.

	\item \textbf{Precondizione:}
	      \begin{itemize}
		      \item L'\gls{Utente base}$^G$ sta effettuando un \gls{Ordinazione}$^G$ collaborativa dei pasti (vedi \autoref{usecase:Creazione dell'\gls{Ordinazione}$^G$ collaborativa dei pasti});
		      \item L'\gls{Utente base}$^G$ deve rispettare un limite temporale: se vuole modificare il proprio \gls{Ordine}$^G$ deve farlo ventiquattro ore prima a partire dall'ora stabilita della prenotazione.
	      \end{itemize}

	\item \textbf{Postcondizioni:}
	      \begin{itemize}
		      \item L'\gls{Utente base}$^G$ ha modificato il proprio \gls{Ordine}$^G$.
		      \item Il Sistema aggiorna le informazioni inerenti al suo \gls{Ordine}$^G$.
	      \end{itemize}

	\item \textbf{Scenario principale:}
	      \begin{enumerate}
		      \item L'\gls{Utente base}$^G$ modifica una propria pietanza (vedi \autoref{usecase:Modifica pietanza});
		      \item L'\gls{Utente base}$^G$ conferma il proprio \gls{Ordine}$^G$;
		      \item Il Sistema aggiorna il riepilogo dell'\gls{Ordinazione}$^G$.
	      \end{enumerate}
\end{itemize}

\subsubusecasebase{Modifica pietanza}
\label{usecase:Modifica pietanza}
\begin{itemize}
	\item \textbf{\gls{Attore}$^G$ principale:} \gls{Utente base}$^G$.

	\item \textbf{Precondizioni:}  L'\gls{Utente base}$^G$ si trova nella sezione Modifica della propria \gls{Ordinazione}$^G$ (vedi \autoref{usecase:Modifica della propria \gls{Ordinazione}$^G$}).


	\item \textbf{Postcondizione:} L'\gls{Utente base}$^G$ ha modificato una pietanza.

	\item \textbf{Scenario principale:}
	      \begin{enumerate}
		      \item L'\gls{Utente base}$^G$ seleziona tra le pietanze del suo \gls{Ordine}$^G$ una che vuole modificare, ovvero:
		            \begin{itemize}
			            \item Modifica della quantità della pietanza selezionata.
			            \item Rimozione di ingredienti della pietanza selezionata.
			            \item Aggiunta di ingredienti della pietanza selezionata.
		            \end{itemize}
		      \item L'\gls{Utente base}$^G$ conferma la sua modifica;
		      \item Il Sistema registra la sua modifica.
	      \end{enumerate}
\end{itemize}
