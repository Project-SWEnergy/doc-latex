\usecasegenerico{Effettua registrazione \gls{Utente ristoratore}$^G$}
\label{usecase:Effettua registrazione \gls{Utente ristoratore}$^G$}
\begin{itemize}

	\item \textbf{\gls{Attore}$^G$ principale:} \gls{Utente generico}$^G$.
	\item \textbf{\gls{Attore}$^G$ secondario:} Sistema di autenticazione esterno$^G$. 

	\item \textbf{Precondizioni:} 
	\begin{itemize}
        \item  L'\gls{Utente generico}$^G$ è connesso al Sistema.
        \item  L'\gls{Utente generico}$^G$ desidera registrarsi come \gls{Utente ristoratore}$^G$.
    \end{itemize}
    

	\item \textbf{Postcondizioni:} 
    \begin{itemize}
        \item  Tutte le informazioni inserite durante la fase di registrazione sono state verificate dal Sistema.
        \item  L'\gls{Utente generico}$^G$ ha creato un \textit{account} come \gls{Utente ristoratore}$^G$.
    \end{itemize}

	\item \textbf{Scenario principale:}
	\begin{enumerate}

            \item Se l'\gls{Utente generico}$^G$ ha selezionato la creazione attraverso terze parti, il Sistema reindirizza l'utente alla pagina del Sistema esterno di terze parti (vedi \autoref{usecase:Effettua accesso per terze parti});
            \item Se l'\gls{Utente generico}$^G$ ha scelto l'accesso tradizionale, dovrà inserire:
            \begin{itemize}
                \item Nome.
                \item Cognome.
                \item \textit{Email}.
                \item \textit{Password}.
            \end{itemize}

            \item Indipendentemente dalla modalità di creazione dell' \textit{account}, l'utente dovrà infine inserire le seguenti informazioni relative al ristorante:
                \begin{itemize}
                    \item Denominazione.
                    \item Recapito.
                    \item Orari di apertura.
                    \item Numero dei coperti disponibili.
                    \item Tipologia di c\gls{UC}$^G$ina.
                \end{itemize}
	\end{enumerate}
	
\end{itemize}