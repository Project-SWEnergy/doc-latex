\section{Modello di sviluppo}

\subsection{Modello incrementale}
SWEnergy ha deciso di adottare il modello di sviluppo incrementale, che prevede
la suddivisione del progetto in incrementi. Al completamento di ciascun
incremento viene rilasciata una nuova versione del prodotto, che è mostrata al
proponente durante le riunioni di revisione.\\
Poiché il \textit{team} non ha mai
lavorato in ambito professionale, il modello di sviluppo è ispirato al
\textit{framework} Scrum con \textit{sprint} di due settimane, con qualche
modifica per adattarlo alle esigenze del progetto. In particolare,
SWEnergy introd\gls{\gls{UC}$^G$E} una retrospettiva \textit{in media res} per valutare il
lavoro svolto e apportare eventuali modifiche al processo di sviluppo.
In aggiunta, non sono previste i \textit{daily stand-up meeting}, in quanto il
gruppo ritiene che la cadenza dei suddetti sia troppo elevata considerando
che il progetto è svolto da studenti universitari e non da lavoratori a tempo
pieno.

\subsection{Iterazioni}

\subsubsection{\textit{Sprint}}
Uno \textit{sprint} ha una durata di due settimane. Durante questo periodo, il
gruppo si impegna a sviluppare l'incremento del prodotto concordato con il
proponente.

La durata di uno \textit{sprint}
consente di ricevere \textit{\gls{Feedback}$^G$} frequenti dal proponente e di apportare
modifiche al prodotto in modo tempestivo.
permette al \textit{team} di effettuare un mini-\textit{sprint} di una
settimana;
in aggiunta, permette al \textit{team} di risolvere eventuali problemi o dubbi
con il proponente in modo dinamico, flessibile e tempestivo.
I mini-\textit{sprint} aumentano la frequenza delle retrospettive e mantengono
il gruppo focalizzato sul lavoro da svolgere. Permettono di valutare lo
svolgimento delle attività durante lo \textit{sprint} e di apportare modifiche
al processo di sviluppo adattando il lavoro alle esigenze del progetto.

\subsubsection{Mini-\textit{sprint}}
Questa iterazione emula il \textit{framework} Scrum, con una cadenza
settimanale. Si tratta di uno \textit{sprint}
interno al gruppo: il proponente non viene coinvolto. Al termine di un
mini-\textit{sprint} potrebbe corrispondere un cambio dei ruoli, in base alle
esigenze del progetto e del gruppo.

\subsection{Eventi}

\subsubsection{SAL}
Lo \gls{Stato}$^G$ di Avanzamento del Lavoro (SAL) è un incontro con il proponente che
avviene ogni due settimane di venerdì. Queste riunioni sono utili per
condividere i \textit{\gls{Feedback}$^G$} in entrambe le direzioni: il proponente può
valutare il lavoro svolto dal gruppo; e SWEnergy può esprimere le proprie
opinioni sul prodotto. Di seguito sono riportate le attività principali che
avvengono durante un SAL:

\begin{itemize}
	\item \textbf{\textit{Sprint review}}: Il gruppo presenta il lavoro svolto
	      durante lo
	      \textit{sprint} e il proponente fornisce dei \textit{\gls{Feedback}$^G$} sul
	      prodotto. In aggiunta, il gruppo può porre domande per chiarire
	      eventuali dubbi in merito ai requisiti e alle funzionalità richieste,
	      oppure in merito all'implementazione di queste ultime.

	\item \textbf{\textit{Sprint retrospective}}: Il gruppo discute sulle modalità
	      di lavoro. In particolare, valuta se il processo di sviluppo
	      è \gls{Stato}$^G$ efficace ed efficiente ed in quale modo sia migliorabile.
	      Sono chiesti consigli al proponente in merito all'organizzazione del
	      lavoro; sono riportati eventuali problemi riscontrati durante lo
	      \textit{sprint} e sono proposte soluzioni per risolverli.

	\item \textbf{\textit{Sprint planning}}: Il gruppo e il proponente concordano
	      l'incremento del prodotto da sviluppare durante lo \textit{sprint}
	      s\gls{UC}$^G$cessivo; ovvero che cosa inserire nello \textit{sprint backlog}.
\end{itemize}

\subsubsection{\textit{Stand-up}}
Il nome \textit{stand-up} è ispirato ai \textit{daily stand-up meeting} del
\textit{framework} Scrum. Si noti che l'incontro con il proponente avviene un
venerdì su due; mentre le \textit{stand-up}, gli incontri all'inizio
e al termine di un mini-\textit{sprint}, hanno luogo ogni domenica,
per dare modo al responsabile di considerare i \textit{\gls{Feedback}$^G$} del proponente
e di pianificare l'iterazione s\gls{UC}$^G$cessiva. L'organizzazione viene poi discussa
durante la \textit{stand-up}. Di seguito sono riportate le attività principali
che avvengono durante una \textit{stand-up}:

\begin{itemize}
	\item \textbf{\textit{Brainstorming}}: Il responsabile riassume il lavoro
	      svolto durante la settimana e ciascun membro del gruppo può arricchire
	      la spiegazione con le proprie esperienze, per esempio, descrivendo le
	      difficoltà incontrate e le soluzioni adottate.

	\item \textbf{Retrospettiva}: Il gruppo discute sulle modalità di
	      lavoro. In particolare, valuta se il processo di sviluppo è \gls{Stato}$^G$
	      efficace ed efficiente ed in quale modo sia migliorabile. Sono
	      riportati eventuali problemi riscontrati durante il
	      mini-\textit{sprint} e sono proposte soluzioni per risolverli.
	      Eventualmente si prende nota dei problemi per discuterne con il
	      proponente durante il SAL; oppure per domandare consigli al
	      committente.

	\item \textbf{Pianificazione}: Il responsabile mostra la pianificazione del
	      mini-\textit{sprint} s\gls{UC}$^G$cessivo. I membri del gruppo possono
	      intervenire per proporre miglioramenti o per chiarire eventuali dubbi.
	      Infine, il responsabile assegna i compiti ai membri del gruppo,
	      secondo le loro disponibilità, capacità e preferenze.
\end{itemize}

Risulta utile dividere le \textit{stand-up} in due gruppi: le \textit{stand-up}
che avvengono la domenica s\gls{UC}$^G$cessiva al SAL sono dedicate alla pianificazione
del mini-\textit{sprint} s\gls{UC}$^G$cessivo; mentre le \textit{stand-up} che avvengono
la domenica s\gls{UC}$^G$cessiva sono principalemente dedicate alla retrospettiva del
mini-\textit{sprint} appena concluso. Queste ultime sono utilizzate anche per
organizzare il SAL s\gls{UC}$^G$cessivo: il responsabile raccoglie i problemi riscontrati;
organizza le domande da porre al proponente; e pianifica l'\gls{Ordine}$^G$ del giorno del
SAL. Il \textit{team} può richiedere qualche modifica all'\gls{Ordine}$^G$ del giorno. In
questo modo, è possibile anticipare eventuali problemi e risolverli prima del
SAL, evitando di sprecare tempo prezioso. In aggiunta, il responsabile può
condividere l'\gls{Ordine}$^G$ del giorno con il proponente prima del SAL. Si noti
che è prevista uno \textit{stand-up}, della durata di circa 30
minuti, anche tra il giovedì o il venerdì precedente al SAL, per effetturare un
veloce \textit{brainsorming}, raccogliere le ultime domande e modificare
l'\gls{Ordine}$^G$ del giorno in base alle risposte del proponente e alle necessità del
gruppo.

\subsection{Motivazioni}

SWEnergy ha deciso di organizzarsi come descritto in precedenza più per
necessità che per scelta: durante i corsi di Ingegneria del Software e di Metodi
e Tecnologie per lo Sviluppo Software, i membri del gruppo hanno appreso i
concetti fondamentali del \textit{framework} Scrum. Dunque, il \textit{team} ha
deciso di metterli in pratica. Come già accennato, il gruppo non ha mai lavorato
in un ambito professionale e non ha esperienza con i metodi di lavoro ed
organizzativi. In aggiunta, il proponente ha richiesto una pianificazione di
almeno due settimane.
Con la seguente organizzazione, SWEnergy spera di riuscire a soddisfare le
esigenze del proponente mitigando i rischi individuati durante l'analisi dei
rischi. Il \textit{framework} Scrum dovrebbe fornire i seguenti vantaggi:

\begin{enumerate}
	\item \textbf{Flessibilità}: Il gruppo può adattare il processo di sviluppo
	      alle esigenze del progetto; in particolare, può modificare la
	      pianificazione in base alle esigenze del proponente e ai danni
	      riscontrati.

	\item \textbf{Comunicazione trasparente}: Il gruppo rilascia incrementi del
	      prodotto in modo tempestivo. In aggiunta, il proponente valuta il
	      lavoro svolto dal gruppo e fornisce \textit{\gls{Feedback}$^G$} in itinere;
	      infine rimane aggiornato in merito allo \textit{status quo} del
	      progetto.

	\item \textbf{Miglioramento continuo}: Le retrospettive permettono al
	      gruppo di valutare il processo di sviluppo e di apportare
	      modifiche per migliorarlo. In aggiunta, il gruppo può
	      discutere con il proponente o con il committente per ricevere consigli
	      e suggerimenti per migliorare l'organizzazione e il metodo di lavoro.

	\item \textbf{Monitoraggio costante}: La pianificazione basata sugli
	      \textit{sprint} permette di identificare e affrontare rischi in modo
	      tempestivo, rid\gls{\gls{UC}$^G$E}ndo la possibilità di ritardi gravi e un rincaro del
	      progetto.
\end{enumerate}
