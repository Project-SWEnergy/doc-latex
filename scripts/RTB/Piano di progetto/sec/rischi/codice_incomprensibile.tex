\risktech{Codice incomprensibile}
\label{risk:codice incomprensibile}
\begin{itemize}
	\item \textbf{Descrizione}: il codice prodotto da qualche membro del gruppo
	      è di difficile comprensione per gli altri membri del gruppo.
	\item \textbf{Identificazione}:
	      \begin{itemize}
		      \item \textit{code review}: durante la verifica del codice, i
		            verificatori possono riscontrare difficoltà nella comprensione
		            del codice;

		      \item \textit{testing}: durante la fase di testing, i tester possono
		            riscontrare difficoltà nella comprensione del codice;

		      \item dopo un lasso di tempo ampio, i membri del gruppo possono
		            riscontrare difficoltà nella comprensione del codice.
	      \end{itemize}
	\item \textbf{\gls{Mitigazione}$^G$}:
	      \begin{itemize}
		      \item \textit{Way of working}: il gruppo stila delle linee guida
		            da seguire per la stesura del codice, in modo da uniformare
		            la stesura del codice e facilitarne la comprensione.

		      \item documentazione: il codice deve essere documentato in modo
		            chiaro e preciso, in modo da facilitarne la comprensione. Per
		            maggiori chiarimenti si rimanda al "\textit{Way of working}".

		      \item \textit{testing}: il codice deve essere te\gls{Stato}$^G$ in modo
		            approfondito, per facilitarne la comprensione e illustrarne
		            i casi d'uso.

		      \item librerie apposite: il \textit{source code} comprende delle
		            librerie apposite per aiutare la stesura della
		            documentazione, per esempio \textit{OpenAPI} per la
		            documentazione delle API.
	      \end{itemize}
\end{itemize}
