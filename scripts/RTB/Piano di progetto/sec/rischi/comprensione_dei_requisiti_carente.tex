\riskplan{Comprensione dei requisiti carente}
\label{risk:comprensione dei requisiti carente}
\begin{itemize}
	\item \textbf{Descrizione}:
	      Il gruppo o qualche suo membro potrebbe non essere in grado di
	      comprendere i requisiti del progetto, oppure potrebbe riscontrare
	      delle difficoltà a causa di una cattiva comprensione dei requisiti.
	\item \textbf{Identificazione}:
	      \begin{itemize}
		      \item dubbi: i membri del gruppo hanno dei dubbi in merito ai
		            requisiti;

		      \item dibattiti sui requisiti: i membri del gruppo
		            discutono tra loro in merito ai requisiti;

		      \item discrepanza nella progettazione: i membri del gruppo
		            progettano in modo diverso, a causa di una cattiva
		            comprensione dei requisiti.
	      \end{itemize}
	\item \textbf{\gls{Mitigazione}$^G$}:
	      \begin{itemize}
		      \item Dibattito interno: SWEnergy si è diviso in coppie per
		            approfondire i casi d'uso e i requisiti del progetto. Poi si
		            è tenuta una riunione interna in cui ciasuna coppia ha
		            esposto i propri dubbi e le proprie considerazioni. In
		            questo modo, si è cercato di chiarire i dubbi e di
		            uniformare la comprensione dei requisiti.

		      \item "Analisi dei requisiti": il metodo più formale per ovviare a
		            questa situazione risulta essere l'"Analisi dei requisiti".
		            I requisiti dovrebbero essere chiari e completi. Inoltre,
		            il documento contiene i casi d'uso, che aiutano a
		            comprendere meglio i requisiti concordati con il proponente.

		      \item dialogo con il proponente: si discute con il proponente in
		            merito ai requisiti, per chiarire eventuali dubbi e per
		            decidere in maggiore dettaglio le funzionalità del prodotto.

		      \item messaggi tempestivi con il proponente: in caso di dubbi
		            semplici e veloci da risolvere, si inviano dei messaggi al
		            proponente, per ottenere una risposta tempestiva.
	      \end{itemize}
\end{itemize}
