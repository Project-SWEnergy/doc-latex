\riskcom{Comunicazione interna carente}
\label{risk:comunicazione interna carente}
\begin{itemize}
	\item \textbf{Descrizione}:
	      La comunicazione interna non è efficace ed efficiente, causando riunioni
	      interne più lunghe del previsto e rallentando le attività.
	\item \textbf{Identificazione}:
	      \begin{itemize}
		      \item dubbi ripetuti: durante le riunioni interne, i membri del
		            gruppo possono porre domande già presentate in precedenza;

		      \item riunioni interne lunghe: le riunioni interne possono
		            protrarsi oltre il tempo previsto;

		      \item qualche membro del gruppo non sa che cosa deve fare;

		      \item qualche membro del gruppo non sa come fare qualcosa;

		      \item fraintendimenti frequenti: i membri del gruppo possono
		            fraintendersi frequentemente.

		      \item durante le retrospettive, i membri del gruppo possono
		            lamentarsi di una comunicazione interna carente.
	      \end{itemize}
	\item \textbf{\gls{Mitigazione}$^G$}:
	      \begin{itemize}
		      \item "\textit{Way of working}": il gruppo stila delle linee guida
		            da seguire per la comunicazione interna, in modo da
		            uniformare la comunicazione interna e facilitarla. Sono
		            dunque creati dei canali di comunicazione come
		            \textit{\gls{Telegram}$^G$}, \textit{\gls{Discord}$^G$} oppure su
		            \textit{\gls{\gls{Git}$^G$Hub}};

		      \item documentazione: il gruppo stila una documentazione
		            adeguata per facilitare la comunicazione interna. A seconda
		            dell'argomento la documentazione può avere diverse forme.
		            Ciascun macro argomento deve avere una sezione all'interno
		            del "\textit{Way of working}" in cui viene descritta la
		            documentazione adatta;

		      \item \textit{meeting} frequenti: il gruppo si impegna a
		            tenere riunioni interne frequenti, in modo da ridurre la
		            durata delle riunioni interne e facilitare la comunicazione
		            interna;

		      \item \gls{Ordine}$^G$ del giorno: ciascuna riunione deve avere l'\gls{Ordine}$^G$ del
		            giorno ben definito, per discutere di tutti gli argomenti
		            utili allo sviluppo del progetto e per definire la durata di
		            ciascuno dei punti dell'\gls{Ordine}$^G$ del giorno;

		      \item retrospettiva: durante la retrospettiva, il gruppo deve
		            pensare a soluzioni \textit{ad hoc} per migliorare la
		            comunicazione interna.
	      \end{itemize}
\end{itemize}
