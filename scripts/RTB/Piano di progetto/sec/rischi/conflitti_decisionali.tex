\riskcom{Conflitti decisionali}
\label{risk:conflitti decisionali}
\begin{itemize}
	\item \textbf{Descrizione}:
	      Il gruppo potrebbe dilungarsi nella discussione di una sola idea, senza
	      raggiungere una decisione finale.
	\item \textbf{Identificazione}:
	      \begin{itemize}
		      \item un punto dell'\gls{Ordine}$^G$ del giorno subisce un ritardo grave;
	      \end{itemize}
	\item \textbf{\gls{Mitigazione}$^G$}:
	      \begin{itemize}
		      \item dibattito: i membri del gruppo discutono riguardo
		            all'importanza del punto dell'\gls{Ordine}$^G$ del giorno, per capire se
		            è necessario approfondire la discussione o meno;

		      \item approfondimento: se il punto dell'\gls{Ordine}$^G$ del giorno è
		            ritenuto importante, almeno due membri del gruppo si impegnano
		            a studiare i pro ed i contro delle varie soluzioni possibili.
		            Può essere chiesto un supporto al proponente oppure al
		            committente per chiarire i dubbi;

		      \item votazione: alla fine del dibattito i membri del gruppo
		            votano per la soluzione che ritengono più opportuna. La
		            votazione si ritiene conclusa quando la maggioranza dei
		            membri del gruppo ha espresso la propria preferenza e il
		            risultato non è un pareggio.

		      \item il responsabile ha il compito di vigilare sul corretto
		            svolgimento del dibattito e della votazione, in modo da
		            evitare che si dilunghi troppo.
	      \end{itemize}
\end{itemize}
