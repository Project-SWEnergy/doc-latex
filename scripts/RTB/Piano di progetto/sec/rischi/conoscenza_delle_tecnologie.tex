\risktech{Conoscenza delle tecnologie carente}
\label{risk:conoscenza tecnologie carente}
\begin{itemize}
	\item \textbf{Descrizione}:
	      nello sviluppo del progetto, si può incorrere nella situazione in cui
	      almeno qualche membro non conosce almeno una tecnologia adottata dal
	      gruppo e necessaria per lo sviluppo del progetto.

	\item \textbf{Identificazione}: il \textit{team} ha individuato le
	      tecnologie conosciute dal gruppo. Con il proponente sono state
	      discusse e concordate le tecnologie da utilizzare per lo sviluppo del
	      progetto. In questo modo, sono state individuate le tecnologie
	      non conosciute dal gruppo.

	\item \textbf{\gls{Mitigazione}$^G$}:
	      \begin{itemize}
		      \item \textit{workshop} interni: il \textit{team} sceglie
		            una o due persone per ogni tecnologia non conosciuta dal
		            gruppo. Le persone scelte si occupano di approfondire la
		            tecnologia e di organizzare un \textit{workshop} interno.
		            Le persone scelte svilupperanno inoltre degli esempi di
		            codice per illustrare l'utilizzo della tecnologia e degli
		            appunti da condividere;

		      \item seminari con il proponente: il \textit{team} partecipa a
		            dei seminari organizzati con il proponente, per approfondire
		            le tecnologie non conosciute dal gruppo. Il proponente
		            spiegherà le tecnologie e fornirà degli esempi di codice
		            per illustrarne l'utilizzo;

		      \item dialogo con il proponente: il \textit{team} può
		            contattare il proponente per chiedere chiarimenti sulle
		            tecnologie non conosciute dal gruppo.

		      \item documentazione: il \textit{team} può consultare la
		            documentazione ufficiale delle tecnologie non conosciute
		            dal gruppo.

		      \item \textit{pair programming}: il codice viene sviluppato con
		            almeno un altro membro del gruppo. Le modalità di lavoro
		            sono meglio descritte nel documento "\textit{Way of
			            working}".

		      \item \textit{code review}: il codice viene revisionato da
		            almeno un altro membro del gruppo. Le modalità di lavoro
		            sono meglio descritte nel documento "\textit{Way of
			            working}".

		      \item divisione del \textit{front-end} e del
		            \textit{back-end}: il \textit{team} si divide in due
		            sottogruppi, uno che si occupa del \textit{front-end} e
		            l'altro del \textit{back-end}. In questo modo, si diminuisce
		            l'\textit{overhead} di comunicazione e di cambio di contesto
		            tra le due tecnologie. I due gruppi si scambiano i ruoli al
		            termine della prima fase del progetto: RTB.
	      \end{itemize}
\end{itemize}
